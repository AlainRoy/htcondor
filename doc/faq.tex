
\index{Condor!FAQ|(}
\index{Condor!Frequently Asked Questions|(}
\index{FAQ|(}
\index{Frequently Asked Questions|(}

This is where you can find quick answers to some commonly asked
questions about Condor.

%%%%%%%%%%%%%%%%%%%%%%%%%%%%%%%%%%%%%%%%%%%%%%%%
\section{Obtaining \& Installing Condor}
%%%%%%%%%%%%%%%%%%%%%%%%%%%%%%%%%%%%%%%%%%%%%%%%

\index{FAQ!installing Condor}
%%%%%%%%%%%%%%%%%%%%%%%%%%%%%%%%%%%%%%%%%%%%%%%%
\subsection*{Where can I download Condor?}
%%%%%%%%%%%%%%%%%%%%%%%%%%%%%%%%%%%%%%%%%%%%%%%%
\index{Condor!downloading}
\index{Condor!distribution}
\index{Condor!getting}
\index{Condor!binaries}

Condor can be downloaded from the mirrors listed at
\URL{http://www.cs.wisc.edu/condor/downloads}.  

%%%%%%%%%%%%%%%%%%%%%%%%%%%%%%%%%%%%%%%%%%%%%%%%
\subsection*{When I click to download Condor, it sends me back to the downloads page!}
%%%%%%%%%%%%%%%%%%%%%%%%%%%%%%%%%%%%%%%%%%%%%%%%

If you are trying to download Condor through a web proxy, try
disabling it.
Our web site uses the ``referring page'' as you navigate through our
download menus in order to give you the right version of Condor, but
sometimes proxies block this information from reaching our web site.

%%%%%%%%%%%%%%%%%%%%%%%%%%%%%%%%%%%%%%%%%%%%%%%%
\subsection*{What platforms are supported?}
%%%%%%%%%%%%%%%%%%%%%%%%%%%%%%%%%%%%%%%%%%%%%%%%

Supported platforms are listed in section~\ref{sec:Availability}, on
page~\pageref{sec:Availability}.
There is also platform-specific information at
Chapter~\ref{platforms} on page~\pageref{platforms}.

%%%%%%%%%%%%%%%%%%%%%%%%%%%%%%%%%%%%%%%%%%%%%%%%
\subsection*{Can I get the source code?}
%%%%%%%%%%%%%%%%%%%%%%%%%%%%%%%%%%%%%%%%%%%%%%%%
\index{Condor!source code}

For Condor version 7.0.0 and later releases,
the Condor source code is available for 
public download with the binary distributions.

%%%%%%%%%%%%%%%%%%%%%%%%%%%%%%%%%%%%%%%%%%%%%%%%
\subsection*{What is Personal Condor?}
%%%%%%%%%%%%%%%%%%%%%%%%%%%%%%%%%%%%%%%%%%%%%%%%
\index{Condor!Personal}
\index{Personal Condor}

Personal Condor is a term used to describe a specific style of Condor
installation suited for individual users who do not have their own
pool of machines, but want to submit Condor jobs to run elsewhere.

A Personal Condor is essentially a one-machine, self-contained Condor
pool which can use \Term{flocking} to access resources in other Condor
pools.
See Section~\ref{sec:Flocking}, on page~\pageref{sec:Flocking} for
more information on flocking.


%%%%%%%%%%%%%%%%%%%%%%%%%%%%%%%%%%%%%%%%%%%%%%%%
\subsection*{What do I do now? My installation of Condor does not work.}
%%%%%%%%%%%%%%%%%%%%%%%%%%%%%%%%%%%%%%%%%%%%%%%%

\index{port usage!FAQ on communication errors}
What to do to get Condor running properly depends on what sort of
error occurs. 
One common error category are communication errors.
Condor daemon log files report a failure to bind.
For example:

\footnotesize
\begin{verbatim}
(date and time) Failed to bind to command ReliSock
\end{verbatim}
\normalsize

Or, the errors in the various log files may be of the form:

\footnotesize
\begin{verbatim}
(date and time) Error sending update to collector(s)
(date and time) Can't send end_of_message
(date and time) Error sending UDP update to the collector

(date and time) failed to update central manager

(date and time) Can't send EOM to the collector
\end{verbatim}
\normalsize

This problem can also be observed by running \Condor{status}.
It will give a message of the form:
\footnotesize
\begin{verbatim}
Error:  Could not fetch ads --- error communication error
\end{verbatim}
\normalsize

To solve this problem, understand that
Condor uses the first network interface it sees on the machine.
Since machines often have more than one interface,
this problem usually implies that the wrong network
interface is being used.  It also may be the case that
the system simply has the wrong IP address configured.

It is incorrect to use the localhost network interface.
This has IP address 127.0.0.1 on all machines.
To check if this incorrect IP address is being used,
look at the contents of the
CollectorLog file on the pool's
your central manager right after it is started.  
The contents will be of the form:

\footnotesize
\begin{verbatim}
5/25 15:39:33 ******************************************************
5/25 15:39:33 ** condor_collector (CONDOR_COLLECTOR) STARTING UP
5/25 15:39:33 ** $CondorVersion: 6.2.0 Mar 16 2001 $
5/25 15:39:33 ** $CondorPlatform: INTEL-LINUX-GLIBC21 $
5/25 15:39:33 ** PID = 18658
5/25 15:39:33 ******************************************************
5/25 15:39:33 DaemonCore: Command Socket at <128.105.101.15:9618>
\end{verbatim}
\normalsize

The last line tells the IP address and port the collector has
bound to and is listening on.
If the IP address is 127.0.0.1, then Condor is definitely using the wrong
network interface.

There are two solutions to this problem.
One solution changes the order of the network interfaces.
The preferred solution
sets which network interface Condor should use
by adding the following parameter to the
local Condor configuration file:

\begin{verbatim}
NETWORK_INTERFACE = machine-ip-address
\end{verbatim}

Where \verb@machine-ip-address@ is the IP address of the interface you wish
Condor to use.

%%%%%%%%%%%%%%%%%%%%%%%%%%%%%%%%%%%%%%%%%%%%%%%%
\subsection*{After an installation of Condor, why do the daemons refuse to start?}
%%%%%%%%%%%%%%%%%%%%%%%%%%%%%%%%%%%%%%%%%%%%%%%%

This message appears in the log files:
\footnotesize
\begin{verbatim}
ERROR "The following configuration macros appear to contain default values 
that must be changed before Condor will run.  These macros are:
hostallow_write 
(found on line 1853 of /scratch/adesmet/TRUNK/work/src/localdir/condor_config)"
at line 217 in file condor_config.C
\end{verbatim}
\normalsize

As of Condor 6.8.0, if 
Condor sees the bare key word: 
\Expr{YOU\_MUST\_CHANGE\_THIS\_INVALID\_CONDOR\_CONFIGURATION\_VALUE}
as the value of a configuration file entry,
Condor daemons will log the given error message and exit.

By default, an installation of Condor 6.8.0 and later releases
will have the
configuration file entry \MacroNI{HOSTALLOW\_WRITE} set to the above sentinel
value. 
The Condor administrator must alter this value to be the correct domain
or IP addresses that the administrator desires.
The wild card character (\verb@*@) may be used to define this entry,
but that allows anyone, from anywhere,
to submit jobs into the pool.
A better value will be of the form \Expr{*.domainname.com}.

%%%%%%%%%%%%%%%%%%%%%%%%%%%%%%%%%%%%%%%%%%%%%%%%
\subsection*{Why do standard universe jobs never run after an upgrade?}
%%%%%%%%%%%%%%%%%%%%%%%%%%%%%%%%%%%%%%%%%%%%%%%%
\label{sec:checkpoint-platform-faq}

Standard universe jobs that remain in the job queue across an upgrade
from any Condor release previous to 6.7.15
to any Condor release of 6.7.15 or more recent
cannot run.
They are missing a required ClassAd attribute
(\Attr{LastCheckpointPlatform}) added for
all standard universe jobs as of Condor version 6.7.15.
This new attribute describes the platform where a job was
running when it produced a checkpoint.
The attribute is utilized to identify platforms capable of 
continuing the job (using the checkpoint).

This attribute becomes necessary due to bugs in some Linux kernels.
A standard universe job may be continued on some, but not all
Linux machines.
And, the \Attr{CkptOpSys} attribute is not specific enough
to be utilized.

There are two possible solutions for these standard universe jobs that
cannot run, yet are in the queue:

\begin{enumerate}
  \item Remove and resubmit the standard universe jobs that
  remain in the queue across the upgrade. 
  This includes all standard universe jobs that have flocked in to 
  the pool.
  Note that the resubmitted jobs will start over again from the beginning.

  \item For each standard universe job in the queue,
  modify its job ClassAd such that it can possibly run within the
  upgraded pool.
  If the job has already run and produced a checkpoint on a machine
  before the upgrade, determine the machine that produced the checkpoint
  using the \Attr{LastRemoteHost} attribute
  in the job's ClassAd.
  Then look at that machine's ClassAd (after the upgrade) to
  determine and extract the value of the \Attr{CheckpointPlatform} attribute.
  Add this (using \Condor{qedit}) as the value of the 
  new attribute \Attr{LastCheckpointPlatform} in the job's ClassAd.
  Note that this operation must also have to be performed on standard
  universe jobs flocking in to an upgraded pool. 
  It is recommended that pools that flock between each other upgrade to a
  post 6.7.15 version of Condor.
\end{enumerate}

Note that if the upgrade to Condor takes place at the same time
as a platform change (such as booting an upgraded kernel),
there is no way to properly set the \Attr{LastCheckpointPlatform} attribute.
The only option is to remove and resubmit the standard universe jobs.

%%%%%%%%%%%%%%%%%%%%%%%%%%%%%%%%%%%%%%%%%%%%%%%%
\section{Setting up Condor}
%%%%%%%%%%%%%%%%%%%%%%%%%%%%%%%%%%%%%%%%%%%%%%%%

%%%%%%%%%%%%%%%%%%%%%%%%%%%%%%%%%%%%%%%%%%%%%%%%
\subsection*{How do I set up a central manager on a machine with multiple network interfaces?}
%%%%%%%%%%%%%%%%%%%%%%%%%%%%%%%%%%%%%%%%%%%%%%%%

Please see section~\ref{sec:Multiple-Interfaces} on 
page~\pageref{sec:Multiple-Interfaces}.

%%%%%%%%%%%%%%%%%%%%%%%%%%%%%%%%%%%%%%%%%%%%%%%%
\subsection*{How do I get more than one job to run on my SMP machine?}
%%%%%%%%%%%%%%%%%%%%%%%%%%%%%%%%%%%%%%%%%%%%%%%%

Condor will automatically recognize a SMP machine and advertise each
CPU of the machine separately.
For more details, see section~\ref{sec:Configuring-SMP} on
page~\pageref{sec:Configuring-SMP}.

%%%%%%%%%%%%%%%%%%%%%%%%%%%%%%%%%%%%%%%%%%%%%%%%
\subsection*{How do I configure a separate policy for the CPUs of an SMP machine?}
%%%%%%%%%%%%%%%%%%%%%%%%%%%%%%%%%%%%%%%%%%%%%%%%

Please see section~\ref{sec:Configuring-SMP} on
page~\pageref{sec:Configuring-SMP} for a lengthy discussion on
this topic.

%%%%%%%%%%%%%%%%%%%%%%%%%%%%%%%%%%%%%%%%%%%%%%%%
\subsection*{How do I set up my machines so that only specific users' jobs will run on them?}
%%%%%%%%%%%%%%%%%%%%%%%%%%%%%%%%%%%%%%%%%%%%%%%%
\index{running a job!on only certain machines}

Restrictions on what jobs will run on a given resource are
enforced by only starting jobs that meet specific constraints,
and these constraints are specified as part of the configuration.

To specify that a given machine should only run certain users' jobs,
and always run the jobs regardless of other activity on the machine,
load average, etc.,
place the following entry in the
machine's Condor configuration file:

\footnotesize
\begin{verbatim}
START = ( (User == "userfoo@baz.edu") || \
          (User == "userbar@baz.edu") )
\end{verbatim}
\normalsize

A more likely scenario is that the machine is restricted to run
only specific users' jobs, contingent on the machine not having
other interactive activity and not being heavily loaded.
The following entries are in the machine's Condor configuration file. 
Note that extra configuration variables are defined to make 
the \MacroNI{START} variable easier to read.

\footnotesize
\begin{verbatim}
# Only start jobs if:
# 1) the job is owned by the allowed users, AND
# 2) the keyboard has been idle long enough, AND
# 3) the load average is low enough OR the machine is currently
#    running a Condor job, and would therefore accept running
#    a different one
AllowedUser    = ( (User == "userfoo@baz.edu") || \
                   (User == "userbar@baz.edu") )
KeyboardUnused = (KeyboardIdle > $(StartIdleTime))
NoOwnerLoad    = ($(CPUIdle) || (State != "Unclaimed" && State != "Owner"))
START          = $(AllowedUser) && $(KeyboardUnused) && $(NoOwnerLoad)
\end{verbatim}
\normalsize

To configure multiple machines to do so, create a common
configuration file containing this entry for them to share.

%%%%%%%%%%%%%%%%%%%%%%%%%%%%%%%%%%%%%%%%%%%%%%%%
\subsection*{How do I configure Condor to run my jobs only on machines that have the right packages installed?}
%%%%%%%%%%%%%%%%%%%%%%%%%%%%%%%%%%%%%%%%%%%%%%%%

This is a two-step process.
First, you need to tell the machines to report that they have special
software installed, and second, you need to tell the jobs to require
machines that have that software.

To tell the machines to report the presence of special software, first
add a parameter to their configuration files like so:

\begin{verbatim}
HAS_MY_SOFTWARE = True
\end{verbatim}

And then, if there are already \MacroNI{STARTD\_ATTRS} defined in that file, add
\MacroNI{HAS\_MY\_SOFTWARE} to them, or, if not, add the line:

\footnotesize
\begin{verbatim}
STARTD_ATTRS = HAS_MY_SOFTWARE, $(STARTD_ATTRS)
\end{verbatim}
\normalsize

\Note For these changes to take effect, each \Condor{startd} you update
needs to be reconfigured with \Condor{reconfig} -startd.

Next, to tell your jobs to only run on machines that have this
software, add a requirements statement to their submit files like so:

\footnotesize
\begin{verbatim}
Requirements = (HAS_MY_SOFTWARE =?= True)
\end{verbatim}
\normalsize

\Note Be sure to use =?= instead of == so that if a machine doesn't have
the HAS\_MY\_SOFTWARE parameter defined, the job's Requirements
expression will not evaluate to ``undefined'', preventing it from
running anywhere!


%%%%%%%%%%%%%%%%%%%%%%%%%%%%%%%%%%%%%%%%%%%%%%%%
\subsection*{How do I configure Condor to only run jobs at night?}
%%%%%%%%%%%%%%%%%%%%%%%%%%%%%%%%%%%%%%%%%%%%%%%%
\index{running a job!only at night|}
\index{running a job!at certain times of day|}

A commonly requested policy for running batch jobs is to only allow
them to run at night, or at other pre-specified times of the day.
Condor allows you to configure this policy with the use of the
\Attr{ClockMin} and \Attr{ClockDay} \Condor{startd} attributes.  
A complete example of how to use these attributes for this kind of
policy is discussed in subsubsection~\ref{sec:Time of Day Policy} on
page~\pageref{sec:Time of Day Policy}.


%%%%%%%%%%%%%%%%%%%%%%%%%%%%%%%%%%%%%%%%%%%%%%%%
\subsection*{How do I configure Condor such that all machines do not produce checkpoints at the same time?}
%%%%%%%%%%%%%%%%%%%%%%%%%%%%%%%%%%%%%%%%%%%%%%%%
\label{sec:randomintegerusage}
\index{checkpoint!periodic}
\index{\$RANDOM\_INTEGER()!in configuration}
If machines are configured to produce checkpoints at fixed intervals,
a large number of jobs are queued (submitted) at the same time,
and these jobs start on machines at about the same time,
then all these jobs will be trying to write out their checkpoints
at the same time.
It is likely to cause rather poor performance during this burst of
writing.

The \Macro{RANDOM\_INTEGER()} macro can help in this instance.
Instead of defining \Macro{PERIODIC\_CHECKPOINT} to be a fixed
interval, each machine is configured to randomly choose 
one of a set of intervals.
For example, to set a machine's interval for producing checkpoints
to within the range of two to three hours, use the following
configuration:
\footnotesize
\begin{verbatim}
PERIODIC_CHECKPOINT = $(LastCkpt) > ( 2 * $(HOUR) + \
      $RANDOM_INTEGER(0,60,10) * $(MINUTE) )
\end{verbatim}
\normalsize

The interval used is set at configuration time.
Each machine is randomly assigned a different interval 
(2 hours, 2 hours and 10 minutes, 2 hours and 20 minutes, etc.)
at which to produce checkpoints.
Therefore, the various machines will not all attempt to
produce checkpoints at the same time.

%%%%%%%%%%%%%%%%%%%%%%%%%%%%%%%%%%%%%%%%%%%%%%%%
\subsection*{Why will the \Condor{master} not run when a local configuration file is missing?}
%%%%%%%%%%%%%%%%%%%%%%%%%%%%%%%%%%%%%%%%%%%%%%%%

If a \Macro{LOCAL\_CONFIG\_FILE} 
is specified in the global configuration file,
but the specified file does not exist,
the \Condor{master} will not start up, and it prints a variation
of the following example message.

\footnotesize
\begin{verbatim}
ERROR: Can't read config file /mnt/condor/hosts/bagel/condor_config.local
\end{verbatim}
\normalsize

This is not a bug; it is a feature!
Condor has always worked this way on purpose.
There is a potentially
large security hole if Condor is configured to read from a file that
does not exist.
By creating that file, a malicious user could
change all sorts of Condor settings.
This is an easy way
to gain root access to a machine,
where the daemons are running as root.

The intent is that
if you've set up your global configuration file to read
from a local configuration file, and the local file is not there,
then something is wrong.
It is better for the \Condor{master} to exit right away and
log an error message than to start up.

If the \Condor{master} continued with the local configuration file
missing, either A) someone could breach security or B) you will have
potentially important configuration information missing.
Consider the example where the local configuration file was on an NFS
partition and the server was down. 
There would be all sorts of
really important stuff in the local configuration file,
and Condor might do bad things if it started without those settings.  

If supplied it with an empty file, the \Condor{master} works fine.



%%%%%%%%%%%%%%%%%%%%%%%%%%%%%%%%%%%%%%%%%%%%%%%%
\section{Running Condor Jobs}
%%%%%%%%%%%%%%%%%%%%%%%%%%%%%%%%%%%%%%%%%%%%%%%%

%%%%%%%%%%%%%%%%%%%%%%%%%%%%%%%%%%%%%%%%%%%%%%%%
\index{job!not running, why?}
\subsection*{Why aren't any or all of my jobs running?}
%%%%%%%%%%%%%%%%%%%%%%%%%%%%%%%%%%%%%%%%%%%%%%%%

Please see 
Section~\ref{sec:job-not-running}, on page~\pageref{sec:job-not-running}
for information on why a job might not be running.


%%%%%%%%%%%%%%%%%%%%%%%%%%%%%%%%%%%%%%%%%%%%%%%%
\subsection*{I'm at the University of Wisconsin-Madison Computer Science Dept., and I am having problems!}
%%%%%%%%%%%%%%%%%%%%%%%%%%%%%%%%%%%%%%%%%%%%%%%%

Please see the web page \URL{http://www.cs.wisc.edu/condor/uwcs}.
As
it explains, your home directory is in AFS, which by default has
access control restrictions which can prevent Condor jobs from running
properly.
The above URL will explain how to solve the problem.

%%%%%%%%%%%%%%%%%%%%%%%%%%%%%%%%%%%%%%%%%%%%%%%%
\subsection*{I'm getting a lot of e-mail from Condor.  Can I just delete it all?}
%%%%%%%%%%%%%%%%%%%%%%%%%%%%%%%%%%%%%%%%%%%%%%%%

Generally you shouldn't ignore \Bold{all} of the mail Condor sends,
but you can reduce the amount you get by telling Condor that you don't
want to be notified every time a job successfully completes, only when
a job experiences an error.
To do this, include a line in your submit file like the following:

\begin{verbatim}
Notification = Error
\end{verbatim}

See the Notification parameter in the \Condor{q} man page on
page~\pageref{man-condor-submit-notification} of this manual for more
information.

%%%%%%%%%%%%%%%%%%%%%%%%%%%%%%%%%%%%%%%%%%%%%%%%
\subsection*{Why will my vanilla jobs only run on the machine where I submitted them from?}
%%%%%%%%%%%%%%%%%%%%%%%%%%%%%%%%%%%%%%%%%%%%%%%%

Check the following:
\begin {enumerate}

\item{Did you submit the job from a local file system that other
computers can't access?}

See Section~\ref{sec:Shared-Filesystem-Config-File-Entries}, on
page~\pageref{sec:Shared-Filesystem-Config-File-Entries}.

\item{Did you set a special requirements expression for 
vanilla jobs that's preventing them from running but not other jobs?}

See Section~\ref{sec:Shared-Filesystem-Config-File-Entries}, on
page~\pageref{sec:Shared-Filesystem-Config-File-Entries}.

\item{Is Condor running as a non-root user?}

See Section~\ref{sec:Non-Root}, on page~\pageref{sec:Non-Root}.

\end{enumerate}

%%%%%%%%%%%%%%%%%%%%%%%%%%%%%%%%%%%%%%%%%%%%%%%%
\subsection*{Why does the \Attr{requirements} expression for the job I submitted\\
have extra things that I did not put in my submit description file?}
%%%%%%%%%%%%%%%%%%%%%%%%%%%%%%%%%%%%%%%%%%%%%%%%
\index{requirements attribute!automatic extensions}
There are several extensions to the submitted \Attr{requirements}
that are automatically added by Condor.
Here is a list:
\begin{itemize}
  \item Condor automatically adds \Attr{arch} and \Attr{opsys} if 
  not specified in the submit description file. It is assumed that
  the executable needs to execute on the same platform as the machine
  on which the job is submitted.

  \item Condor automatically adds the expression
  \Expr{(Memory * 1024 > ImageSize)}.
  This ensures that the job will run on a machine with at
  least as much physical memory as the memory footprint of the job.

  \item Condor automatically adds the expression
  \Expr{(Disk >= DiskUsage)} if not already specified.
  This ensures that the job will run on a machine with enough disk
  space for the job's local I/O (if there is any).

  \item A pool administrator may define configuration variables that
  cause expressions to be added to \Attr{requirements}.
  These configuration variables are \MacroNI{APPEND\_REQUIREMENTS},
  \MacroNI{APPEND\_REQ\_VANILLA}, and \MacroNI{APPEND\_REQ\_STANDARD}.
  These configuration variables give
  pool administrators the flexibility to set policy for a local pool.

  \item Older versions of Condor needed to add confusing clauses
  about WINNT and the FileSystemDomain to vanilla universe jobs.
  This made sure that the jobs ran on a machine where files were
  accessible.
  The Windows version supported automatically transferring files
  with the vanilla job,
  while the Unix version relied on a shared file system.
  Since the Unix version of Condor now supports transferring files,
  these expressions are no longer added to the
  \Attr{requirements} for a job.
\end{itemize}


%%%%%%%%%%%%%%%%%%%%%%%%%%%%%%%%%%%%%%%%%%%%%%%%
\subsection*{When I use \Condor{compile} to produce a job, I get an error that says, "Internal ld was not invoked!". What does this mean?}
%%%%%%%%%%%%%%%%%%%%%%%%%%%%%%%%%%%%%%%%%%%%%%%%
\index{condor\_compile@\Condor{compile}}

\Condor{compile} enforces a specific behavior in the compilers and
linkers that it supports
(for example \Prog{gcc}, \Prog{g77}, \Prog{cc}, \Prog{CC}, \Prog{ld})
where a special linker script
provided by Condor must be invoked during the final linking stages of
the supplied compiler or linker.

In some rare cases,
as with \Prog{gcc} compiled with
the options \Opt{--with-as} or \Opt{--with-ld},
the enforcement mechanism
we rely upon to have \Prog{gcc}
choose our supplied linker script is not honored
by the compiler.
When this happens, an executable is produced,
but the executable is devoid of the
Condor libraries which both identify it as a Condor executable linked
for the standard universe and implement the feature sets of remote I/O
and transparent process checkpointing and migration.

Often, the only fix in order to use the compiler desired,
is to reconfigure and recompile the compiler itself,
such that it does not use the errant options mentioned. 

With Condor's standard universe,
we highly recommend that your source files
are compiled with the supported compiler for your platform.
See
section~\ref{sec:Availability}
for the list of supported compilers.
For a Linux platform, the supported compiler
is the default compiler that came with the distribution.
It is often found in the directory \File{/usr/bin}.

%%%%%%%%%%%%%%%%%%%%%%%%%%%%%%%%%%%%%%%%%%%%%%%%
\subsection*{Why might my job be preempted (evicted)?}
%%%%%%%%%%%%%%%%%%%%%%%%%%%%%%%%%%%%%%%%%%%%%%%%

There are four circumstances under which Condor may evict a job.
They are controlled by different expressions.

Reason number 1 is the user priority:
controlled by the \Attr{PREEMPTION\_REQUIREMENTS}
expression in the configuration file.
If there is a job from a 
higher priority user sitting idle,
the \Condor{negotiator} daemon may evict 
a currently running job submitted from a lower priority user if 
\Attr{PREEMPTION\_REQUIREMENTS} is True.
For more on user priorities,
see section~\ref{sec:Priorities} and
section~\ref{sec:UserPrio}.

Reason number 2 is the owner (machine) policy:
controlled by the \Attr{PREEMPT} expression in the configuration file.
When a job is running and the \Attr{PREEMPT} expression
evaluates to True,
the \Condor{startd} will evict the job.
The \Attr{PREEMPT} expression should reflect 
the requirements under which the machine owner will not permit
a job to continue to run.
For example, a policy to evict a currently running job when a key is hit
or when it is the 9:00am work arrival time,
would be expressed in the \Attr{PREEMPT} expression 
and enforced by the \Condor{startd}.
For more on the \Attr{PREEMPT} expression,
see section~\ref{sec:Configuring-Policy}.

Reason number 3 is the owner (machine) preference:
controlled by the \Attr{RANK} expression in the 
configuration file (sometimes called the startd rank or machine rank).
The \Attr{RANK} expression is evaluated as a floating point number.
When one job is running, a second idle job that evaluates to a higher
\Attr{RANK} value 
tells the \Condor{startd} to prefer the second job over the first.
Therefore, the \Condor{startd} will evict the first 
job so that it can start running the second (preferred) job.
For more on \Attr{RANK},
see section~\ref{sec:Configuring-Policy}.

Reason number 4 is if Condor is to be shutdown:
on a machine that is currently running a job.
Condor evicts the currently running job before proceeding
with the shutdown.

%%%%%%%%%%%%%%%%%%%%%%%%%%%%%%%%%%%%%%%%%%%%%%%%
\subsection*{Condor does not stop the Condor jobs running on my Linux machine
when I use my keyboard and mouse.  Is there a bug?}
%%%%%%%%%%%%%%%%%%%%%%%%%%%%%%%%%%%%%%%%%%%%%%%%
\index{Linux!keyboard and mouse activity}

There is no bug in Condor.
Unfortunately,
recent Linux 2.4.x and all Linux 2.6.x kernels through 
version 2.6.10
do not post proper state information,
such that Condor can detect keyboard and mouse activity.
Condor 
implements workarounds to piece together the needed
state information for PS/2 devices.
A better fix of the problem utilizes the 
kernel patch linked to from the directions posted at
\URL{http://www.cs.wisc.edu/condor/kernel.patch.html}.
This patch works better for PS/2 devices, and
may also work for USB devices.
A future version of Condor will implement better recognition
of USB devices,
such that the kernel patch will also definitively work for USB devices.


%%%%%%%%%%%%%%%%%%%%%%%%%%%%%%%%%%%%%%%%%%%%%%%%
\subsection*{What signals get sent to my jobs when Condor needs to preempt or kill them, or when I remove them from the queue?  Can I tell Condor which signals to send?}
%%%%%%%%%%%%%%%%%%%%%%%%%%%%%%%%%%%%%%%%%%%%%%%%

The answer is dependent on the universe of the jobs.

Under the scheduler universe,
the signal jobs get upon \Condor{rm} can be set by
the user in the submit description file with the form of
\begin{verbatim}
remove_kill_sig = SIGWHATEVER
\end{verbatim}
If this command is not defined, 
Condor further looks for a command 
in the submit description file with the form
\begin{verbatim}
kill_sig = SIGWHATEVER
\end{verbatim}
And, if that command is also not given,
Condor uses SIGTERM.

For all other universes, the jobs get the value of
the submit description file command
\verb@kill_sig@, which is SIGTERM by default.

If a job is killed or evicted, the job is sent a
\verb@kill_sig@, 
unless it is on the receiving end of a hard kill,
in which case it gets SIGKILL.

Under all universes,
the signal is sent only to the parent PID of the job,
namely, the first child of the \Condor{starter}.
If the child itself is forking,
the child must catch and forward signals as appropriate.
This in turn depends on the user's desired behavior.
The exception to this is (again) where the job is receiving
a hard kill.
Condor sends the value SIGKILL to all the PIDs in the family.

%%%%%%%%%%%%%%%%%%%%%%%%%%%%%%%%%%%%%%%%%%%%%%%%
\subsection*{Why does my Linux job have an enormous ImageSize and
refuse to run anymore?}
%%%%%%%%%%%%%%%%%%%%%%%%%%%%%%%%%%%%%%%%%%%%%%%%

\index{job!image size}
\index{ImageSize}

Sometimes Linux jobs run, are preempted and can not start again because
Condor thinks the image size of the job is too big. This is because
Condor has a problem calculating the image size of a program on Linux
that uses threads. It is particularly noticeable in the Java universe,
but it also happens in the vanilla universe. It is not an issue in the
standard universe, because threaded programs are not allowed.

On Linux, each thread appears to consume as much memory as the entire
program consumes, so the image size appears to be (number-of-threads *
image-size-of-program). If your program uses a lot of threads, your
apparent image size balloons. You can see the image size that Condor
believes your program has by using the -l option to \Condor{q}, and
looking at the \Attr{ImageSize} attribute.

When you submit your job, Condor creates or extends the requirements
for your job. In particular, it adds a requirement that you job must
run on a machine with sufficient memory:

\footnotesize
\begin{verbatim}
Requirements = ... ((Memory * 1024) >= ImageSize) ...
\end{verbatim}
\normalsize

Note that memory is the execution machine's memory in Mbytes,
while \Attr{ImageSize} is in Kbytes.
\Attr{ImageSize} is not a perfect measure of the memory requirements of a job.
It over-counts memory that is shared between processes.
It may appear quite large if the job uses \Procedure{mmap} on a large file.
It does not account for memory that the job uses indirectly in the operating
system's file system cache.

In the Requirements expression above, 
Condor added \Expr{(Memory * 1024) >= ImageSize)} on behalf of the job.
To prevent Condor from doing this,
provide your own expression about memory in the submit description file,
as in this example:

\begin{verbatim}
Requirements = Memory > 1024
\end{verbatim}

You will need to change the value 1024 to a reasonably good estimate of 
the actual
memory requirements of the program, in Mbytes. This example says that
the program requires 1 Gbyte of memory. If you underestimate the
memory your application needs, you may have bad performance if the job
runs on machines that have insufficient memory.

In addition, if you have modified your machine policies to preempt
jobs when \Attr{ImageSize} is large,
you will need to change those policies.

%%%%%%%%%%%%%%%%%%%%%%%%%%%%%%%%%%%%%%%%%%%%%%%%
\subsection*{Why does the time output from \Condor{status} appear
as [?????] ? }
%%%%%%%%%%%%%%%%%%%%%%%%%%%%%%%%%%%%%%%%%%%%%%%%
\index{timing information!incorrect}
\index{clock skew}
\index{skew in timing information}

Condor collects timing information for a large variety of uses.
Collection of the data relies on accurate times.
Being a distributed system, clock skew among machines causes 
errant timing calculations.
Values can be reported too large or too small, with the possibility
of calculating negative timing values.

This problem may be seen by the user when looking at the output
of \Condor{status}.
If the \Opt{ActivityTime} field appears
as [?????],
then this calculated statistic was negative.
\Condor{status} recognizes that a negative amount of time will
be nonsense to report, and instead displays this string. 

The solution to the problem is to synchronize the clocks on
these machines.
An administrator can do this using a tool such as \Prog{ntp}.

%%%%%%%%%%%%%%%%%%%%%%%%%%%%%%%%%%%%%%%%%%%%%%%%
\subsection*{The user condor's home directory cannot be found.  Why?}
%%%%%%%%%%%%%%%%%%%%%%%%%%%%%%%%%%%%%%%%%%%%%%%%

\index{NIS!Condor must be dynamically linked}
\index{user condor!home directory not found}
This problem may be observed after installation, when attempting
to execute
\footnotesize
\begin{verbatim}
~condor/condor/bin/condor_config_val  -tilde
\end{verbatim}
\normalsize
and there is a user named condor.
The command prints a message such as
\footnotesize
\begin{verbatim}
     Error: Specified -tilde but can't find condor's home directory
\end{verbatim}
\normalsize

In this case, the difficulty stems from 
using NIS,
because the Condor daemons fail to communicate properly with NIS to get
account information.
To fix the problem, a dynamically linked version of Condor must
be installed.

%%%%%%%%%%%%%%%%%%%%%%%%%%%%%%%%%%%%%%%%%%%%%%%%
\subsection*{Condor commands (including \Condor{q}) are really slow. What is going on?}
%%%%%%%%%%%%%%%%%%%%%%%%%%%%%%%%%%%%%%%%%%%%%%%%
\index{Condor commands!really slow; why?}

Some Condor programs will react slowly if they expect to find a
\Condor{collector} daemon, yet cannot contact one.
Notably, \Condor{q} can be very slow.
The \Condor{schedd} daemon will also be slow,
and it will log lots of harmless messages complaining.
If you are not running a \Condor{collector} daemon,
it is important that the configuration variable
\Macro{COLLECTOR\_HOST} be set to nothing.
This is typically done by setting \MacroNI{CONDOR\_HOST} with
\footnotesize
\begin{verbatim}
CONDOR_HOST=
COLLECTOR_HOST=$(CONDOR_HOST)
\end{verbatim}
\normalsize
or
\footnotesize
\begin{verbatim}
COLLECTOR_HOST=
\end{verbatim}
\normalsize

%%%%%%%%%%%%%%%%%%%%%%%%%%%%%%%%%%%%%%%%%%%%%%%%
\subsection*{Where are my missing files?  The command \Expr{when\_to\_transfer\_output = ON\_EXIT\_OR\_EVICT} is in the submit description file.}
%%%%%%%%%%%%%%%%%%%%%%%%%%%%%%%%%%%%%%%%%%%%%%%%
\index{submit commands!when\_to\_transfer\_output}
\index{file transfer mechanism!missing files}
\index{transferring files}
Although it may appear as if files are missing,
they are not.
The transfer does take place whenever a job is 
preempted by another job, vacates the machine, or is killed.
Look for the files in the directory defined by
the \MacroNI{SPOOL} configuration variable.
See
section~\ref{sec:file-transfer-if-when}, on
page~\pageref{sec:file-transfer-if-when} for details on the naming
of the intermediate files.

%%%%%%%%%%%%%%%%%%%%%%%%%%%%%%%%%%%%%%%%%%%%%%%%
\subsection*{\label{sec:vmware-symlink-bug}Why are my vm universe VMware jobs failing and being put on hold?}
%%%%%%%%%%%%%%%%%%%%%%%%%%%%%%%%%%%%%%%%%%%%%%%%

Strange behavior has been noted when Condor tries to run a 
\SubmitCmd{vm} universe VMware
job using a path to a VMX file that contains a symbolic link.
An example of an error message that may appear in such a job's user log:
\begin{verbatim}
Error from starter on master_vmuniverse_strtd@nostos.cs.wisc
.edu: register(/scratch/gquinn/condor/git/CONDOR_SRC/src/con
dor_tests/31426/31426vmuniverse/execute/dir_31534/vmN3hylp_c
ondor.vmx) = 1/Error: Command failed: A file was not found/(
ERROR) Can't create snapshot for vm(/scratch/gquinn/condor/g
it/CONDOR_SRC/src/condor_tests/31426/31426vmuniverse/execute
/dir_31534/vmN3hylp_condor.vmx)
\end{verbatim}
To work around this problem:
\begin{itemize}
\item If using file transfer
(the submit description file contains
\SubmitCmd{vmware\_should\_transfer\_files = true}),
then modify any
configuration variable \Macro{EXECUTE} values on all execute machines,
such that they do not contain symbolic link path components.
\item If using a shared file system, ensure that the
submit description file command
\SubmitCmd{vmware\_dir} does not use
symbolic link path name components.
\end{itemize}

%%%%%%%%%%%%%%%%%%%%%%%%%%%%%%%%%%%%%%%%%%%%%%%%
\section{Condor on Windows}
%%%%%%%%%%%%%%%%%%%%%%%%%%%%%%%%%%%%%%%%%%%%%%%%

\index{FAQ!Condor on Windows machines}
%%%%%%%%%%%%%%%%%%%%%%%%%%%%%%%%%%%%%%%%%%%%%%%%
\subsection*{Will Condor work on a network of mixed Unix and Windows machines?}
%%%%%%%%%%%%%%%%%%%%%%%%%%%%%%%%%%%%%%%%%%%%%%%%

You can have a Condor pool that consists of both Unix and Windows machines.

Your central manager can be either Windows or Unix.  For example,
even if you had a pool consisting strictly of Unix machines, you could
use a Windows box for your central manager, and vice versa.

Submitted jobs can originate from either a 
Windows \Bold{or} a Unix machine,
and be destined to run on Windows
\Bold{or} a Unix machine.
Note that there are still restrictions on the supported universes
for jobs executed on Windows machines.

So, in summary:

\begin{enumerate}

\item{A single Condor pool can consist of both Windows and Unix
machines.}

\item{It does not matter at all if your Central Manager is Unix or Windows.}

\item{Unix machines can submit jobs to run on other Unix or Windows
machines.}

\item{Windows machines can submit jobs to run on other Windows
or Unix machines.}

\end{enumerate}


%%%%%%%%%%%%%%%%%%%%%%%%%%%%%%%%%%%%%%%%%%%%%%%%
\subsection*{What versions of Windows will Condor run on?}
%%%%%%%%%%%%%%%%%%%%%%%%%%%%%%%%%%%%%%%%%%%%%%%%

See Section~\ref{sec:Availability}, on
page~\pageref{sec:Availability}.


%%%%%%%%%%%%%%%%%%%%%%%%%%%%%%%%%%%%%%%%%%%%%%%%
\subsection*{My Windows program works fine when executed on its own, but it
does not work when submitted to Condor.}
%%%%%%%%%%%%%%%%%%%%%%%%%%%%%%%%%%%%%%%%%%%%%%%%

\underline{First}, make sure that the program really does work
outside of Condor under Windows,
that the disk is not full,
and that the system is not out of user resources.

\underline{As the next consideration},
know that 
some Windows programs do not run properly because they are dynamically linked,
and they cannot find the \File{.dll} files that they depend on.
Version 6.4.x of Condor sets the \Env{PATH} to be empty when
running a job.
To avoid these difficulties, do one of the following
\begin{enumerate}
\item statically link the application
\item wrap the job in a script that sets up the environment
\item submit the job from a correctly-set environment with the command
\begin{verbatim}
getenv = true
\end{verbatim}
in the submit description file.
This will copy your environment into the job's environment.
\item send the required \File{.dll} files along with the job
using the submit description file command \Opt{transfer\_input\_files}.
\end{enumerate}


%%%%%%%%%%%%%%%%%%%%%%%%%%%%%%%%%%%%%%%%%%%%%%%%
\subsection*{Why is the \Condor{master} daemon failing to start, giving an error about\\
   	"In StartServiceCtrlDispatcher, Error number: 1063"?}
%%%%%%%%%%%%%%%%%%%%%%%%%%%%%%%%%%%%%%%%%%%%%%%%
In Condor for Windows, the \Condor{master} daemon is started as a service.
Therefore,
starting the \Condor{master} daemon as you would on Unix will not work.
Start Condor on Windows machines using either
\begin{verbatim}
	net start condor
\end{verbatim}
or start the Condor service from the Service Control Manager located in
the Windows Control Panel.

%%%%%%%%%%%%%%%%%%%%%%%%%%%%%%%%%%%%%%%%%%%%%%%%
\subsection*{Jobs submitted from Windows give an error referring to a credential.}
%%%%%%%%%%%%%%%%%%%%%%%%%%%%%%%%%%%%%%%%%%%%%%%%

Jobs submitted from a Windows machine require a stashed password in
order for Condor to perform certain operations on the user's behalf.
Refer to section \ref{sec:windows-sps} for information about password
storage on Windows.  The command which stashes a password for a user
is \Condor{store\_cred}.  See the manual page on on
page~\pageref{man-condor-store-cred} for usage details.
\index{job!credential error on Windows}

The error message that Condor gives if a user has not stashed a
password is of the form:
\footnotesize
\begin{verbatim}
ERROR: No credential stored for username@machinename

        Correct this by running:
	        condor_store_cred add
\end{verbatim}
\normalsize

%%%%%%%%%%%%%%%%%%%%%%%%%%%%%%%%%%%%%%%%%%%%%%%%
\subsection*{Jobs submitted from Unix to execute on Windows do not work properly.}
%%%%%%%%%%%%%%%%%%%%%%%%%%%%%%%%%%%%%%%%%%%%%%%%

A difficulty with defaults causes jobs submitted from Unix for execution
on a Windows platform to remain in the queue, but make no progress.
For jobs with this problem, log files will contain error messages
pointing to shadow exceptions.

This difficulty stems from the defaults for whether file transfer
takes place.
The workaround for this problem is to place the lines
\begin{verbatim}
   should_transfer_files = YES
   when_to_transfer_output = ON_EXIT
\end{verbatim}
into the submit description file for jobs submitted from a Unix
machine for execution on a Windows machine.


%%%%%%%%%%%%%%%%%%%%%%%%%%%%%%%%%%%%%%%%%%%%%%%%
\subsection*{When I run \Condor{status} I get a communication error, or the Condor daemon log files report a failure to bind.}
%%%%%%%%%%%%%%%%%%%%%%%%%%%%%%%%%%%%%%%%%%%%%%%%

Condor uses the first network interface it sees on your machine.
This problem usually means you have an extra, inactive network
interface (such as a RAS dial up interface) defined before the
regular network interface.

To solve this problem, either change the order of the network
interfaces in the Control Panel, or explicitly set which network
interface Condor should use by adding the following definition to the
Condor configuration file:

\begin{verbatim}
NETWORK_INTERFACE = <ip-address>
\end{verbatim}

Where \verb@<ip-address>@ is the IP address of the interface that
Condor is to use.

%%%%%%%%%%%%%%%%%%%%%%%%%%%%%%%%%%%%%%%%%%%%%%%%
\subsection*{My job starts but exits right away with status 128.}
%%%%%%%%%%%%%%%%%%%%%%%%%%%%%%%%%%%%%%%%%%%%%%%%
\index{job!exiting with status 128 \(NT\)}

This can occur when the machine your job is running on is missing a
DLL (Dynamically Linked Library) required by your program.
The solution is to find the DLL file the program needs and put it in
the TRANSFER\_INPUT\_FILES list in the job's submit file.

To find out what DLLs your program depends on, right-click the program
in Explorer, choose Quickview, and look under ``Import List''.


%%%%%%%%%%%%%%%%%%%%%%%%%%%%%%%%%%%%%%%%%%%%%%%%
\subsection*{How can I access network files with Condor on Windows?}
%%%%%%%%%%%%%%%%%%%%%%%%%%%%%%%%%%%%%%%%%%%%%%%%

Five methods for making access of network files work with Condor
are given in 
section~\ref{sec:network-files-solutions}.

%%%%%%%%%%%%%%%%%%%%%%%%%%%%%%%%%%%%%%%%%%%%%%%%
\subsection*{What is wrong when \Condor{off} cannot find my host, and \Condor{status} does not give me a complete host name?}
%%%%%%%%%%%%%%%%%%%%%%%%%%%%%%%%%%%%%%%%%%%%%%%%

Given the command
\begin{verbatim}
  condor_off hostname2
\end{verbatim}
an error message of the form
\begin{verbatim}
  Can't find address for master hostname2.somewhere.edu
\end{verbatim}
appears.
Yet, when looking at the host names with
\begin{verbatim}
  condor_status -master
\end{verbatim}
the output is of the form 
\begin{verbatim}
  hostname1.somewhere.edu
  hostname2
  hostname3.somewhere.edu
\end{verbatim}

To correct this incomplete host name, add an entry to the
configuration file for
\Macro{DEFAULT\_DOMAIN\_NAME} 
that specifies the domain name to be used.
For the example given, the configuration entry will be
\begin{verbatim}
  DEFAULT_DOMAIN_NAME = somewhere.edu
\end{verbatim}

After adding this configuration file entry, use \Condor{restart}
to restart the Condor daemons and effect the change.

%To correct this incomplete host name on Windows 2000 or XP,
%use the ``Append parent suffixes of the primary DNS suffix''
%checkbox for the TCP/IP Advanced Properties.
%Disable and reenable the connection to make the change take
%effect.
%
%To correct this incomplete host name on Windows NT Version 4,
%set the domain in the TCP/IP Properties dialog box.
%Restart Condor after making this change.

%%%%%%%%%%%%%%%%%%%%%%%%%%%%%%%%%%%%%%%%%%%%%%%%
\subsection*{Does \MacroNI{USER\_JOB\_WRAPPER} work on Windows machines?}
%%%%%%%%%%%%%%%%%%%%%%%%%%%%%%%%%%%%%%%%%%%%%%%%
The \Macro{USER\_JOB\_WRAPPER} configuration variable
does work on Windows machines.
The wrapper must be either a
batch script with a file name extension of \File{.bat} or \File{.cmd},
or an
executable with a file name extension of \File{.exe} or \File{.com}.

An example of a batch script sets environment variables:
\footnotesize
\begin{verbatim}
REM set some environment variables
set LICENSE_SERVER=192.168.1.202:5012
set MY_PARAMS=2

REM Run the actual job now
%*
\end{verbatim}
\normalsize


%%%%%%%%%%%%%%%%%%%%%%%%%%%%%%%%%%%%%%%%%%%%%%%%
\subsection*{\Condor{store\_cred} is failing, and I'm sure I'm typing my password correctly.}
%%%%%%%%%%%%%%%%%%%%%%%%%%%%%%%%%%%%%%%%%%%%%%%%

First, make sure the \Condor{schedd} daemon is running.

Next, check the log file written by the \Condor{schedd} daemon.
It will contain more detailed information about the failure.
Frequently, the error is a result of 
\verb@PERMISSION DENIED@ errors.
More information about proper configuration of 
security settings is on page~\pageref{sec:Host-Security}.


%%%%%%%%%%%%%%%%%%%%%%%%%%%%%%%%%%%%%%%%%%%%%%%%
\subsection*{My submit machine cannot have more than 120 jobs running concurrently. Why?}
%%%%%%%%%%%%%%%%%%%%%%%%%%%%%%%%%%%%%%%%%%%%%%%%

\index{Windows!out of desktop heap}
Windows is likely to be running out of desktop heap. 
Confirm this to be the case
by looking in the log for the \Condor{schedd} daemon
to see if \Condor{shadow} daemons are immediately
exiting with status 128.
If this is the case, increase the desktop heap size.
Open the registry key:

\footnotesize
\begin{verbatim}
HKEY_LOCAL_MACHINE\System\CurrentControlSet\Control\Session Manager\SubSystems\Windows
\end{verbatim}
\normalsize

The SharedSection value can have three values separated by commas.
The third value controls the desktop heap size for non-interactive desktops,
which the Condor service uses.
The default is 512 (Kbytes).
60 \Condor{shadow} daemons consume about 256 Kbytes,
hence 120 shadows can run with the default value.
To be able to run a maximum of 300 \Condor{shadow} daemons,
set this value at 1280.

Reboot the system for the changes to take effect.
For more information,
see Microsoft Article Q184802.

%%%%%%%%%%%%%%%%%%%%%%%%%%%%%%%%%%%%%%%%%%%%%%%%
\subsection*{Why do Condor daemons exit after logging a 10038 (WSAENOTSOCK) error on some machines?}
%%%%%%%%%%%%%%%%%%%%%%%%%%%%%%%%%%%%%%%%%%%%%%%%

Usually when Condor daemons exit in this manner, it is because the system in
question has a non-standard Winsock Layered Service Provider (LSP) installed
on it. An LSP is, in effect, a plug-in for the TCP/IP protocol stack.
LSPs have been 
installed as part of anti-virus software and other security-related
packages.

There are several tools available to check your system for the
presence of LSPs. One with which we have had success is \Prog{LSP-Fix},
available at \URL{http://www.cexx.org/lspfix.htm}.
Any non-Microsoft LSPs identified by
this tool may potentially be causing the WSAENOTSOCK error in Condor.
Although the \Prog{LSP-Fix} tool allows the direct removal of an LSP,
it is likely advisable to completely remove the application for which
the LSP is a part via the Control Panel.

Another approach is to completely reset the TCP/IP stack to its
original state.  This can be done using the \verb@netsh@ tool:
\begin{verbatim}
netsh int ip reset reset-stack.log
\end{verbatim}
The command will return the TCP/IP stack back to the state is was
in when the OS was first installed.  The log file defined above will
record all the configuration changes made by \verb@netsh@.

%%%%%%%%%%%%%%%%%%%%%%%%%%%%%%%%%%%%%%%%%%%%%%%%
\subsection*{Why do Condor daemons exit with "Unexpected performance counter size", "unable to spawn the ProcD" or "loadavg thread died, restarting. (exit code=2)" errors?}
%%%%%%%%%%%%%%%%%%%%%%%%%%%%%%%%%%%%%%%%%%%%%%%%

Condor on Windows platforms relies on built-in performance counters
for its operation. 
If performance counters that Condor requires are disabled,
daemons may exit with a message such as

\footnotesize
\begin{verbatim}
1/26 09:16:42 (fd:2) (pid:5732) ERROR: "Unexpected performance counter
    size for total CPU: 0 (expected: 8)" at line 2846 in file
    ..\src\condor_procapi\procapi.cpp
\end{verbatim}
\normalsize

or

\footnotesize
\begin{verbatim}
1/20 15:29:14 (pid:2484) ERROR "unable to spawn the ProcD" at line 136
    in file ..\src\condor_c++_util\proc_family_proxy.C
\end{verbatim}
\normalsize

and even

\footnotesize
\begin{verbatim}
4/16 10:49:13 loadavg thread died, restarting. (exit code=2)
\end{verbatim}
\normalsize

To enable the performance counters, check the registry key
\footnotesize
\begin{verbatim}
HKEY_LOCAL_MACHINE\SYSTEM\CurrentControlSet\Services\PerfProc\Performance
\end{verbatim}
\normalsize
If a value for \verb@Disable Performance Counters@ exists, delete it or set
it to \verb@0@.

%%%%%%%%%%%%%%%%%%%%%%%%%%%%%%%%%%%%%%%%%%%%%%%%
\subsection*{Why does the Windows Installer fail with ``Error 2738. Could not access VBScript run time for custom action''?}
%%%%%%%%%%%%%%%%%%%%%%%%%%%%%%%%%%%%%%%%%%%%%%%%

This error results when the VBScript engine is not registered.
Since Condor's installer depends on the VBScript engine for custom steps,
the installer will fail if it cannot find the VBScript engine.

The fix is to register the VMScript engine.
With Administrative privilege:

\begin{enumerate}
\item Launch the Command Prompt (\Prog{cmd.exe}) as the Administrator. 
\item At the Command Prompt, change directories to the \File{System32} folder, 
within the Windows folder.
\item Issue the command 
\begin{verbatim}
  regsvr32 vbscript.dll
\end{verbatim}
\end{enumerate}

If successful, the message
\begin{verbatim}
  DllRegisterServer in vbscript.dll succeeded.  
\end{verbatim}
is printed.

%%%%%%%%%%%%%%%%%%%%%%%%%%%%%%%%%%%%%%%%%%%%%%%%
\subsection*{Why does Condor sometimes fail to parse floating point numbers?}
%%%%%%%%%%%%%%%%%%%%%%%%%%%%%%%%%%%%%%%%%%%%%%%%

Condor assumes that all floating point numbers are of the form x.y, which,
depending on a computer's current locale, may not always be the case. This
problem occurs even if Condor is running under an account that has had the
locale configured correctly.  The problem lies in the template user account
which is used to create Condor's dynamic accounts. Even if the entire
system is configured to use a new locale, this template account seems to
retain the original system locale. The following steps can be used fix
this problem.

To create a default user profile, you must be logged on as 
\textbf{Administrator} or be a member of the \textbf{Administrators} group.
Create a new user profile for all new user accounts on a computer to be
based on. To create subsequent profiles, you can use the new user account
as a template. Here is how to use the new user profile as a template account
to use as a new user's profile:
\begin{enumerate}
\item \textbf{Log on} to the computer as the new user, and customize the
desktop if appropriate. 
\item Optionally, install and configure any applications to be shared
by user accounts made from this template. 
\item \textbf{Log off}, and then log on as the \textbf{Administrator}. 
\item In the \textbf{Control Panel}, open the \textbf{System} Control Panel
 applet. 
 \begin{itemize}
 \item On \textbf{Vista} click on the 
 \textbf{Advanced system settings} \textbf{Task} listed in the left pane.
 \end{itemize} 
\item On the \textbf{Advanced} tab, under \textbf{User Profiles}, 
click \textbf{Settings}. 
\item Under \textbf{Profiles stored on this computer}, select the user
you created to be the template, and then click \textbf{Copy To}. 
\item To create the default user profile for the computer, type the path
to the default user:
 \begin{itemize}
 \item On Windows 2000: \verb@%WinDir%\Profiles\Default@;
 \item On Windows XP: \verb@%SystemDrive%\Documents and Settings\Defualt@;
 \item On Vista: \verb@%SystemDrive%\Users\Default@.
 \end{itemize}
\item In the \textbf{Copy To} dialog box, under \textbf{Permitted to use},
click \textbf{Change}. 
\item In the \textbf{Select User or Group} dialog box, in the 
\textbf{Enter the object name to select} text box, type: \textbf{Everyone}
and click \textbf{OK}.
\item Click \textbf{OK} to dismiss the \textbf{Copy To} dialog box.
\item Click \textbf{OK} again to dismiss the \textbf{User Profiles} dialog box.
\item Finally, click \textbf{OK} one last time to dismiss the
\textbf{System Properties} dialog.
\end{enumerate}

If Condor has already created some dynamic accounts, you will need to remove
them so that Condor can re-create them with the new template account.

%%%%%%%%%%%%%%%%%%%%%%%%%%%%%%%%%%%%%%%%%%%%%%%%
\section{Grid Computing}
%%%%%%%%%%%%%%%%%%%%%%%%%%%%%%%%%%%%%%%%%%%%%%%%

\index{grid computing!FAQs}

%%%%%%%%%%%%%%%%%%%%%%%%%%%%%%%%%%%%%%%%%%%%%%%%
\subsection*{What must be installed to access grid resources?}
%%%%%%%%%%%%%%%%%%%%%%%%%%%%%%%%%%%%%%%%%%%%%%%%
A single machine with Condor installed such that jobs may
be submitted is the minimum software necessary.
If matchmaking or glidein is desired,
then a single machine must not only be running Condor
such that jobs may be submitted,
but also fill the role of a central manager.
A Personal Condor installation may satisfy both.

%%%%%%%%%%%%%%%%%%%%%%%%%%%%%%%%%%%%%%%%%%%%%%%%
\subsection*{I am the administrator at Physics, and I have a 64-node cluster
running Condor.
The administrator at Chemistry is also running Condor on her 64-node cluster.
We would like to be able to share resources.
How do we do this?}
%%%%%%%%%%%%%%%%%%%%%%%%%%%%%%%%%%%%%%%%%%%%%%%%

Condor's flocking feature
allows multiple Condor pools to share resources.
By setting configuration variables within each pool,
jobs may be executed on either cluster.
See the manual section on flocking, section~\ref{sec:Flocking},
for details.

%%%%%%%%%%%%%%%%%%%%%%%%%%%%%%%%%%%%%%%%%%%%%%%%
\subsection*{What is glidein?}
%%%%%%%%%%%%%%%%%%%%%%%%%%%%%%%%%%%%%%%%%%%%%%%%

\index{glidein}
Glidein provides a way to temporarily add a resource
to a local Condor pool.
Glidein uses Globus resource-management software to run jobs
on the resource.
Those jobs are initially portions of Condor
software, such that Condor is running on the resource,
configured to be part of the local pool.
Then, Condor may execute the user's jobs.
There are several benefits to working in this way.
Standard universe jobs may be submitted to run on the resource.
Condor can also dynamically schedule jobs across the grid.

See the section on Glidein, section~\ref{sec:Glidein} of the manual
for further information.


%%%%%%%%%%%%%%%%%%%%%%%%%%%%%%%%%%%%%%%%%%%%%%%%
\subsection*{Using my Globus gatekeeper to submit jobs to the Condor pool
does not work.  What is wrong?}
%%%%%%%%%%%%%%%%%%%%%%%%%%%%%%%%%%%%%%%%%%%%%%%%
\index{Globus!gatekeeper errors}

The Condor configuration file is in a non-standard location,
and the Globus software does not know how to locate it,
when you see either of the following error messages.

\underline{first error message}
\footnotesize
\begin{verbatim}
% globus-job-run \
  globus-gate-keeper.example.com/jobmanager-condor /bin/date

Neither the environment variable CONDOR_CONFIG, /etc/condor/,
nor ~condor/ contain a condor_config file.  Either set
CONDOR_CONFIG to point to a valid config file, or put a
"condor_config" file in /etc/condor or ~condor/ Exiting.

GRAM Job failed because the job failed when the job manager
attempted to run it (error code 17)
\end{verbatim}
\normalsize

\underline{second error message}
\footnotesize
\begin{verbatim}
% globus-job-run \
   globus-gate-keeper.example.com/jobmanager-condor /bin/date

ERROR: Can't find address of local schedd GRAM Job failed
because the job failed when the job manager attempted to run it
(error code 17)
\end{verbatim}
\normalsize

As described in
section~\ref{sec:Preparing-to-Install}, 
Condor searches for its configuration file using the following
ordering.
\begin{enumerate}
\item File specified in the \Env{CONDOR\_CONFIG} environment variable
\item \File{\$(HOME)/.condor/condor\_config}
\item \File{/etc/condor/condor\_config}
\item \File{\Tilde condor/condor\_config}
\item \File{\MacroUNI{GLOBUS\_LOCATION}/etc/condor\_config}
\end{enumerate}

Presuming the configuration file is not in a standard location,
you will need to set the \Env{CONDOR\_CONFIG} environment variable
\index{environment variables!CONDOR\_CONFIG}
by hand, or set it in an initialization script.
One of the following solutions for an initialization may be used.
\begin{enumerate}
\item 
Wherever \Prog{globus-gatekeeper} is launched,
replace it with a minimal shell script that sets
\Env{CONDOR\_CONFIG} and then starts \Prog{globus-gatekeeper}.
Something like the following should work:

\footnotesize
\begin{verbatim}
  #! /bin/sh
  CONDOR_CONFIG=/path/to/condor_config
  export CONDOR_CONFIG
  exec /path/to/globus/sbin/globus-gatekeeper "$@"
\end{verbatim}
\normalsize
\item 
If you are starting \Prog{globus-gatekeeper} using \Prog{inetd},
\Prog{xinetd}, or a similar program,
set the environment variable there.
If you are using \Prog{inetd}, you can use the \Prog{env} program
to set the environment.
This example does this;
the example is shown on multiple lines,
but it will be all on one line in the \Prog{inetd} configuration. 
\footnotesize
\begin{verbatim}
globus-gatekeeper stream tcp nowait root /usr/bin/env
env CONDOR_CONFIG=/path/to/condor_config
/path/to/globus/sbin/globus-gatekeeper
-co /path/to/globus/etc/globus-gatekeeper.conf
\end{verbatim}
\normalsize
If you're using \Prog{xinetd}, add an env setting
something like the following:
\footnotesize
\begin{verbatim}
service gsigatekeeper
{
    env = CONDOR_CONFIG=/path/to/condor_config
    cps = 1000 1
    disable = no
    instances = UNLIMITED
    max_load = 300
    nice = 10
    protocol = tcp
    server = /path/to/globus/sbin/globus-gatekeeper
    server_args = -conf /path/to/globus/etc/globus-gatekeeper.conf
    socket_type = stream
    user = root
    wait = no
}
\end{verbatim}
\normalsize

\end{enumerate}

%%%%%%%%%%%%%%%%%%%%%%%%%%%%%%%%%%%%%%%%%%%%%%%%
\section{Managing Large Workflows}
%%%%%%%%%%%%%%%%%%%%%%%%%%%%%%%%%%%%%%%%%%%%%%%%

%%%%%%%%%%%%%%%%%%%%%%%%%%%%%%%%%%%%%%%%%%%%%%%%
\subsection*{How do I get meaningful output from \Condor{q} with so many jobs in the queue?}
%%%%%%%%%%%%%%%%%%%%%%%%%%%%%%%%%%%%%%%%%%%%%%%%
 
There are several ways to constrain the output of \Condor{q} when there are
lots and lots of jobs in the queue.
To see only the jobs that are currently running:
\begin{verbatim}
  condor_q -run
\end{verbatim}
To see only the jobs that are currently on hold:
\begin{verbatim}
  condor_q -hold
\end{verbatim}
To see other output,  combine options.
For example, to see only running jobs submitted by A. Einstein
that belong to cluster 77:
\begin{verbatim}
  condor_q -run einstein 77
\end{verbatim}
Another example uses the \Opt{-constraint} option to \Condor{q}.
To see only the jobs in the queue that started running,
but were interrupted and then started again from the beginning,
perhaps more than once: 
\footnotesize
\begin{verbatim}
  condor_q -constraint 'NumJobStarts > 1'
\end{verbatim}
\normalsize
Complete details of \Condor{q} are contained in the manual page
at page~\pageref{man-condor-q}.

%%%%%%%%%%%%%%%%%%%%%%%%%%%%%%%%%%%%%%%%%%%%%%%%
\subsection*{What does Condor offer that can help with running
a large number of jobs?}
%%%%%%%%%%%%%%%%%%%%%%%%%%%%%%%%%%%%%%%%%%%%%%%%

Many of the features of DAGMan are targeted at helping with the 
administration and running of large numbers of jobs.
See section~\ref{sec:DAGLotsaJobs} at page~\pageref{sec:DAGLotsaJobs}.
 

%%%%%%%%%%%%%%%%%%%%%%%%%%%%%%%%%%%%%%%%%%%%%%%%
\section{Troubleshooting}
%%%%%%%%%%%%%%%%%%%%%%%%%%%%%%%%%%%%%%%%%%%%%%%%

%%%%%%%%%%%%%%%%%%%%%%%%%%%%%%%%%%%%%%%%%%%%%%%%
\subsection*{If I see \texttt{PERMISSION DENIED} in my log files,
what does that mean?}
%%%%%%%%%%%%%%%%%%%%%%%%%%%%%%%%%%%%%%%%%%%%%%%%
\index{permission denied}

Most likely, the Condor installation has been misconfigured
and Condor's access control security functionality is preventing
daemons and tools from communicating with each other.
Other symptoms of this problem include Condor tools (such as
\Condor{status} and \Condor{q}) not producing any output, or commands
that appear to have no effect (for example, \Condor{off} or
\Condor{on}). 

The solution is to properly configure the \Macro{HOSTALLOW\_*} and
\Macro{HOSTDENY\_*} settings (for host/IP based authentication) or to
configure strong authentication and set \Macro{ALLOW\_*} and
\Macro{DENY\_*} as appropriate.
Host-based authentication is described in
section~\ref{sec:Host-Security} on page~\pageref{sec:Host-Security}.
Information about other forms of authentication is provided in 
section~\ref{sec:Config-Security} on page~\pageref{sec:Config-Security}.

%%%%%%%%%%%%%%%%%%%%%%%%%%%%%%%%%%%%%%%%%%%%%%%%
\subsection*{What happens if the central manager crashes?}
%%%%%%%%%%%%%%%%%%%%%%%%%%%%%%%%%%%%%%%%%%%%%%%%
\index{crashes}
\index{recovery from crashes}

If the central manager crashes, jobs that are already running will
continue to run unaffected.
Queued jobs will remain in the queue unharmed, but can not begin
running until the central manager is restarted and begins matchmaking
again.
Nothing special needs to be done after the central manager is brought
back on line.

%%%%%%%%%%%%%%%%%%%%%%%%%%%%%%%%%%%%%%%%%%%%%%%%
\subsection*{Why did the \Condor{schedd} daemon die and restart?}
%%%%%%%%%%%%%%%%%%%%%%%%%%%%%%%%%%%%%%%%%%%%%%%%
\index{condor\_schedd daemon!receiving signal 25}

The \Condor{schedd} daemon receives signal 25,
dies, and is restarted when the
history file reaches a 2 Gbyte size limit.
On 32-bit OSes, Condor cannot write log files larger than
2 Gbytes.
If you need to keep more than 2 Gbytes of history, you can set a
maximum history file size of 2 Gbytes and multiple rotations of the
file.
For example, to keep 6 Gbytes of history, you would put these lines in
your Condor configuration file:
\begin{verbatim}
ENABLE_HISTORY_ROTATION = True
MAX_HISTORY_LOG = 2000000000
MAX_HISTORY_ROTATIONS = 2
\end{verbatim}

%%%%%%%%%%%%%%%%%%%%%%%%%%%%%%%%%%%%%%%%%%%%%%%%
\subsection*{When I ssh/telnet to a machine to check particulars of how
Condor is doing something, it is always vacating or unclaimed when I
know a job had been running there!}
%%%%%%%%%%%%%%%%%%%%%%%%%%%%%%%%%%%%%%%%%%%%%%%%

Depending on how your policy is set up, Condor will track \emph{any} tty
on the machine for the purpose of determining if a job is to be vacated
or suspended on the machine. It could be the case that after you ssh
there, Condor notices activity on the tty allocated to your connection
and then vacates the job.

%%%%%%%%%%%%%%%%%%%%%%%%%%%%%%%%%%%%%%%%%%%%%%%%
\subsection*{What is wrong? I get no output from \Condor{status}, but the Condor daemons are running.}
%%%%%%%%%%%%%%%%%%%%%%%%%%%%%%%%%%%%%%%%%%%%%%%%

\underline{One likely error message} within the collector log of the form
\footnotesize
\begin{verbatim}
DaemonCore: PERMISSION DENIED to host <xxx.xxx.xxx.xxx> for command 0 (UPDATE_STARTD_AD)
\end{verbatim}
\normalsize
indicates a permissions problem.
The \Condor{startd} daemons do not have write permission to the
\Condor{collector} daemon.
This could be because
you used domain names in your \MacroNI{HOSTALLOW\_WRITE} and/or
\MacroNI{HOSTDENY\_WRITE} configuration macros,
but the domain name server (DNS) is not properly configured at your site.
Without the proper configuration, Condor cannot resolve
the IP addresses of your machines
into fully-qualified domain names (an inverse look up).
If this is the problem, then the solution takes one of two forms:
\begin{enumerate}
\item Fix the DNS so that inverse look ups (trying to get the domain name
   from an IP address) works for your machines.  You can
   either fix the DNS itself,
   or use the \MacroNI{DEFAULT\_DOMAIN\_NAME} setting in your Condor
         configuration file.
\item Use numeric IP addresses in the \MacroNI{HOSTALLOW\_WRITE} and/or
   \MacroNI{HOSTDENY\_WRITE} configuration macros
   instead of domain names.
   As an example of this, assume your site has a machine such as
   foo.your.domain.com, and it has two subnets, with IP addresses
   129.131.133.10, and 129.131.132.10.
   If the configuration macro is set as 

\begin{verbatim}
 HOSTALLOW_WRITE = *.your.domain.com
\end{verbatim}

   and this does not work, use

\begin{verbatim}
 HOSTALLOW_WRITE = 192.131.133.*, 192.131.132.*
\end{verbatim}
\end{enumerate}

\underline{Alternatively}, this permissions problem
may be caused by being too restrictive in the setting of
your \MacroNI{HOSTALLOW\_WRITE} and/or
\MacroNI{HOSTDENY\_WRITE} configuration macros.
If it is, then the solution is to change the macros,
for example from
\begin{verbatim}
 HOSTALLOW_WRITE = condor.your.domain.com
\end{verbatim}
to
\begin{verbatim}
 HOSTALLOW_WRITE = *.your.domain.com
\end{verbatim}
or possibly
\footnotesize
\begin{verbatim}
 HOSTALLOW_WRITE = condor.your.domain.com, foo.your.domain.com, \
 bar.your.domain.com 
\end{verbatim}
\normalsize


\underline{Another likely error message} within the collector log of the form
\footnotesize
\begin{verbatim}
DaemonCore: PERMISSION DENIED to host <xxx.xxx.xxx.xxx> for command 5 (QUERY_STARTD_ADS)
\end{verbatim}
\normalsize
indicates a similar problem as above, but read permission
is the problem (as opposed to write permission).
Use the solutions given above.

%%%%%%%%%%%%%%%%%%%%%%%%%%%%%%%%%%%%%%%%%%%%%%%%
\subsection*{Why does Condor leave mail processes around?}
%%%%%%%%%%%%%%%%%%%%%%%%%%%%%%%%%%%%%%%%%%%%%%%%

Under FreeBSD and Mac OSX operating systems,
misconfiguration of of a system's outgoing mail causes
Condor to inadvertently leave paused and zombie mail
processes around when Condor attempts to send notification e-mail.
The solution to this problem is
to correct the mailer configuration.

Execute the following command as the user under which Condor
daemons run to determine whether outgoing e-mail works.

\begin{verbatim}
$ uname -a | mail -v your@emailaddress.com
\end{verbatim}

If no e-mail arrives, then outgoing e-mail does not work
correctly.

Note that this problem does not manifest itself
on non-BSD Unix platforms, such as Linux.

%%%%%%%%%%%%%%%%%%%%%%%%%%%%%%%%%%%%%%%%%%%%%%%%
\subsection*{\label{sec:xen-jiffies-bug}Why are there spurious Condor errors on some machines running Xen kernels?}
%%%%%%%%%%%%%%%%%%%%%%%%%%%%%%%%%%%%%%%%%%%%%%%%

Some older Xen kernels had a problem where the kernel's jiffy counter
could jump backwards in time. This breaks an assumption made by the
\Condor{procd}. This problem can only be worked around by upgrading
the Xen kernel to a version that fixes the issue with the jiffy counter.
Running Condor on an affected Xen kernel often results in failures
of the following forms in Condor daemon log files:
\begin{itemize}
\item \verb@error: parent process's birthday is later than our own@
\item \verb@ERROR: No family with the given PID is registered@
\end{itemize}

%%%%%%%%%%%%%%%%%%%%%%%%%%%%%%%%%%%%%%%%%%%%%%%%
\section{Other questions}
%%%%%%%%%%%%%%%%%%%%%%%%%%%%%%%%%%%%%%%%%%%%%%%%


%%%%%%%%%%%%%%%%%%%%%%%%%%%%%%%%%%%%%%%%%%%%%%%%
\subsection*{Is there a Condor mailing-list?}
%%%%%%%%%%%%%%%%%%%%%%%%%%%%%%%%%%%%%%%%%%%%%%%%
\index{Condor!mailing lists}
\index{mailing lists}
\index{Condor!new versions, notification of}
\index{Condor!contact information}

Yes. There are two useful mailing lists.
First, we run an extremely low traffic mailing list solely to announce new
versions of Condor.
Follow the instructions for Condor World at
\URL{http://www.cs.wisc.edu/condor/mail-lists/}.
Second, our users can be extremely knowledgeable,
and they help each other solve problems
using the Condor Users mailing list.
Again, follow the instructions for Condor Users at
\URL{http://www.cs.wisc.edu/condor/mail-lists/}.



%%%%%%%%%%%%%%%%%%%%%%%%%%%%%%%%%%%%%%%%%%%%%%%%
\subsection*{My question isn't in the FAQ!}
%%%%%%%%%%%%%%%%%%%%%%%%%%%%%%%%%%%%%%%%%%%%%%%%

If you have any questions that are not listed in this FAQ, try looking
through the rest of the manual.
Try joining the Condor Users mailing list, where our users
support each other in finding answers to problems.
Follow the instructions at
\URL{http://www.cs.wisc.edu/condor/mail-lists/}.
If you still can't find an answer, feel free to contact us at
\Email{condor-admin@cs.wisc.edu}.

Note that Condor's free e-mail support is provided on a best-effort
basis, and at times we may not be able to provide a timely response.
If guaranteed support is important to you, please inquire about our
paid support services.



\index{Condor!FAQ|)}
\index{Condor!Frequently Asked Questions|)}
\index{FAQ|)}
\index{Frequently Asked Questions|)}
