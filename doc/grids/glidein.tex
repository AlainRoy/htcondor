%%%%%%%%%%%%%%%%%%%%%%%%%%%%%%%%%%%%%%%%%%%%%%%%%%
\section{\label{sec:Glidein}Glidein}
%%%%%%%%%%%%%%%%%%%%%%%%%%%%%%%%%%%%%%%%%%%%%%%%%%
\index{universe!Globus}
\index{Globus}
\index{Condor commands!condor\_glidein}
\index{glidein}
\index{grid computing!glidein}

Glidein is a mechanism by which one or more Grid resources (remote machines)
temporarily join a local Condor pool. 
The program \Condor{glidein} is used to add a machine
to a Condor pool.
During the period of time when the added resource is
part of the local pool, the resource is visible 
to users of the pool, but, by default, the resource is only available for
use by the user
that added the resource to the pool.

After glidein, the user may submit jobs for execution on the
added resource the same way that all Condor jobs are submitted.
To force a submitted job to run on the added resource, the
submit description file could contain a requirement that the job run 
specifically on the added resource.


%%%%%%%%%%%%%%%%%%%%%%%%%%%%%%%%%%%%%%%%%%%%%%%%%%
\subsubsection{\Condor{glidein} Requirements}
%%%%%%%%%%%%%%%%%%%%%%%%%%%%%%%%%%%%%%%%%%%%%%%%%%

The local Condor pool configuration file(s) must 
give \Macro{HOSTALLOW\_WRITE} permission
to every resource that will be added using \Condor{glidein}. 
Wildcards are permitted in this specification.
For example, you can add every machine at
cs.wisc.edu by adding *.cs.wisc.edu to the
\MacroNI{HOSTALLOW\_WRITE} list.
Recall that you must run \Condor{reconfig} for configuration
file changes to take effect.

If it is undesirable to modify the security settings on
your primary Condor pool, you can simply run your own
personal Condor pool (which may exist entirely on a single
machine and coexist with other instances of Condor).  The glidein
resources may then join this personal Condor pool, because
you can set the security settings however you want.  Using
flocking, you can still have your jobs run on the combination
of your personal glidein pool and any other Condor pools to
which you have access.

%%%%%%%%%%%%%%%%%%%%%%%%%%%%%%%%%%%%%%%%%%%%%%%%%%
\subsubsection{What \Condor{glidein} Does}
%%%%%%%%%%%%%%%%%%%%%%%%%%%%%%%%%%%%%%%%%%%%%%%%%%

\Condor{glidein} first contacts the Globus resource and checks for the
presence of the necessary configuration files and Condor executables.
If the executables are not present for the machine architecture,
operating system version, and Condor version required,
\Condor{glidein} will attempt an automatic installation.  If you
need more control over how this works, see page~\pageref{man-condor-glidein}.

When the files are correctly in place,
Condor daemons are started.
\Condor{glidein} does this by creating a submit description file for
\Condor{submit}, which runs the \Condor{master} under the Globus
universe.
Once \Condor{master} begins running, it runs \Condor{startd}, which phones
home to your \Condor{collector} to join your pool.
The Condor daemons exit gracefully when no jobs run on the daemons for a
configurable period of time. The default length of time is 20 minutes.

By default, the \Expr{START}
expression for the \Condor{startd} daemon requires that the username
of the person running \Condor{glidein} matches the username of the jobs
submitted through Condor.

Here is an example of how a glidein resource appears, similar to how
any other machine appears in your Condor pool.  The name just has a
slightly different form, in order to handle the possibility of
multiple instances of glidein daemons inhabiting a multi-processor
machine.

\footnotesize
\begin{verbatim}
% condor_status | grep denal
7591386@denal IRIX65      SGI    Unclaimed  Idle       3.700  24064  0+00:06:35
\end{verbatim}
\normalsize

Once the Globus resource has been added to the local Condor
pool with \Condor{glidein},
job(s) may be submitted.
To force a job to run on the Globus resource,
specify that Globus resource as a machine requirement
in the submit description file. 
Here is an example from within the submit description file
that forces submission to the Globus resource denali.mcs.anl.gov:
\begin{verbatim}
      requirements = ( machine == "denali.mcs.anl.gov" ) \
         && FileSystemDomain != "" \
         && Arch != "" && OpSys != ""
\end{verbatim}
This example requires that the job run only on denali.mcs.anl.gov,
and it prevents Condor from inserting the filesystem domain,
architecture, and operating system attributes as requirements
in the matchmaking process.
Condor must be told not to use the submission machine's
attributes in those cases
where the Globus resource's attributes
do not match the submission machine's attributes and your job
really is capable of running on the target machine.  You
may want to use Condor's file-transfer capabilities in order
to copy input and output files back and forth between the submission
and execution machine.
