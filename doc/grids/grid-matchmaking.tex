%%%%%%%%%%%%%%%%%%%%%%%%%%%%%%%%%%%%%%%%%%%%%%%%%%
\subsection{\label{sec:Condor-G-Matchmaking}Condor-G-Matchmaking}
%%%%%%%%%%%%%%%%%%%%%%%%%%%%%%%%%%%%%%%%%%%%%%%%%%
\index{universe!Globus}
\index{Globus}
\index{matchmaking!on the Grid}
\index{grid computing!matchmaking}

In it simplest usage, Condor-G allows users to specify the single
grid site they wish to submit their job to.
Often this is sufficient: perhaps a user knows exactly which
grid site they wish to use,
or a higher-level resource broker
(such as the European Data Grid's resource broker)
has decided which grid site should be used.
But when users have a variety of sites to choose from and there
is no other resource broker to make the decision, Condor-G can use
matchmaking to decide which grid site a job should run on. 

Please note that Condor-G's matchmaking ability is relatively
new. Work is being done to improve it and make it easier to use. For
now, please expect some rough edges. 

Condor-G uses the same matchmaking mechanism that Condor uses: the
\Condor{collector} and \Condor{negotiator} daemons, which are described in
Section~\ref{sec:Condor-Daemons}. 

Two changes are required to use Condor-G's matchmaking.
First,
advertise grid sites that are available so that they are
known and considered during the matchmaking process.
This is accomplished by writing ClassAd attributes and
using \Condor{advertise} to place the attributes into the
ClassAd used in matchmaking.
The second change is to the
submit description file.
This file needs to specify requirements that describe what
type of grid site can be used, instead of identifying a specific grid site.

% Karen had editted to this point.

%%%%%%%%%%%%%%%%%%%%%%%%%%%%%%%%%%%%%%%%%%%%%%%%%%
\subsubsection{Advertising grid sites to Condor-G}
%%%%%%%%%%%%%%%%%%%%%%%%%%%%%%%%%%%%%%%%%%%%%%%%%%

Each grid site that is available for matching purposes needs to be
advertised to the \Condor{collector}. Normally in Condor this is done
with the \Condor{startd} daemon, and you do not normally need to be
aware of the contents of this advertisement. Currently, there is no
equivalent to the \Condor{startd} daemon for advertising grid sites,
so you need have a deeper understanding. 

To properly advertise a grid site, a ClassAd need to be sent
periodically to the \Condor{collector}. A ClassAd is a list of
attributes and values that describe a job, a machine, or a grid
site. ClassAds are briefly described in
Section~\ref{sec:matchmaking-with-classads} and some of the common
attributes of machine ClassAds are described in
Section~\ref{user-man-machad}.

When you advertise a grid site, it looks very similar to a ClassAd for
a machine. In fact, the \Condor{collector} will believe it is a
machine, but with a different set of attributes. 

To advertise a grid site, you first need to describe the site in a
file. Here is a sample ClassAd that describes a grid site:

\footnotesize
\begin{verbatim}
# This is a comment
MyType                = "Machine"
TargetType            = "Job"
Name                  = "Example1_Gatekeeper"
gatekeeper_url        = "grid.example.com/jobmanager"
Requirements          = (CurMatches < 10) && (TARGET.JobUniverse == 9)
Rank                  = 0.000000
CurrentRank           = 0.000000
WantAdRevaluate       = True
UpdateSequenceNumber  = 4
CurMatches            = 0
\end{verbatim}
\normalsize

Let's look at each line:

\begin{verbatim}
# This is a comment
\end{verbatim}

Your file can have comments that begin with the hash mark (\#). 

\begin{verbatim}
MyType                = "Machine"
\end{verbatim}

Your grid site is pretending to be a single machine, for the purpose
of matchmaking. \Attr{MyType} is an attribute that the \Condor{negotiator}
daemon
will expect to be a string. Strings must be surrounded by double-quote
marks, as in this example. You may have surprising, unintuitive errors
if they are not quoted. You will always want \Attr{MyType} to be
``Machine''. 

\begin{verbatim}
TargetType            = "Job"
\end{verbatim}

This is an attribute that says the grid site (machine) wants to be
matched with a job. Leave this as it is. 


\footnotesize
\begin{verbatim}
Name                  = "Example1_Gatekeeper"
\end{verbatim}
\normalsize

You will want a unique name for each grid site. Any name is fine, as long as
it is quoted.

\footnotesize
\begin{verbatim}
gatekeeper_url        = "grid.example.com/jobmanager"
\end{verbatim}
\normalsize

This is the Globus gatekeeper contact string for your grid site. It is
probably a machine name followed by a slash followed by the name of
the jobmanager. If you have different job managers, you can only
specify one per ClassAd. 

\begin{verbatim}
UpdateSequenceNumber  = 4
\end{verbatim}

UpdateSequenceNumber is a positive number that must increase each time
you advertise a grid site. Normally you advertise your grid site
every five minutes. The \Condor{collector} daemon will discard a grid site's
ClassAd after 15 minutes if there have been no updates. A good number
to set this to is the current time in seconds (the epoch, as given by
the C \Procedure{time} function call), but if you are worried about your clock
running backward, you can set it to whatever you like. If ClassAds are
received with a sequence number older than the last ClassAd, they are
ignored. 

\begin{verbatim}
CurMatches            = 0
\end{verbatim}

This number is incremented each time a match is made for this grid
site. Unlike a normal machine ClassAd that can only be matched against
once, grid site advertisements can be matched against many time. 

You will probably want to set this number to be the number of grid
jobs that you have running on your site, and keep it updated each time
you submit a new ClassAd. If you do not specify CurMatches, Condor
will assume it is 0.

Condor will increment this number every time it makes a match against
a grid site.

\footnotesize
\begin{verbatim}
Requirements          = (CurMatches < 10) && (TARGET.JobUniverse == 9)
\end{verbatim}
\normalsize

These are the requirements that the grid site insists must be true
before it will accept a job. These could refer to features of the
job's ClassAd. In this case, we will take any globus-universe job, as
long we have less than 10 matches currently. This will ensure that
Condor-G will only run 10 jobs at your site---assuming that you keep
CurMatches up to date when jobs finish. Of course, you can edit this
statement to have different requirements. For example, if you want to
accept all jobs, you can have \ShortExpr{Requirements = True}.

\footnotesize
\begin{verbatim}
Rank                  = 0.000000
CurrentRank           = 0.000000
\end{verbatim}
\normalsize

This is a numerical ranking that will be assigned to a job. Right now
it is not used, but should be set to 0. 

\begin{verbatim}
WantAdRevaluate       = True
\end{verbatim}

The \Attr{WantAdRevaluate} attribute distinguishes grid site
ClassAds from normal machine ClassAds and allows multiple matches to
be made against a single site. It should be in your ad and should be
true. Note that True is not in quotes, and it should not be.

You can add other attributes to your ClassAd, to make it easy for a
job to decide which grid site it wants to use. For instance, if you
have pre-installed the Bamboozle software environment on your grid
site, you could advertise, \ShortExpr{HaveBamboozle = True} and
\ShortExpr{BamboozleVersion = 10}. Jobs can require a grid site that has
Bamboozle installed by extending their requirements with
\ShortExpr{HaveBamboozle == True}. (Note the double equal sign in the
requirements.) 

As an aside, we recommend that jobs that need specific applications
should bring them with them instead of relying on having them
pre-installed at a Grid site. You will have more reliable execution if
you do. 

Once you have a file that describes your site, you need to send it to
the \Condor{collector} daemon. For this, use \Condor{advertise}.
We recommend that you write a script to create the file
containing the ClassAd, then run the script every five minutes with
\Prog{cron}. The script should probably update the \Attr{CurMatches}
variable, if you
want to restrict the number of grid jobs that can be submitted at one
time. 

For \Condor{advertise}, specify \Arg{UPDATE\_STARTD\_AD} for
the update command. For example, if your ClassAd is specified in a
file named \File{grid-ad} you would do:

\footnotesize
\begin{verbatim}
    condor_advertise UPDATE_STARTD_AD grid-ad
\end{verbatim}
\normalsize

\Condor{advertise} usually uses UDP to transmit your ClassAd. In
wide-area networks, this may be insufficient. You can use TCP by
specifying the \Opt{-tcp} option. 

%%%%%%%%%%%%%%%%%%%%%%%%%%%%%%%%%%%%%%%%%%%%%%%%%%
\subsubsection{Submitting Condor-G jobs that use matchmaking}
%%%%%%%%%%%%%%%%%%%%%%%%%%%%%%%%%%%%%%%%%%%%%%%%%%

Submitting a job to Condor-G that requires matchmaking is
straightforward. Instead of specifying a particular scheduler with
globussheduler like this:

\footnotesize
\begin{verbatim}
globusscheduler = grid.example.com/jobmanager
\end{verbatim}
\normalsize

you instead specify requirements and tell Condor-G where to find the
gatekeeper URL in the grid site ClassAd:

\footnotesize
\begin{verbatim}
globusscheduler = $$(gatekeeper_url)
requirements    = TARGET.gatekeeper_url =!= UNDEFINED
\end{verbatim}
\normalsize

This will allow to run at any grid site, and will extract the
gatekeeper\_url attribute from the ClassAd. There is no magic meaning
behind gatekeeper\_url---you could use GatekeeperContactString if you
desired, as long as it is the same in both the job description and the
grid site ClassAd. 

The requirements specified here are a bit simple. Perhaps you only
want to run at a site that has the Bamboozle software installed, and
the sites that have it installed specify ``HaveBamboozle = True'', as
described above. A complete job description may look like this:

\footnotesize
\begin{verbatim}
universe        = globus
executable      = analyze_bamboozle_data
output          = aaa.$(Cluster).out
error           = aaa.$(Cluster).err
log             = aaa.log
globusscheduler = $$(gatekeeper_url)
requirements    = (HaveBamboozle == True) && (TARGET.gatekeeper_url =!= UNDEFINED)
leave_in_queue  = jobstatus == 4
queue
\end{verbatim}
\normalsize

%%%%%%%%%%%%%%%%%%%%%%%%%%%%%%%%%%%%%%%%%%%%%%%%%%
\subsubsection{Advanced usage}
%%%%%%%%%%%%%%%%%%%%%%%%%%%%%%%%%%%%%%%%%%%%%%%%%%

What if a job fails to run at a grid site due to an error? It will be
returned to the queue, and Condor will attempt to match it and
re-run it at another site. Condor isn't very clever about avoiding
sites that may be bad, but you can give it some assistance. Let's say
that you want to avoid running at the last grid site you ran at. You
could add this to your job description:

\footnotesize
\begin{verbatim}
match_list_length = 1
Rank              = TARGET.Name != LastMatchName0
\end{verbatim}
\normalsize

This will prefer to run at a grid site that was not just tried, but it
will allow the job to be run there if there is no other option. 

When you specify \Opt{match\_list\_length}, you provide an integer N, and
Condor will keep track of the last N matches. The oldest match will be
LastMatchName0, and next oldest will be LastMatchName1, and so on. (See
the \Condor{submit} manual page for more details.) The Rank expression
allows you to specify a numerical ranking for different matches. When
combined with \Opt{match\_list\_length}, you can prefer to avoid sites that
you have already run at. 

In addition, \Condor{submit} has two options to help you control
Condor-G job resubmissions and rematching.  See \Opt{globus\_resubmit} and
\Opt{globus\_rematch} in the \Condor{submit} manual page.
These options are independent of \Opt{match\_list\_length}.

There are some new attributes that will be added to the Job ClassAd,
and may be useful to you when you write your rank, requirements,
globus\_resubmit or globus\_rematch option. Please refer to
Section~\ref{user-man-jobad} and read about the following option:

\begin{itemize}
\item NumJobMatches
\item NumGlobusSubmits
\item NumSystemHolds
\item HoldReason
\item ReleaseReason
\item EnteredCurrentStatus
\item LastMatchTime
\item LastRejMatchTime
\item LastRejMatchReason
\end{itemize}

The following example of a command within the submit description file
releases jobs 5 minutes after being held,
increasing the time between releases by 5 minutes each time.
It will continue to retry up to 4 times per Globus
submission, plus 4.
The plus 4 is necessary in case
the job goes on hold before being submitted to Globus, although
this is unlikely.

\footnotesize
\begin{verbatim}
periodic_release = ( NumSystemHolds <= ((NumGlobusSubmits * 4) + 4) ) \
   && (NumGlobusSubmits < 4) && \
   ( HoldReason != "via condor_hold (by user $ENV(USER))" ) && \
   ((CurrentTime - EnteredCurrentStatus) > ( NumSystemHolds *60*5 ))
\end{verbatim}
\normalsize

The following example forces Globus resubmission after a job has
been held 4 times per Globus submission.

\footnotesize
\begin{verbatim}
globus_resubmit = NumSystemHolds == (NumGlobusSubmits + 1) * 4
\end{verbatim}
\normalsize

If you are concerned about unknown or malicious grid sites reporting
to your \Condor{collector}, you should use Condor's security options,
documented in Section~\ref{sec:Security}.
