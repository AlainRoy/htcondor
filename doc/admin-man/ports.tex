%%%%%%%%%%%%%%%%%%%%%%%%%%%%%%%%%%%%%%%%%%%%%%%%%%%%%%%%%%%%%%%%%%%%%%%%%%%
\subsection{\label{sec:Port-Details}Port Usage in Special Environments }
%%%%%%%%%%%%%%%%%%%%%%%%%%%%%%%%%%%%%%%%%%%%%%%%%%%%%%%%%%%%%%%%%%%%%%%%%%

\index{port usage}

%%%%%%%%%%%%%%%%%%%%%%%%%%%%%%%%%%%%%%%%%%%%%%%%%%%%%%%%%%%%%%%%%%%%%%%%%%%
\subsubsection{\label{sec:Ports-NonStandard}Non Standard Port Usage}
%%%%%%%%%%%%%%%%%%%%%%%%%%%%%%%%%%%%%%%%%%%%%%%%%%%%%%%%%%%%%%%%%%%%%%%%%%%
\index{port usage!nonstandard ports for central managers}
By default,
Condor uses port 9618 for the \Condor{collector} daemon.
To use a non well-known port number for this daemon,
the configuration variables that tell Condor these communication
details are modified.
Instead of
\begin{verbatim}
CONDOR_HOST = machX.cs.wisc.edu
COLLECTOR_HOST = $(CONDOR_HOST)
\end{verbatim}
the configuration might be
\footnotesize
\begin{verbatim}
CONDOR_HOST = machX.cs.wisc.edu
COLLECTOR_HOST = $(CONDOR_HOST):9650
\end{verbatim}
\normalsize

If a non well-known port is defined, the same value of
\MacroNI{COLLECTOR\_HOST} (including the port) must be used for all
machines in the Condor pool.
Therefore, this setting should be modified in the global
configuration file (\File{condor\_config} file),
or the value must be duplicated across
all configuration files in the pool if a single configuration file
is not being shared.

On single-machine pools, 
it is permitted to configure the
\Condor{collector} daemon
to use a dynamically assigned port,
as given out by the operating system.
This prevents port conflicts with other services on the same machine.
However, a dynamically assigned port is only to be used on
single-machine Condor pools,
and only if the
\Macro{COLLECTOR\_ADDRESS\_FILE} 
configuration variable has also been defined.
This mechanism allows all of the Condor daemons and tools running on
the same machine to find the real port for the \Condor{collector}
daemon,
even when this port is not defined in the
configuration file and is not known in advance.

To enable this daemon to use a dynamic port,
the port number is set to 0 in the \Macro{COLLECTOR\_HOST}
variable.
The \MacroNI{COLLECTOR\_ADDRESS\_FILE}
configuration variable must also be defined,
as it provides a known file where the IP address
and port information that is dynamically generated will be stored.
All Condor clients know to look at the
information stored in this file.
For example:
\footnotesize
\begin{verbatim}
COLLECTOR_HOST = $(CONDOR_HOST):0
COLLECTOR_ADDRESS_FILE = $(LOG)/.collector_address
\end{verbatim}
\normalsize

\Note Using a port of 0 for the \Condor{collector}
and specifying a
\MacroNI{COLLECTOR\_ADDRESS\_FILE}
only works in Condor version 6.6.8 or later in the 6.6 stable series,
and in version 6.7.4 or later in the 6.7 development series.
Do not attempt to use this setting with older versions of Condor.

Configuration definition of \MacroNI{COLLECTOR\_ADDRESS\_FILE}
is in section~\ref{param:SubsysAddressFile} on
page~\pageref{param:SubsysAddressFile},
and
\MacroNI{COLLECTOR\_HOST}
is in
section~\ref{param:CollectorHost} on
page~\pageref{param:CollectorHost}.



%%%%%%%%%%%%%%%%%%%%%%%%%%%%%%%%%%%%%%%%%%%%%%%%%%%%%%%%%%%%%%%%%%%%%%%%%%%
\subsubsection{\label{sec:Ports-Firewalls}Firewalls}
%%%%%%%%%%%%%%%%%%%%%%%%%%%%%%%%%%%%%%%%%%%%%%%%%%%%%%%%%%%%%%%%%%%%%%%%%%%

\index{port usage!firewalls}
If a Condor pool is completely behind a firewall,
then no special consideration is needed.
However, if there is a firewall between the machines within
a Condor pool, then
configuration variables may be set to force the usage of
specific ports and to utilize a specific range of ports.

By default,
Condor uses port 9618 for the \Condor{collector} daemon,
and dynamic (apparently random) ports for everything else.
See section~\ref{sec:Ports-NonStandard},
if a dynamic port is desired for the
\Condor{collector} daemon.

The configuration variables
\Macro{HIGHPORT} and \Macro{LOWPORT} facilitate setting a restricted
range of ports that Condor will use.
This may be useful when some machines are behind a firewall.
The configuration macros
\MacroNI{HIGHPORT} and \MacroNI{LOWPORT} 
will restrict dynamic ports to the range specified.
The configuration variables are fully defined
in section~\ref{sec:Condor-wide-Config-File-Entries}.
Note that both \MacroNI{HIGHPORT} and \MacroNI{LOWPORT} must be at 
least 1024 for Condor version 6.6.8.

The total number of ports needed depends on the size of the pool,
the usage of the machines within the pool (which machines
run which daemons),
and the number of jobs that may execute at one time.
Here we discuss how many ports are used by each
participant in the system.

The central manager of the pool needs
\Expr{5 + \MacroNI{NEGOTIATOR\_SOCKET\_CACHE\_SIZE}}
ports for daemon communication,
where 
\Macro{NEGOTIATOR\_SOCKET\_CACHE\_SIZE}
is specified in the
configuration or defaults to the value 16.

Each execute machine (those machines running a \Condor{startd} daemon)
requires
\Expr{ 5 + (5 * number of virtual machines advertised by that machine)}
ports.
By default, the number of virtual machines advertised
will equal the number of physical CPUs in that machine.

Submit machines (those machines running a \Condor{schedd} daemon)
require
\Expr{ 5 + (5 *  \MacroNI{MAX\_JOBS\_RUNNING})} ports.
The configuration variable \Macro{MAX\_JOBS\_RUNNING}
limits (on a per-machine basis, if desired)
the maximum number of jobs.
Without this configuration macro,
the maximum number of jobs that could be simultaneously
executing at one time
is a function of the number of reachable execute machines. 

Also be aware that \MacroNI{HIGHPORT} and \MacroNI{LOWPORT}
only impact dynamic port selection used by the Condor system,
and they do not impact port selection used by jobs submitted to Condor.
Thus, jobs submitted to Condor that may create
network connections may not work in a port restricted environment.
For this reason, specifying \MacroNI{HIGHPORT} and \MacroNI{LOWPORT}
is not going to produce the
expected results if a user submit jobs to be executed under
the PVM or MPI job universes.

Where desired, a local
configuration for machines \emph{not} behind a firewall
can override the usage of \MacroNI{HIGHPORT} and \MacroNI{LOWPORT},
such that the ports used for these machines are not restricted.
This can be accomplished by adding the following to the
local configuration file of those machines not
behind a firewall:
\begin{verbatim}
HIGHPORT = UNDEFINED
LOWPORT  = UNDEFINED
\end{verbatim}


If the maximum number of ports allocated using 
\MacroNI{HIGHPORT} and \MacroNI{LOWPORT}
is too few,
socket binding errors of the form
\footnotesize
\begin{verbatim}
failed to bind any port within <$LOWPORT> - <$HIGHPORT>
\end{verbatim}
\normalsize
are like to appear repeatedly in log files.

%%%%%%%%%%%%%%%%%%%%%%%%%%%%%%%%%%%%%%%%%%%%%%%%%%%%%%%%%%%%%%%%%%%%%%%%%%%
\subsubsection{\label{sec:Ports-MultipleCollectors}Multiple Collectors}
%%%%%%%%%%%%%%%%%%%%%%%%%%%%%%%%%%%%%%%%%%%%%%%%%%%%%%%%%%%%%%%%%%%%%%%%%%%
\index{port usage!multiple collectors}
\Todo


%%%%%%%%%%%%%%%%%%%%%%%%%%%%%%%%%%%%%%%%%%%%%%%%%%%%%%%%%%%%%%%%%%%%%%%%%%%
\subsubsection{\label{sec:Ports-Conflicts}Port Conflicts}
%%%%%%%%%%%%%%%%%%%%%%%%%%%%%%%%%%%%%%%%%%%%%%%%%%%%%%%%%%%%%%%%%%%%%%%%%%%
\index{port usage!conflicts}
\Todo

