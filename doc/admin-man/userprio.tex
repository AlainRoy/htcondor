%%%%%%%%%%%%%%%%%%%%%%%%%%%%%%%%%%%%%%%%%%%%%%%%%%%%%%%%%%%%%%%%%%%%%%
\section{\label{sec:UserPrio}User Priorities}
%%%%%%%%%%%%%%%%%%%%%%%%%%%%%%%%%%%%%%%%%%%%%%%%%%%%%%%%%%%%%%%%%%%%%%

% Karen's understanding of this stuff, in preparation for a re-write
% of the section:

% Users request machines (by submitting jobs).
% Each user has a calculated priority.
%   A larger priority is worse.
%   This priority essentially tells how many machines the user is
%   currently using.  The priority can be made worse (larger number)
%   by the settings of various configuration variables.
% During each negotiation cycle, all machines are allocated (presuming
%   that there are more requests than machines).
% Each user is allocated machines in a ratio of 1/users's priority.
% Within the negotiation cycle, each user is given an initial
%   allocation of machines.  From there, remaining unallocated machines
%   are divided up among users that want more.  In a round robin
%   manner, each user is allocated a fraction of the remaining
%   unallocated machines.  this fraction is 1/user's priority.

\index{priority!in machine allocation}
\index{user priority}
Condor uses priorities to determine machine allocation for jobs.
This section details the priorities.

For accounting purposes, each user is identified by username@uid\_domain.
Each user is assigned a priority value even if submitting jobs from
different machines in the same domain, or even if submitting from multiple
machines in the different domains.

The numerical priority value assigned to a user is inversely related to the 
\emph{goodness} of the priority.
A user with a numerical priority of 5 gets 
more resources than a user with a numerical priority of 50.
There are two 
priority values assigned to Condor users:
\begin{itemize}
	\item Real User Priority (RUP), which measures resource usage of the 
		user.
	\item Effective User Priority (EUP), which determines the number of
		resources the user can get.
\end{itemize}
This section describes these two priorities and how they affect resource
allocations in Condor.
Documentation on configuring and controlling 
priorities may be found in section~\ref{sec:Negotiator-Config-File-Entries}.

\subsection{Real User Priority (RUP)}
\index{real user priority (RUP)}
\index{user priority!real (RUP)}
A user's RUP measures the resource usage of the user 
through time.
Every user begins with a RUP of one half (0.5), and
at steady state, the RUP of a user equilibrates to the number of resources 
used by that user.  Therefore, if a specific user continuously uses exactly 
ten resources for a long period of time, the RUP of that user stabilizes at 
ten.

However, if the user decreases the number of resources used, the RUP
gets better.  The rate at which the priority value decays 
can be set by the macro \Macro{PRIORITY\_HALFLIFE}, a time period 
defined in seconds.   Intuitively, if the \Macro{PRIORITY\_HALFLIFE} in a pool 
is set to 86400 (one day), and if a user whose RUP was 10 removes all his 
jobs, the user's RUP would be 5 one day later, 2.5 two days later,
and so on.

\subsection{Effective User Priority (EUP)}
\index{effective user priority (EUP)}
\index{user priority!effective (EUP)}
The effective user priority (EUP) of a user is used to determine
how many resources that user may receive.
The EUP is linearly related to the RUP
by a \emph{priority factor} which may be defined on a per-user basis.
Unless otherwise configured, the priority factor for all users is 1.0,
and so the EUP is the same as the the RUP.
However, if desired, the priority factors of
specific users (such as remote submitters) can be increased so that 
others are served preferentially.

The number of resources that a user may receive is inversely related
to the ratio between the EUPs of submitting users.
Therefore user $A$ with EUP=5 will receive
twice as many resources as user $B$ with EUP=10 and four times as many 
resources as user $C$ with EUP=20.
However, if $A$ does not use the full number
of allocated resources,
the available resources are repartitioned and distributed among
remaining users according to the inverse ratio rule.

% editted to here

Condor supplies mechanisms to directly support two policies in which EUP may
be useful:
\begin{description}
	\item[Nice users]  A job may be submitted with the parameter 
	\AdAttr{nice\_user} set to TRUE in the submit command file.
	A nice user job gets its RUP boosted by the 
	\Macro{NICE\_USER\_PRIO\_FACTOR} priority factor specified in the 
	configuration file, leading to a (usually very large) EUP.
	This corresponds to a low priority for resources.
	These jobs are therefore equivalent to Unix background jobs,
	which use resources not used by other Condor users.

	\item[Remote Users] The flocking feature of Condor (see
	section~\ref{sec:Flocking}) allows the \Condor{schedd} to
	submit to more than one pool.
	In addition, the submit-only feature allows a user to run a
	\Condor{schedd} that is submitting jobs into another pool.
	In such situations, submitters from other domains
	can submit to the local pool.
	It is often desirable to have Condor treat local users
	preferentially over these remote users.
	If configured, Condor will boost the RUPs of remote users by
	\Macro{REMOTE\_PRIO\_FACTOR}
	specified in the configuration file,
	thereby lowering their priority for resources.
\end{description}

The priority boost factors for individual users can be set with the 
\Opt{setfactor} option of \Condor{userprio}.
Details may be found in the \Condor{userprio} manual page 
on page~\pageref{man-condor-userprio}.

\subsection{Priorities and Preemption}
\index{preemption!priority}
Priorities are used to ensure that users get their fair share of resources.  
The priority values are used at allocation time.
In addition, Condor preempts machine claims and reallocates them when
conditions change.

To ensure that preemptions do not lead to \Term{thrashing},
a \Macro{PREEMPTION\_REQUIREMENTS} expression is defined to specify the
conditions that must be met for a preemption to occur.
It is usually defined to deny preemption if a current running job
has been running for a relatively short period of time.
This effectively limits the number of preemptions per resource per time
interval.

Note that \MacroNI{PREEMPTION\_REQUIREMENTS} only applies to preemptions
due to user priority.  It does not have any effect if the machine rank
expression prefers a different job, or if the startd policy expression
causes the job to vacate due to other activity on the machine.

\subsection{Priority Calculation}
This section may be skipped if the reader so feels, but for the curious,
here is Condor's priority calculation algorithm.

The RUP of a user $u$ at time $t$, $\pi_r(u,t)$, is calculated 
every time interval $\delta t$ using the formula 
$$\pi_r(u,t) = \beta\times\pi(u,t-\delta t) + (1-\beta)\times\rho(u,t)$$
where $\rho(u,t)$ is the number of resources used by user $u$ at time $t$,
and $\beta=0.5^{{\delta t}/h}$. $h$ is the half life period set by 
\Macro{PRIORITY\_HALFLIFE}.

The EUP of user $u$ at time $t$, $\pi_e(u,t)$
is calculated by
$$\pi_e(u,t) = \pi_r(u,t)\times f(u,t)$$
where $f(u,t)$ is the priority boost factor for user $u$ at time $t$.

As mentioned previously, the RUP calculation is designed so that at steady
state, each user's RUP stabilizes at the number of resources used by that user. 
The definition of $\beta$ ensures that the calculation of $\pi_r(u,t)$ can be 
calculated over non-uniform time intervals $\delta t$ without affecting the 
calculation.  The time interval $\delta t$ varies due to events internal to 
the system, but Condor guarantees that unless the central manager machine is 
down, no matches will be unaccounted for due to this variance.

% Derek's explanation:
%  > Preferably the user priority is determined by the number of
%  > processors jobs of the user currently occupy, i.e., the "history"
%  > should not play a role.
%  
%  this is the responsibility of the condor "accountant", which lives
%  inside the condor_negotiator daemon.  the knob you want to turn is
%  called "PRIORITY_HALFLIFE".  think of your user priority as a
%  radioactive substance. :) consider a priority that exactly matches
%  your current resource usage the "stable state", and a priority
%  "contaminated" with past usage "radioactive."  if it's got a long
%  halflife, it takes a long time for your priority to decay back to
%  "normal".  if the halflife is very short, it'll decay very quickly,
%  and will remain very close to your current usage.  so, just set
%  PRIORITY_HALFLIFE to a small floating point value (like 0.0001), and
%  your user priority should always match your current usage.  if you're
%  not using any resources, your priority will go back to the baseline
%  value instantly.

\subsection{Group Accounting}
\label{sec:group-accounting}

By default, Condor does all accounting on a per-user basis, and this
accounting is primarily used to compute priorities for Condor's
fair-share scheduling algorithms. However, accounting can also be done
on a per-group basis.  Multiple users can all submit jobs into the
same accounting group, and all of the jobs will be treated with the
same priority.

To use group-based instead of user-based accounting for a specific
job, the user must insert an attribute into the job ClassAd which
defines what accounting group to use for the job.  To do so, the user
must put the following line into their job's submit description file:
\begin{verbatim}
    +AccountingGroup = "group_physics"
\end{verbatim}

The AccountingGroup attribute is a string and must be enclosed in
double quote marks.  It can have a maximum length of 40 characters.
Other than that, no changes or user intervention are required.  This
name should *not* be qualified with a domain.  However, certain parts
of the Condor system append the UID\_DOMAIN (as specified in the
configuration file on the submit machine) to this string and use that.
For example, if the UID\_DOMAIN is ``example.com'' and the above
AccountingGroup was specified, \condor{userprio} would show statistics
for this accounting group as:
\begin{verbatim}
                                        Effective
    User Name                           Priority
    ------------------------------      ---------
    group_physics@example.com                0.50
    user@example.com                        23.11
    heavyuser@example.com                  111.13
    ...
\end{verbatim}

Additionally, the \condor{userprio} command allows administrators to
remove an entity from the accounting system in Condor.  For example,
if all the jobs from a given AccountingGroup are completed and the
administrator wishes to remove that group from the system, they may do
so using the new ``-delete" option to \condor{userprio}.  However, the
administrator must use the fully-qualified version of the
AccountingGroup. For example:
\begin{verbatim}
    condor_userprio -delete group_physics@example.com
\end{verbatim}

Note that the name of the group can include a period (.), and it has
no special interpretation for group accounting. In the next section,
group users are introduced, and a period has special meaning. However,
that meaning is only for group quotas, and not group accounting. 

\subsection{Group Quotas}
\label{sec:group-quotas}

User priorities are insufficient for some Condor pools. In some cases,
different groups might own computers in the pool, though they are
willing to share them. For example, imagine a Condor pool with thirty
computers. Twenty computers are owned by the physics group and ten
computers are owned by the chemistry group. The computers could be
configured so that only the physicists can access the physics
computers and only the chemists can access the chemistry computers,
but this is needlessly constraining: if the computers are similar, the
users do not really care which computers they get, as long as they get
enough. Group quotas provide the flexibility needed for this task: it
allows policies like ``the physicists can use up to twenty of the
computers in the pool, and they can be any of the computers that are
available''.

Group quotas require the notion of group users. The syntax for
specifying a group user is ``group.user''. For example, you might have
``group\_physics.roy'' and ``group\_chemistry.smith''. The period (.)
character indicates that a user is actually a group user for
Condor. Group names are not required to begin with ``group'' as in
these examples, but it is be a useful convention because group names
cannot be the same as user names. The configuration file specifies the
number of virtual machines that are in the quota for each group. The
total quota for all groups must be less than or equal to the number of
virtual machines in the entire pool. Any computers not accounted for
by the quota are allocated to the ``none'' group, which is the general
group of users you have if you do not specify any group quotas.

Note that accounting will be done per group-user (the whole name), not
per group (the portion before the period). The group accounting does
not treat group users differently than group, but they are all groups
as far as it is concerned. Group users as defined here only have an
impact for quotas.

Negotiation is changed when group quotas are used: Condor negotiates
first for groups, then for independent job submitters.  When jobs are
submitted from multiple groups, Condor negotiates for the group with
the smallest percentage of resource usage first, and the group with
the greatest percentage of resource usage last.  For example, suppose
user jobs are submitted for group user ``group\_physics.roy'',
``group\_chemistry.roy'', and non-group user "roy".  Suppose also that
the physics group is 80\% utilized, and the chemistory group is 5\%
utilized.  Condor will negotiate for the ``group\_chemistry.roy''
first, then ``group\_physics.roy'', then ``roy''.  For independent job
submitters, the classic Condor user fair share algorithm still
applies.

Several configuration variables affect group quotas:

\begin{description}
  
\item[\Macro{GROUP\_NAMES}]
    A list of the recognized group names, separated by a comma,
	case insensitive.  If undefined (the default), then group support
	is disabled.  Note: these names must not conflict with any
	user names. That is, if you have a ``physics'' group, you should
    not have a ``physics'' user.  Any group that is defined here must
    also have quota, or it will be ignored. Example: 
\begin{verbatim}
    GROUP_NAMES = group_physics, group_chemistry 
\end{verbatim}

\item[\Macro{GROUP\_QUOTA\_<groupname>}]
    This is a static quota that specifies the exact number of machines
	owned by this group.  The number of machines must be an integer
	greater than or equal to zero. Note that Condor does not verify
	that the quota values are reasonable or consistent.  
    % When both are defined, static quotas supersede dynamic quotas. 
    Example:

\begin{verbatim}
    GROUP_QUOTA_group_physics = 20
    GROUP_QUOTA_group_chemistry = 10
\end{verbatim}

%\item[\Macro{GROUP\_DYNAMIC\_MACH\_CONSTRAINT}]
%
%Filter the number of machines for use with dynamic groups accounting
%quotas, below.  Without this filter, dynamic quotas are calculated
%from the total of all machines, in all states, that have reported to
%the collector, which is probably not what you want.  Use this filter to
%restrict the number of machines reserved for calculating dynamic
%quotas.  The value is a ClassAd expression, and is evaluated every
%negotiation cycle.  For example, the following expression will only
%consider unclaimed Intel/Linux machines for dynamic quotas.
%
%\begin{verbatim}
%    GROUP_DYNAMIC_MACH_CONSTRAINT = State == "Unclaimed" && \
%    Arch == "INTEL" && OpSys == "LINUX"
%\end{verbatim}

%will only considered. If all computers should be considered for
%dynamic quotas, then the following expression could be used:

%% \begin{verbatim}
%%     GROUP_DYNAMIC_MACH_CONSTRAINT = TRUE
%% \end{verbatim}


%% \item[\Macro{GROUP\_QUOTA\_DYNAMIC\_<groupname>}]

%% Specify a dynamic group accounting quota.  For example, the following
%% specifies that a quota of 25\% of the total dynamic quota machines are
%% reserved for members of the group\_biology group.

%% \begin{verbatim}
%% 	GROUP_QUOTA_DYNAMIC_group_biology = 0.25
%% \end{verbatim}

%% The total dynamic quota machine count is determined using
%% \Macro{GROUP\_DYNAMIC\_MACH\_CONSTRAINT above}. The group name must
%% also be specified in the \Macro{GROUP\_NAMES} list. The value must be
%% positive float between 0.0 and 1.0. Condor does not verify that the
%% quota value is reasonable, nor does Condor verify that all specified
%% quotas are consistent.  This parameter is evaluated whenever Condor
%% negotiates for the group.  When both are defined, static quotas
%% supersede dynamic quotas.

\item[\Macro{GROUP\_PRIO\_FACTOR\_<groupname>}]

    Specify the default user priority factor for groupname.  The
    group\_name must also be specified in the \Macro{GROUP\_NAMES}
    list.  \Macro{GROUP\_PRIO\_FACTOR\_<groupname>} is evaluated when
    the negotiator first negotiates for the user as a member of
    the group.  All members of the group inherit the default
    priority factor.  The value is a floating point, and must be
    greater than or equal to 1.0.  For example, the following setting
    specifies that all member of the group named group\_physics
    inherit a default user priority factor of 2.0:
\begin{verbatim}
    GROUP_PRIO_FACTOR_HACKERS = 2.0
\end{verbatim}

\item[\Macro{GROUP\_AUTOREGROUP}]
    Set this to either True or False.  Defaults to
	false.  If true, then users who submitted to a specific group will
	also negotiate a second time with the "none" group, to be
    considered with the independent job submitters. 
	This allows group submitted jobs to be matched with idle machines
	even if the group is over its quota.

\item[\Macro{NEGOTIATOR\_CONSIDER\_PREEMPTION}]
    Defaults to true.  If set to false, then the negotiator can run
	faster and also with better spinning pie accuracy.  \emph{Only set
	this to false if \Macro{PREEMPTION\_REQUIREMENTS} is false, and
	if all \condor{startd} rank expressions are false.} This should
	only be used by expert 	users.
\end{description}

Users must specify what group they are part of. They do this in their
submit file with AccountingGroup attribute, which must be
specified with the plus sign and with quotes around the name. For
example:
\begin{verbatim}
+AccountingGroup = "group_physics.roy"
\end{verbatim}

Note that there is no verification that a user in the group that he
claims to be. We rely on societal pressure for enforcement. 

