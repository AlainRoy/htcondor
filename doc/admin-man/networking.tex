%%%%%%%%%%%%%%%%%%%%%%%%%%%%%%%%%%%%%%%%%%%%%%%%%%%%%%%%%%%%%%%%%%%%%%
\section{\label{sec:Networking}Networking}
%%%%%%%%%%%%%%%%%%%%%%%%%%%%%%%%%%%%%%%%%%%%%%%%%%%%%%%%%%%%%%%%%%%%%%
\index{network}
\index{networking}

This section contains most of the information about network
communication in Condor.
The topics addressed include how network ports are used; how Condor
behaves on machines with multiple network interfaces and IP addresses;
and how to function in a pool that spans firewalls and private
networks.
Each issue is covered in a separate section, where we describe the
default behavior, and explain how the administrator can configure
alternatives.

The previous section of the manual on security contains some
information that is relevant to the discussion of network
communication which will not be duplicated here, so please
section~\ref{sec:Security} as well.

Firewalls, private networks, and network address translation (NAT)
pose special problems for Condor.
There are currently two main mechanisms for dealing with firewalls
within Condor:

\begin{enumerate}

\item Restricting Condor to use a specific range of port numbers, and
  allow connections through the firewall that use any port within the
  range

\item Using \Term{Generic Connection Brokering} (GCB)

\end{enumerate}

Each method has its own pros and cons, and they are both described in
more detail in separate sections below.


%%%%%%%%%%%%%%%%%%%%%%%%%%%%%%%%%%%%%%%%%%%%%%%%%%%%%%%%%%%%%%%%%%%%%%
% all of these define their own \subsection, so include them directly
%%%%%%%%%%%%%%%%%%%%%%%%%%%%%%%%%%%%%%%%%%%%%%%%%%%%%%%%%%%%%%%%%%%%%%

%%%%%%%%%%%%%%%%%%%%%%%%%%%%%%%%%%%%%%%%%%%%%%%%%%%%%%%%%%%%%%%%%%%%%%%%%%%
\subsection{\label{sec:Port-Details}Port Usage in Special Environments }
%%%%%%%%%%%%%%%%%%%%%%%%%%%%%%%%%%%%%%%%%%%%%%%%%%%%%%%%%%%%%%%%%%%%%%%%%%

\index{port usage}

%%%%%%%%%%%%%%%%%%%%%%%%%%%%%%%%%%%%%%%%%%%%%%%%%%%%%%%%%%%%%%%%%%%%%%%%%%%
\subsubsection{\label{sec:Ports-NonStandard}Non Standard Port Usage}
%%%%%%%%%%%%%%%%%%%%%%%%%%%%%%%%%%%%%%%%%%%%%%%%%%%%%%%%%%%%%%%%%%%%%%%%%%%
\index{port usage!nonstandard ports for central managers}
By default,
Condor uses port 9618 for the \Condor{collector} daemon.
To use a non well-known port number for this daemon,
the configuration variables that tell Condor these communication
details are modified.
Instead of
\begin{verbatim}
CONDOR_HOST = machX.cs.wisc.edu
COLLECTOR_HOST = $(CONDOR_HOST)
\end{verbatim}
the configuration might be
\footnotesize
\begin{verbatim}
CONDOR_HOST = machX.cs.wisc.edu
COLLECTOR_HOST = $(CONDOR_HOST):9650
\end{verbatim}
\normalsize

If a non well-known port is defined, the same value of
\MacroNI{COLLECTOR\_HOST} (including the port) must be used for all
machines in the Condor pool.
Therefore, this setting should be modified in the global
configuration file (\File{condor\_config} file),
or the value must be duplicated across
all configuration files in the pool if a single configuration file
is not being shared.

On single-machine pools, 
it is permitted to configure the
\Condor{collector} daemon
to use a dynamically assigned port,
as given out by the operating system.
This prevents port conflicts with other services on the same machine.
However, a dynamically assigned port is only to be used on
single-machine Condor pools,
and only if the
\Macro{COLLECTOR\_ADDRESS\_FILE} 
configuration variable has also been defined.
This mechanism allows all of the Condor daemons and tools running on
the same machine to find the real port for the \Condor{collector}
daemon,
even when this port is not defined in the
configuration file and is not known in advance.

To enable this daemon to use a dynamic port,
the port number is set to 0 in the \Macro{COLLECTOR\_HOST}
variable.
The \MacroNI{COLLECTOR\_ADDRESS\_FILE}
configuration variable must also be defined,
as it provides a known file where the IP address
and port information that is dynamically generated will be stored.
All Condor clients know to look at the
information stored in this file.
For example:
\footnotesize
\begin{verbatim}
COLLECTOR_HOST = $(CONDOR_HOST):0
COLLECTOR_ADDRESS_FILE = $(LOG)/.collector_address
\end{verbatim}
\normalsize

\Note Using a port of 0 for the \Condor{collector}
and specifying a
\MacroNI{COLLECTOR\_ADDRESS\_FILE}
only works in Condor version 6.6.8 or later in the 6.6 stable series,
and in version 6.7.4 or later in the 6.7 development series.
Do not attempt to use this setting with older versions of Condor.

Configuration definition of \MacroNI{COLLECTOR\_ADDRESS\_FILE}
is in section~\ref{param:SubsysAddressFile} on
page~\pageref{param:SubsysAddressFile},
and
\MacroNI{COLLECTOR\_HOST}
is in
section~\ref{param:CollectorHost} on
page~\pageref{param:CollectorHost}.



%%%%%%%%%%%%%%%%%%%%%%%%%%%%%%%%%%%%%%%%%%%%%%%%%%%%%%%%%%%%%%%%%%%%%%%%%%%
\subsubsection{\label{sec:Ports-Firewalls}Firewalls}
%%%%%%%%%%%%%%%%%%%%%%%%%%%%%%%%%%%%%%%%%%%%%%%%%%%%%%%%%%%%%%%%%%%%%%%%%%%

\index{port usage!firewalls}
If a Condor pool is completely behind a firewall,
then no special consideration is needed.
However, if there is a firewall between the machines within
a Condor pool, then
configuration variables may be set to force the usage of
specific ports and to utilize a specific range of ports.

By default,
Condor uses port 9618 for the \Condor{collector} daemon,
and dynamic (apparently random) ports for everything else.
See section~\ref{sec:Ports-NonStandard},
if a dynamic port is desired for the
\Condor{collector} daemon.

The configuration variables
\Macro{HIGHPORT} and \Macro{LOWPORT} facilitate setting a restricted
range of ports that Condor will use.
This may be useful when some machines are behind a firewall.
The configuration macros
\MacroNI{HIGHPORT} and \MacroNI{LOWPORT} 
will restrict dynamic ports to the range specified.
The configuration variables are fully defined
in section~\ref{sec:Condor-wide-Config-File-Entries}.
Note that both \MacroNI{HIGHPORT} and \MacroNI{LOWPORT} must be at 
least 1024 for Condor version 6.6.8.

The total number of ports needed depends on the size of the pool,
the usage of the machines within the pool (which machines
run which daemons),
and the number of jobs that may execute at one time.
Here we discuss how many ports are used by each
participant in the system.

The central manager of the pool needs
\Expr{5 + \MacroNI{NEGOTIATOR\_SOCKET\_CACHE\_SIZE}}
ports for daemon communication,
where 
\Macro{NEGOTIATOR\_SOCKET\_CACHE\_SIZE}
is specified in the
configuration or defaults to the value 16.

Each execute machine (those machines running a \Condor{startd} daemon)
requires
\Expr{ 5 + (5 * number of virtual machines advertised by that machine)}
ports.
By default, the number of virtual machines advertised
will equal the number of physical CPUs in that machine.

Submit machines (those machines running a \Condor{schedd} daemon)
require
\Expr{ 5 + (5 *  \MacroNI{MAX\_JOBS\_RUNNING})} ports.
The configuration variable \Macro{MAX\_JOBS\_RUNNING}
limits (on a per-machine basis, if desired)
the maximum number of jobs.
Without this configuration macro,
the maximum number of jobs that could be simultaneously
executing at one time
is a function of the number of reachable execute machines. 

Also be aware that \MacroNI{HIGHPORT} and \MacroNI{LOWPORT}
only impact dynamic port selection used by the Condor system,
and they do not impact port selection used by jobs submitted to Condor.
Thus, jobs submitted to Condor that may create
network connections may not work in a port restricted environment.
For this reason, specifying \MacroNI{HIGHPORT} and \MacroNI{LOWPORT}
is not going to produce the
expected results if a user submit jobs to be executed under
the PVM or MPI job universes.

Where desired, a local
configuration for machines \emph{not} behind a firewall
can override the usage of \MacroNI{HIGHPORT} and \MacroNI{LOWPORT},
such that the ports used for these machines are not restricted.
This can be accomplished by adding the following to the
local configuration file of those machines not
behind a firewall:
\begin{verbatim}
HIGHPORT = UNDEFINED
LOWPORT  = UNDEFINED
\end{verbatim}


If the maximum number of ports allocated using 
\MacroNI{HIGHPORT} and \MacroNI{LOWPORT}
is too few,
socket binding errors of the form
\footnotesize
\begin{verbatim}
failed to bind any port within <$LOWPORT> - <$HIGHPORT>
\end{verbatim}
\normalsize
are like to appear repeatedly in log files.

%%%%%%%%%%%%%%%%%%%%%%%%%%%%%%%%%%%%%%%%%%%%%%%%%%%%%%%%%%%%%%%%%%%%%%%%%%%
\subsubsection{\label{sec:Ports-MultipleCollectors}Multiple Collectors}
%%%%%%%%%%%%%%%%%%%%%%%%%%%%%%%%%%%%%%%%%%%%%%%%%%%%%%%%%%%%%%%%%%%%%%%%%%%
\index{port usage!multiple collectors}
\Todo


%%%%%%%%%%%%%%%%%%%%%%%%%%%%%%%%%%%%%%%%%%%%%%%%%%%%%%%%%%%%%%%%%%%%%%%%%%%
\subsubsection{\label{sec:Ports-Conflicts}Port Conflicts}
%%%%%%%%%%%%%%%%%%%%%%%%%%%%%%%%%%%%%%%%%%%%%%%%%%%%%%%%%%%%%%%%%%%%%%%%%%%
\index{port usage!conflicts}
\Todo



%%%%%%%%%%%%%%%%%%%%%%%%%%%%%%%%%%%%%%%%%%%%%%%%%%%%%%%%%%%%%%%%%%%%%%%%%%%
\subsection{\label{sec:Multiple-Interfaces}Configuring Condor for
Machines With Multiple Network Interfaces } 
%%%%%%%%%%%%%%%%%%%%%%%%%%%%%%%%%%%%%%%%%%%%%%%%%%%%%%%%%%%%%%%%%%%%%%%%%%

% CMNT I'm going to leave the original documentation here in case someone
% CMNT needs the CM_IP_ADDR stuff better documented. For now, though, I've
% CMNT removed it from the documentation because the code should deal with
% CMNT CONDOR_HOST being a straight ip_addr just fine. The new docs I've
% CMNT written don't even reference CM_IP_ADDR because it seems the code
% CMNT doesn't need it.

%Beginning with Condor version 6.1.5, Condor can run on machines with
%multiple network interfaces.
%Basically, you tell each host with multiple interfaces which IP
%address you want the host to use for ingoing and outgoing Condor
%network communication.
%You do this by setting the \Macro{NETWORK\_INTERFACE} parameter in
%the local config file for each host you need to.
%There are a few other special cases you might have to deal with,
%described below.
%
%If your Central Manager is on a machine with multiple interfaces,
%instead of defining the \Macro{COLLECTOR\_HOST} or
%\Macro{NEGOTIATOR\_HOST} parameters (which are usually both defined in
%terms of \Macro{CONDOR\_HOST}), you should set the
%\Macro{CM\_IP\_ADDR}.
%
%\Warn The default \Macro{HOSTALLOW\_ADMINISTRATOR} setting in the
%config file references \MacroU{CONDOR\_HOST}, and the default
%\Macro{HOSTALLOW\_NEGOTIATOR} setting references
%\MacroU{NEGOTIATOR\_HOST}.
%So you'll need to change both of these settings to reference
%\MacroU{CM\_IP\_ADDR} instead.   
%
%If your Checkpoint Server is on a machine with multiple interfaces,
%the only way to get things to work is if your different interfaces
%have different hostnames associated with them, and you set
%\Macro{CKPT\_SERVER\_HOST} to the hostname that corresponds with the
%IP address you want to use.  
%You will still need to specify \Macro{NETWORK\_INTERFACE} in the local
%config file for your Checkpoint Server.
%
%XXX

\index{multiple network interfaces}
\index{network interfaces!multiple}
\index{NICs}

Beginning with Condor version 6.1.5, Condor can run on machines with
multiple network interfaces.
However, starting with Condor version 6.7.13, new functionality is
available that allows for even better support for multi-homed
machines, the configuration setting \MacroNI{BIND\_ALL\_INTERFACES}.
Further improvements to this new functionality will remove the need
for any special configuration in the common case, but for now, care
must still be given to machines with multiple network interfaces, even
when using this new setting.

First, we will describe the new functionality available in version
6.7.13 and beyond. 
Then, we include some common scenarios that you might encounter and
how you go about solving them using the old functionality.
Given the current limitations in the new functionality, some of the
considerations described below for the old functionality still apply. 


%%%%%%%%%%%%%%%%%%%%%%%%%%%%%%%%%%%%%%%%%%%%%%%%%%%%%%%%%%%%
\subsubsection{\label{sec:Using-BindAllInterfaces}Using 
\Macro{BIND\_ALL\_INTERFACES}}
%%%%%%%%%%%%%%%%%%%%%%%%%%%%%%%%%%%%%%%%%%%%%%%%%%%%%%%%%%%%

Starting with version 6.7.13, Condor hosts can be configured such that
whenever they call \Syscall{bind}, they use all network interfaces on
the machine.  
This means that outbound connections will always use the appropriate
interface to connect to a remote host, instead of being forced to use
an interface that might not have a route to the given destination.
Furthermore, sockets where the daemon listens for incoming connections 
will be bound to all interfaces on the machine.
This means that so long as remote clients know the right port, they can
use any IP address on the machine and contact a given Condor daemon.

To enable this functionality, you must define the following in your
Condor configuration file:

\begin{verbatim}
BIND_ALL_INTERFACES = TRUE
\end{verbatim}

However, this functionality has certain limitations, which is why it
is not enabled by default.

\begin{description}

\item[Does not work with Kerberos.] 
  Every Kerberos ticket contains a specific IP address within it.
  Any socket that attempts to authenticate using Kerberos must be
  bound to the same IP address as the one in the ticket.
  Once you enable \MacroNI{BIND\_ALL\_INTERFACES}, outbound
  connections from a multi-homed machine could now potentially
  originate over any of the interfaces.
  Therefore, the IP address of the outbound connection and the IP
  address in the Kerberos ticket won't necessarily match, and the
  authentication will fail.
  Sites using Kerberos authentication on multi-homed machines are
  strongly encouraged to not enable \MacroNI{BIND\_ALL\_INTERFACES},
  at least until we provide a way for Condor's Kerberos functionality
  to support using multiple Kerberos tickets and finding the right one
  to match the IP address a given socket is bound to. 

\item[Opens up a potential security risk.]
  For example, imagine a multi-homed machine sitting on a network
  boundary.
  One interface is on the public Internet, while the other connects to
  a private network, where a compute farm is running.
  If the entire Condor pool is running on the private network, there's
  no reason to have Condor daemons listening on ports on the public
  Internet, and that potentially opens up the risk of hackers trying
  to compromise the security of your pool.
  If you do not need Condor daemons to be listening on the public
  interface, there's no reason to have them do it, and it creates a
  possible security risk.

\item[Only one IP address will be advertised.]
  At present, even though a given Condor daemon will be listening to
  ports on multiple interfaces, each with their own IP address,
  there's currently no mechanism for that daemon to advertise all of
  the possible IP addresses where it can be contacted.
  Therefore, Condor clients (other Condor daemons or tools) won't
  necessarily able to locate and communicate with a given server
  running on a multi-homed machine where
  \MacroNI{BIND\_ALL\_INTERFACES} has been enabled.

\end{description}

This final limitation on IP address advertising requires more
explanation.
Currently, Condor daemons can only advertise a single IP address in
the classad they send to their \Condor{collector}.
Condor tools and other daemons only know how to look up a single IP
address and attempt to use that when connecting to the daemon.
So, even if the daemon is listening on 2 or more different interfaces,
each with a separate IP, the daemon must choose what IP address to
publicly advertise so that other daemons and tools can locate it.

By default, Condor will just advertise the first IP address it finds
when looking up the host information for itself.  
However, the \Macro{NETWORK\_INTERFACE} setting can still be used to
specify what IP address Condor should advertise, even if
\MacroNI{BIND\_ALL\_INTERFACES} is set to \verb@TRUE@.
Therefore, some of the considerations described below regarding what
interface should be used in various situations still apply when
decided what interface should be advertised.

Sites that make heavy use of private networks and multi-homed machines
should consider if using Generic Connection Brokering, or GCB, is
right for them.
More information about GCB and Condor can be found in
section~\ref{sec:GCB} on page~\pageref{sec:GCB}.


%%%%%%%%%%%%%%%%%%%%%%%%%%%%%%%%%%%%%%%%%%%%%%%%%%%%%%%%%%%%
\subsubsection{Central Manager with Two or More NICs}
%%%%%%%%%%%%%%%%%%%%%%%%%%%%%%%%%%%%%%%%%%%%%%%%%%%%%%%%%%%%

Often users of Condor wish to set up ``compute farms'' where there is one
machine with two network interface cards (one for the public Internet,
and one for the private net). It is convenient to set up the ``head''
node as a central manager in most cases and so here are the instructions
required to do so.

Setting up the central manager on a machine with more than one NIC can
be a little confusing because there are a few external variables
that could make the process difficult. One of the biggest mistakes
in getting this to work is that either one of the separate interfaces is
not active, or the host/domain names associated with the interfaces are
incorrectly configured. 

Given that the interfaces are up and functioning, and they have good
host/domain names associated with them here is how to configure Condor:

In this example, \Bold{farm-server.farm.org} maps to the private interface.

On the central manager's global (to the cluster) configuration file: \\
\Macro{CONDOR\_HOST} = \Bold{farm-server.farm.org}

On your central manager's local configuration file: \\
\MacroNI{NETWORK\_INTERFACE} = ip address of \Bold{farm-server.farm.org} \\
\MacroNI{NEGOTIATOR} = \MacroUNI{SBIN}/condor\_negotiator \\
\MacroNI{COLLECTOR} = \MacroUNI{SBIN}/condor\_collector \\
\MacroNI{DAEMON\_LIST} = \MacroNI{MASTER}, \MacroNI{COLLECTOR}, \MacroNI{NEGOTIATOR}, \MacroNI{SCHEDD}, \MacroNI{STARTD}

If your central manager and farm machines are all NT, then you only have
vanilla universe and it will work now.  However, if you have this setup
for UNIX, then at this point, standard universe jobs should be able to
function in the pool, but if you did not configure the \Macro{UID\_DOMAIN}
macro to be homogeneous across the farm machines, the standard universe
jobs will run as \Bold{nobody} on the farm machines.

In order to get vanilla jobs and file server load balancing for standard
universe jobs working (under Unix), do some more work both in
the cluster you have put together and in Condor to make everything work.
First, you need a file server (which could also be the central manager) to
serve files to all of the farm machines. This could be NFS or AFS, it does
not really matter to Condor. The mount point of the directories you wish
your users to use must be the same across all of the farm machines. Now,
configure \Macro{UID\_DOMAIN} and \Macro{FILESYSTEM\_DOMAIN} to be
homogeneous across the farm machines and the central manager. Now, you
will have to inform Condor that an NFS or AFS filesystem exists and that
is done in this manner. In the global (to the farm) configuration file:

\begin{verbatim}
# If you have NFS
USE_NFS = True
# If you have AFS
HAS_AFS = True
USE_AFS = True
# if you want both NFS and AFS, then enable both sets above
\end{verbatim}

Now, if you've set up your cluster so that it is possible for a machine
name to never have a domain name (for example: there is machine
name but no fully qualified domain name in \File{/etc/hosts}), you must
configure \Macro{DEFAULT\_DOMAIN\_NAME} to be the domain that you wish
to be added on to the end of your host name.


%%%%%%%%%%%%%%%%%%%%%%%%%%%%%%%%%%%%%%%%%%%%%%%%%%%%%%%%%%%%
\subsubsection{A Client Machine with Multiple Interfaces}
%%%%%%%%%%%%%%%%%%%%%%%%%%%%%%%%%%%%%%%%%%%%%%%%%%%%%%%%%%%%

If you have a client machine with two or more NICs, then there might be
a specific network interface with which you desire a client machine to
communicate with the rest of the Condor pool. In this case, in the local
configuration file for that machine, place: \\ 
\Macro{NETWORK\_INTERFACE} = ip address of interface desired \\


%%%%%%%%%%%%%%%%%%%%%%%%%%%%%%%%%%%%%%%%%%%%%%%%%%%%%%%%%%%%
\subsubsection{A Checkpoint Server on a Machine with Multiple NICs}
%%%%%%%%%%%%%%%%%%%%%%%%%%%%%%%%%%%%%%%%%%%%%%%%%%%%%%%%%%%%

If your Checkpoint Server is on a machine with multiple interfaces,
the only way to get things to work is if your different interfaces
have different host names associated with them, and you set
\Macro{CKPT\_SERVER\_HOST} to the host name that corresponds with the
IP address you want to use in the global configuration file for your pool.
You will still need to specify \Macro{NETWORK\_INTERFACE} in the local
config file for your Checkpoint Server.



%%%%%%%%%%%%%%%%%%%%%%%%%%%%%%%%%%%%%%%%%%%%%%%%%%%%%%%%%%%%%%%%%%%%%%
\subsection{\label{sec:GCB}Generic Connection Brokering (GCB)}
%%%%%%%%%%%%%%%%%%%%%%%%%%%%%%%%%%%%%%%%%%%%%%%%%%%%%%%%%%%%%%%%%%%%%%
\index{GCB (Generic Connection Brokering)}

Generic Connection Brokering, or GCB, is a system for managing network
connections across private network and firewall boundaries.
Starting with Condor version 6.7.13,
Condor's Linux releases are linked with GCB,
and can use GCB functionality to run jobs
(either directly or via flocking)
on pools that span public and private networks.

While GCB provides numerous advantages over restricting Condor to use
a range of ports which are then opened on the firewall (see
section~\ref{sec:Ports-Firewalls} on
page~\pageref{sec:Ports-Firewalls}),
GCB is also a very complicated system, with major
implications for Condor's networking and security functionality.
Therefore, sites must carefully weigh the 
advantages and disadvantages
of attempting
to configure and use GCB before making a decision.

Advantages:
\begin{itemize}

\item Better connectivity. GCB works with pools that have multiple
  private networks (even multiple private networks that use the same
  IP addresses (for example, 192.168.2.*).
  GCB also works with sites that use network address translation
  (NAT). 

\item More secure. Administrators never need to allow inbound
  connections through the firewall.
  With GCB, only outbound connections from behind the firewall must be
  allowed (which is a standard firewall configuration).
  It is possible to trade decreased performance for better security, and
  configure the firewall to only allow outbound connections to a
  single public IP address.

\item Does not require \Login{root} access to any machines.
  All parts of a GCB system can be run as an unprivileged user, and in
  the common case, no changes to the firewall configuration are
  required.

\end{itemize}

Disadvantages:
\begin{itemize}

\item The GCB broker 
  (section~\ref{sec:GCB-Broker-Intro} describes the broker)
  node(s) is a potential failure point to the pool.
  Any private nodes that want to communicate outside their own network
  must be represented by a GCB broker.
  This machine must be highly reliable, since if the broker is ever
  down, all inbound communication with the private nodes is
  impossible.
  Furthermore, no other Condor services should be run on a GCB broker
  (for example, the Condor pool's central manager).
  While it is possible to do so, it is not recommended.
  In general, no other services should be run on the machine at all,
  and the host should be dedicated to the task of serving as a GCB
  broker.

\item All Condor nodes behind a given firewall share a single IP
  address (the public IP address of their GCB broker).
  All Condor daemons using a GCB broker will advertise themselves with
  this single IP address, and in some cases, connections to/from those
  daemons will actually originate at the broker.
  This has implications for Condor's host/IP based security,
  and the general level of confusion for users and
  administrators of the pool.
  Debugging problems will be more difficult, as any log messages which 
  only print the IP address (not the name and/or port) will become ambiguous.
  Even log or error messages that include the port will not necessarily
  be helpful, as it is difficult to correlate ports on the broker
  with the corresponding private nodes.
  
\item Can not function with Kerberos authentication.
  Kerberos tickets include the IP address of the machine where they
  were created.
  However, when Condor daemons are using GCB, they use a different IP
  address, and therefore, any attempt to authenticate
  using Kerberos will fail, as Kerberos will consider this a (poor)
  attempt to fool it into using an invalid host principle.

\item Scalability and performance degradation: 
  \begin{itemize}
  \item Connections are more expensive to establish.
  \item In some cases, connections must be forwarded through a proxy
    server on the GCB broker.
  \item Each network port on each private node must correspond to a
    unique port on the broker host, so there is a fixed limit to how
    many private nodes a given broker can service (which is a function
    of the number of ports each private node requires and the total
    number of available ports on the broker).
  \item Each private node must maintain an open TCP connection to its
    GCB broker.  GCB will attempt to recover in the case of the socket
    being closed, but this means the broker must have at least as many
    sockets open as there are private nodes.
  \end{itemize}

\item It is more complex to configure and debug.

\end{itemize}

Given the increased complexity, use of GCB requires a careful
read of this entire manual section, followed by a thorough
installation.
%% Derek's use of non-subtle advice:
%% Please do not skim the information in this section, attempt to quickly
%% install GCB, and then ask for help when things go wrong.

Details of GCB and how it works can be found at the GCB
homepage:

\URL{http://www.cs.wisc.edu/\~{}sschang/firewall/gcb}

This information is useful for understanding the technical
details of how GCB works, and the various parts of the system.
While some of the information is partly out of date (especially the
discussion of how to configure GCB) most of the sections are perfectly
accurate and worth reading.
Ignore the section on ``GCBnize'', which describes
how to get a given application to use GCB, as 
Condor does this for you.

The rest of this section gives the details for configuring a
Condor pool to use GCB.
It is divided into the following topics:

\begin{itemize}
\item Introduction to the GCB broker
\item Configuring the GCB broker
\item Spawning a GCB broker (with a \Condor{master} or using \Prog{initd})
\item How to configure Condor machines to use GCB
\item Configuring the GCB routing table
\item Implications for Condor's host/IP security settings
\item Implications for other Condor configuration settings
\end{itemize}


%%%%%%%%%%%%%%%%%%%%%%%%%%%%%%%%%%%%%%%%%%%%%%%%%%%%%%%%%%%%%%%%%%%%%%
\subsubsection{\label{sec:GCB-Broker-Intro}Introduction to the GCB Broker}
%%%%%%%%%%%%%%%%%%%%%%%%%%%%%%%%%%%%%%%%%%%%%%%%%%%%%%%%%%%%%%%%%%%%%%

\index{GCB (Generic Connection Brokering)!broker}
\index{GCB (Generic Connection Brokering)!inagent}
At the heart of GCB is a logical entity known as a \Term{broker} or
\Term{inagent}.
In reality, the entity is made up of daemon
processes running on
the same machine comprised of the \Prog{gcb\_broker} and a set of
\Prog{gcb\_relay\_server} processes, each one spawned by the
\Prog{gcb\_broker}.

Every private network using GCB
must have at least one broker to arrange connections.
%Every private network in a given network topology trying to use GCB
%must have at least one corresponding broker to arrange connections for it.
The broker must be installed on a machine that nodes in both the
public and the private (firewalled) network can directly talk to.
The broker need not be able to initiate connections to the
private nodes.  
It can take advantage of the case where it can
initiate connections to the private nodes, and that will improve
performance. 
The broker is generally installed on a 
machine with multiple network interfaces
(on the network boundary) or just outside
of a network that allows outbound connections.
If the private network contains many hosts, sites can configure
multiple GCB brokers, and partition the private nodes so that different
subsets of the nodes use different brokers.

For a more thorough explanation of what a GCB broker is, check out:
\URL{http://www.cs.wisc.edu/\~{}sschang/firewall/gcb/mechanism.htm}

A GCB broker should generally be installed on a dedicated machine.
These are machines that are not running other Condor daemons or services.
If running any other Condor service 
(for example, the central manager of the pool)
on the same machine as the GCB broker,
all other machines attempting
to use this Condor service
(for example, to connect to the \Condor{collector} or \Condor{negotiator})
will incur additional connection costs and latency.
It is possible that future versions of GCB and Condor will be able to
overcome these limitations, but for now, we recommend that a broker
is run on a dedicated machine with no other Condor daemons (except
perhaps a single \Condor{master} used to spawn the \Prog{gcb\_broker}
daemon, as described below).

In principle, a GCB broker is a network element that functions almost
like a router.
It allows certain connections through the firewall by redirecting
connections or forwarding connections.
In general, it is not a good idea to run a lot of other services on
the network elements, especially not services like Condor which can
spawn arbitrary jobs.
Furthermore, the GCB broker relies on listening to many network
ports.
If other applications are running on the same host as the broker,
problems exist
where the broker does not have enough network
ports available to forward all the connections that might be required
of it.
Also, all nodes inside a private network rely on the GCB broker for
all incoming communication.
For performance reasons, avoid forcing the GCB broker to
contend with other processes for system resources, such that it is always
available to handle communication requests.
There is nothing in GCB or Condor requiring
the broker to run on a separate machine, 
but it is the recommended configuration.

\index{GCB broker!ports 65432 and 65430}
The \Prog{gcb\_broker} daemon listens on two hard-coded,
fixed ports (65432 and 65430).
A future version of Condor and GCB will remove this limitation.
However, for now, to run a \Prog{gcb\_broker} on a
given host, ensure that ports 65432 and 65430 are not already
in use. 

If \Login{root} access on a machine where a GCB 
broker is planned, one good option is to have \Prog{initd} configured to
spawn (and respawn) the \Prog{gcb\_broker} binary (which is located in
the \Release{libexec} directory).
This way, the \Prog{gcb\_broker} will be automatically restarted on
reboots, or in the event that the broker itself crashes or is killed.
Without \Login{root} access, use a \Condor{master} to
manage the \Prog{gcb\_broker} binary. 

%%%%%%%%%%%%%%%%%%%%%%%%%%%%%%%%%%%%%%%%%%%%%%%%%%
\subsubsection{\label{sec:GCB-Broker-Config}
Configuring the GCB broker}
%%%%%%%%%%%%%%%%%%%%%%%%%%%%%%%%%%%%%%%%%%%%%%%%%%

\index{GCB broker!configuration}
Since the \Prog{gcb\_broker} and \Prog{gcb\_relay\_server} are not
Condor daemons, they do not read the Condor configuration
files.
Therefore, they must be configured by other means, namely the
environment and through the use of command-line arguments.

There is one required command-line argument for the \Prog{gcb\_broker}.
This argument defines the public IP address this broker will use to
represent itself and any private network nodes that are configured to
use this broker.
This information is defined with \Opt{-i xxx.xxx.xxx.xxx} on the
command-line when the \Prog{gcb\_broker} is executed.
If the broker is being setup outside the private network, it is likely
that the machine will only have one IP address, which is clearly the
one to use.
However, if the broker is being run on a machine on the
network boundary (a multi-homed machine with interfaces into both the
private and public networks), be sure to use the IP address of the
interface on the public network.

Additionally, specify environment variables to control
how the \Prog{gcb\_broker} (and the \Prog{gcb\_relay\_server}
processes it spawns) will behave.
Some of these settings can also be specified as command-line
arguments to the \Prog{gcb\_broker}.
All of them have reasonable defaults if not defined.

\begin{itemize}

\item General daemon behavior

\begin{description}

\item The environment variable \Env{GCB\_RELAY\_SERVER} \label{Env:GCB-relay-server}
  defines the full path to the \File{gcb\_relay\_server} binary
  the broker should use.
  The command-line override for this is \Opt{-r /full/path/to/relayserver}.
  If not set either on the command-line or in the environment,
  the \Prog{gcb\_broker} process will search for a program named
  \File{gcb\_relay\_server} in the same directory where the
  \File{gcb\_broker} binary is located, and attempt to use that one.

\item The environment variable \Env{GCB\_ACTIVE\_TO\_CLIENT}
  \label{Env:GCB-active-to-client}
  is a boolean that defines whether the GCB broker can directly talk to servers
  running inside the network that it manages
  The value must be \verb@yes@ or \verb@no@, case sensitive.
  \Env{GCB\_ACTIVE\_TO\_CLIENT} should be set to \verb@yes@ only if
  this GCB broker is running on a network boundary and can connect to
  both the private and public nodes.
  If the broker is running in the public network, it should be left
  undefined or set to \verb@no@.

\end{description}

\item Log file locations

\begin{description}

\item The environment variable \Env{GCB\_LOG\_DIR} \label{Env:GCB-log-dir}
  defines a directory to use for all GCB-related log files.
  If defined, and the per-daemon log file settings (described
  below) are not defined, the broker will write to
  \verb@$GCB_LOG_DIR/BrokerLog@ and the relay server will write to
  \verb@$GCB_LOG_DIR/RelayServerLog.<pid>@

\item The environment variable \Env{GCB\_BROKER\_LOG} \label{Env:GCB-broker-log}
  defines the full path for the GCB broker's log file.
  The command-line override is \Opt{-l /full/path/to/log/file}.
  This definition overrides \Env{GCB\_LOG\_DIR}.

\item The environment variable \Env{GCB\_RELAY\_SERVER\_LOG}
  \label{Env:GCB-relay-server-log}
  defines the full path to the GCB relay server's log file.
  Each relay server writes its own log file, so the actual filename
  will be: \verb@$GCB_RELAY_SERVER_LOG.<pid>@ where \verb@<pid>@ is
  replaced with the process id of the corresponding
  \Prog{gcb\_relay\_server}.
  When defined, this setting overrides \Env{GCB\_LOG\_DIR}.

\end{description}

\item Verbose logging 

\begin{description}

\item The environment variable \Env{GCB\_DEBUG\_LEVEL} 
  \label{Env:GCB-DEBUG-LEVEL}
  controls how verbose all the GCB daemon's log files should be.
  Can be either \verb@fulldebug@ (more verbose) or \verb@basic@.
  This defines logging behavior for all GCB daemons, unless
  the following daemon-specific settings are defined.

\item The environment variable \Env{GCB\_BROKER\_DEBUG}
  \label{Env:GCB-BROKER-DEBUG}
  controls verbose logging specifically for the GCB broker.
  The command-line override for this is \Opt{-d level}.
  Overrides \Env{GCB\_DEBUG\_LEVEL}. 

\item The environment variable \Env{GCB\_RELAY\_SERVER\_DEBUG} 
  \label{Env:GCB-RELAY-SERVER-DEBUG}
  controls verbose logging specifically for the GCB relay server.  
  Overrides \Env{GCB\_DEBUG\_LEVEL}. 

\end{description}

\item Maximum log file size

\begin{description}

\item The environment variable \Env{GCB\_MAX\_LOG} \label{Env:GCB-MAX-LOG}
  defines the maximum size in bytes of all GCB log files.
  When the log file reaches this size, the content of the file will be
  moved to \File{filename.old}, and a new log is started.
  This defines logging behavior for all GCB daemons, unless
  the following daemon-specific settings are used.

\item The environment variable \Env{GCB\_BROKER\_MAX\_LOG} \label{Env:GCB-BROKER-MAX-LOG}
  defines the maximum size in bytes of the GCB broker log file.

\item The environment variable \Env{GCB\_RELAY\_SERVER\_MAX\_LOG} 
  \label{Env:GCB-RELAY-SERVER-MAX-LOG}
  defines the maximum size in bytes of the GCB relay server log file.

\end{description}

\end{itemize}

%%%%%%%%%%%%%%%%%%%%%%%%%%%%%%%%%%%%%%%%%%%%%%%%%%
\subsubsection{\label{sec:GCB-broker-spawn}
Spawning the GCB Broker}
%%%%%%%%%%%%%%%%%%%%%%%%%%%%%%%%%%%%%%%%%%%%%%%%%%

\index{GCB broker!how to spawn the broker}
There are two ways to spawn the GCB broker:

\begin{itemize}
\item Use a \Condor{master}.

To spawn the GCB broker with a \Condor{master}, here are 
the recommended \File{condor\_config} settings that will work:

\footnotesize
\begin{verbatim}
# Specify that you only want the master and the broker running
DAEMON_LIST = MASTER, GCB_BROKER

# Define the path to the broker binary for the master to spawn
GCB_BROKER = $(RELEASE_DIR)/libexec/gcb_broker

# Define the path to the release_server binary for the broker to use 
GCB_RELAY = $(RELEASE_DIR)/libexec/gcb_relay_server

# Setup the gcb_broker's environment.  We use a macro to build up the
# environment we want in pieces, and then finally define
# GCB_BROKER_ENVIRONMENT, the setting that condor_master uses.

# Initialize an empty macro
GCB_BROKER_ENV =

# (recommended) Provide the full path to the gcb_relay_server
GCB_BROKER_ENV = $(GCB_BROKER_ENV);GCB_RELAY_SERVER=$(GCB_RELAY)

# (recommended) Tell GCB to write all log files into the Condor log
# directory (the directory used by the condor_master itself)
GCB_BROKER_ENV = $(GCB_BROKER_ENV);GCB_LOG_DIR=$(LOG)
# Or, you can specify a log file seperately for each GCB daemon:
#GCB_BROKER_ENV = $(GCB_BROKER_ENV);GCB_BROKER_LOG=$(LOG)/GCB_Broker_Log
#GCB_BROKER_ENV = $(GCB_BROKER_ENV);GCB_RELAY_SERVER_LOG=$(LOG)/GCB_RS_Log

# (optional -- only set if true) Tell the GCB broker that it can
# directly connect to machines in the private network which it is
# handling communication for.  This should only be enabled if the GCB
# broker is running directly on a network boundry and can open direct
# connections to the private nodes.
#GCB_BROKER_ENV = $(GCB_BROKER_ENV);GCB_ACTIVE_TO_CLIENT=yes

# (optional) turn on verbose logging for all of GCB
#GCB_BROKER_ENV = $(GCB_BROKER_ENV);GCB_DEBUG_LEVEL=fulldebug
# Or, you can turn this on seperately for each GCB daemon:
#GCB_BROKER_ENV = $(GCB_BROKER_ENV);GCB_BROKER_DEBUG=fulldebug
#GCB_BROKER_ENV = $(GCB_BROKER_ENV);GCB_RELAY_SERVER_DEBUG=fulldebug

# (optional) specify the maximum log file size (in bytes)
#GCB_BROKER_ENV = $(GCB_BROKER_ENV);GCB_MAX_LOG=640000
# Or, you can define this seperately for each GCB daemon:
#GCB_BROKER_ENV = $(GCB_BROKER_ENV);GCB_BROKER_MAX_LOG=640000
#GCB_BROKER_ENV = $(GCB_BROKER_ENV);GCB_RELAY_SERVER_MAX_LOG=640000

# Finally, set the value the condor_master really uses
GCB_BROKER_ENVIRONMENT = $(GCB_BROKER_ENV)

# If your Condor installation on this host already has a public
# interface as the default (either because it is the first interface
# listed in this machine's host entry, or because you've already
# defined NETWORK_INTERFACE), you can just use Condor's special macro
# that holds the IP address for this.
GCB_BROKER_IP = $(ip_address)
# Otherwise, you could define it yourself with your real public IP:
# GCB_BROKER_IP = 123.123.123.123

# (required) define the command-line arguments for the broker 
GCB_BROKER_ARGS = -i $(GCB_BROKER_IP)
\end{verbatim}
\normalsize

Once those settings are in place, either spawn or restart the
\Condor{master} and the \Prog{gcb\_broker} should be started.
Ensure the broker is running by reading the log file
specified with \Env{GCB\_BROKER\_LOG}, or in
\File{\MacroUNI{LOG}/BrokerLog} if using the default.


%%%%%%%%%%%%%%%%%%%%%%%%%%%%%%%%%%%%%%%%%%%%%%%%%%
%%\subsubsection{\label{sec:GCB-initd-spawn}
%%Using \Prog{initd} to spawn the GCB broker}
%%%%%%%%%%%%%%%%%%%%%%%%%%%%%%%%%%%%%%%%%%%%%%%%%%

\item Use \Prog{initd}.

The system's initd may be used to manage the
\Prog{gcb\_broker} without \Login{root} access
and without running the \Condor{master} on the broker node.
Generally, this involves adding a line to the \File{/etc/inittab}
file.
Some sites use other means to manage and generate the
\File{/etc/inittab}, such as \Prog{cfengine} or other system configuration
management tools, so check with the local system administrator
to be sure.
An example line might be something like:

\footnotesize
\begin{verbatim}
GB:23:respawn:/path/to/gcb_broker -i 123.123.123.123 -r /path/to/relay_server
\end{verbatim}
\normalsize

It may be easier to wrap the \Prog{gcb\_broker} binary
in a shell script, in order to change the command-line arguments (and
set environment variables) without having to edit \File{/etc/inittab}
all the time.
This will be similar to:

\footnotesize
\begin{verbatim}
GB:23:respawn:/opt/condor-6.7.13/libexec/gcb_broker.sh
\end{verbatim}
\normalsize

Then, create the wrapper, as similar to: 

\footnotesize
\begin{verbatim}
#!/bin/sh

libexec=/opt/condor-6.7.13/libexec
ip=123.123.123.123
relay=$libexec/gcb_relay_server

exec $libexec/gcb_broker -i $ip -r $relay
\end{verbatim}
\normalsize

You will probably also want to set some environment variables to tell
the GCB daemons where to write their log files (\Env{GCB\_LOG\_DIR}),
and possibly some of the other variables described above.

Either way, after updating the \File{/etc/inittab}, send
the \Prog{initd} process (always PID 1) a \verb@SIGHUP@ signal, and it
will re-read the \File{inittab} and spawn the \Prog{gcb\_broker}.

\end{itemize}

%%%%%%%%%%%%%%%%%%%%%%%%%%%%%%%%%%%%%%%%%%%%%%%%%%%%%%%%%%%%%%%%%%%%%%
\subsubsection{\label{sec:GCB-condor-config}
Configuring Condor nodes to be GCB clients}
%%%%%%%%%%%%%%%%%%%%%%%%%%%%%%%%%%%%%%%%%%%%%%%%%%%%%%%%%%%%%%%%%%%%%%

% EDITTED to this point.
\index{GCB (Generic Connection Brokering)!Condor client configuration}
In general, before configuring any Condor node to use GCB, the GCB
broker node(s) for your pool must be setup and running.
So, if you haven't already, go back and read the sections describing
the broker, how to configure it, and how to spawn it.

To enable GCB on a given Condor host, you must set the following
Condor configuration attributes:

\footnotesize
\begin{verbatim}
# Tell Condor to use a network remapping service (currently only GCB
# is supported, but in the future, there might be other options)
NET_REMAP_ENABLE = true
NET_REMAP_SERVICE = GCB
\end{verbatim}
\normalsize

Only GCB clients inside a private network need to define the following
setting, which specifies the IP address of the broker serving this
network.
Note that this IP must be the same as the IP that was specified on the
broker's command-line with the \Opt{-i} option.

\footnotesize
\begin{verbatim}
# Public IP address (in standard dot notation) of the GCB broker
# serving this private node.
NET_REMAP_INAGENT = xxx.xxx.xxx.xxx
\end{verbatim}
\normalsize

Obviously, because the \MacroNI{NET\_REMAP\_INAGENT} setting is only
valid on private nodes, it should not be defined in a global
\File{condor\_config} file.
Furthermore, if you have a large number of hosts in a given private
network, and you choose to run multiple brokers to alleviate the
scalability issues, each subset of your private nodes that uses a
specific broker will need a different value for this setting.

Finally, if you choose to setup a GCB routing table (which is
recommended but optional, and described below in a separate section),
you must tell Condor daemons where to find their table.
You do so by defining the following setting:

\footnotesize
\begin{verbatim}
# The full path to the routing table used by GCB
NET_REMAP_ROUTE = /full/path/to/GCB-routing-table
\end{verbatim}
\normalsize

Once the \MacroNI{NET\_REMAP\_ENABLE} setting (described above) is
defined, the \Macro{BIND\_ALL\_INTERFACES} setting is set
automatically.
More information about this setting can be found in
section~\ref{sec:Using-BindAllInterfaces} on
page~\pageref{sec:Using-BindAllInterfaces}.
It would not hurt to put the following in your config file near the
other GCB-related settings, just so you remember this is happening:

\footnotesize
\begin{verbatim}
# Tell Condor to bind to all network interfaces, instead of a single
# interface.
BIND_ALL_INTERFACES = true
\end{verbatim}
\normalsize

Once a GCB broker is setup and running to manage connections for a
each private network, and the Condor installation for all the nodes in
either private and public networks are configured to enable GCB, you
can restart your Condor daemons and all of the different machines
should be able to communicate with each other.


%%%%%%%%%%%%%%%%%%%%%%%%%%%%%%%%%%%%%%%%%%%%%%%%%%%%%%%%%%%%%%%%%%%%%%
\subsubsection{\label{sec:GCB-routing-table}Configuring the GCB
  routing table} 
%%%%%%%%%%%%%%%%%%%%%%%%%%%%%%%%%%%%%%%%%%%%%%%%%%%%%%%%%%%%%%%%%%%%%%

\index{GCB (Generic Connection Brokering)!GCB routing table configuration}
By default, a GCB-enabled application will always attempt to directly
connect to a given IP/port pair.
In the case of private nodes being represented by a GCB broker, the
IP/port will be a proxy socket on the broker node, not the real
address of the private node.
When the GCB broker receives a direct connection to one of its proxy 
sockets, it will notify the corresponding private node, which will
establish a new connection to the broker.
The broker will then forward packets between these two sockets,
establishing a communication pathway into the private node.
This allows even clients which are not linked with GCB to communicate
through GCB to private nodes.

However, this mechanism is expensive and unnecessary in the case of
GCB-aware clients trying to connect to private nodes that can directly
communicate with the public host.
The alternative is to contact the GCB broker's command interface (the
fixed port where it is listening for GCB management commands), and use
a GCB-specific protocol to request a connection to the given IP/port.
In this case, the GCB broker will notify the private node to directly
connect to the public client (technically, to a new socket created by
the GCB client library linked in with the client's application), and a
direct socket between the two is established, removing the need for
packet forwarding between the proxy sockets at the GCB broker.

On the other hand, in cases where a direct connection from the client
to a given server is possible (e.g. two GCB-aware clients in the same
public network attempted to communicate with each other), it is
expensive and unnecessary to attempt to contact a GCB broker, and the
client should connect directly.

To allow a GCB-enabled client to know if it should make a direct
connection (which might involve packet forwarding through proxy
sockets), or if it should use the GCB protocol to communicate with the
broker's command port and arrange a direct socket,
GCB provides a \Term{routing table}.
Using this table, an administrator can define what IP addresses should
be considered private nodes where the GCB connection protocol will be
used, and what nodes are public, where a direct connection (without
incurring the latency of contacting the GCB broker, only to find out
there's no information about the given IP/port) should be made
immediately. 

If the attempt to contact the GCB broker for a given IP/port fails, or
if the desired port is not being managed by the broker, the GCB client
library making the connection will fallback and attempt a direct
connection.
Therefore, configuring a GCB routing table is not required for
communication to work within a GCB-enabled environment.
However, the GCB routing table can significantly improve performance
for communication with private nodes being represented by a GCB
broker. 

One confusing aspect of GCB is that all of the nodes on a private
network will think their IP address is the address of their GCB
broker.
Therefore, all the Condor daemons on a private network will advertise
themselves with the same IP address (though the broker will map the
different ports to different nodes within the private network).
Therefore, a given node in the public network needs to be told that if
it is contacting this IP address, it should know that the IP is really
a GCB broker representing a node in the private network, so that it
can contact the broker to arrange a single socket from the private
node to the public one, instead of relying on forwarding packets
between proxy sockets at the broker.
However, any other addresses (for example, other public IPs) can be
contacted directly, without going through a GCB broker.
Similarly, other nodes within the same private network will still be
advertising their address with their GCB broker's public IP.
So, nodes within the same private network also have to know that the
public IP of the broker is really a GCB broker, yet all other public
IPs are valid for direct communication.

In general, all connections can be made directly, except to a host
represented by a GCB broker.
Furthermore, the default behavior of the GCB client library is to make
a direct connection.
So, the routing table is just a (somewhat complicated) way to tell a
given GCB installation what GCB brokers it might have to communicate
with, and that it should directly communicate with anything else.
In practice, the routing table should just have a single entry for
each GCB broker in your system.
Future versions of GCB will be able to make use of more complicated
routing behavior, which is why the full routing table infrastructure
described below is implemented, even if the current version of GCB is
not taking advantage of all of it.


%%%%%%%%%%%%%%%%%%%%%%%%%%%%%%%%%%%%%%%%%%%%%%%%%%%%%%%%%%%%%%%%%%%%%%
\Bold{Format of the GCB routing table}
%%%%%%%%%%%%%%%%%%%%%%%%%%%%%%%%%%%%%%%%%%%%%%%%%%%%%%%%%%%%%%%%%%%%%%

The routing table should be stored in a plain ASCII text file.
Each line of the file contains one rule.
Each rule consists of a \Term{target} and a \Term{method}.
The target specifies destination IP address(es) to match and method
defines what mechanism must be used to connect to the given target.
The target must be specified as a valid IP string in the standard
dotted notation, ``/'', and an integer \Term{mask}.
The mask specifies how many bits of a given destination IP and the IP
of the target must match.
Method must be either \verb@GCB@ or \verb@direct@.
GCB stops searching the table as soon as it finds a matching rule,
therefore you must put specific rules before generic ones.
The default if no rule is matched is to use direct communication.
Some examples and the corresponding routing tables you would use will
hopefully make this syntax more clear.


%%%%%%%%%%%%%%%%%%%%%%%%%%%%%%%%%%%%%%%%%%%%%%%%%%%%%%%%%%%%%%%%%%%%%%
\Bold{Simple GCB routing table example (1 private, 1 public)}
%%%%%%%%%%%%%%%%%%%%%%%%%%%%%%%%%%%%%%%%%%%%%%%%%%%%%%%%%%%%%%%%%%%%%%

Say you have a private network with a bunch of nodes whose IP
addresses are \verb@192.168.2.*@, some nodes in a public network 
whose IP addresses are \verb@123.123.123.*@, and a broker for the 192
network running on \verb@123.123.123.123@.
In that case, the routing table for both the public and private nodes
should be:

\begin{verbatim}
123.123.123.123/32 GCB
\end{verbatim}

The rule says that for IP addresses where all 32 bits exactly match
the address \verb@123.123.123.123@ first communicate with GCB broker.

Since the default is to directly connect if no rule in the routing
table matches a given target IP, this single rule is all that is
required.
However, to illustrate how the routing table syntax works, the
following routing table is equivalent:

\begin{verbatim}
123.123.123.123/32 GCB
*/0 direct
\end{verbatim}

Any attempt to connect to \verb@123.123.123.123@ will still use GCB
(since that's the first rule in the file), but all other IP addresses
will connect directly.
This table explicitly defines GCB's default behavior.

%%%%%%%%%%%%%%%%%%%%%%%%%%%%%%%%%%%%%%%%%%%%%%%%%%%%%%%%%%%%%%%%%%%%%%
\Bold{More complex GCB routing table example (2 private, 1 public)}
%%%%%%%%%%%%%%%%%%%%%%%%%%%%%%%%%%%%%%%%%%%%%%%%%%%%%%%%%%%%%%%%%%%%%%

Here's a slightly more complicated case: a single Condor pool that
spans a public network and 2 private networks.
Say you have two separate private networks, each where the machines
have private addresses like \verb@192.168.2.*@.
Let's call one of these private networks \verb@A@ and the other one
\verb@B@. 
You've still got a public network and nodes with IP addresses like
\verb@123.123.123.*@.
The GCB broker for nodes in the \verb@A@ network will be at
\verb@123.123.123.65@, and the broker for the \verb@B@ nodes will be
at \verb@123.123.123.66@.
All of the nodes want to be able to talk to each other.
In this case, nodes in the \verb@A@ private network advertise
themselves as \verb@123.123.123.65@, so any node, regardless of being
in A, B, or the public network, must treat that IP as a GCB broker.
Similarly, nodes in the \verb@B@ network advertise themselves as
\verb@123.123.123.66@, so any node, in any network, needs to treat
that as a GCB broker (which it is).
All other connections from any node can be made directly.
Therefore, here is the appropriate routing table for all nodes:

\begin{verbatim}
123.123.123.65/32 GCB
123.123.123.66/32 GCB
\end{verbatim}


%%%%%%%%%%%%%%%%%%%%%%%%%%%%%%%%%%%%%%%%%%%%%%%%%%%%%%%%%%%%%%%%%%%%%%
\subsubsection{\label{sec:GCB-host-security-implications}Implications
of Using GCB for Condor's Host/IP Based Security} 
%%%%%%%%%%%%%%%%%%%%%%%%%%%%%%%%%%%%%%%%%%%%%%%%%%%%%%%%%%%%%%%%%%%%%%

\index{GCB (Generic Connection Brokering)!security implications}
Whenever a connection comes into a Condor daemon's command socket,
Condor will either allow or refuse the command depending on the IP
address of the incoming socket.
For more information about this host-based security in Condor, see
section~\ref{sec:Host-Security} on page~\pageref{sec:Host-Security}.
Because of the way GCB changes the IP addresses that are used and
advertised by GCB-enabled clients, and since all nodes being
represented by a GCB broker are represented by different ports on the
broker node (a process known as \Term{address leasing}), using GCB has
implications for this authorization process.

Depending on the communication pathway used by a GCB-enabled Condor
client (either a tool or another Condor daemon) to connect to a given
Condor server daemon, and where in the network each side of the
connection resides, the IP address of the resulting socket actually
used will be very different.

In the case of a private client (i.e. a client behind a firewall,
which may or may not be using NAT and a fully private, non-routable IP
address) attempting to connect to a server, there are three
possibilities: 

\begin{itemize}

  \item Direct connection to another node within the private network:
  the server will see the private IP address of the client.

  \item Direct outbound connection to a public node: if NAT is being
  used, the server will see the IP of the NAT server for the private
  network.
  If there's no NAT, and the firewall is just blocking connections in
  one direction but not re-writing IP addresses, the server will see
  the client's real IP address.

  \item Connection to a host in a different private network that must
  be relayed through the GCB broker: the server will see the IP
  address of the broker representing the server.
  This is just another instance of the private server case which is
  described below in more detail.

\end{itemize}

Therefore, any public server that wants to allow a command from a
given client must have any or all of the various IP addresses
mentioned above in the appropriate \MacroNI{HOSTALLOW} settings.  In
practice, that means opening up the \MacroNI{HOSTALLOW} settings to
include not just the actual IP addresses of each node, but also the IP
address of the various GCB brokers in use, and potentially, the public
IP address of the NAT host for each private network.

However, given that all private nodes which are represented by a
given GCB broker could potentially make connections to any other
host using the GCB broker's IP address (whenever proxy socket
forwarding is being used), if a single private node is being granted
a certain level of permission in your pool, all of the private nodes
using the same broker will have the same level of permission.
This is particularly important if you attempt to grant
\Macro{HOSTALLOW\_ADMINISTRATOR} or \Macro{HOSTALLOW\_CONFIG}
privileges to a private node represented by a GCB broker.

In the case of a public client attempting to connect to a private
server, there are only two possible cases:

\begin{itemize}

  \item GCB broker can arrange a direct socket from the private node:
  the private server will see the real public IP of the client.

  \item GCB broker must forward packets from a proxy socket (because
  of a non-GCB aware public client, a misconfigured or missing GCB
  routing table, or a client in a different private network): the
  private server will see the IP address of its own GCB broker.
  In the case where the GCB broker is running directly on the network
  boundry, the private server will see the broker's private IP
  address (even if the broker is also listening on the public
  interface and the leased addresses it provides use the public IP). 
  If the broker is running entirely in the public network and cannot
  directly connect to the private nodes, the private server will see
  the remote connection as coming from the broker's public IP
  address.

\end{itemize}

This second case is particularly troubling.
Since there are legitimate circumstances where a private server would
need to use a forwarded proxy socket from its GCB broker, in general,
the server should allow requests originating from its GCB broker.
But, precisely because of the proxy forwarding, that means any client
on Earth that can connect to the GCB broker would be allowed into the
private server if IP-based authorization was the only defense.

The final host-based security setting that requires special mention is
\Macro{HOSTALLOW\_NEGOTIATOR}.
If the \Condor{negotiator} for your pool is running on a private node
being represented by a GCB broker, you must make a couple of
modifications to the default value.
For the purposes of Condor's host-based security, the
\Condor{negotiator} acts as a client when communicating with each 
\Condor{schedd} in your pool which has idle jobs that need to be
matched with available resources.
Therefore, all the possible cases of a private client attempting to
connect to a given server apply to a private \Condor{negotiator}.
In practice, that means adding the public IP of the broker, the real
private IP of the negotiator host, and possibly the the public IP of
the NAT host for this private network, to your
\MacroNI{HOSTALLOW\_NEGOTIATOR} setting.
Unfortunately, that means *any* host behind the same NAT host or using
the same GCB broker will be authorized as if it was your
\Condor{negotiator}. 

Future versions of GCB and Condor will hopefully add some form of
authentication and authorization to the GCB broker itself, to help
alleviate these problems.
Until then, sites using GCB are encouraged to use GSI strong
authentication (since Kerberos also depends on IP addresses and is
therefore incompatible with GCB) to rely on an authorization system
that isn't effected by address leasing.
This is especially true for sites that (foolishly) choose to run their
central manager on a private node.


%%%%%%%%%%%%%%%%%%%%%%%%%%%%%%%%%%%%%%%%%%%%%%%%%%%%%%%%%%%%%%%%%%%%%%
\subsubsection{\label{sec:GCB-config-implications}Implications of
Using GCB for Other Condor Configuration Settings} 
%%%%%%%%%%%%%%%%%%%%%%%%%%%%%%%%%%%%%%%%%%%%%%%%%%%%%%%%%%%%%%%%%%%%%%

Using GCB and address leasing has implications for a few other Condor
configuration settings.
Each one is described below. 

\begin{description}

% i'm intentionally using \Macro, not \MacroNI for each of these,
% since they should show up in the index...

\item[\Macro{COLLECTOR\_HOST}]
  If the \Condor{collector} for your pool is running on a private node
  being represented by a GCB broker, the \MacroNI{COLLECTOR\_HOST}
  must be set to the host name or IP address of the GCB broker machine,
  NOT the real host name/IP of the private node where the daemons are
  actually running.
  When the \Condor{collector} on the private node attempts to
  \Syscall{bind} to its command port (9618 by default), it will
  request port 9618 on the GCB broker node, instead.
  So, you do not need to worry about the port, but you definitely have
  to worry about the host name or IP address.
  When public nodes want to communicate with the \Condor{collector},
  they must go through the GCB broker, anyway.
  In theory, other nodes inside the same private network could be told
  to directly use the private IP of the collector host, but that's
  unnecessary and would probably lead to other confusion and
  configuration problems.

  However, because the \Condor{collector} is listening on a fixed
  port, and that single port is reserved on the GCB broker node, no
  two private nodes using the same broker can attempt to use the same
  port for their \Condor{collector}.
  Therefore, any site that is attempting to setup multiple pools
  within the same private network is strongly encouraged to setup
  separate GCB brokers for each pool.
  Otherwise, they must configure one or both of the pools to use a
  non-standard port for their \Condor{collector} which adds yet more
  complication to an already complicated situation. 

\item[\Macro{CKPT\_SERVER\_HOST}]
  Much like the case for \MacroNI{COLLECTOR\_HOST} described above,
  a checkpoint server on a private node will have to lease a port on
  the GCB broker node.
  However, the checkpoint server also uses a fixed port, and unlike
  the \Condor{collector}, there's no way to configure an alternate
  value.
  Therefore, only a single checkpoint server can be run behind a given
  GCB broker.
  Once again, if multiple checkpoint servers are required, multiple
  GCB brokers should be deployed and configured.
  Furthermore, the host name of the GCB broker should be used as the
  value for \MacroNI{CKPT\_SERVER\_HOST}, not the real IP or host name
  of the private node where the \Condor{ckpt\_server} is running.

\item[\Macro{SEC\_DEFAULT\_AUTHENTICATION\_METHODS}]
  Once you enable GCB for your pool, you can no longer use
  \verb@KERBEROS@ as one of your authentication methods.
  As previously discussed, the IP addresses used in various
  circumstances will not be the real IP addresses of your machines.
  However, Kerberos stores the IP address of each host as part of the
  Kerberos ticket, and if a connection attempts to authenticate with
  Kerberos and the two IP addresses (the underlying socket and the one
  listed in the ticket) don't match, Kerberos will refuse to
  authenticate.

\end{description}

Due to the complications and security limitations that arise from
running a central manager on a private node represented by GCB (both
regarding the \MacroNI{COLLECTOR\_HOST} just described, and
\MacroNI{HOSTALLOW\_NEGOTIATOR} from the previous section), we
recommend that sites avoid locating a central manager on a private
host if at all possible.



%%%%%%%%%%%%%%%%%%%%%%%%%%%%%%%%%%%%%%%%%%%%%%%%%%%%%%%%%%%%%%%%%%%%%%%%%%%
\subsection{\label{sec:tcp-collector-update}Using TCP to Send Updates to
the \Condor{collector}}
%%%%%%%%%%%%%%%%%%%%%%%%%%%%%%%%%%%%%%%%%%%%%%%%%%%%%%%%%%%%%%%%%%%%%%%%%%

\index{TCP}
\index{TCP!sending updates}
\index{UDP}
\index{UDP!lost datagrams}
\index{condor\_collector}

TCP sockets are reliable, connection-based sockets that guarantee
the delivery of any data sent.
However, TCP sockets are fairly expensive to establish, and there is more
network overhead involved in sending and receiving messages.

UDP sockets are datagrams, and are not reliable.
There is very little overhead in establishing or using a UDP socket,
but there is also no guarantee that the data will be delivered.
All previous Condor versions used UDP sockets to send updates to
the \Condor{collector}, and this did not cause problems.

Beginning with version 6.5.0, Condor can be configured to use TCP
sockets to send updates to the \Condor{collector} instead of
UDP datagrams.
It is \emph{not} intended for most sites.
This feature is targeted at sites where UDP updates are
lost because of the underlying network.
Most Condor administrators that believe this is a good idea for
their site are wrong.
Do not enable this feature just because it sounds like a good idea.
The only cases where an administrator would want this feature are if
the ClassAd updates are consistently not getting to the
\Condor{collector}.
An example where this may happen is if the pool is comprised of
machines across a wide area network (WAN) where UDP packets are
frequently dropped.

Configuration variables are set to enable the use of TCP sockets.
There are two variables that an
administrator must define to enable this feature:

\begin{description}

\item[\Macro{UPDATE\_COLLECTOR\_WITH\_TCP}]
  When set to \Expr{True}, the Condor daemons to use TCP to
  update the \Condor{collector}, instead of the default UDP.
  Defaults to \Expr{False}.

\item[\Macro{COLLECTOR\_SOCKET\_CACHE\_SIZE}] 
  Specifies the number of TCP sockets cached at the \Condor{collector}.
  The default value for this setting is 0, with no cache enabled.

\end{description}

The use of a cache allows Condor to leave established TCP sockets open,
facilitating much better performance.
Subsequent updates can reuse an already open socket.
The work to establish a TCP connection may be lengthy,
including authentication and setting up encryption.
Therefore, Condor requires that
a socket cache be defined if TCP updates are to be used.
TCP updates will be refused by the \Condor{collector} daemon
if a cache is not enabled.

Each Condor daemon will have 1 socket open to the \Condor{collector}.
So, in a pool with N machines, each of them running a \Condor{master},
\Condor{schedd}, and \Condor{startd}, the \Condor{collector} would
need a socket cache that has at least 3*N entries.
Machines running Personal Condor in the pool need
an additional two entries (for the \Condor{master} and
\Condor{schedd}) for each Personal Condor installation.

Every cache entry utilizes a file descriptor within the
\Condor{collector} daemon.
Therefore, be careful not to define a cache that
is larger than the number of file descriptors the underlying operating
system allocates for a single process.

\Note At this time, \MacroNI{UPDATE\_COLLECTOR\_WITH\_TCP}, only
affects the main \Condor{collector} for the site, not any sites that
a \Condor{schedd} might flock to.




% NAT -- Network address translation
% when access to internet uses a single IP addr/port,
%  but there are multiple computers communicating by this single
%  place
% The NAT is an extra layer that translates the multiple addr/port
%  to the single, and visa versa (from the single to one of the
%  multiple).

% Lore from Derek:
% however, "nat" is also one of those strange condor-team terms (like
% "frank") that has it's own, special meaning. :)
%
% a "nat" is a unit for measuring productivity (or lack thereof) in
% condor work over time.  it can be normalized to any time unit you
% want, by dividing the amount of work accomplished into the time scale
% you want.  1 nat is *very* little work over a given time period,
% almost too small to measure unless you use a long time scale.  at the
% time it first came up, we decided that jim basney *sleeps* at about 40
% nats, just for comparison. :)

