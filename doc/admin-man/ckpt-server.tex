%%%%%%%%%%%%%%%%%%%%%%%%%%%%%%%%%%%%%%%%%%%%%%%%%%%%%%%%%%%%%%%%%%%%%%
\subsection{\label{sec:Ckpt-Server}
Installing a Checkpoint Server}
%%%%%%%%%%%%%%%%%%%%%%%%%%%%%%%%%%%%%%%%%%%%%%%%%%%%%%%%%%%%%%%%%%%%%%

The Checkpoint Server is a daemon that can be installed on a server to
handle all of the checkpoints that a Condor pool will create.
This machine should have a large amount of disk space available, and
should have a fast connection to your machines.

\Note It is a good idea to pick a very stable machine for your checkpoint
server.
If the checkpoint server crashes, the Condor system will continue to
operate, though poorly.  
While the Condor system will recover from a checkpoint server crash
as best it can, there are two problems that can (and will) occur:
\begin{enumerate}

\item If the checkpoint server is not functioning, when jobs need to
checkpoint, they cannot do so.
The jobs will keep trying to contact the checkpoint server, backing
off exponentially in the time they wait between attempts.
Normally, jobs only have a limited time to checkpoint before they are
kicked off the machine.
So, if the server is down for a long period of time, chances are that
you'll loose a lot of work by jobs being killed without writing a
checkpoint. 

\item When the jobs wish to start, if they cannot be retrieved from
the checkpoint server, they will either have to be restarted from
scratch, or the job will sit there, waiting for the server to come
back on-line.
You can control this behavior with the
\Macro{MAX\_DISCARDED\_RUN\_TIME} parameter in your config file (see
section~\ref{param:MaxDiscardedRunTime} on
page~\pageref{param:MaxDiscardedRunTime} for details).
Basically, this represents the maximum amount of CPU time you're
willing to discard by starting a job over from scratch if the
checkpoint server isn't responding to requests.

\end{enumerate}

%%%%%%%%%%%%%%%%%%%%%%%%%%%%%%%%%%%%%%%%%%%%%%%%%%%%%%%%%%%%%%%%%%%%%%
\subsubsection{\label{sec:Prepare-Ckpt-Server}
Preparing to Install a Checkpoint Server} 
%%%%%%%%%%%%%%%%%%%%%%%%%%%%%%%%%%%%%%%%%%%%%%%%%%%%%%%%%%%%%%%%%%%%%%

Because of the problems that exist if your pool is configured to use a
checkpoint server and that server is down, it is advisable to shut
your pool down before doing any maintenance on your checkpoint
server.  
See section~\ref{sec:Pool-Management} on
page~\pageref{sec:Pool-Management} for details on how to do that. 

If you are installing a checkpoint server for the first time,
you must make sure there are no jobs in your pool before you start.
If you have jobs in your queues, with checkpoint files on the local
spool directories of your submit machines, those jobs will never run
if your submit machines are configured to use a checkpoint server and
the checkpoint files cannot be found on the server.  
You can either remove jobs from your queues, or let them complete
before you begin the installation of the checkpoint server.

%%%%%%%%%%%%%%%%%%%%%%%%%%%%%%%%%%%%%%%%%%%%%%%%%%%%%%%%%%%%%%%%%%%%%%
\subsubsection{\label{sec:Install-Ckpt-Server-Module}
Installing the Checkpoint Server Module} 
%%%%%%%%%%%%%%%%%%%%%%%%%%%%%%%%%%%%%%%%%%%%%%%%%%%%%%%%%%%%%%%%%%%%%%

To install a checkpoint server, download the appropriate binary
contrib module for the platform your server will run on.
When you uncompress and untar that file, you'll have a directory that
contains a \File{README}, \File{ckpt\_server.tar}, and so on.
The \File{ckpt\_server.tar} acts much like the \File{release.tar} file
from a main release.
This archive contains these files:
\begin{verbatim}
        sbin/condor_ckpt_server
        sbin/condor_cleanckpts
        etc/examples/condor_config.local.ckpt.server
\end{verbatim}
These are all new files, not found in the main release, so you can
safely untar the archive directly into your existing release
directory. 
\File{\condor{ckpt\_server}} is the checkpoint server binary.
\File{\condor{cleanckpts}} is a script that can be periodically run to
remove stale checkpoint files from your server.  
Normally, the checkpoint server cleans all old files by itself.  
However, in certain error situations, stale files can be left that are
no longer needed. 
So, you may want to put a cron job in place that calls
\Condor{cleanckpts} every week or so, just to be safe.
The example config file is described below.

Once you have unpacked the contrib module, you have a few more steps
you must complete.
Each will be discussed in their own section:
\begin{enumerate}
\item Configure the checkpoint server.
\item Spawn the checkpoint server.
\item Configure your pool to use the checkpoint server.
\end{enumerate}

%%%%%%%%%%%%%%%%%%%%%%%%%%%%%%%%%%%%%%%%%%%%%%%%%%%%%%%%%%%%%%%%%%%%%%
\subsubsection{\label{sec:Configure-Ckpt-Server}
Configuring a Checkpoint Server} 
%%%%%%%%%%%%%%%%%%%%%%%%%%%%%%%%%%%%%%%%%%%%%%%%%%%%%%%%%%%%%%%%%%%%%%

There are a few settings you must place in the local config file of
your checkpoint server.  
The file \File{etc/examples/condor\_config.local.ckpt.server} contains
all such settings, and you can just insert it into the local
configuration file of your checkpoint server machine. 

There is one setting that you must customize, and that is
\Macro{CKPT\_SERVER\_DIR}.  
\label{param:CkptServerDir}
The \Macro{CKPT\_SERVER\_DIR} defines where your checkpoint files
should be located. 
It is better if this is on a very fast local file system (preferably a
RAID). 
The speed of this file system will have a direct impact on the speed
at which your checkpoint files can be retrieved from the remote
machines. 

The other optional settings are:
\begin{description}

\item[\Macro{DAEMON\_LIST}] (Described in
section~\ref{sec:Master-Config-File-Entries}).  
If you want the checkpoint server managed by the \Condor{master}, the
\Macro{DAEMON\_LIST} entry must have MASTER and CKPT\_SERVER.
Add STARTD if you want to allow jobs to run on your checkpoint server.
Similarly, add SCHEDD if you would like to submit jobs from your
checkpoint server. 

\end{description}

The rest of these settings are the checkpoint-server specific versions
of the Condor logging entries, described in
section~\ref{sec:Daemon-Logging-Config-File-Entries} on
page~\pageref{sec:Daemon-Logging-Config-File-Entries}.
\begin{description}

\item[\Macro{CKPT\_SERVER\_LOG}] The CKPT\_SERVER\_LOG is where the
checkpoint server log gets put.

\item[\Macro{MAX\_CKPT\_SERVER\_LOG}] Use this item to configure the
size of the checkpoint server log before it is rotated.

\item[\Macro{CKPT\_SERVER\_DEBUG}] The amount of information you would
like printed in your logfile.
Currently, the only debug level supported is \Dflag{ALWAYS}.

\end{description}

%%%%%%%%%%%%%%%%%%%%%%%%%%%%%%%%%%%%%%%%%%%%%%%%%%%%%%%%%%%%%%%%%%%%%%
\subsubsection{\label{sec:Spawn-Ckpt-Server} 
Spawning a Checkpoint Server} 
%%%%%%%%%%%%%%%%%%%%%%%%%%%%%%%%%%%%%%%%%%%%%%%%%%%%%%%%%%%%%%%%%%%%%%

To spawn a checkpoint server once it is configured to run on a given
machine, all you have to do is restart Condor on that host to enable
the \Condor{master} to notice the new configuration.
You can do this by sending a \Condor{restart} command from any machine
with ``administrator'' access to your pool.
See section~\ref{sec:Host-Security} on
page~\pageref{sec:Host-Security} for full details about IP/host-based
security in Condor.

%%%%%%%%%%%%%%%%%%%%%%%%%%%%%%%%%%%%%%%%%%%%%%%%%%%%%%%%%%%%%%%%%%%%%%
\subsubsection{\label{sec:Configure-Pool-Ckpt-Server} 
Configuring your Pool to Use the Checkpoint Server}
%%%%%%%%%%%%%%%%%%%%%%%%%%%%%%%%%%%%%%%%%%%%%%%%%%%%%%%%%%%%%%%%%%%%%%

Once the checkpoint server is installed and running, you just have to
change a few settings in your global config file to let your pool know
about your new server:
\begin{description}

\item[\Macro{USE\_CKPT\_SERVER}] This parameter should be set to
``True''.

\item[\Macro{CKPT\_SERVER\_HOST}] This parameter should be set to
the full hostname of the machine that is now running your checkpoint
server.  

\end{description}

Once these settings are in place, you simply have to send a
\Condor{reconfig} to all machines in your pool so the changes take
effect.
This is described in section~\ref{sec:Reconfigure-Pool} on
page~\pageref{sec:Reconfigure-Pool}.

