%%%%%%%%%%%%%%%%%%%%%%%%%%%%%%%%%%%%%%%%%%%%%%%%%%%%%%%%%%%%%%%%%%%%%%
\section{\label{sec:Pool-Management}Pool Management}
%%%%%%%%%%%%%%%%%%%%%%%%%%%%%%%%%%%%%%%%%%%%%%%%%%%%%%%%%%%%%%%%%%%%%%

Condor provides administrative tools to help with
pool management.
The following sections describe some of these various tasks.

All of the commands described in this section must be run from an
authorized machine. 
An authorized machine is one that is listed in the 
\Macro{HOSTALLOW\_ADMINISTRATOR} configuration variable,
such that IP/host-based security allows the
administrator commands to be serviced.
See section~\ref{sec:Host-Security} on
page~\pageref{sec:Host-Security} for full details about IP/host-based
security in Condor.

%%%%%%%%%%%%%%%%%%%%%%%%%%%%%%%%%%%%%%%%%%%%%%%%%%%%%%%%%%%%%%%%%%%%%%
\subsection{\label{sec:Pool-Shutdown-and-Restart}
Shutting Down and Restarting a Condor Pool}
%%%%%%%%%%%%%%%%%%%%%%%%%%%%%%%%%%%%%%%%%%%%%%%%%%%%%%%%%%%%%%%%%%%%%%

The installation of new binaries is a situation in which the
shutdown and restart of an entire Condor pool is appropriate.
It is generally
best to make sure no jobs are running, shut down Condor, and then
install the new daemons.

%%%%%%%%%%%%%%%%%%%%%%%%%%%%%%%%%%%%%%%%%%%%%%%%%%%%%%%%%%%%%%%%%%%%%%
\subsubsection{\label{sec:Pool-Shutdown}Shutting Down a Condor Pool}
%%%%%%%%%%%%%%%%%%%%%%%%%%%%%%%%%%%%%%%%%%%%%%%%%%%%%%%%%%%%%%%%%%%%%%

The best way to shut down a pool is to take advantage of the remote
administration capabilities of the \Condor{master}.
The first step is to save the IP address and port of the
\Condor{master} daemon on all of the machines to a file, so that 
even if the case that the \Condor{collector} is shut down, one can still send
administrator commands to the different machines.
Use the following command:
\footnotesize
\begin{verbatim}
  % condor_status -master -format "%s\n" MasterIpAddr > addresses
\end{verbatim}
\normalsize

The first step to shutting down the pool is to stop any currently
running jobs, and give them a chance to produce a checkpoint.
Depending on the size of the pool, the network infrastructure, and
the image-size of the standard jobs running on the pool,
this may logically be
a slow process, only vacating one host at a time.
Either shut down hosts that have jobs submitted (in which case
all the jobs from that host will try to produce a checkpoint simultaneously),
or shut down individual hosts that are running jobs.
To shutdown a host, issue the command:
\begin{verbatim}
  % condor_off hostname
\end{verbatim}
where \Opt{hostname} is the name of the host to be shut down.
This only works so long as the \Condor{collector} is still
running.
Once Condor is shut down on the central manager,
rely on the \File{addresses} file already created.

If all the running jobs have produced a checkpoint and stopped,
or if not
worried about the network load caused by shutting down
everything at once, it is safe to turn off all daemons on all machines
in the pool.
Do this with a single command, issued from an authorized
administrator machine:
\begin{verbatim}
  % condor_off -all
\end{verbatim}

\Condor{off} will shut down all the daemons, but leave the
\Condor{master} running, so that a future \Condor{on} will work.

Once all of the Condor daemons (except the \Condor{master}) on each
host is turned off, all is done.
It is now safe to install new binaries, move the checkpoint server
to another host, or any other task that requires the pool to be
shut down to successfully complete.

If planning to install a new \Condor{master} binary, be
sure to read the following section to learn of the special considerations 
associated with
this somewhat delicate task.

%%%%%%%%%%%%%%%%%%%%%%%%%%%%%%%%%%%%%%%%%%%%%%%%%%%%%%%%%%%%%%%%%%%%%%
\subsubsection{\label{sec:New-Master}Installing a New \Condor{master}}
%%%%%%%%%%%%%%%%%%%%%%%%%%%%%%%%%%%%%%%%%%%%%%%%%%%%%%%%%%%%%%%%%%%%%%

To install a new \Condor{master} binary, follow a
a few more steps.
When the \Condor{master} restarts, it will listen on a new port,
so the \File{addresses} file will contain stale information.
Moreover, when the \Condor{master} restarts,
it does not know of the previously issued
\Condor{off} command, and will just start up all the daemons
it is configured to spawn.
It must be explicitly told otherwise.

If it is desired that the pool completely restart itself whenever the
\Condor{master} notices its new binary, 
then neither of these issues are of any
concern: skip this (and the next) section.
Just be sure installing the new \Condor{master} binary is the final step,
and once the new binary is in place, the pool will
restart itself over the next 5 minutes (whenever all the \Condor{master}
daemons
notice the new binary, which they each check for once every 5 minutes
by default).

However, to have absolute control over when the rest of
the daemons restart, take a few steps:

\begin{enumerate}
\item Place the following in the global configuration file:
\begin{verbatim}
  START_DAEMONS = False
\end{verbatim}
This will make sure that when the master restarts itself, it 
does not also start up the rest of its daemons.
\item Install the new \Condor{master} binary.
\item Start up Condor on the central manager machine.
This is done manually by logging into the machine and
sending commands locally.
First, send a \Condor{restart} to make sure you have the new \Condor{master},
then send a \Condor{on} to start up the other daemons (including, most
importantly, the \Condor{collector}).
\item Wait 5 minutes, such that all the \Condor{master} daemons have a chance to
notice the new binary, restart themselves, and send an update with
their new address.  Make sure that: 
\begin{verbatim}
  % condor_status -master
\end{verbatim}
lists all the machines in the pool.
\item Remove the special setting from the global configuration file.
\item Recreate the \File{addresses} file as described above:
\footnotesize
\begin{verbatim}
  % condor_status -master -format "%s\n" MasterIpAddr > addresses
\end{verbatim}
\normalsize
\end{enumerate}

Once the new master is in place,
and you are ready to start up the
pool again, restart your whole pool by following the
steps in the next section.

%%%%%%%%%%%%%%%%%%%%%%%%%%%%%%%%%%%%%%%%%%%%%%%%%%%%%%%%%%%%%%%%%%%%%%
\subsubsection{\label{sec:Pool-Restart}Restarting your Condor Pool}
%%%%%%%%%%%%%%%%%%%%%%%%%%%%%%%%%%%%%%%%%%%%%%%%%%%%%%%%%%%%%%%%%%%%%%

Once all preliminary tasks are done and
it is time to restart the pool, send a
\Condor{on} to all the \Condor{master} daemons on each host.
Do this with a single command, issued from an authorized
administrator machine:
\begin{verbatim}
  % condor_on `cat addresses`
\end{verbatim}
At this point, all the daemons should now be restarted, and the pool
will be back on its way.

%%%%%%%%%%%%%%%%%%%%%%%%%%%%%%%%%%%%%%%%%%%%%%%%%%%%%%%%%%%%%%%%%%%%%%
\subsection{\label{sec:Reconfigure-Pool}Reconfiguring Your Condor Pool}
%%%%%%%%%%%%%%%%%%%%%%%%%%%%%%%%%%%%%%%%%%%%%%%%%%%%%%%%%%%%%%%%%%%%%%

To change a global configuration file setting and have all the
machines start to use the new setting, send a
\Condor{reconfig} command to each host.
Do this with a single command, issued from an authorized
administrator machine:
\begin{verbatim}
  % condor_reconfig -all
\end{verbatim}

If the global configuration file is not shared among all the machines
(using a shared file system), the change must be made to each
copy of the global configuration file before issuing the \Condor{reconfig}
command.


%%%%%%%%%%%%%%%%%%%%%%%%%%%%%%%%%%%%%%%%%%%%%%%%%%%%%%%%%%%%%%%%%%%%%%
%%%%%%%%%%%%%%%%%%%%%%%%%%%%%%%%%%%%%%%%%%%%%%%%%%%%%%%%%%%%%%%%%%%%%%
\subsection{\label{sec:Dynamic-Attributes}Using Dynamic Attributes}
%%%%%%%%%%%%%%%%%%%%%%%%%%%%%%%%%%%%%%%%%%%%%%%%%%%%%%%%%%%%%%%%%%%%%%

\index{pool management!with dynamic attributes}

\Todo

%%%%%%%%%%%%%%%%%%%%%%%%%%%%%%%%%%%%%%%%%%%%%%%%%%%%%%%%%%%%%%%%%%%%%%
