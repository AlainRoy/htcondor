%%%%%%%%%%%%%%%%%%%%%%%%%%%%%%%%%%%%%%%%%%%%%%%%%%%%%%%%%%%%%%%%%%%%%%%%%%%
\subsection{\label{sec:Backfill}Configuring Condor for Running Backfill Jobs} 
%%%%%%%%%%%%%%%%%%%%%%%%%%%%%%%%%%%%%%%%%%%%%%%%%%%%%%%%%%%%%%%%%%%%%%%%%%

\index{Backfill}

Beginning with Condor version 6.7.17, Condor can be configured to run
backfill jobs whenever the \Condor{startd} has no other work to
perform.
These jobs are considered the lowest possible priority, but when
machines would otherwise be idle, the resources can be put to good 
use.

Currently, Condor only supports using the Berkeley Open Infrastructure
for Network Computing (BOINC) to provide the backfill jobs.
More information about BOINC is available at
\URL{http://boinc.berkeley.edu}.

The rest of this section will provide an overview of how backfill jobs
work in Condor, details for configuring the policy for when backfill
jobs are started or killed, and details on how to configure Condor to
spawn the BOINC client to perform the work.


%%%%%%%%%%%%%%%%%%%%%%%%%%%%%%%%%%%%%%%%%%%%%%%%%%%%%%%%%%%%%%%%%%%%%%
\subsubsection{\label{sec:Backfill-Overview}Overview of Backfill jobs
in Condor}
%%%%%%%%%%%%%%%%%%%%%%%%%%%%%%%%%%%%%%%%%%%%%%%%%%%%%%%%%%%%%%%%%%%%%%

\index{Backfill!Overview}

Whenever a resource controlled by Condor is in the Unclaimed/Idle
state, it is totally idle: neither the interactive user nor a Condor
job is performing any work.
Machines in this state can be configured to enter the \Term{Backfill}
state, which means the resource will attempt to perform a background
computation to keep itself busy until other work arrives (either a 
user returning to use the machine interactively, or a normal Condor
job).
Once a resource enters the Backfill state, the \Condor{startd} will
attempt to spawn another program, called a \Term{backfill client}, to
launch and manage the backfill computation.
When other work arrives, the \Condor{startd} will kill the backfill
client and clean up any processes it has spawned, freeing the machine
resources for the new, higher priority task.
More details about the different states a Condor resource can enter
and all of the possible transitions between them are described in
section~\ref{sec:Configuring-Policy} beginning on
page~\pageref{sec:Configuring-Policy}, especially
sections~\ref{sec:States}, \ref{sec:Activities}, and
\ref{sec:State-and-Activity-Transitions}.

At this point, the only backfill system supported by Condor is BOINC. 
The \Condor{startd} has the ability to start and stop the BOINC client
program at the appropriate times, but otherwise provides no additional
services to configure the BOINC computations themselves.
Future versions of Condor might provide additional functionality to
make it easier to manage BOINC computations from within the Condor
configuration settings.
For now, the BOINC client must be manually installed and configured
outside of Condor on each backfill-enabled machine.


%%%%%%%%%%%%%%%%%%%%%%%%%%%%%%%%%%%%%%%%%%%%%%%%%%%%%%%%%%%%%%%%%%%%%%
\subsubsection{\label{sec:Backfill-Policy}Defining the Backfill Policy}
%%%%%%%%%%%%%%%%%%%%%%%%%%%%%%%%%%%%%%%%%%%%%%%%%%%%%%%%%%%%%%%%%%%%%%

\index{Backfill!Defining Condor policy}

There are a small set of policy expressions that determine if a
\Condor{startd} will attempt to spawn backfill jobs at all, and if so,
to control the transitions in to and out of the Backfill state.
This section briefly lists these expressions.
More detail can be found in
section~\ref{sec:Startd-Config-File-Entries} on
page~\pageref{sec:Startd-Config-File-Entries}.

\begin{description}

\item[\Macro{ENABLE\_BACKFILL}] A boolean value to determine if any
  backfill functionality should be used.
  The default is \Expr{False}.

\item[\Macro{BACKFILL\_SYSTEM}] A string that defines what backfill
  system to use for spawning and managing backfill computations.
  Currently, the only supported value for this is \AdStr{BOINC}.
  
\item[\Macro{START\_BACKFILL}] A boolean expression to control if a
  Condor resource should start a backfill computation.
  This is only evaluated when the machine is in the Unclaimed/Idle
  state and the \MacroNI{ENABLE\_BACKFILL} expression is \Expr{True}.

\item[\Macro{EVICT\_BACKFILL}] A boolean expression that is evaluated
  whenever a Condor resource is in the Backfill state which, when
  \Expr{True}, indicates the machine should immediately kill the
  currently running backfill computation and return to the Owner
  state.

\end{description}

The following examples show some possible uses of these settings:

\footnotesize
\begin{verbatim}
# Turn on backfill functionality, and use BOINC
ENABLE_BACKFILL = TRUE
BACKFILL_SYSTEM = BOINC

# Spawn a backfill job if we've been Unclaimed for more than 5
# minutes 
START_BACKFILL = $(StateTimer) > (5 * $(MINUTE))

# Evict a backfill job if the machine is busy (based on keyboard
# activity or cpu load)
EVICT_BACKFILL = $(MachineBusy)
\end{verbatim}
\normalsize


%%%%%%%%%%%%%%%%%%%%%%%%%%%%%%%%%%%%%%%%%%%%%%%%%%%%%%%%%%%%%%%%%%%%%%
\subsubsection{\label{sec:Backfill-BOINC-overview}Overview of the
 BOINC system}
%%%%%%%%%%%%%%%%%%%%%%%%%%%%%%%%%%%%%%%%%%%%%%%%%%%%%%%%%%%%%%%%%%%%%%

\index{Backfill!BOINC Overview}

The BOINC system is a distributed computing environment for solving
large scale scientific problems.
A detailed explanation of this system is beyond the scope of this
manual.
Thorough documentation about BOINC is available at their website:
\URL{http://boinc.berkeley.edu}.
However, a brief overview is provided here for sites interested in
using BOINC with Condor to manage backfill jobs. 

BOINC grew out of the relatively famous SETI@home computation, where
volunteers would install special client software (in the form of a
screen saver) that would contact a centralized server to download work
units.
Each work unit contained a set of radio telescope data and the
computation tried to find patterns in the data, a sign of intelligent
life elsewhere in the universe (hence the name: ``Search for Extra
Terrestrial Intelligence at home'').
BOINC is developed by the Space Sciences Lab at the University of
California, Berkeley, by the same people who created SETI@home.
However, instead of being tied to the specific radio telescope
application, BOINC is a generic infrastructure where many different
kinds of scientific computations can be solved.
The current generation of SETI@home now runs on top of BOINC, along
with various physics, biology, climatology, and other applications.

The basic computational model for BOINC and the original SETI@home is
the same: volunteers install BOINC client software which will run
whenever the machine would otherwise be idle.
However, the BOINC installation on any given machine must be
configured so that it knows what computations to work for (each
computation is referred to as a \Term{project} using BOINC's
terminology), instead of always working on a hard coded computation.
A given BOINC client can be configured to donate all of its cycles to
a single project, or to split the cycles between projects so that, on
average, the desired percentage of the computational power is
allocated to each project.
Once the client software (a program called the \Prog{boinc\_client})
starts running, it will attempt to contact a centralized server for
each project it has been configured to work for.
The BOINC software will download the appropriate platform-specific
application binary and some work units from the central server for
each project.
Whenever the client software completes a given work unit, it will once
again attempt to connect to that project's central server to upload
the results and download more work.

BOINC participants must register at the centralized server for each
project they wish to donate cycles to.
The process produces a unique identifier so that the work performed by
a given client can be credited to a specific user.
BOINC keeps track of the work units completed by each user, so that
users providing the most cycles get the highest rankings (and
therefore, bragging rights).

Because BOINC already handles the problems of distributing the
application binaries for each scientific computation, the work units,
and compiling the results, it is a perfect system for managing
backfill computations in Condor.
Many of the applications that run on top of BOINC do their own
application-specific checkpointing, so even if the
\Prog{boinc\_client} is killed (for example, when a Condor job arrives
at a machine, or if the interactive user returns) an entire work unit
won't necessarily be lost.


%%%%%%%%%%%%%%%%%%%%%%%%%%%%%%%%%%%%%%%%%%%%%%%%%%%%%%%%%%%%%%%%%%%%%%
\subsubsection{\label{sec:Backfill-BOINC-install}Installing the BOINC client
software}
%%%%%%%%%%%%%%%%%%%%%%%%%%%%%%%%%%%%%%%%%%%%%%%%%%%%%%%%%%%%%%%%%%%%%%

\index{Backfill!BOINC Installation}

If a working installation of BOINC currently exists on machines
where backfill is desired,
skip the remainder of this section.
Continue reading with the section titled ``Configuring the BOINC
client under Condor''.

In Condor \VersionNotice, the BOINC client software that actually
spawns and manages the backfill computations (the
\Prog{boinc\_client}) must be manually downloaded, installed and
configured outside of Condor.
Hopefully in future versions, the Condor package will include the
\Prog{boinc\_client}, and there will be a way to automatically install
and configure the BOINC software together with Condor.

The \Prog{boinc\_client} executables can be obtained at one of the
following locations:
\begin{description}
\item[\URL{http://boinc.berkeley.edu/download.php}]
  This is the official BOINC download site, which provides binaries
  for MacOS 10.3 or higher, Linux/x86, Solaris/SPARC and Windows/x86.
  From the download table, use the ``Recommended version'', and use 
  the ``Core client only (command-line)'' package when available.

\item[\URL{http://boinc.berkeley.edu/download\_other.php}]
  This page contains links to sites that distribute
  \Prog{boinc\_client} binaries for other platforms beyond the
  officially supported ones.
\end{description}

Once the BOINC client software has been downloaded, the
\Prog{boinc\_client} binary should be placed in a location where the
Condor daemons can use it.
The path will be specified via a Condor configuration setting,
\Macro{BOINC\_Executable}, described below.

Additionally, a local directory on each machine should be created
where the BOINC system can write files it needs.
This directory must not be shared by multiple instances of the BOINC
software, just like the \File{spool} or \File{execute} directories
used by Condor.
This location of this directory is defined using the
\Macro{BOINC\_InitialDir} macro, described below.
The directory must be writable by whatever user the
\Prog{boinc\_client} will run as.
This user is either the same as the user the Condor daemons are
running as (if Condor is not running as root), or a user defined via
the \Macro{BOINC\_Owner} setting described below.

Finally, Condor administrators wishing to use BOINC for backfill jobs
must create accounts at the various BOINC projects they want to donate
cycles to.
The details of this process vary from project to project.
Beware that this step must be done manually, as the BOINC software
spawned by Condor (the \Prog{boinc\_client}) can not automatically
register a user at a given project (unlike the more fancy GUI version
of the BOINC client software which many users run as a screen saver). 
For example, to configure machines to perform work for the
Einstein@home project (a physics experiment run by the University of
Wisconsin at Milwaukee) Condor administrators should go to
\URL{http://einstein.phys.uwm.edu/create\_account\_form.php}, fill in
the web form, and generate a new Einstein@home identity.
This identity takes the form of a project URL (such as
\URL{http://einstein.phys.uwm.edu}) followed by an \Term{account key},
which is a long string of letters and numbers that is used as a unique
identifier. 
This URL and account key will be needed when configuring Condor to use
BOINC for backfill computations (described in the next section).


%%%%%%%%%%%%%%%%%%%%%%%%%%%%%%%%%%%%%%%%%%%%%%%%%%%%%%%%%%%%%%%%%%%%%%
\subsubsection{\label{sec:Backfill-BOINC-Condor}Configuring the BOINC client
under Condor}
%%%%%%%%%%%%%%%%%%%%%%%%%%%%%%%%%%%%%%%%%%%%%%%%%%%%%%%%%%%%%%%%%%%%%%

\index{Backfill!BOINC Configuration in Condor}

This section assumes that the BOINC client software has already been
installed on a given machine, that the BOINC projects to join have
been selected, and that a unique project account key has been created
for each project.
If any of these steps has not been completed, please read the previous
section titled ``Installing the BOINC client software''

Whenever the \Condor{startd} decides to spawn the \Prog{boinc\_client}
to perform backfill computations (when \Macro{ENABLE\_BACKFILL} is
\Expr{True}, when the resource is in Unclaimed/Idle, and when the
\Macro{START\_BACKFILL} expression evaluates to \Expr{True}), it will
spawn a \Condor{starter} to directly launch and monitor the
\Prog{boinc\_client} program.
This \Condor{starter} is just like the one used to spawn normal Condor
jobs.
In fact, the argv[0] of the \Prog{boinc\_client} will be renamed to
``\condor{exec}'', as described in section~\ref{sec:renaming-argv} on 
page~\pageref{sec:renaming-argv}.

The \Condor{starter} for spawning the \Prog{boinc\_client} reads
values out of the Condor configuration files to define the job it
should run, as opposed to getting these values from a job classified
ad in the case of a normal Condor job.
All of the configuration settings to control things like the path to
the \Prog{boinc\_client} binary to use, the command-line arguments,
the initial working directory, and so on, are prefixed with the string
\AdStr{BOINC\_}.
Each possible setting is described below: 

Required settings:

\begin{description}

\item[\Macro{BOINC\_Executable}] \label{param:BoincExecutable} The
  full path to the \Prog{boinc\_client} binary to use.

\item[\Macro{BOINC\_InitialDir}] \label{param:BoincInitialDir} The
  full path to the local directory where BOINC should run.

\item[\Macro{BOINC\_Universe}] \label{param:BoincUniverse} The Condor
  universe used for running the \Prog{boinc\_client} program.
  This \Bold{must} be set to \AdStr{vanilla} for BOINC to work under
  Condor.

\item[\Macro{BOINC\_Owner}] \label{param:BoincOwner} What user the
  \Prog{boinc\_client} program should be run as.
  This macro is only used if the Condor daemons are running as root.
  In this case, the \Condor{starter} must be told what user identity
  to switch to before spawning the \Prog{boinc\_client}.
  This can be any valid user on the local system, but it must have
  write permission in whatever directory is specified in
  \MacroNI{BOINC\_InitialDir}).

\end{description}

Optional settings:

\begin{description}

\item[\Macro{BOINC\_Arguments}] \label{param:BoincArguments}
  Command-line arguments that should be passed to the
  \Prog{boinc\_client} program.
  For example, one way to specify the BOINC project to join is to use 
  the \Opt{--attach\_project} argument to specify a project URL and
  account key.
  For example:

\footnotesize
\begin{verbatim}
BOINC_Arguments = --attach_project http://einstein.phys.uwm.edu [account_key] 
\end{verbatim}
\normalsize

\item[\Macro{BOINC\_Environment}] \label{param:BoincEnvironment}
  Environment variables that should be set for the
  \Prog{boinc\_client}.

\item[\Macro{BOINC\_Output}] \label{param:BoincOutput} Full path to
  the file where STDOUT from the \Prog{boinc\_client} should be
  written.
  If this macro is not defined, STDOUT will be discarded.

\item[\Macro{BOINC\_Error}] \label{param:BoincError} Full path to
  the file where STDERR from the \Prog{boinc\_client} should be
  written.
  If this macro is not defined, STDERR will be discarded.

\end{description}


The following example shows one possible usage of these settings:

\footnotesize
\begin{verbatim}
# Define a shared macro that can be used to define other settings.
# This directory must be manually created before attempting to run
# any backfill jobs.
BOINC_HOME = $(LOCAL_DIR)/boinc

# Path to the boinc_client to use, and required universe setting
BOINC_Executable = /usr/local/bin/boinc_client
BOINC_Universe = vanilla

# What initial working directory should BOINC use?
BOINC_InitialDir = $(BOINC_HOME)

# Save STDOUT and STDERR
BOINC_Output = $(BOINC_HOME)/boinc.out
BOINC_Error = $(BOINC_HOME)/boinc.err
\end{verbatim}
\normalsize

If the Condor daemons reading this configuration are running as root,
an additional macro must be defined:

\footnotesize
\begin{verbatim}
# What user should be used for the boinc_client?
BOINC_Owner = nobody
\end{verbatim}
\normalsize

In this case, Condor would spawn the \Prog{boinc\_client} as
\Username{nobody}, so the directory specified in \MacroNI{BOINC\_HOME}
would have to be writable by the \Username{nobody} user.

A better choice would probably be to create a separate user account
just for running BOINC jobs, so that the local BOINC installation is
not writable by other processes running as \Username{nobody}.
Alternatively, the \MacroNI{BOINC\_Owner} could be set to
\Username{daemon}. 

\noindent \Bold{Attaching to a specific BOINC project}

There are a few ways to attach a Condor/BOINC installation to a given
BOINC project:
\begin{itemize}

\item The \Opt{--attach\_project} argument to the \Prog{boinc\_client}
  program, defined via the \MacroNI{BOINC\_Arguments} setting
  (described above). 

\item The \Prog{boinc\_cmd} command-line tool can perform various
  BOINC administrative tasks, including attaching to a BOINC project.
  Using \Prog{boinc\_cmd}, the appropriate argument to use is called
  \Opt{--project\_attach}.

\item The \Prog{boinc\_cmd} command-line tool can perform various
  BOINC administrative tasks, including attaching to a BOINC project.
  Using \Prog{boinc\_cmd}, the appropriate argument to is called
  \Opt{--project\_attach}.

\end{itemize}

In all of these cases, BOINC will write some files to the local BOINC
directory on the machine, and then future invocations of the
\Prog{boinc\_client} will already be attached to the appropriate
project(s).
More information about participating in multiple BOINC projects can be
found at \URL{http://boinc.berkeley.edu/multiple\_projects.php}.

