%%%%%%%%%%%%%%%%%%%%%%%%%%%%%%%%%%%%%%%%%%%%%%%%%%%%%%%%%%%%%%%%%%%%%%
\subsection{\label{sec:SMP-install}Installing The SMP-Startd Contrib
Module} 
%%%%%%%%%%%%%%%%%%%%%%%%%%%%%%%%%%%%%%%%%%%%%%%%%%%%%%%%%%%%%%%%%%%%%%

Basically, the ``SMP-Startd Contrib module'' is just a selection of a
few files needed to run the 6.1 startd in your existing 6.0 pool.
For documentation on the new startd or the supporting files, see the
version 6.1 manual.

See section~\ref{sec:install} on page~\pageref{sec:install} for
complete details on how to install Condor.
In particular, you should read the first few sections that discuss
\Term{release directories}, pool layout, and so on.

To install the SMP-startd from the separate contrib module, you must
first download the appropriate binary modules for each of your
platforms.  
Once you uncompress and untar the module, you will have a directory
with an \File{smp\_startd.tar} file, a \File{README}, and so on.
The \File{smp\_startd.tar} acts much like the \File{release.tar} file
for a main release.
It contains all the binaries and supporting files you would install in
your release directory:
\begin{verbatim}
        sbin/condor_startd
        sbin/condor_starter
        sbin/condor_preen
        bin/condor_status
        etc/examples/condor_config.local.smp
\end{verbatim}

\Condor{preen} and \Condor{status} are both fully backwards
compatible, so you can use the new version for your entire pool
without changing any of your config files.  
They each have just been enhanced to handle the SMP startd.
See the version 6.1 man pages on each for details.
The \Condor{starter} is also backwards compatible, so you probably
want to install it pool-wide, as well.

The SMP startd is backwards compatible, only in the sense that it
still runs and works just fine on single-CPU machines.
However, it uses different policy expressions to control its policy,
so in this (more important) sense, it is not backwards compatible.
For this reason, you must have some separate config file settings in
effect on machines running the new version.
Therefore, you must decide if you want to convert all your machines to
the new version, or only convert your SMP machines.
If you just convert the SMP machines, you can put the new settings in
the local config file for each SMP machine.
If you convert all your machines, you will want to put the new
settings into your global config file.

%%%%%%%%%%%%%%%%%%%%%%%%%%%%%%%%%%%%%%%%%%%%%%%%%%%%%%%%%%%%%%%%%%%%%%
\subsubsection{\label{sec:SMP-full-install}Installing Pool-Wide}
%%%%%%%%%%%%%%%%%%%%%%%%%%%%%%%%%%%%%%%%%%%%%%%%%%%%%%%%%%%%%%%%%%%%%%

Since you are installing new daemon binaries for all hosts in your
pool, it's generally a good idea to make sure no jobs are running and
all the Condor daemons are shut off before you begin.
Please see section~\ref{sec:Pool-Management} on
page~\pageref{sec:Pool-Management} for details on how to do this.

You may want to keep your old binaries around, just to be safe.
Simply move the existing \Condor{startd}, \Condor{starter},
\Condor{preen} out of the way (for example, to
``\condor{startd}.old'') in the sbin directory, and move
\Condor{status} out of the way in bin.   

You can simply untar the \File{smp\_startd.tar} file into your release
directory, and it will install the new versions (and overwrite your
existing binaries if you haven't moved them out of the way).
Once the new binaries are in place, all you need to do is add the new
settings for the SMP startd to your global config file.

Once the binaries and config settings are in place, you can restart
your pool, as described in section~\ref{sec:Pool-Restart} on
page~\pageref{sec:Pool-Restart} on ``Restarting Your Condor Pool''. 

%%%%%%%%%%%%%%%%%%%%%%%%%%%%%%%%%%%%%%%%%%%%%%%%%%%%%%%%%%%%%%%%%%%%%%
\subsubsection{\label{sec:SMP-partial-install}Installing Only on SMP
Machines} 
%%%%%%%%%%%%%%%%%%%%%%%%%%%%%%%%%%%%%%%%%%%%%%%%%%%%%%%%%%%%%%%%%%%%%%

If you only want to run the new startd on your SMP machines, you
should untar the \File{smp\_startd.tar} file into some temporary
location. 
Copy the \File{sbin/condor\_startd} file into
\Release{sbin/condor\_startd.smp}.
You can simply overwrite \Release{sbin/condor\_preen} and
\Release{bin/condor\_status} with the new versions.
In case you have any currently running \condor{starter} processes, you
should move the existing binary to \condor{starter.old} with ``mv'' so
that you don't get starters that crash with SIGILL or SIGBUS.  
Once you have moved the existing starter out of the way, you can
install the new version from your scratch directory.

Once you've got all the new binaries installed, all you need to do is
edit the local config file for each SMP host in your pool to add the 
SMP-specific settings described below.
In addition, you will need to add the line:
\begin{verbatim}
        STARTD = $(SBIN)/condor_startd.smp
\end{verbatim}
to let the \Condor{master} know you want the new version spawned on
that host.

Once the binaries are all in place and all the configuration settings
are done, you can send a \Condor{reconfig} command to your SMP hosts
(from any machine listed in the \Macro{HOST\_ALLOW\_ADMINISTRATOR}
setting in your config files), the \Condor{master} should notice the
new binaries on the SMP machines, and spawn them.

%%%%%%%%%%%%%%%%%%%%%%%%%%%%%%%%%%%%%%%%%%%%%%%%%%%%%%%%%%%%%%%%%%%%%%
\subsubsection{\label{sec:SMP-config-install}Notes on SMP Startd
configuration} 
%%%%%%%%%%%%%%%%%%%%%%%%%%%%%%%%%%%%%%%%%%%%%%%%%%%%%%%%%%%%%%%%%%%%%%

All documentation for the new Startd can be found in the version 6.1
manual.  
In the \File{etc/examples/condor\_config.local.smp} file, you will see
all the new config file settings you must define or change with the
new version.
Mainly, these are the new policy expressions.  
Look in the version 6.1 manual, in the ``Configuring The Startd
Policy'' section for complete details on how to configure the policy
for the 6.1 startd.
In particular, you probably want to read
the section titled ``Differences from the Version 6.0 Policy
Settings'' to see how the new policy expressions differ from previous
versions. 
These changes are not SMP-specific, they just make writing more
complicated policies much easier.
Given the wide range of SMP machines, from dual-CPU desktop
workstations, up to giant, 128-node super computers, more flexibility
in writing complicated policies is a big help.

In addition to the new policy expressions, there are a few settings
that control how the SMP startd's view of the machine state effects
each of the virtual machines it is representing.  
See the section ``Configuring The Startd for SMP Machines'' for full
details on configuring these other settings of the SMP startd.

Finally, on SMP machines, each running node has its own
\Condor{starter}, and each starter maintains its own log file with a
different name.
Therefore, you want to list which files \Condor{preen} should remove
from the \File{log} directory, instead of having to list the files you
want to keep.
To do this, you specify a \Macro{INVALID\_LOG\_FILES} setting instead
of a \Macro{VALID\_LOG\_FILES} setting.
In both install cases, since you are using the new \Condor{preen} in
your whole pool, you should add the following to your global config
file:
\begin{verbatim}
        INVALID_LOG_FILES = core
\end{verbatim}
since core files are the only unwanted things that might show up in
your \File{log} directory. 
