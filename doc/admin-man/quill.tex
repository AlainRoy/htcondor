%%%%%%%%%%%%%%%%%%%%%%%%%%%%%%%%%%%%%%%
\section{\label{sec:Quill}Quill}
%%%%%%%%%%%%%%%%%%%%%%%%%%%%%%%%%%%%%%%
\index{Quill|(}

Quill builds and maintains a mirror database of a Condor job queue.
The \Condor{quill} daemon implements it,
and the \Condor{q} and \Condor{history} tools use it.

%%%%%%%%%%%%%%%%%%%%%%%%%%%%%%%%%%%%%%%
\subsection{\label{sec:Quill-Installation}Installation and Configuration}
%%%%%%%%%%%%%%%%%%%%%%%%%%%%%%%%%%%%%%%

Quill uses the \Prog{PostgreSQL} database management system.
Quill uses the \Prog{PostgreSQL} server as its back end
and client library, 
\Prog{libpq} to talk to the server.
Quill works with \Prog{PostgreSQL}
version 8.0;
it has also been tested with version 7.4.

Obtain \Prog{PostgreSQL} from

\URL{http://www.postgresql.org/ftp/source/}

Installation instructions are detailed in:
\URL{http://www.postgresql.org/docs/8.0/static/installation.html}

Configure \Prog{PostgreSQL} after installation:

\begin{enumerate}
\item The \Condor{quill} daemon and client tools connect
to the database as users \Username{quillreader} and 
\Username{quillwriter}.
These are database users, not operating system users.
The two types of users are quite different from each other.
If these data base users do not exist,
add them using the 
\Prog{createuser} command supplied with the installation.
Assign them with appropriate passwords;
these passwords will be used by the Quill tools to connect
to the database in a secure way.
User \Username{quillreader} should not be allowed to create
more databases nor create more users.
User \Username{quillwriter} should
not be allowed to create more users,
however it should be allowed to create more databases.
The following commands create the two users
with the appropriate permissions,
and be ready to enter the corresponding passwords when prompted.

\footnotesize
\begin{verbatim}
/path/to/postgreSQL/bin/directory/createuser quillreader \
	--no-createdb --no-adduser --pwprompt

/path/to/postgreSQL/bin/directory/createuser quillwriter \
	--createdb --no-adduser --pwprompt
\end{verbatim}
\normalsize


\item Configure to accept TCP/IP connections.
For \Prog{PostgreSQL} version 8,
use the \Code{listen\_addresses} variable in 
\File{postgresql.conf} file as a guide.
For example,
\Code{listen\_addresses = '*'}
means listen on any IP interface.
In \Prog{PostgreSQL} version 7,
this was accomplished by setting 
\Code{tcpip\_socket=true} in the \File{postgresql.conf} file.


\item Configure \Prog{PostgreSQL} to accept TCP/IP connections from 
specific hosts.
Modify the \File{pg\_hba.conf} file 
(which usually resides in the \Prog{PostgreSQL} server's data directory).
Access is required by the \Condor{quill} daemon,
as well as the database users
\Username{quillreader} and \Username{quillwriter}.
For example, to give
database users \Username{quillreader} and \Username{quillwriter}
password-enabled access to all databases on current machine from any
other machine in the network, add the following:

\begin{tabular}{llllll}
host&all&quillreader&128.105.0.0&255.255.0.0&password\\
host&all&quillwriter&128.105.0.0&255.255.0.0&password
\end{tabular}

Note that in addition to the database specified by
the configuration variable \MacroNI{QUILL\_DB\_NAME},
the \Condor{quill} daemon also needs access to the database
"template1".
In order to create the database in the first place, 
the \Condor{quill} daemon needs to connect to the database.

\item The \Condor{quill} daemon needs read and write access
to the database.
It connects as user \Username{quillwriter},
who has owner privileges to the database.
Since this gives all access to the \Username{quillwriter} user,
this password cannot be stored in a public place 
(such as the \Condor{collector}).
For this reason, the \Username{quillwriter} password is stored
in a file named \File{.quillwritepassword} in the Condor spool directory.
Appropriate protections on this file guarantee secure access to the database.
This file must be created and protected by the site administrator;
if this file does not exist as and where expected, the \Condor{quill}
daemon logs an error and exits.

\end{enumerate}


Condor must also be configured to use Quill.

\begin{description}
\item Add \Expr{QUILL} to the variable \MacroNI{DAEMON\_LIST}.
\item Add the file \File{.quillwritepassword} to the 
  \MacroNI{VALID\_SPOOL\_FILES} variable, since \Condor{preen} must
  be told not to delete this file.
\item Tell Condor where Quill resides, as well as specify Quill's
  start-up arguments:
\begin{verbatim}
QUILL = $(SBIN)/condor_quill
QUILL_ARGS = -f
\end{verbatim}
\item Identify Quill's log file:
\begin{verbatim}
QUILL_LOG = $(LOG)/QuillLog
\end{verbatim}
\item Set up all other configuration variables that are specific
  to the installation.
\footnotesize
\begin{verbatim}
QUILL_ENABLED           = TRUE
QUILL_NAME              = some-unique-quill-name.cs.wisc.edu
QUILL_DB_NAME           = database-for-some-unique-quill-name
QUILL_DB_IP_ADDR        = databaseipaddress:port
# the following parameter's units is in seconds
QUILL_POLLING_PERIOD    = 10
# the following parameter's units is in hours
QUILL_HISTORY_CLEANING_INTERVAL = 24
# the following parameter's units is in days
QUILL_HISTORY_DURATION 	= 30
QUILL_IS_REMOTELY_QUERYABLE = TRUE
QUILL_DB_QUERY_PASSWORD =  password-for-database-user-quillreader
QUILL_ADDRESS_FILE      = $(LOG)/.quill_address
\end{verbatim}
\normalsize

\end{description}


Descriptions of these and other configuration variables are in
section~\ref{sec:Quill-Config-File-Entries}.
Here are further brief details:

\begin{itemize}

\item \Macro{QUILL\_DB\_<NAME>} and \Macro{QUILL\_DB\_IP\_ADDR}\\
These two variables are used to determine the location of the database
server that this Quill would talk to, and the name of the database that
it creates.  More than one Quill server can talk to the same database
server.  This can be done by simply letting all the 
\MacroNI{QUILL\_DB\_IP\_ADDR} point to the same database server.

If more than one Quill server are sharing the same database
server, then the \MacroNI{QUILL\_DB\_NAME} variable for all of them should
be unique.  Otherwise, there would be record overwriting and corruption
of job queue information.

\item \Macro{QUILL\_POLLING\_PERIOD}\\
This controls the frequency with which Quill polls the
\File{job\_queue.log} file.  By default, it is 10 seconds.  Since Quill
works by periodically sniffing the log file for updates and then sending
those updates to the database, this variable controls the trade off between
the currency of query results and Quill's load on the system--usually
negligible.

\item \Macro{QUILL\_HISTORY\_CLEANING\_INTERVAL} and 
		\Macro{QUILL\_HISTORY\_DURATION}\\
These two variables control the deletion of historical jobs from the
database.  \MacroNI{QUILL\_HISTORY\_DURATION} is the number of days
after completion (more precisely, the number of days since the history ad got 
into the history database - those two might be different if a job is completed 
but stays in the queue for a while) that a given job will stay in the database.  
So all jobs beyond \MacroNI{QUILL\_HISTORY\_DURATION} will be deleted.  Now,
scanning the entire database for old jobs can get pretty expensive,
so the other variable \MacroNI{QUILL\_HISTORY\_CLEANING\_INTERVAL}
is the number of hours between two successive scans.  By default,
\MacroNI{QUILL\_HISTORY\_DURATION} is set to 180 days and
\MacroNI{QUILL\_HISTORY\_CLEANING\_INTERVAL} is set to 24 hours.

\item \Macro{QUILL\_IS\_REMOTELY\_QUERYABLE}\\
Thanks to 
\Prog{PostgreSQL},
one can now remotely query both the job queue and the
history tables. This variable controls whether this remote querying 
feature should be enabled.  By default it is TRUE.  Note that even if 
this is FALSE, one can still query the job queue in the remote schedd
This variable only controls whether the database tables are remotely queryable.

\item \Macro{QUILL\_DB\_QUERY\_PASSWORD}\\
In order for the query tools to connect to a database, it needs to provide
the password that is assigned to database user \Username{quillreader}. 
This variable is then advertised by the \Condor{quill} daemon
to the \Condor{collector}.  
This facility enables remote querying: remote \Condor{q} query tools first 
ask the \Condor{collector} for the password associated with a particular Quill database 
and then query that database.  Users who do not have access to the collector 
cannot view the password and as such cannot query the database.  Again, this 
password just provides 'read' access to the database.

\item \Macro{QUILL\_ADDRESS\_FILE}\\
When Quill starts up, it can place it's address (IP and port)
into a file.  This way, tools running on the local machine do not
need to query the central manager to find Quill.  This 
feature can be turned off by commenting out the variable.

\end{itemize}


%%%%%%%%%%%%%%%%%%%%%%%%%%%%%%%%%%%%%%%
\subsection{\label{sec:Quill-Example}Four Usage Examples}
%%%%%%%%%%%%%%%%%%%%%%%%%%%%%%%%%%%%%%%


\begin{enumerate}
\item Query a remote Quill daemon on \File{regular.cs.wisc.edu}
for all the jobs in the queue
\begin{verbatim}
	condor_q -name quill@regular.cs.wisc.edu
	condor_q -name schedd@regular.cs.wisc.edu

\end{verbatim}
There are two ways to get to a Quill daemon: directly using its name as 
specified in the \MacroNI{QUILL\_NAME} configuration variable, or indirectly
by querying the \Condor{schedd} daemon using its name.
In the latter case, \Condor{q} will detect 
if that \Condor{schedd} daemon is being serviced by a database, and if so, directly query it.
In both cases, the IP address and port of the database server hosting the data of 
this particular remote Quill daemon can be figured out by the \MacroNI{QUILL\_DB\_IP\_ADDR} 
and \MacroNI{QUILL\_DB\_NAME} variables specified in the \MacroNI{QUILL\_AD}
sent by the quill daemon to the collector and in the \MacroNI{SCHEDD\_AD} sent by
the \Condor{schedd} daemon.  

\item Query a remote Quill daemon on \File{regular.cs.wisc.edu} for all historical 
jobs belonging to owner einstein.
\begin{verbatim}
	condor_history -name quill@regular.cs.wisc.edu einstein
\end{verbatim}

\item Query the local Quill daemon for the average time spent in the queue 
for all non-completed jobs. 
\begin{verbatim}
	condor_q -avgqueuetime 
\end{verbatim}
The averate queue time is defined as the average of
\Expr{(currenttime - jobsubmissiontime)} over all jobs which are neither
completed \Expr{(JobStatus == 4)} or removed \Expr{(JobStatus == 3)}.

\item Query the local Quill daemon for all historical jobs completed since 
Apr 1, 2005 at 13h 00m.
\begin{verbatim}
	condor_history -completedsince '04/01/2005 13:00'
\end{verbatim}
It fetches all jobs
which got into the 'Completed' state on or after the
specified time stamp.  It use the \Prog{PostgreSQL} date/time
syntax rules, as it encompasses most format options.  See
\URL{http://www.postgresql.org/docs/8.0/static/datatype-datetime.html\#AEN4516}
for the various time stamp formats.

\end{enumerate}

%%%%%%%%%%%%%%%%%%%%%%%%%%%%%%%%%%%%%%%
\subsection{\label{sec:Quill-Schema}Quill and Its RDBMS Schema}
%%%%%%%%%%%%%%%%%%%%%%%%%%%%%%%%%%%%%%%

With only 7 tables and 2 views, Quill uses a relatively simple database
schema.  These can be broadly divided into tables used to store job
queue information and those used to store historical information.

The job queue part of the schema closely follows Condor's ClassAd data
model. For example, each row in these tables describe an <attribute,value>
pair of the classad.  Additionally, just as how Condor distinguishes a
ClusterAd from a ProcAd where the former stores attributes common to all
jobs within a cluster whereas the latter stores attributes specific to
each job, the schema also makes this distinction.  Finally, numerical
and string valued attributes are stored separately.

Thus, there are four tables:

\begin{itemize}

\item 
	\SQLTableDef{ClusterAds\_Str}
		{cid int, 
		attr text, 
		val text, 
		primary key (cid, attr)}

\item 
	\SQLTableDef{ClusterAds\_Num}
		{cid int, 
		attr text, 
		val double precision, 
		primary key (cid, attr)}

\item 
	\SQLTableDef{ProcAds\_Str}
		{cid int, 
		pid int, 
		attr text, 
		val text, 
		primary key (cid, pid, attr)}

\item 
	\SQLTableDef{ProcAds\_Num}
		{cid int, 
		pid int, 
		attr text, 
		val double precision,
		primary key (cid, pid, attr)}

\end{itemize}

In addition to the <attribute, value>, each row contains the cluster-id
(cid) and in the case of the ProcAd tables, also the proc-id (pid).

Since each ClassAd would be split into potentially two tables (string
and numeric), there are views that unify them into a single entity in
order to simplify queries.

Here are the view definitions:

\begin{itemize}
\item Definition of ClusterAds view\\
	\SQLViewDef{ClusterAds}
		{select cid, 
		attr, 
		val from ClusterAds\_Str UNION ALL
		select cid, 
		attr, 
		cast(val as text) from ClusterAds\_Num}


\item Definition of ProcAds view\\
	\SQLViewDef{ProcAds}
		{select cid, 
		pid, 
		attr, 
		val from ProcAds\_Str UNION ALL
		select cid, 
		pid, 
		attr, 
		cast(val as text) from ProcAds\_Num}

\end{itemize}

Finally, the job queue part of the schema also contains a table that
stores metadata information related to the \File{job\_queue.log} file.

\begin{itemize}
\item \SQLTableDef{JobQueuePollingInfo}
        {last\_file\_mtime         BIGINT,
        last\_file\_size          BIGINT,
        last\_next\_cmd\_offset    BIGINT,
        last\_cmd\_offset         BIGINT,
        last\_cmd\_type           SMALLINT,
        last\_cmd\_key            text,
        last\_cmd\_mytype         text,
        last\_cmd\_targettype     text,
        last\_cmd\_name           text,
        last\_cmd\_value          text}
\end{itemize}
	
At all times, there is only 1 row in this table and it describes
information related to the last time Quill polled the \File{job\_queue.log} file.

\begin{itemize}
\item \Bold{last\_file\_mtime} and \Bold{last\_file\_size}\\
	The last modified time and size of the file.

\item \Bold{last\_cmd\_offset} and \Bold{last\_next\_cmd\_offset}\\
	The offsets of the record last read from the file and its successive record.

\item \Bold{last\_cmd\_type}\\
	The command type (101, 102, etc.) of the record.

\item	\Bold{last\_cmd\_key}, 
		\Bold{last\_cmd\_mytype}, 
		\Bold{last\_cmd\_targettype},
		\Bold{last\_cmd\_name},
		and
		\Bold{last\_cmd\_value}\\
	Together, these attributes define the record itself.	The key
	refers to the combined "cid.pid" pair, mytype and target usually
	contains Job and Machine respectively, and finally the name and
	value contains the <attribute,value> pair.
\end{itemize}

The historical information on the other hand is slightly differently
designed.  Instead of a purely vertical data model (each row is a
$<$attribute,value$>$ pair), we have two tables that together represent the
complete job classad.  Their schema is as follows:

\begin{enumerate}

\item \SQLTableDef{History\_Horizontal}
        {cid                  int,
        pid                  int,
	EnteredHistoryTable  timestamp with time zone,
        Owner                text,
        QDate                int,
        RemoteWallClockTime  int,
        RemoteUserCpu        float,
        RemoteSysCpu         float,
        ImageSize            int,
        JobStatus            int,
        JobPrio              int,
        Cmd                  text,
        CompletionDate       int,
        LastRemoteHost       text,
        primary key(cid,pid)}

\item \SQLTableDef{History\_Vertical}
	{cid int, pid int, attr text, val text, primary key
	(cid, pid, attr)}

\end{enumerate}

Each historical job ad is divided into its horizontal and vertical
counterparts.  This division was made because of query performance
reasons.  While its easier to store ClassAds in a vertical table,
queries on vertical tables generally perform worse than those on
horizontal tables since the latter has lot fewer records.  However, in
Condor, since job ads do not have a fixed schema (users can define their
own attributes), a purely horizontal schema would end up having a lot
of null values. As such, we have a hybrid schema where attributes on
which queries are frequently performed (via \Condor{history}) are put
in the \SQLTable{History\_Horizontal} table and the other attributes
are stored vertically (just as in the Cluster/Proc tables above) in the
\SQLTable{History\_Vertical} table. Also \SQLTable{History\_Horizontal}
contains all the attributes needed to service the short form of the
\Condor{history} command (that is, without the -l option).

The resulting hybrid schema has proven to be the most efficient in
servicing \Condor{history} queries.  The job queue tables (Cluster and
Proc) were not designed in this hybrid manner because job queues aren't
as large as history; just a vertical schema worked great.


%%%%%%%%%%%%%%%%%%%%%%%%%%%%%%%%%%%%%%%
\subsection{\label{sec:Quill-Security}Quill and Security}
%%%%%%%%%%%%%%%%%%%%%%%%%%%%%%%%%%%%%%%

There are several layers of security in Quill, some provided by Condor and
others provided by the database.  First, all accesses to the database
are password-protected.

\begin{enumerate}
\item The query tools, \Condor{q} and
\Condor{history} connect to the database as user \Username{quillreader}.
The password for this user can vary from one database to another and
as such, each Quill daemon advertises this password to the collector.
The query tools then obtain this password from the collector and
connect successfully to the database.  Access to the database by the
\Username{quillreader} user is read-only, as this is sufficient for the
query tools.  The \Condor{quill} daemon ensures this protected access using the sql
GRANT command when it first creates the tables in the database.  Note that
access to the \Username{quillreader} password itself can be blocked by
blocking access to the collector, a feature already supported in Condor.

\item The \Condor{quill} daemon, on the other hand, needs read and write access
to the database.  As such, it connects as user \Username{quillwriter},
who has owner privileges to the database.  Since this gives all
access to the \Username{quillwriter} user, this password cannot
be stored in a public place (such as the collector).  For this
reason, the \Username{quillwriter} password is stored in a file called
\File{.quillwritepassword} in the Condor spool directory.
Appropriate protections on this file guarantee secure access to the database.
This file must be created and protected by the site administrator;
if this file does not exist as and where expected, the \Condor{quill}
daemon logs an error and exits.

\item The \Attr{IsRemotelyQueryable} attribute in the Quill ClassAd advertised
by the Quill daemon to the collector can be used by site administrators
to disallow the database from being read by all remote Condor query tools.

\end{enumerate}

