%%%%%%%%%%%%%%%%%%%%%%%%%%%%%%%%%%%%%%%%%%%%%%%%%%%%%%%%%%%%%%%%%%%%%%
\subsection{\label{sec:Contrib-CondorView-Install}
Installing CondorView Contrib Modules}
%%%%%%%%%%%%%%%%%%%%%%%%%%%%%%%%%%%%%%%%%%%%%%%%%%%%%%%%%%%%%%%%%%%%%%

To install CondorView for your pool, you really need two things:
\begin{enumerate}
\item The CondorView server, which collects historical information.
\item The CondorView client, a Java applett that views this data.
\end{enumerate}

Since these are totally seperate modules, they will each be handled in
their own sections.

%%%%%%%%%%%%%%%%%%%%%%%%%%%%%%%%%%%%%%%%%%%%%%%%%%%%%%%%%%%%%%%%%%%%%%
\subsection{\label{sec:CondorView-Server-Install}
Installing the CondorView Server Module}
%%%%%%%%%%%%%%%%%%%%%%%%%%%%%%%%%%%%%%%%%%%%%%%%%%%%%%%%%%%%%%%%%%%%%%

The CondorView server is just an enhanced version of the
\Condor{collector} which can log information to disk, providing a 
persistant, historical database of your pool state.
This includes machine state, as well as the state of jobs submitted by
users, and so on.
This enhanced \Condor{collector} is simply the version 6.1 development
series, but it can be installed in a 6.0 pool.
The historical information logging can be turned on or off, so you can
install the CondorView collector without using up disk space for
historical information if you don't want it.

To install the CondorView server, you must download the appropriate
binary module for whatever platform you are going to run your
CondorView server on.
This does not have to be the same platform as your exisiting central
manager (see below).
Once you uncompress and untar the module, you will have a directory
with a \File{view\_server.tar} file, a \File{README}, and so on.
The \File{view\_server.tar} acts much like the \File{release.tar} file
for a main release of Condor.
It contains all the binaries and supporting files you would install in
your release directory:
\begin{verbatim}
        sbin/condor_collector
        sbin/condor_stats
        etc/examples/condor_config.local.view_server
\end{verbatim}

You have two options to choose from when deciding how to install this
enhanced \Condor{collector} in your pool:
\begin{enumerate}
\item Replace your exisiting \Condor{collector} and use the new
version for both historical information and the regular role the 
collector plays in your pool.
\item Install the new \Condor{collector} and run it on a seperate host
from your main \Condor{collector} and configure your machines to send
updates to both collectors.
\end{enumerate}

If you replace your existing collector with the enhanced version,
because it is development code, there might be a bug or problem that
would cause problems for your pool.
On the other hand, if you install the enhanced version on a seperate
host, if there are problems, only CondorView will be affected, not
your entire pool.
However, installing the CondorView collector on a seperate host
generates more network traffic (from all the duplicate updates that
are sent from each machine in your pool to both collectors).
In addition, the installation procedure to have both collectors
running is a more complicated process.
You will just have to decide for yourself which solution you feel more
comfortable with.

Before we discuss the details of one type of installation or the
other, we explain the steps you must take in either case.

%%%%%%%%%%%%%%%%%%%%%%%%%%%%%%%%%%%%%%%%%%%%%%%%%%%%%%%%%%%%%%%%%%%%%%
\subsubsection{\label{sec:CondorView-Server-Setup}
Setting up the CondorView Server Module} 
%%%%%%%%%%%%%%%%%%%%%%%%%%%%%%%%%%%%%%%%%%%%%%%%%%%%%%%%%%%%%%%%%%%%%%

Before you install the CondorView collector (as described in the
following sections), you have to add a few settings to the local
config file of that machine to enable historical data collection.
These settings are described in detail in the Condor Version 6.1
Administrator's Manual, in the section ``\condor{collector} Config File
Entries''.
However, a short explaination of the ones you must customize is
provided below. 
These entries are also explained in the
\File{etc/examples/condor\_config.local.view\_server} file, included
in the contrib module.
You should just insert that file into the local config file for your
CondorView collector host and customize as appropriate at your site.  
\begin{description}

\item[\Macro{POOL\_HISTORY\_DIR}] This is the directory where
historical data will be stored.
There is a configurable limit to the maximum space required for all
the files created by the CondorView server
(\Macro{POOL\_HISTORY\_MAX\_STORAGE}). 
This directory must be writable by whatever user the CondorView
collector is running as (usually "condor").  

\Note This should be a seperate directory, not the same as either the
\File{Spool} or \File{Log} directories you have already setup for
Condor. 
There are a few problems putting these files into either of those
directories.

\item[\Macro{KEEP\_POOL\_HISTORY}] This is a boolean that determines
if the CondorView collector should store the historical information.
It is false by default, which is why you must specify it as true in
your local config file.

\end{description}

Once these settings are in place in the local config file for your
CondorView server host, you must to create the directory you specified
in \Macro{POOL\_HISTORY\_DIR} and make it writable by whomever your
CondorView collector is running as.
This would be the same user that owns the \File{CollectorLog} file in
your \File{Log} directory (usually, ``condor'').

Once those steps are completed, you are ready to install the new
binaries and you will begin collecting historical information.
Then, you should install the CondorView client contrib module which
contains the tools used to query and display this information.

%%%%%%%%%%%%%%%%%%%%%%%%%%%%%%%%%%%%%%%%%%%%%%%%%%%%%%%%%%%%%%%%%%%%%%
\subsubsection{\label{sec:CondorView-Server-Only}
CondorView Collector as Your Only Collector} 
%%%%%%%%%%%%%%%%%%%%%%%%%%%%%%%%%%%%%%%%%%%%%%%%%%%%%%%%%%%%%%%%%%%%%%

To install the new CondorView collector as your main collector, you
simply have to replace your existing binary with the new one, found in
the \File{view\_server.tar} file.
All you need to do is move your existing \File{\condor{collector}}
binary out of the way with the ``mv'' command.
For example:
\begin{verbatim}
        % cd /full/path/to/your/release/directory
        % cd sbin
        % mv condor_collector condor_collector.old
\end{verbatim}
Then, from that same directory, you just have to untar the
\File{view\_server.tar} file, into your release directory, which will
install a new \File{\condor{collector}} binary, \File{\condor{stats}},
a tool that can be used to query this collector for historical
information, and an example config file.
Within 5 minutes, the \Condor{master} will notice the new timestamp on
your \File{\condor{collector}} binary, shutdown your existing
collector, and spawn the new version.
You will see messages about this in the log file for your
\Condor{master} (usually \File{MasterLog} in your \File{log}
directory).
Once the new collector is running, it is safe to remove your old
binary, though you may want to keep it around in case you have
problems with the new version and want to revert back.

%%%%%%%%%%%%%%%%%%%%%%%%%%%%%%%%%%%%%%%%%%%%%%%%%%%%%%%%%%%%%%%%%%%%%%
\subsubsection{\label{sec:CondorView-Server-Both}
CondorView Collector in Addition to Your Main Collector} 
%%%%%%%%%%%%%%%%%%%%%%%%%%%%%%%%%%%%%%%%%%%%%%%%%%%%%%%%%%%%%%%%%%%%%%

To install the CondorView collector in addition to your regular
collector requires a little extra work.
First, you should untar the \File{view\_server.tar} file into some
temporary location (not your main release directory).
Copy the \File{sbin/\condor{collector}} file out of there, and into
your main release directory's sbin with a new name (such as
\File{\condor{collector}.view\_server}).
You will also want to copy the \File{\condor{stats}} program to your
sbin release directory. 

Next, you must configure whatever host is going to run your seperate
CondorView server to spawn this new collector in addition to whatever
other daemons it's running.
You do this by adding ``COLLECTOR'' to the \Macro{DAEMON\_LIST} on
this machine, and defining what ``COLLECTOR'' means.
For example:
\begin{verbatim}
        DAEMON_LIST = MASTER, STARTD, SCHEDD, COLLECTOR
        COLLECTOR = $(SBIN)/condor_collector.view_server
\end{verbatim}
For this change to take effect, you must actually re-start the
\Condor{master} on this host (which you can do with the
\Condor{restart} command, if you run that command from a machine with 
``ADMINISTRATOR'' access to your pool.
(See section~\ref{sec:Host-Security} on
page~\pageref{sec:Host-Security} for full details of IP/host-based
security in Condor).

Finally, you must tell all the machines in your pool to start sending
updates to both collectors.
You do this by specifying the following setting in your global config
file:
\begin{verbatim}
        CONDOR_VIEW_HOST = full.hostname
\end{verbatim}
where ``full.hostname'' is the full hostname of the machine where you
are running your CondorView collector.

Once this setting is in place, you must send a \Condor{reconfig} to
your entire pool.  The easiest way to do this is:
\begin{verbatim}
        % condor_reconfig `condor_status -master`
\end{verbatim}
Again, this command must be run from a trusted ``administrator''
machine for it to work.  

%%%%%%%%%%%%%%%%%%%%%%%%%%%%%%%%%%%%%%%%%%%%%%%%%%%%%%%%%%%%%%%%%%%%%%
\subsection{\label{sec:CondorView-Client-Install}
Installing the CondorView Client Contrib Module} 
%%%%%%%%%%%%%%%%%%%%%%%%%%%%%%%%%%%%%%%%%%%%%%%%%%%%%%%%%%%%%%%%%%%%%%

\Todo