%%%%%%%%%%%%%%%%%%%%%%%%%%%%%%%%%%%%%%%%%%%%%%%%%%%%%%%%%%%%%%%%%%%%%%
\subsection{\label{sec:GCB}Generic Connection Brokering (GCB)}
%%%%%%%%%%%%%%%%%%%%%%%%%%%%%%%%%%%%%%%%%%%%%%%%%%%%%%%%%%%%%%%%%%%%%%

Generic Connection Brokering, or GCB, is a system for managing network
connections across private networks and firewalls.
Starting in version 6.7.13, Condor's Linux releases are linked with
GCB, and can use GCB functionality to run jobs across pools that span
public and private networks (either directly or via flocking).

While GCB provides numerous advantages over restricting Condor to use
a range of ports which are then opened on the firewall (see
section~\ref{sec:Ports-Firewalls} on
page~\pageref{sec:Ports-Firewalls}),
GCB is also a very complicated system, with a lot of major
implications for Condor's networking and security functionality.
Therefore, sites must carefully weigh the pros and cons of attempting
to configure and use GCB before making a decision.

Pros:
\begin{itemize}

\item Better connectivity: GCB works with pools that have multiple
  private networks (even multiple private networks that use the same
  IP addresses (e.g. 192.168.2.*).
  GCB also works with sites that use network address translation
  (NAT). 

\item More secure: administrators never need to allow inbound
  connections through their firewall.
  With GCB, only outbound connections from behind the firewall must be
  allowed (which is a very standard firewall configuration).
  You can even trade decreased performance for better security and
  configure your firewall to only allow outbound connections to a
  single public IP address.

\item Does not require root access to any machines.
  All parts of a GCB system can be run as an unprivileged user, and in
  the common case, no changes to the firewall configuration are
  required.

\end{itemize}

Cons:
\begin{itemize}

\item Adds new potential failure points to your pool: the GCB broker
  node(s).
  Any private nodes that want to communicate outside their own network
  must be represented by a GCB broker (described below).
  This machine must be highly reliable, since if the broker is ever
  down, all inbound communication with the private nodes is
  impossible.
  Furthermore, no other Condor services should be run on a GCB broker
  (for example, the pool central manager).
  While it's possible to do so, this is not recommended (the reasons
  are explained in section~\ref{sec:GCB-Broker-Intro}, which describes
  the GCB broker in full detail). 
  In general, no other services should be run on the machine at all,
  and the host should be dedicated to the task of serving as a GCB
  broker.

\item All Condor nodes behind a given firewall share a single IP
  address (the public IP address of their GCB broker).
  All Condor daemons using a GCB broker will advertise themselves with
  this single IP address, and in some cases, connections to/from those
  daemons will actually originate at the broker.
  This has all sorts of implications for Condor's host/IP based
  security, and the general level of confusion for users and
  administrators of the pool.
  Debugging problems will be more difficult, as any log messages which 
  only print the IP (not the name and/or port) will become ambiguous.
  Even log or error messages that include the port won't necessarily
  be helpful, since it is difficult to correlate ports on the broker
  with the corresponding private nodes.
  
\item Can not function with Kerberos authentication.
  Kerberos tickets include the IP address of the machine where they
  were created.
  However, when Condor daemons are using GCB, they use a different IP
  address (see above), and therefore, any attempt to authenticate
  using Kerberos will fail, as Kerberos will consider this a (poor)
  attempt to fool it into using an invalid host principle.

\item Scalability and performance degradation: 
  \begin{itemize}
  \item Connections are more expensive to establish
  \item In some cases, connections must be forwarded through a proxy
    server on the GCB broker
  \item Each network port on each private node must correspond to a
    unique port on the broker host, so there is a fixed limit to how
    many private nodes a given broker can service (which is a function
    of the number of ports each private node requires and the total
    number of available ports on the broker)
  \item Each private node must maintain an open TCP connection to its
    GCB broker.  GCB will attempt to recover in the case of the socket
    being closed, but this means the broker must have at least as many
    sockets open as there are private nodes.
  \end{itemize}

\item More complex to configure and debug

\end{itemize}

Given the increased complexity, any site that decides to use GCB must
read this manual section in its entirety before having any hope of
configuring the system properly.
Please do not skim the information in this section, attempt to quickly
install GCB, and then ask for help when things go wrong.

More information about GCB and how it works can be found at the GCB
homepage:

\URL{http://www.cs.wisc.edu/\~{}sschang/firewall/gcb}

This information is very useful for understanding the technical
details of how GCB works, the various parts of the system, and so on.
While some of the information is partly out of date (especially the
discussion of how to configure GCB) most of the sections are perfectly
accurate and worth reading.
You will not need to read the section on ``GCBnize'', which describes
how to get a given application to use GCB, since we have already done
that to Condor for you.

The rest of this section will give the details for configuring a
Condor pool to use GCB.
It is divided into the following topics:

\begin{itemize}
\item Introduction to the GCB broker
\item Configuring the GCB broker
\item Using a \Condor{master} to spawn the GCB broker
\item Using \Prog{initd} to spawn the GCB broker
\item How to configure Condor machines to use GCB
\item Configuring the GCB routing table
\item Implications for Condor's host/IP security settings
\item Implications for other Condor configuration settings
\end{itemize}


%%%%%%%%%%%%%%%%%%%%%%%%%%%%%%%%%%%%%%%%%%%%%%%%%%%%%%%%%%%%%%%%%%%%%%
\subsubsection{\label{sec:GCB-Broker-Intro}Introduction to the GCB Broker}
%%%%%%%%%%%%%%%%%%%%%%%%%%%%%%%%%%%%%%%%%%%%%%%%%%%%%%%%%%%%%%%%%%%%%%

At the heart of GCB is a logical entity known as a \Term{broker} or
\Term{inagent}.
In reality, it is made up of two different daemon processes running on
the same machine (the \Prog{gcb\_broker}, and a set of
\Prog{gcb\_relay\_server} processes, each one spawned by the
\Prog{gcb\_broker}).

Every private network in a given network topology trying to use GCB
must have at least one corresponding broker to arrange connections for
it.
The broker must be installed on a machine that nodes in both the
public and the private (firewalled) network can directly talk to.
However, the broker need not be able to initiate connections to the
private nodes (though it can take advantage of the case where it can
initiate connections to the private nodes and that will improve
performance). 
Therefore, the broker is generally installed directly on the network
boundary (a machine with multiple network interfaces) or just outside
of a network that allows outbound connections.
If the private network contains many hosts, sites can configure
multiple GCB brokers and partition the private nodes so that different
subsets of the nodes use different brokers.

For a more thorough explanation of what a GCB broker is, check out:
\URL{http://www.cs.wisc.edu/\~{}sschang/firewall/gcb/mechanism.htm}

The GCB broker should generally be installed on dedicated machines
(i.e. machines that aren't running other Condor daemons or services).
If you run any other Condor service on the GCB broker node (for
example, the central manager of your pool) all Condor nodes attempting
to use this service (for example, to connect to the \Condor{collector}
or \Condor{negotiator}) will incur additional connection costs and
latency.
It's possible that future versions of GCB and Condor will be able to
overcome these limitations, but for now, we recommend that the broker
is run on a dedicated machine with no other Condor daemons (except
perhaps a single \Condor{master} used to spawn the \Prog{gcb\_broker}
daemon, as described below).

In principle, a GCB broker is a network element that functions almost
like a router.
It allows certain connections through your firewall by redirecting
connections, and even forwarding connections.
In general, it's not a good idea to run a lot of other services on
your network elements, especially not services like Condor which can
spawn arbitrary jobs.
Furthermore, the GCB broker relies on listening to a lot of network
ports.
If other applications are running on the same host as the broker, you
can run into problems where the broker doesn't have enough network
ports available to forward all the connections that might be required
of it.
Also, all nodes inside a private network rely on the GCB broker for
all incoming communication.
For performance reasons, you should avoid the GCB broker having to
contend with other processes for system resources so that it is always
available to handle communication requests.
However, there is nothing in GCB or Condor that requires
the broker runs on a separate machine, that's just how we recommend
you configure it if at all possible.

The \Prog{gcb\_broker} daemon (currently) listens on two hard-coded,
fixed ports (65432 and 65430).
Future versions of Condor and GCB will remove this limitation.
However, for now, if you're going to run a \Prog{gcb\_broker} on a
given host, you must ensure that ports 65432 and 65430 are not already
in use. 

If you have root access on a machine where you would like to run a GCB
broker, one good option is to have your \Prog{initd} configured to
spawn (and respawn) the \Prog{gcb\_broker} binary (which is located in
the \Release{libexec} directory).
This way, the \Prog{gcb\_broker} will be automatically restarted on
reboots, or in the event that the broker itself crashes or is killed.
If you do not have root access, you can also use a \Condor{master} to
manage your \Prog{gcb\_broker} binary.  Each of these options are
described below in more detail.

%%%%%%%%%%%%%%%%%%%%%%%%%%%%%%%%%%%%%%%%%%%%%%%%%%
\subsubsection{\label{sec:GCB-Broker-Config}
Configuring the GCB broker}
%%%%%%%%%%%%%%%%%%%%%%%%%%%%%%%%%%%%%%%%%%%%%%%%%%

Since the \Prog{gcb\_broker} and \Prog{gcb\_relay\_server} are not
regular Condor daemons, they do not read the Condor configuration
files.
Therefore, they must be configured by other means, namely the
environment and through the use of command-line arguments.
Luckily, there are very few things to control with these daemons, so
the fact they can't use the Condor configuration files is not a major
limitation.

There is one required command-line argument for the \Prog{gcb\_broker}.
This argument defines the public IP address this broker will use to
represent itself and any private network nodes that are configured to
use this broker.
This information is defined with \Opt{-i xxx.xxx.xxx.xxx} on the
command-line when the \Prog{gcb\_broker} is executed.
If the broker is being setup outside the private network, it's likely
that the machine will only have one IP address, which is obviously the
one to use.
However, if the broker is being run on a machine directly on the
network boundary (a multi-homed machine with interfaces into both the
private and public networks), be sure to use the IP address of the
interface on the public network.

Additionally, you can specify some environment variables to control
how the broker (and relay\_server processes it spawns) will behave.
Some of these settings can also be specified as command-line
arguments to the \Prog{gcb\_broker}.
All of them have reasonable defaults if not defined.

\begin{itemize}

\item General daemon behavior

\begin{description}

\item \Env{GCB\_RELAY\_SERVER} \label{Env:GCB-relay-server}
  Full path to the \File{gcb\_relay\_server} binary your broker should
  use.
  The command-line override for this is \Opt{-r /full/path/to/relayserver}.
  If this is not set either on the command-line or in the environment,
  the \Prog{gcb\_broker} process will search for a program named
  \File{gcb\_relay\_server} in the same directory where the
  \File{gcb\_broker} binary is located and attempt to use that.

\item \Env{GCB\_ACTIVE\_TO\_CLIENT} \label{Env:GCB-active-to-client}
  Boolean that defines whether GCB broker can directly talk to servers
  running inside the network that it manages (\verb@yes@ or \verb@no@,
  case sensitive).
  \Env{GCB\_ACTIVE\_TO\_CLIENT} should be set to \verb@yes@ only if
  this GCB broker is running on a network boundary and can connect to
  both the private and public nodes.
  If the broker is running in the public network, it should be left
  undefined or set to \verb@no@.

\end{description}

\item Log file locations

\begin{description}

\item \Env{GCB\_LOG\_DIR} \label{Env:GCB-log-dir}
  Directory to use for all GCB-related log files.
  If this is set and the per-daemon log file settings (described
  below) are not defined, the broker will write to
  \verb@$GCB_LOG_DIR/BrokerLog@ and the relay server will write to
  \verb@$GCB_LOG_DIR/RelayServerLog.<pid>@

\item \Env{GCB\_BROKER\_LOG} \label{Env:GCB-broker-log}
  The full path for the GCB broker's log file.
  The command-line override for this is \Opt{-l /full/path/to/log/file}.
  This setting overrides \Env{GCB\_LOG\_DIR} (described above).

\item \Env{GCB\_RELAY\_SERVER\_LOG} \label{Env:GCB-relay-server-log}
  The full path for the GCB relay server's log file.
  Each relay server writes its own log file, so the actual filename
  will be: \verb@$GCB_RELAY_SERVER_LOG.<pid>@ where \verb@<pid>@ is
  replaced with the process id of the corresponding
  \Prog{gcb\_relay\_server}.
  This setting overrides \Env{GCB\_LOG\_DIR} (described above) if
  defined.

\end{description}

\item Verbose logging 

\begin{description}

\item \Env{GCB\_DEBUG\_LEVEL} \label{Env:GCB-DEBUG-LEVEL}
  Controls how verbose all the GCB daemon's log files should be.
  Can be either \verb@fulldebug@ (more verbose) or \verb@basic@.
  This setting defines logging behavior for all GCB daemons, unless
  the following daemon-specific settings are defined.

\item \Env{GCB\_BROKER\_DEBUG} \label{Env:GCB-BROKER-DEBUG}
  Controls verbose logging for just the GCB broker.
  The command-line override for this is \Opt{-d level}.
  Overrides \Env{GCB\_DEBUG\_LEVEL}. 

\item \Env{GCB\_RELAY\_SERVER\_DEBUG} \label{Env:GCB-RELAY-SERVER-DEBUG}
  Controls verbose logging for just the GCB relay server.  
  Overrides \Env{GCB\_DEBUG\_LEVEL}. 

\end{description}

\item Maximum log file size

\begin{description}

\item \Env{GCB\_MAX\_LOG} \label{Env:GCB-MAX-LOG}
  The maximum size of all GCB log files.
  When the log file reaches this size, the content of the file will be
  moved to \File{filename.old} and a new log will be started.
  This setting defines logging behavior for all GCB daemons, unless
  the following daemon-specific settings are used.

\item \Env{GCB\_BROKER\_MAX\_LOG} \label{Env:GCB-BROKER-MAX-LOG}
  The maximum size of the GCB broker log file.

\item \Env{GCB\_RELAY\_SERVER\_MAX\_LOG} \label{Env:GCB-RELAY-SERVER-MAX-LOG}
  The maximum size of the GCB relay server log file.

\end{description}

\end{itemize}

%%%%%%%%%%%%%%%%%%%%%%%%%%%%%%%%%%%%%%%%%%%%%%%%%%
\subsubsection{\label{sec:GCB-condor-master-spawn}
Using \Condor{master} to spawn the GCB Broker}
%%%%%%%%%%%%%%%%%%%%%%%%%%%%%%%%%%%%%%%%%%%%%%%%%%

If you wish to spawn the GCB broker with a \Condor{master}, here are 
the recommended \File{condor\_config} settings that will work:

\footnotesize
\begin{verbatim}
# Specify that you only want the master and the broker running
DAEMON_LIST = MASTER, GCB_BROKER

# Define the path to the broker binary for the master to spawn
GCB_BROKER = $(RELEASE_DIR)/libexec/gcb_broker

# Define the path to the release_server binary for the broker to use 
GCB_RELAY = $(RELEASE_DIR)/libexec/gcb_relay_server

# Setup the gcb_broker's environment.  We use a macro to build up the
# environment we want in pieces, and then finally define
# GCB_BROKER_ENVIRONMENT, the setting that condor_master uses.

# Initialize an empty macro
GCB_BROKER_ENV =

# (recommended) Provide the full path to the gcb_relay_server
GCB_BROKER_ENV = $(GCB_BROKER_ENV);GCB_RELAY_SERVER=$(GCB_RELAY)

# (recommended) Tell GCB to write all log files into the Condor log
# directory (the directory used by the condor_master itself)
GCB_BROKER_ENV = $(GCB_BROKER_ENV);GCB_LOG_DIR=$(LOG)
# Or, you can specify a log file seperately for each GCB daemon:
#GCB_BROKER_ENV = $(GCB_BROKER_ENV);GCB_BROKER_LOG=$(LOG)/GCB_Broker_Log
#GCB_BROKER_ENV = $(GCB_BROKER_ENV);GCB_RELAY_SERVER_LOG=$(LOG)/GCB_RS_Log

# (optional -- only set if true) Tell the GCB broker that it can
# directly connect to machines in the private network which it is
# handling communication for.  This should only be enabled if the GCB
# broker is running directly on a network boundry and can open direct
# connections to the private nodes.
#GCB_BROKER_ENV = $(GCB_BROKER_ENV);GCB_ACTIVE_TO_CLIENT=yes

# (optional) turn on verbose logging for all of GCB
#GCB_BROKER_ENV = $(GCB_BROKER_ENV);GCB_DEBUG_LEVEL=fulldebug
# Or, you can turn this on seperately for each GCB daemon:
#GCB_BROKER_ENV = $(GCB_BROKER_ENV);GCB_BROKER_DEBUG=fulldebug
#GCB_BROKER_ENV = $(GCB_BROKER_ENV);GCB_RELAY_SERVER_DEBUG=fulldebug

# (optional) specify the maximum log file size (in bytes)
#GCB_BROKER_ENV = $(GCB_BROKER_ENV);GCB_MAX_LOG=640000
# Or, you can define this seperately for each GCB daemon:
#GCB_BROKER_ENV = $(GCB_BROKER_ENV);GCB_BROKER_MAX_LOG=640000
#GCB_BROKER_ENV = $(GCB_BROKER_ENV);GCB_RELAY_SERVER_MAX_LOG=640000

# Finally, set the value the condor_master really uses
GCB_BROKER_ENVIRONMENT = $(GCB_BROKER_ENV)

# If your Condor installation on this host already has a public
# interface as the default (either because it is the first interface
# listed in this machine's host entry, or because you've already
# defined NETWORK_INTERFACE), you can just use Condor's special macro
# that holds the IP address for this.
GCB_BROKER_IP = $(ip_address)
# Otherwise, you could define it yourself with your real public IP:
# GCB_BROKER_IP = 123.123.123.123

# (required) define the command-line arguments for the broker 
GCB_BROKER_ARGS = -i $(GCB_BROKER_IP)
\end{verbatim}
\normalsize

Once those settings are in place, you can either spawn or restart the
\Condor{master} and the \Prog{gcb\_broker} should be started.
You can ensure the broker is running by reading the log file you
specified with \Env{GCB\_BROKER\_LOG}, or in
\File{\MacroUNI{LOG}/BrokerLog} if you're using the default.


%%%%%%%%%%%%%%%%%%%%%%%%%%%%%%%%%%%%%%%%%%%%%%%%%%
\subsubsection{\label{sec:GCB-initd-spawn}
Using \Prog{initd} to spawn the GCB broker}
%%%%%%%%%%%%%%%%%%%%%%%%%%%%%%%%%%%%%%%%%%%%%%%%%%

If you have root access and do not wish to install and use a
\Condor{master} on your broker node, you can use the system's initd to
manage the \Prog{gcb\_broker} for you.
Generally, this involves adding a line to your \File{/etc/inittab}
file.
However, some sites use other means to manage and generate the
\File{/etc/inittab}, such as \Prog{cfengine} or other system configuration
management tools, so check with your local system administrator (if
that's not you) to be sure.
An example line might be something like:

\footnotesize
\begin{verbatim}
GB:23:respawn:/path/to/gcb_broker -i 123.123.123.123 -r /path/to/relay_server
\end{verbatim}
\normalsize

You might also find it's easier to wrap the \Prog{gcb\_broker} binary
in a shell script, so you can change the command-line arguments (and
set environment variables) without having to edit \File{/etc/inittab}
all the time.
Something like:

\footnotesize
\begin{verbatim}
GB:23:respawn:/opt/condor-6.7.13/libexec/gcb_broker.sh
\end{verbatim}
\normalsize

Then, you'd just have to create the wrapper, something like: 

\footnotesize
\begin{verbatim}
#!/bin/sh

libexec=/opt/condor-6.7.13/libexec
ip=123.123.123.123
relay=$libexec/gcb_relay_server

exec $libexec/gcb_broker -i $ip -r $relay
\end{verbatim}
\normalsize

You'll probably also want to set some environment variables to tell
the GCB daemons where to write their log files (\Env{GCB\_LOG\_DIR}),
and possibly some of the other variables described above.

Either way, once you've updated the \File{/etc/inittab}, you just send
the \Prog{initd} process (always PID 1) a \verb@SIGHUP@ signal and it
will re-read the \File{inittab} and spawn your \Prog{gcb\_broker}.


%%%%%%%%%%%%%%%%%%%%%%%%%%%%%%%%%%%%%%%%%%%%%%%%%%%%%%%%%%%%%%%%%%%%%%
\subsubsection{\label{sec:GCB-condor-config}
Configuring Condor nodes to be GCB clients}
%%%%%%%%%%%%%%%%%%%%%%%%%%%%%%%%%%%%%%%%%%%%%%%%%%%%%%%%%%%%%%%%%%%%%%

In general, before configuring any Condor node to use GCB, the GCB
broker node(s) for your pool must be setup and running.
So, if you haven't already, go back and read the sections describing
the broker, how to configure it, and how to spawn it.

To enable GCB on a given Condor host, you must set the following
Condor configuration attributes:

\footnotesize
\begin{verbatim}
# Tell Condor to use a network remapping service (currently only GCB
# is supported, but in the future, there might be other options)
NET_REMAP_ENABLE = true
NET_REMAP_SERVICE = GCB
\end{verbatim}
\normalsize

Only GCB clients inside a private network need to define the following
setting, which specifies the IP address of the broker serving this
network.
Note that this IP must be the same as the IP that was specified on the
broker's command-line with the \Opt{-i} option.

\footnotesize
\begin{verbatim}
# Public IP address (in standard dot notation) of the GCB broker
# serving this private node.
NET_REMAP_INAGENT = xxx.xxx.xxx.xxx
\end{verbatim}
\normalsize

Obviously, because the \MacroNI{NET\_REMAP\_INAGENT} setting is only
valid on private nodes, it should not be defined in a global
\File{condor\_config} file.
Furthermore, if you have a large number of hosts in a given private
network, and you choose to run multiple brokers to alleviate the
scalability issues, each subset of your private nodes that uses a
specific broker will need a different value for this setting.

Finally, if you choose to setup a GCB routing table (which is
recommended but optional, and described below in a separate section),
you must tell Condor daemons where to find their table.
You do so by defining the following setting:

\footnotesize
\begin{verbatim}
# The full path to the routing table used by GCB
NET_REMAP_ROUTE = /full/path/to/GCB-routing-table
\end{verbatim}
\normalsize

Once the \MacroNI{NET\_REMAP\_ENABLE} setting (described above) is
defined, the \Macro{BIND\_ALL\_INTERFACES} setting is set
automatically.
% TODO: add reference to other section here
It would not hurt to put the following in your config file near the
other GCB-related settings, just so you remember this is happening:

\footnotesize
\begin{verbatim}
# Tell Condor to bind to all network interfaces, instead of a single
# interface.
BIND_ALL_INTERFACES = true
\end{verbatim}
\normalsize

Once a GCB broker is setup and running to manage connections for a
each private network, and the Condor installation for all the nodes in
either private and public networks are configured to enable GCB, you
can restart your Condor daemons and all of the different machines
should be able to communicate with each other.


%%%%%%%%%%%%%%%%%%%%%%%%%%%%%%%%%%%%%%%%%%%%%%%%%%%%%%%%%%%%%%%%%%%%%%
\subsubsection{\label{sec:GCB-routing-table}Configuring the GCB
  routing table} 
%%%%%%%%%%%%%%%%%%%%%%%%%%%%%%%%%%%%%%%%%%%%%%%%%%%%%%%%%%%%%%%%%%%%%%

By default, a GCB-enabled application will always attempt to directly
connect to a given IP/port pair.
In the case of private nodes being represented by a GCB broker, the
IP/port will be a proxy socket on the broker node, not the real
address of the private node.
When the GCB broker receives a direct connection to one of its proxy 
sockets, it will notify the corresponding private node, which will
establish a new connection to the broker.
The broker will then forward packets between these two sockets,
establishing a communication pathway into the private node.
This allows even clients which are not linked with GCB to communicate
through GCB to private nodes.

However, this mechanism is expensive and unnecessary in the case of
GCB-aware clients trying to connect to private nodes that can directly
communicate with the public host.
The alternative is to contact the GCB broker's command interface (the
fixed port where it is listening for GCB management commands), and use
a GCB-specific protocol to request a connection to the given IP/port.
In this case, the GCB broker will notify the private node to directly
connect to the public client (technically, to a new socket created by
the GCB client library linked in with the client's application), and a
direct socket between the two is established, removing the need for
packet forwarding between the proxy sockets at the GCB broker.

On the other hand, in cases where a direct connection from the client
to a given server is possible (e.g. two GCB-aware clients in the same
public network attempted to communicate with each other), it is
expensive and unnecessary to attempt to contact a GCB broker, and the
client should connect directly.

To allow a GCB-enabled client to know if it should make a direct
connection (which might involve packet forwarding through proxy
sockets), or if it should use the GCB protocol to communicate with the
broker's command port and arrange a direct socket,
GCB provides a \Term{routing table}.
Using this table, an administrator can define what IP addresses should
be considered private nodes where the GCB connection protocol will be
used, and what nodes are public, where a direct connection (without
incurring the latency of contacting the GCB broker, only to find out
there's no information about the given IP/port) should be made
immediately. 

If the attempt to contact the GCB broker for a given IP/port fails, or
if the desired port is not being managed by the broker, the GCB client
library making the connection will fallback and attempt a direct
connection.
Therefore, configuring a GCB routing table is not required for
communication to work within a GCB-enabled environment.
However, the GCB routing table can significantly improve performance
for communication with private nodes being represented by a GCB
broker. 

One confusing aspect of GCB is that all of the nodes on a private
network will think their IP address is the address of their GCB
broker.
Therefore, all the Condor daemons on a private network will advertise
themselves with the same IP address (though the broker will map the
different ports to different nodes within the private network).
Therefore, a given node in the public network needs to be told that if
it is contacting this IP address, it should know that the IP is really
a GCB broker representing a node in the private network, so that it
can contact the broker to arrange a single socket from the private
node to the public one, instead of relying on forwarding packets
between proxy sockets at the broker.
However, any other addresses (for example, other public IPs) can be
contacted directly, without going through a GCB broker.
Similarly, other nodes within the same private network will still be
advertising their address with their GCB broker's public IP.
So, nodes within the same private network also have to know that the
public IP of the broker is really a GCB broker, yet all other public
IPs are valid for direct communication.

In general, all connections can be made directly, except to a host
represented by a GCB broker.
Furthermore, the default behavior of the GCB client library is to make
a direct connection.
So, the routing table is just a (somewhat complicated) way to tell a
given GCB installation what GCB brokers it might have to communicate
with, and that it should directly communicate with anything else.
In practice, the routing table should just have a single entry for
each GCB broker in your system.
Future versions of GCB will be able to make use of more complicated
routing behavior, which is why the full routing table infrastructure
described below is implemented, even if the current version of GCB is
not taking advantage of all of it.


%%%%%%%%%%%%%%%%%%%%%%%%%%%%%%%%%%%%%%%%%%%%%%%%%%%%%%%%%%%%%%%%%%%%%%
\Bold{Format of the GCB routing table}
%%%%%%%%%%%%%%%%%%%%%%%%%%%%%%%%%%%%%%%%%%%%%%%%%%%%%%%%%%%%%%%%%%%%%%

The routing table should be stored in a plain ASCII text file.
Each line of the file contains one rule.
Each rule consists of a \Term{target} and a \Term{method}.
The target specifies destination IP address(es) to match and method
defines what mechanism must be used to connect to the given target.
The target must be specified as a valid IP string in the standard
dotted notation, ``/'', and an integer \Term{mask}.
The mask specifies how many bits of a given destination IP and the IP
of the target must match.
Method must be either \verb@GCB@ or \verb@direct@.
GCB stops searching the table as soon as it finds a matching rule,
therefore you must put specific rules before generic ones.
The default if no rule is matched is to use direct communication.
Some examples and the corresponding routing tables you would use will
hopefully make this syntax more clear.


%%%%%%%%%%%%%%%%%%%%%%%%%%%%%%%%%%%%%%%%%%%%%%%%%%%%%%%%%%%%%%%%%%%%%%
\Bold{Simple GCB routing table example (1 private, 1 public)}
%%%%%%%%%%%%%%%%%%%%%%%%%%%%%%%%%%%%%%%%%%%%%%%%%%%%%%%%%%%%%%%%%%%%%%

Say you have a private network with a bunch of nodes whose IP
addresses are \verb@192.168.2.*@, some nodes in a public network 
whose IP addresses are \verb@123.123.123.*@, and a broker for the 192
network running on \verb@123.123.123.123@.
In that case, the routing table for both the public and private nodes
should be:

\begin{verbatim}
123.123.123.123/32 GCB
\end{verbatim}

The rule says that for IP addresses where all 32 bits exactly match
the address \verb@123.123.123.123@ first communicate with GCB broker.

Since the default is to directly connect if no rule in the routing
table matches a given target IP, this single rule is all that is
required.
However, to illustrate how the routing table syntax works, the
following routing table is equivalent:

\begin{verbatim}
123.123.123.123/32 GCB
*/0 direct
\end{verbatim}

Any attempt to connect to \verb@123.123.123.123@ will still use GCB
(since that's the first rule in the file), but all other IP addresses
will connect directly.
This table explicitly defines GCB's default behavior.

%%%%%%%%%%%%%%%%%%%%%%%%%%%%%%%%%%%%%%%%%%%%%%%%%%%%%%%%%%%%%%%%%%%%%%
\Bold{More complex GCB routing table example (2 private, 1 public)}
%%%%%%%%%%%%%%%%%%%%%%%%%%%%%%%%%%%%%%%%%%%%%%%%%%%%%%%%%%%%%%%%%%%%%%

Here's a slightly more complicated case: a single Condor pool that
spans a public network and 2 private networks.
Say you have two separate private networks, each where the machines
have private addresses like \verb@192.168.2.*@.
Let's call one of these private networks \verb@A@ and the other one
\verb@B@. 
You've still got a public network and nodes with IP addresses like
\verb@123.123.123.*@.
The GCB broker for nodes in the \verb@A@ network will be at
\verb@123.123.123.65@, and the broker for the \verb@B@ nodes will be
at \verb@123.123.123.66@.
All of the nodes want to be able to talk to each other.
In this case, nodes in the \verb@A@ private network advertise
themselves as \verb@123.123.123.65@, so any node, regardless of being
in A, B, or the public network, must treat that IP as a GCB broker.
Similarly, nodes in the \verb@B@ network advertise themselves as
\verb@123.123.123.66@, so any node, in any network, needs to treat
that as a GCB broker (which it is).
All other connections from any node can be made directly.
Therefore, here is the appropriate routing table for all nodes:

\begin{verbatim}
123.123.123.65/32 GCB
123.123.123.66/32 GCB
\end{verbatim}


%%%%%%%%%%%%%%%%%%%%%%%%%%%%%%%%%%%%%%%%%%%%%%%%%%%%%%%%%%%%%%%%%%%%%%
\subsubsection{\label{sec:GCB-host-security-implications}Implications
of Using GCB for Condor's Host/IP Based Security} 
%%%%%%%%%%%%%%%%%%%%%%%%%%%%%%%%%%%%%%%%%%%%%%%%%%%%%%%%%%%%%%%%%%%%%%

Whenever a connection comes into a Condor daemon's command socket,
Condor will either allow or refuse the command depending on the IP
address of the incoming socket.
For more information about this host-based security in Condor, see
section~\ref{sec:Host-Security} on page~\pageref{sec:Host-Security}.
Because of the way GCB changes the IP addresses that are used and
advertised by GCB-enabled clients, and since all nodes being
represented by a GCB broker are represented by different ports on the
broker node (a process known as \Term{address leasing}), using GCB has
implications for this authorization process.

Depending on the communication pathway used by a GCB-enabled Condor
client (either a tool or another Condor daemon) to connect to a given
Condor server daemon, and where in the network each side of the
connection resides, the IP address of the resulting socket actually
used will be very different.

In the case of a private client attempting to connect to a server,
there are three possibilities:
\begin{itemize}

  \item Direct connection to another node within the private network:
  the server will see the private IP address of the client.

  \item Direct outbound connection to a public node: if NAT is being
  used, the server will see the IP of the NAT server for the private
  network.
  If there's no NAT, and the firewall is just blocking connections in
  one direction but not re-writing IP addresses, the server will see
  the client's real private IP address.

  \item Connection to a host in a different private network that must
  be relayed through the GCB broker: the server will see the IP
  address of the broker representing the server.
  This is just another instance of the private server case which is
  described below in more detail.

\end{itemize}

Therefore, any public server that wants to allow a command from a
given client must have any or all of the various IP addresses
mentioned above in the appropriate \MacroNI{HOSTALLOW} settings.  In
practice, that means opening up the \MacroNI{HOSTALLOW} settings to
include not just the actual IP addresses of each node, but also the IP
address of the various GCB brokers in use, and potentially, the public
IP address of the NAT host for each private network.

However, given that all private nodes which are represented by a
given GCB broker could potentially make connections to any other
host using the GCB broker's IP address (whenever proxy socket
forwarding is being used), if a single private node is being granted
a certain level of permission in your pool, all of the private nodes
using the same broker will have the same level of permission.
This is particularly important if you attempt to grant
\Macro{HOSTALLOW\_ADMINISTRATOR} or \Macro{HOSTALLOW\_CONFIG}
privileges to a private node represented by a GCB broker.

In the case of a public client attempting to connect to a private
server, there are only two possible cases:

\begin{itemize}

  \item GCB broker can arrange a direct socket from the private node:
  the private server will see the real public IP of the client.

  \item GCB broker must forward packets from a proxy socket (because
  of a non-GCB aware public client, a misconfigured or missing GCB
  routing table, or a client in a different private network): the
  private server will see the IP address of its own GCB broker.
  In the case where the GCB broker is running directly on the network
  boundry, the private server will see the broker's private IP
  address (even if the broker is also listening on the public
  interface and the leased addresses it provides use the public IP). 
  If the broker is running entirely in the public network and cannot
  directly connect to the private nodes, the private server will see
  the remote connection as coming from the broker's public IP
  address.

\end{itemize}

This second case is particularly troubling.
Since there are legitimate circumstances where a private server would
need to use a forwarded proxy socket from its GCB broker, in general,
the server should allow requests originating from its GCB broker.
But, precisely because of the proxy forwarding, that means any client
on Earth that can connect to the GCB broker would be allowed into the
private server if IP-based authorization was the only defense.

The final host-based security setting that requires special mention is
\Macro{HOSTALLOW\_NEGOTIATOR}.
If the \Condor{negotiator} for your pool is running on a private node
being represented by a GCB broker, you must make a couple of
modifications to the default value.
For the purposes of Condor's host-based security, the
\Condor{negotiator} acts as a client when communicating with each 
\Condor{schedd} in your pool which has idle jobs that need to be
matched with available resources.
Therefore, all the possible cases of a private client attempting to
connect to a given server apply to a private \Condor{negotiator}.
In practice, that means adding the public IP of the broker, the real
private IP of the negotiator host, and possibly the the public IP of
the NAT host for this private network, to your
\MacroNI{HOSTALLOW\_NEGOTIATOR} setting.
Unfortunately, that means *any* host behind the same NAT host or using
the same GCB broker will be authorized as if it was your
\Condor{negotiator}. 

Future versions of GCB and Condor will hopefully add some form of
authentication and authorization to the GCB broker itself, to help
alleviate these problems.
Until then, sites using GCB are encouraged to use GSI strong
authentication (since Kerberos also depends on IP addresses and is
therefore incompatible with GCB) to rely on an authorization system
that isn't effected by address leasing.
This is especially true for sites that (foolishly) choose to run their
central manager on a private node.


%%%%%%%%%%%%%%%%%%%%%%%%%%%%%%%%%%%%%%%%%%%%%%%%%%%%%%%%%%%%%%%%%%%%%%
\subsubsection{\label{sec:GCB-config-implications}Implications of
Using GCB for Other Condor Configuration Settings} 
%%%%%%%%%%%%%%%%%%%%%%%%%%%%%%%%%%%%%%%%%%%%%%%%%%%%%%%%%%%%%%%%%%%%%%

Using GCB and address leasing has implications for a few other Condor
configuration settings.
Each one is described below. 

\begin{description}

% i'm intentionally using \Macro, not \MacroNI for each of these,
% since these should all show up in the index for each setting...

\item[\Macro{COLLECTOR\_HOST}]
  If the \Condor{collector} for your pool is running on a private node
  being represented by a GCB broker, the \MacroNI{COLLECTOR\_HOST}
  must be set to the host name or IP address of the GCB broker machine,
  NOT the real host name/IP of the private node where the daemons are
  actually running.
  When the \Condor{collector} on the private node attempts to
  \Syscall{bind} to its command port (9618 by default), it will
  request port 9618 on the GCB broker node, instead.
  So, you do not need to worry about the port, but you definitely have
  to worry about the host name or IP address.
  When public nodes want to communicate with the \Condor{collector},
  they must go through the GCB broker, anyway.
  In theory, other nodes inside the same private network could be told
  to directly use the private IP of the collector host, but that's
  unnecessary and would probably lead to other confusion and
  configuration problems.

  However, because the \Condor{collector} is listening on a fixed
  port, and that single port is reserved on the GCB broker node, no
  two private nodes using the same broker can attempt to use the same
  port for their \Condor{collector}.
  Therefore, any site that is attempting to setup multiple pools
  within the same private network is strongly encouraged to setup
  separate GCB brokers for each pool.
  Otherwise, they must configure one or both of the pools to use a
  non-standard port for their \Condor{collector} which adds yet more
  complication to an already complicated situation. 

\item[\Macro{CKPT\_SERVER\_HOST}]
  Much like the case for \MacroNI{COLLECTOR\_HOST} described above,
  a checkpoint server on a private node will have to lease a port on
  the GCB broker node.
  However, the checkpoint server also uses a fixed port, and unlike
  the \Condor{collector}, there's no way to configure an alternate
  value.
  Therefore, only a single checkpoint server can be run behind a given
  GCB broker.
  Once again, if multiple checkpoint servers are required, multiple
  GCB brokers should be deployed and configured.
  Furthermore, the host name of the GCB broker should be used as the
  value for \MacroNI{CKPT\_SERVER\_HOST}, not the real IP or host name
  of the private node where the \Condor{ckpt\_server} is running.

\item[\Macro{SEC\_DEFAULT\_AUTHENTICATION\_METHODS}]
  Once you enable GCB for your pool, you can no longer use
  \verb@KERBEROS@ as one of your authentication methods.
  As previously discussed, the IP addresses used in various
  circumstances will not be the real IP addresses of your machines.
  However, Kerberos stores the IP address of each host as part of the
  Kerberos ticket, and if a connection attempts to authenticate with
  Kerberos and the two IP addresses (the underlying socket and the one
  listed in the ticket) don't match, Kerberos will refuse to
  authenticate.

\end{description}

Due to the complications and security limitations that arise from
running a central manager on a private node represented by GCB (both
regarding the \MacroNI{COLLECTOR\_HOST} just described, and
\MacroNI{HOSTALLOW\_NEGOTIATOR} from the previous section), we
recommend that sites avoid locating a central manager on a private
host if at all possible.

