%%%%%%%%%%%%%%%%%%%%%%%%%%%%%%%%%%%%%%%%%%%%%%%%%%%%%%%%%%%%%%%%%%%%%%%%%%%
\subsection{\label{sec:HTCondor-AFS}Using HTCondor with AFS}
%%%%%%%%%%%%%%%%%%%%%%%%%%%%%%%%%%%%%%%%%%%%%%%%%%%%%%%%%%%%%%%%%%%%%%%%%%%
\index{file system!AFS}

Configuration variables that allow machines to interact with and
use a shared file system are given at
section~\ref{sec:Shared-Filesystem-Config-File-Entries}.

Limitations with AFS occur because
 HTCondor does not currently have a way to authenticate itself to AFS.
This is true of the HTCondor daemons that would like to authenticate as
the AFS user \Login{condor}, and of the \Condor{shadow} which would like to
authenticate as the user who submitted the job it is serving.
Since neither of these things can happen yet, there are special
things to do when interacting with AFS.
Some of this must be done by the administrator(s) installing HTCondor.
Other things must be done by HTCondor users who submit jobs.

%%%%%%%%%%%%%%%%%%%%%%%%%%%%%%%%%%%%%%%%%%%%%%%%%%%%%%%%%%%%%%%%%%%%%%%%%%%
\subsubsection{\label{sec:HTCondor-AFS-Admin}AFS and HTCondor for Administrators}
%%%%%%%%%%%%%%%%%%%%%%%%%%%%%%%%%%%%%%%%%%%%%%%%%%%%%%%%%%%%%%%%%%%%%%%%%%%

The largest result from the lack of authentication with AFS is that
the directory defined by the configuration variable \MacroNI{LOCAL\_DIR}
and its subdirectories \File{log} and \File{spool} on each machine
must be either writable to unauthenticated users, or must not be on AFS.
Making these directories writable a \emph{very} bad security hole,
so it is \emph{not} a viable solution.
Placing \MacroNI{LOCAL\_DIR} onto NFS is acceptable.
To avoid AFS, place the directory defined for \MacroNI{LOCAL\_DIR} on
a local partition on each machine in the pool.
This implies running \Condor{configure} to install the release directory and
configure the pool,
setting the \MacroNI{LOCAL\_DIR} variable to a local partition.
When that is complete, log into each machine in the pool,
and run \Condor{init} to set up the local HTCondor directory.

The directory defined by \MacroNI{RELEASE\_DIR},
which holds all the HTCondor binaries,
libraries, and scripts, can be on AFS.
None of the HTCondor daemons need to write to these files.
They only need to read them.
So, the directory defined by \MacroNI{RELEASE\_DIR} only needs to be
world readable in order to let HTCondor function.
This makes it easier to
upgrade the binaries to a newer version at a later date,
and means that users can find the HTCondor tools in a consistent location
on all the machines in the pool. 
Also, the HTCondor configuration files may be placed in a centralized location.
This is what we do for the UW-Madison's CS department HTCondor pool,
and it works quite well.

Finally, consider setting up some targeted AFS groups to help 
users deal with HTCondor and AFS better.
This is discussed in the following manual subsection.
In short, create an AFS group that
contains all users, authenticated or not,
but which is restricted to a given host or subnet.
These should be made as host-based ACLs with AFS,
but here at UW-Madison, we have had some trouble getting that working.
Instead, we have a special group for all machines in our department.
The users here are required to make their output
directories on AFS writable to any process running on any of our
machines, instead of any process on any machine with AFS on the
Internet.

%%%%%%%%%%%%%%%%%%%%%%%%%%%%%%%%%%%%%%%%%%%%%%%%%%%%%%%%%%%%%%%%%%%%%%%%%%%
\subsubsection{\label{sec:HTCondor-AFS-Users}AFS and HTCondor for Users}
%%%%%%%%%%%%%%%%%%%%%%%%%%%%%%%%%%%%%%%%%%%%%%%%%%%%%%%%%%%%%%%%%%%%%%%%%%%

The \Condor{shadow} daemon runs on the machine where jobs are submitted.
It performs all file system access on behalf of the jobs.
Because the \Condor{shadow} daemon is not authenticated to AFS
as the user who submitted the job, the \Condor{shadow} daemon
will not normally be able to write any output.
Therefore the directories in which the job will be creating output files
will need to be world writable; they need to be writable by
non-authenticated AFS users.
In addition, the program's \File{stdout}, \File{stderr}, log file,
and any file the program explicitly opens
will need to be in a directory that is world-writable.

An administrator may be able to set up special AFS groups that can make 
unauthenticated access to the program's files less scary.
For example, there is
supposed to be a way for AFS to grant access to any unauthenticated
process on a given host. 
If set up,
write access need only be granted to unauthenticated processes 
on the submit machine,
as opposed to any unauthenticated process on the Internet.
Similarly,
unauthenticated read access could be granted only to processes running
on the submit machine.

A solution to this problem is to not use AFS for output files.
If disk space on the submit machine is available in a partition not on AFS,
submit the jobs from there.
While the \Condor{shadow} daemon is not authenticated to AFS,
it does run with the effective UID of the user who submitted the jobs.
So, on a local (or NFS) file system,
the \Condor{shadow} daemon will be able to access the files,
and no special permissions need be granted to anyone other than the job
submitter.
If the HTCondor daemons are not invoked as root however,
the \Condor{shadow} daemon will not be able to run with the submitter's
effective UID, leading to
a similar problem as with files on AFS.

