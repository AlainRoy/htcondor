%%%%%%%%%%%%%%%%%%%%%%%%%%%%%%%%%%%%%%%%%%%%%%%%%%%%%%%%%%%%%%%%%%%%%%
\subsection{\label{sec:Contrib-CondorView-Install}
Configuring The CondorView Server}
%%%%%%%%%%%%%%%%%%%%%%%%%%%%%%%%%%%%%%%%%%%%%%%%%%%%%%%%%%%%%%%%%%%%%%

\index{CondorView!Server}
The CondorView server is an alternate use of the
\Condor{collector}
that logs information on disk, providing a 
persistent, historical database of pool state.
This includes machine state, as well as the state of jobs submitted by
users.

An existing \Condor{collector} may act as the
CondorView collector through configuration.  
This is the simplest situation, because the only change
needed is to turn on the logging of historical information.
The alternative of configuring a new \Condor{collector} to act as the
CondorView collector is slightly more complicated,
while it offers the
advantage that the same CondorView collector may be used
for several pools as desired, to aggregate information into one place.

The following sections describe how to configure a machine to run a
CondorView server and to configure a pool to send updates to it. 


%%%%%%%%%%%%%%%%%%%%%%%%%%%%%%%%%%%%%%%%%%%%%%%%%%%%%%%%%%%%%%%%%%%%%%
\subsubsection{\label{sec:CondorView-Server-Setup}
Configuring a Machine to be a CondorView Server} 
%%%%%%%%%%%%%%%%%%%%%%%%%%%%%%%%%%%%%%%%%%%%%%%%%%%%%%%%%%%%%%%%%%%%%%

\index{CondorView!configuration}

To configure the CondorView collector, a few configuration variables
are added or modified
for the \Condor{collector} chosen to act
as the CondorView collector.
These configuration variables are described in 
section~\ref{sec:Collector-Config-File-Entries} on
page~\pageref{sec:Collector-Config-File-Entries}.
Here are brief explanations of the entries that must be customized:

\begin{description}

\item[\Macro{POOL\_HISTORY\_DIR}] The directory where
historical data will be stored.
This directory must be writable by whatever user the CondorView
collector is running as (usually the user \Login{condor}).  
There is a configurable limit to the maximum space required for all
the files created by the CondorView server called
(\Macro{POOL\_HISTORY\_MAX\_STORAGE}). 

\Note This directory should be separate and different from the
\File{spool} or \File{log} directories already set up for
Condor.
There are a few problems putting these files into either of those
directories.

\item[\Macro{KEEP\_POOL\_HISTORY}] A boolean value that determines
if the CondorView collector should store the historical information.
It is \Expr{False} by default, and must be specified as \Expr{True} in
the local configuration file to enable data collection.

\end{description}

Once these settings are in place in the configuration file for the
CondorView server host, create the directory specified
in \MacroNI{POOL\_HISTORY\_DIR} and make it writable by the user the
CondorView collector is running as.
This is the same user that owns the \File{CollectorLog} file in
the \File{log} directory. The user is usually \Login{condor}.

If using the existing \Condor{collector} as the CondorView collector,
no further configuration is needed.  
To run a different
\Condor{collector} to act as the CondorView collector, configure
Condor to automatically start it.

If using a separate host for the CondorView collector,
to start it, add the value \Expr{COLLECTOR} to
\MacroNI{DAEMON\_LIST}, and restart Condor on that host.
To run the CondorView collector on the same host as another 
\Condor{collector},
ensure that the two \Condor{collector} daemons use different network ports.
Here is an example configuration in which the main \Condor{collector} and the
CondorView collector are started up by the same \Condor{master} daemon on
the same machine.  In this example, the CondorView collector uses
port 12345.

\footnotesize
\begin{verbatim}
  VIEW_SERVER = $(COLLECTOR)
  VIEW_SERVER_ARGS = -f -p 12345
  VIEW_SERVER_ENVIRONMENT = "_CONDOR_COLLECTOR_LOG=$(LOG)/ViewServerLog"
  DAEMON_LIST = MASTER, NEGOTIATOR, COLLECTOR, VIEW_SERVER
\end{verbatim}
\normalsize


For this change to take effect, restart the
\Condor{master} on this host.
This may be accomplished with the \Condor{restart} command,
if the command is run with
administrator access to the pool.


%%%%%%%%%%%%%%%%%%%%%%%%%%%%%%%%%%%%%%%%%%%%%%%%%%%%%%%%%%%%%%%%%%%%%%
\subsubsection{\label{sec:CondorView-Pool-Setup}
Configuring a Pool to Report to the CondorView Server} 
%%%%%%%%%%%%%%%%%%%%%%%%%%%%%%%%%%%%%%%%%%%%%%%%%%%%%%%%%%%%%%%%%%%%%%

For the CondorView server to function, configure the existing collector to
forward ClassAd updates to it.
This configuration is only necessary if 
the CondorView collector is a different collector from the existing
\Condor{collector} for the pool.
All the Condor daemons in the pool send their ClassAd updates to the
regular \Condor{collector}, which in turn will forward them on to the
CondorView server.

Define the following configuration variable:
\footnotesize
\begin{verbatim}
  CONDOR_VIEW_HOST = full.hostname[:portnumber]
\end{verbatim}
\normalsize
where \verb@full.hostname@ is the full host name of the machine 
running the CondorView collector.
The full host name is optionally followed by a colon and
port number.  This is only necessary if the CondorView
collector is configured to use a port number other than the default.

Place this setting in the configuration file used by the existing 
\Condor{collector}.
It is acceptable to place it in the global configuration file.  The
CondorView collector will ignore this setting (as it should) as it notices
that it is being asked to forward ClassAds to itself.

Once the CondorView server is running with this 
change, send a
\Condor{reconfig} command to the main \Condor{collector} for the change to
take effect, so it will begin forwarding updates.  
A query to the CondorView collector will verify that it is working.
A query example:

\footnotesize
\begin{verbatim}
  condor_status -pool condor.view.host[:portnumber]
\end{verbatim}
\normalsize


A \Condor{collector} may also be configured to report to multiple CondorView
servers.  The configuration variable \Macro{CONDOR\_VIEW\_HOST} can be
given as a list of CondorView servers separated by commas and/or spaces.

The following demonstrates an example configuration for two CondorView servers,
where both CondorView servers (and the \Condor{collector}) are running on the
same machine, localhost.localdomain:

\footnotesize
\begin{verbatim}
VIEWSERV01 = $(COLLECTOR)
VIEWSERV01_ARGS = -f -p 12345 -local-name VIEWSERV01
VIEWSERV01_ENVIRONMENT = "_CONDOR_COLLECTOR_LOG=$(LOG)/ViewServerLog01"
VIEWSERV01.POOL_HISTORY_DIR = $(LOCAL_DIR)/poolhist01
VIEWSERV01.KEEP_POOL_HISTORY = TRUE
VIEWSERV01.CONDOR_VIEW_HOST =

VIEWSERV02 = $(COLLECTOR)
VIEWSERV02_ARGS = -f -p 24680 -local-name VIEWSERV02
VIEWSERV02_ENVIRONMENT = "_CONDOR_COLLECTOR_LOG=$(LOG)/ViewServerLog02"
VIEWSERV02.POOL_HISTORY_DIR = $(LOCAL_DIR)/poolhist02
VIEWSERV02.KEEP_POOL_HISTORY = TRUE
VIEWSERV02.CONDOR_VIEW_HOST =

CONDOR_VIEW_HOST = localhost.localdomain:12345 localhost.localdomain:24680
DAEMON_LIST = $(DAEMON_LIST) VIEWSERV01 VIEWSERV02
\end{verbatim}
\normalsize

Note that the value of \Macro{CONDOR\_VIEW\_HOST} for VIEWSERV01 and VIEWSERV02
is unset, to prevent them from inheriting the global value of
\MacroNI{CONDOR\_VIEW\_HOST} and attempting to report to themselves 
or each other.  If the CondorView servers are running on different machines where
there is no global value for \MacroNI{CONDOR\_VIEW\_HOST}, this precaution
is not required.
