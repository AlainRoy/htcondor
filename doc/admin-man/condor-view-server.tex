%%%%%%%%%%%%%%%%%%%%%%%%%%%%%%%%%%%%%%%%%%%%%%%%%%%%%%%%%%%%%%%%%%%%%%
\subsection{\label{sec:Contrib-CondorView-Install}
Configuring The CondorView Server}
%%%%%%%%%%%%%%%%%%%%%%%%%%%%%%%%%%%%%%%%%%%%%%%%%%%%%%%%%%%%%%%%%%%%%%

The CondorView server is an alternate use of the
\Condor{collector}
\index{CondorView}
that logs information on disk, providing a 
persistent, historical database of pool state.
This includes machine state, as well as the state of jobs submitted by
users.
Historical information logging can be turned on or off, so you can
install the CondorView collector without using up disk space for
historical information if you don't want it.

The CondorView collector is a \Condor{collector} that has been specially 
configured \emph{and running on a different machine from the main
\Condor{collector}}. The pool must be configured to send updates to both
the normal collector and the CondorView collector.
Unfortunately, installing the CondorView collector on a separate host
generates more network traffic (from all the duplicate updates that
are sent from each machine in your pool to both collectors).

%%%%%%%%%%%%%%%%%%%%%%%%%%%%%%%%%%%%%%%%%%%%%%%%%%%%%%%%%%%%%%%%%%%%%%
\subsubsection{\label{sec:CondorView-Server-Setup}
Configuring a Machine to be a CondorView Server} 
%%%%%%%%%%%%%%%%%%%%%%%%%%%%%%%%%%%%%%%%%%%%%%%%%%%%%%%%%%%%%%%%%%%%%%

\index{CondorView!installation|(}
Before you configure the CondorView collector (as described in the
following sections), you have to add a few settings to the \emph{local}
configuration file of the chosen machine(that is not the main 
\Condor{collector} machine) to enable historical data collection.
These settings are described in detail in the Condor Version 6.1
Administrator's Manual, in the section ``\condor{collector} Config File
Entries''.
A short explanation of the entries you must customize is
provided below. 

\begin{description}

\item[\Macro{POOL\_HISTORY\_DIR}] This is the directory where
historical data will be stored.
This directory must be writable by whatever user the CondorView
collector is running as (usually the user condor).  
There is a configurable limit to the maximum space required for all
the files created by the CondorView server called
(\Macro{POOL\_HISTORY\_MAX\_STORAGE}). 

\Note This directory should be separate and different from the
\File{spool} or \File{log} directories already set up for
Condor.
There are a few problems putting these files into either of those
directories.

\item[\Macro{KEEP\_POOL\_HISTORY}] This is a boolean value that determines
if the CondorView collector should store the historical information.
It is false by default, which is why you must specify it as true in
your local configuration file to enable data collection.

\end{description}

Once these settings are in place in the local configuration file for your
CondorView server host, you must to create the directory you specified
in \Macro{POOL\_HISTORY\_DIR} and make it writable by the user your
CondorView collector is running as.
This is the same user that owns the \File{CollectorLog} file in
your \File{log} directory. The user is usually condor.

After you've configured the CondorView attributes, you must configure Condor
to automatically start and then begin reporting to the CondorView server.
You do this by adding \Expr{COLLECTOR} to the \Macro{DAEMON\_LIST} on
this machine and defining what \Expr{COLLECTOR} means.
For example:
\begin{verbatim}
        COLLECTOR = $(SBIN)/condor_collector
        DAEMON_LIST = MASTER, STARTD, SCHEDD, COLLECTOR
\end{verbatim}
For this change to take effect, you must re-start the
\Condor{master} on this host (which you can do with the
\Condor{restart} command, if you run the command from a machine with 
administrator access to your pool.
(See section~\ref{sec:Host-Security} on
page~\pageref{sec:Host-Security} for full details of IP/host-based
security in Condor).

As a last step, you tell all the machines in your pool to start sending
updates to both collectors.
You do this by specifying the following setting in your global
configuration file:
\begin{verbatim}
        CONDOR_VIEW_HOST = full.hostname
\end{verbatim}
where \verb@full.hostname@ is the full hostname of the machine where you
are running your CondorView collector.

Once this setting is in place, send a
\Condor{reconfig} to all machines in your pool so the changes take
effect.
This is described in section~\ref{sec:Reconfigure-Pool} on
page~\pageref{sec:Reconfigure-Pool}.

