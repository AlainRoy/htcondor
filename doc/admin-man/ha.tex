%%%%%%%%%%%%%%%%%%%%%%%%%%%%%%%%%%%%%%%%%%%%%%%%%%%%%%%%%%%%%%%%%%%%%%%%%%%
\subsection{\label{sec:HA-schedd} High Availability of the Job Queue} 
%%%%%%%%%%%%%%%%%%%%%%%%%%%%%%%%%%%%%%%%%%%%%%%%%%%%%%%%%%%%%%%%%%%%%%%%%%

\index{High Availability!of job queue}

For a pool where all jobs are submitted through
a single machine in the pool,
and there are lots of jobs,
this machine becoming nonfunctional means that
jobs stop running.
The \Condor{schedd} daemon maintains the job queue.
No job queue due to having a nonfunctional machine
implies that no jobs can be run.
This situation is worsened by using one machine
as the single submission point.
For each Condor job (taken from the queue) that is executed,
a \Condor{shadow} process runs on the machine where submitted
to handle input/output functionality.
If this machine becomes nonfunctional, none of the jobs can
continue.
The entire pool stops running jobs.

The goal of \emph{High Availability} in this special case is
to transfer the \Condor{schedd} daemon to run on another
designated machine.
Jobs caused to stop without finishing can be restarted from the
beginning, or can continue execution using the most recent checkpoint.
New jobs can enter the job queue.
Without \emph{High Availability},
the job queue would remain intact, but further progress on jobs
would wait until the machine running the \Condor{schedd} daemon
became available (after fixing whatever caused it to become
unavailable).

Condor uses its flexible configuration mechanisms to allow
the transfer of the \Condor{schedd} daemon from one machine
to another.
The configuration specifies
which machines are chosen to run the \Condor{schedd} daemon.
To prevent multiple \Condor{schedd} daemons from running at the same time,
a lock (semaphore-like) is held over the job queue.
This synchronizes  the situation in which control is
transferred to a secondary machine,
and the primary machine returns to functionality.
Configuration variables also determine time intervals at which 
the lock expires,
and periods of time that pass between polling to check
for expired locks.

To specify a single machine that would take over, if the
machine running the \Condor{schedd} daemon stops working,
the following additions are made to the local configuration
of any and all machines that are able to run the \Condor{schedd} daemon
(becoming the single pool submission point): 

\begin{verbatim}
MASTER_HA_LIST = SCHEDD
SPOOL = /share/spool
HA_LOCK_URL = file:/share/spool
\end{verbatim}

Configuration macro \Macro{MASTER\_HA\_LIST} identifies the 
\Condor{schedd} daemon as the daemon that is to be watched
to make sure that it is running.
Each machine with this configuration must have access to the
lock (the job queue) which synchronizes which single machine does run the
\Condor{schedd} daemon.
This lock and the job queue must both be located in a shared file space,
and is currently specified only with a file URL.
The configuration specifies the shared space
(\MacroNI{SPOOL}),
and the URL of the lock.

As Condor starts on machines that are configured to run
the single \Condor{schedd} daemon, 
the \Condor{master} daemon of the
first machine that looks at (polls) the lock
and notices that no lock is held.
This implies that no \Condor{schedd} daemon is running.
This \Condor{master} daemon acquires the lock
and runs the \Condor{schedd} daemon.
Other machines with this same capability to run the
\Condor{schedd} daemon look at (poll) the lock, 
but do not run the daemon, as the lock is held.
The machine running the \Condor{schedd} daemon renews the
lock periodically.

If the machine running the \Condor{schedd} daemon fails to renew
the lock (because the machine is not functioning),
the lock times out (becomes stale).
The lock is released by the \Condor{master} daemon
if \Condor{off} or \Prog{condor\_off -schedd} is
executed, or when the \Condor{master} daemon knows that the
\Condor{schedd} daemon is no longer running.
As other machines capable of running the \Condor{schedd} daemon
look at the lock (poll), one machine will be the first
to notice that the lock has timed out or been released.
This machine (correctly) interprets this situation as the
\Condor{schedd} daemon is no longer running.
This machine's \Condor{master} daemon then acquires the lock
and runs the \Condor{schedd} daemon.

See 
section~\ref{sec:Master-Config-File-Entries},
in the section on \Condor{master} Configuration File Macros
for details relating to the configuration variables
used to set timing and polling intervals.

%%%%%%%%%%%%%%%%%%%%%%%%%%%%%%%%%%%%%%%%%%%%%%%%%%%%%%%%%%%%%%%%%%%%%%%%%%%
\subsection{\label{sec:HA-negotiator} High Availability of the
Central Manager} 
%%%%%%%%%%%%%%%%%%%%%%%%%%%%%%%%%%%%%%%%%%%%%%%%%%%%%%%%%%%%%%%%%%%%%%%%%%
\index{High Availability!of central manager}

The \Condor{negotiator} and \Condor{collector} daemons
are the heart of the Condor matchmaking system.
The availability of these daemons is critical to a Condor pool's
functionality.
Both daemons usually run on the same machine,
most often known as the central manager.
The failure of a central manager machine prevents Condor from matching
new jobs and allocating new resources.
High availability of the \Condor{negotiator} and \Condor{collector} daemons
eliminates this problem. 

Configuration allows one of
multiple machines within the pool to function as the central manager.
While there are may be many active \Condor{collector} daemons,
only a single, active \Condor{negotiator} daemon will be running.
The machine with the \Condor{negotiator} daemon running is
the active central manager.
The other potential central managers each
have a \Condor{collector} daemon running;
these are the idle central managers.

All submit and execute machines are configured to report
to all potential central managers machines.

Each potential central manager machine runs the high
availability daemon, \Condor{had}.
\index{Condor daemon!\Condor{had}}
\index{daemon!\Condor{had}}
These daemons communicate with each other,
constantly monitoring the pool to ensure that one active central
manager is available.
If the active central manager machine crashes or is shut down,
these daemons detect the failure,
and they agree on
which of the idle central managers is to become the active one.
A protocol determines this.

In the case of a network partition,
idle \Condor{had} daemons within each partition
detect (by the lack of communication) a partitioning,
and then use the protocol to chose an active central manager.
As long as the partition remains, 
and there exists an idle central manager within the partition,
there will be one active central manager within each partition.
When the network is repaired,
the protocol returns to having one central manager.

Through configuration,
a specific central manager machine may act as the
primary central manager.
While this machine is up and running, 
it functions as the central manager.
After a failure of this primary central manager,
another idle central manager becomes the active one.
When the primary recovers,
it again becomes the central manager.
This is a recommended configuration,
if one of the central managers is a reliable machine,
which is expected to have very short periods of instability.
An alternative configuration allows the promoted 
active central manager 
(in the case that the central manager fails)
to stay active after the failed central manager machine
returns.

This high availability mechanism operates by monitoring
communication between machines.
Note that there is a significant difference in
communications between machines when 
\begin{enumerate}
\item 
a machine is down
\item 
a specific daemon (the \Condor{had} daemon in this case)
is not running, yet the machine is functioning
\end{enumerate}
The high availability mechanism distinguishes between these two,
and it operates based only on first
(when a central manager machine is down).
A lack of executing daemons does \emph{not}
cause the protocol to choose or use a new active central manager. 

%In order for all machines in the pool to take advantage of the high
%availability capability, they all should run the version of Condor
%with implemented High Availability mechanism. However, the machines
%running older versions of Condor will still function properly after
%upgrade, as long as the machine which served a central manager before
%the upgrade is configured as primary central manager. Having said
%that, the machines with the older version will not be able to take
%advantage of the central manager fail-over, and will suffer from the
%old problems in case of failure of the central manager


%%%%%%%%%%%%%%%%%%%%%%%%%%%%%%%%%%%%%%%%%%%%%%%%%%%%%%%%%%%%%%%%%%%%%%%%%%%
\subsubsection{\label{sec:HA-configuration} Configuration} 
%%%%%%%%%%%%%%%%%%%%%%%%%%%%%%%%%%%%%%%%%%%%%%%%%%%%%%%%%%%%%%%%%%%%%%%%%%

The high availability of central manager machines is
enabled through configuration.
It is disabled by default.
All machines in a pool must be configured appropriately
in order to make the high availability mechanism work.
See
section~\ref{sec:HA-Config-File-Entries},
for definitions of these configuration variables.

The stabilization period is the time it takes for the 
\Condor{had} daemons
to detect a change in the pool state such as
a active central manager failure
or network partition, and recover from this change.
It may be computed using the following formula:
\footnotesize
\begin{verbatim}
stabilization period = 12 * (number of central managers) * HAD_CONNECTION_TIMEOUT
\end{verbatim}
\normalsize


To disable the high availability of central managers mechanism,
it is sufficient to remove
\Expr{HAD} and \Expr{NEGOTIATOR} from the
\MacroNI{DAEMON\_LIST} configuration variable on all machines,
leaving only one \Condor{negotiator} in the pool. 

To shut down a currently operating high availability mechanism,
follow the given steps.
All commands must be invoked
from a host which has administrative permissions
on all central managers.
The first two commands kill all \Condor{had}
and all running \Condor{negotiator} daemons.
The last command is invoked on the host where the
single \Condor{negotiator} daemon is to run.

\begin{enumerate}
\item \verb@condor_off -all -neg@
\item \verb@condor_off -all -had@
\item \verb@condor_on -neg@
\end{enumerate}

%%%%%%%%%%%%%%%%%%%%%%%%%%%%%%%%%%%%%%%%%%%%%%%%%%%%%%%%%%%%%%%%%%%%%%%%%%%
\subsubsection{\label{sec:HA-sample-config} Sample Configuration} 
%%%%%%%%%%%%%%%%%%%%%%%%%%%%%%%%%%%%%%%%%%%%%%%%%%%%%%%%%%%%%%%%%%%%%%%%%%%
\index{High Availability!sample configuration}
This section provides sample configurations for high availability.
The two parts to this are the 
configuration for the potential central manager machines,
and the configuration for the
machines within the pool that will \emph{not} be
central managers.

This is a sample configuration relating to the 
high availability of central managers.
This is for the potential central manager machines.

\footnotesize
\begin{verbatim}
##########################################################################
# A sample configuration file for central managers, to enable the        #
# the high availability  mechanism.                                      #
##########################################################################

# unset these two macros 
NEGOTIATOR_HOST=
CONDOR_HOST=

#########################################################################
## THE FOLLOWING MUST BE IDENTICAL ON ALL POTENTIAL CENTRAL MANAGERS.   # 
#########################################################################
## For simplicity in writing other expressions, define a variable
## for each potential central manager in the pool. 
## These are samples.
CENTRAL_MANAGER1 = cm1.cs.technion.ac.il
CENTRAL_MANAGER2 = cm2.cs.technion.ac.il
## A list of all potential central managers in the pool.
COLLECTOR_HOST = $(CENTRAL_MANAGER1),$(CENTRAL_MANAGER2)

## Define the port number on which the condor_had daemon will
## listen.  The port must match the port number used
## for when defining HAD_LIST.  This port number is
## arbitrary; make sure that there is no port number collision
## with other applications.
HAD_PORT = 51450
HAD_ARGS = -p $(HAD_PORT)

## The following list must contain the same addresses in the same order 
## as COLLECTOR_HOST. In addition, for each hostname, it should specify 
## the port number of condor_had daemon running on that host.
## The first machine in the list will be the PRIMARY central manager
## machine, in case HAD_USE_PRIMARY is set to true.
HAD_LIST = $(CENTRAL_MANAGER1):$(HAD_PORT),$(CENTRAL_MANAGER2):$(HAD_PORT)


## HAD connection time.
## Recommended value is 2 if the central managers are on the same subnet.
## Recommended value is 5 if Condor security is enabled.
## Recommended value is 10 if the network is very slow, or
## to reduce the sensitivity of HA daemons to network failures.
HAD_CONNECTION_TIMEOUT = 2

##If true, the first central manager in HAD_LIST is a primary.
HAD_USE_PRIMARY = true

##--------------------------------------------------------------------
##  Host/IP access levels
##--------------------------------------------------------------------

##  What machines have administrative rights for your pool?  This
##  defaults to your central manager.  You should set it to the
##  machine(s) where whoever is the condor administrator(s) works
##  (assuming you trust all the users who log into that/those
##  machine(s), since this is machine-wide access you're granting).
HOSTALLOW_ADMINISTRATOR = $(COLLECTOR_HOST) 


##  Negotiator access.  Machines listed here are trusted central
##  managers.  You should normally not have to change this.
HOSTALLOW_NEGOTIATOR = $(COLLECTOR_HOST)

###################################################################
## THE PARAMETERS BELOW ARE ALLOWED TO BE DIFFERENT ON EACH       #
## CENTRAL MANAGERS                                               #
###################################################################

## The location of executable files
HAD = $(SBIN)/condor_had
## the master should start at least these four daemons
DAEMON_LIST = MASTER, COLLECTOR, NEGOTIATOR, HAD
## DC_Daemon list should contain at least these four
DC_DAEMON_LIST = MASTER, COLLECTOR, NEGOTIATOR, HAD

## The size of the log file
MAX_HAD_LOG = 640000
## debug level 
HAD_DEBUG = D_COMMAND
## location of the condor_had log file
HAD_LOG = $(LOG)/HADLog
\end{verbatim}
\normalsize

Machines that are not potential central managers also 
require configuration.
The following is a sample configuration relating to
high availability for machines that will \emph{not} be central managers.

\footnotesize
\begin{verbatim}
##########################################################################
# Sample configuration relating to high availability for machines        # 
# that DO NOT run the condor_had daemon.                                 #
##########################################################################

#unset these variables 
NEGOTIATOR_HOST =
CONDOR_HOST =

## For simplicity define a variable for each potential central manager
## in the pool. 
CENTRAL_MANAGER1 = cm1.cs.technion.ac.il
CENTRAL_MANAGER2 = cm2.cs.technion.ac.il
## List of all potential central managers in the pool
COLLECTOR_HOST = $(CENTRAL_MANAGER1),$(CENTRAL_MANAGER2)

##--------------------------------------------------------------------
##  Host/IP access levels
##--------------------------------------------------------------------

##  Negotiator access.  Machines listed here are trusted central
##  managers.  You should normally not need to change this.
HOSTALLOW_NEGOTIATOR = $(COLLECTOR_HOST)

##  Now, with flocking (and HA) we need to let the SCHEDD trust the other 
##  negotiators we are flocking with as well.  You should normally 
##  not need to change this.
HOSTALLOW_NEGOTIATOR_SCHEDD = $(COLLECTOR_HOST) 
\end{verbatim}
\normalsize

%%%%%%%%%%%%%%%%%%%%%%%%%%%%%%%%%%%%%%%%%%%%%%%%%%%%%%%%%%%%%%%%%%%%%%%%%%%
%\subsubsection{\label{sec:HA-check-script} Checking the Sanity of the
%Configuration} 
%%%%%%%%%%%%%%%%%%%%%%%%%%%%%%%%%%%%%%%%%%%%%%%%%%%%%%%%%%%%%%%%%%%%%%%%%%
%\index{High Availability!sanity check script}

%A sanity check scrip is provided to check
%that all high-availability-related configuration
%variables on the running Condor pool are configured correctly. 
%The script does not check that daemons are indeed
%running on the machine;
%it only ensures the correctness of configuration variables.
%
%The script finds the addresses of all potential central managers
%by checking the value of the 
%\MacroNI{HAD\_LIST} configuration variable.
%To work properly, the pool must
%
%\begin{itemize}
%\item be running \Condor{master} daemons on each potential
%central manager machine.
%\item have \MacroUNI{SBIN} in the \Env{PATH} environment variable,
%before any other possible system binaries directory.
%\item have Perl version 5.6.0 or higher.
%\end{itemize}
%
%The items tested by the script are
%
%\begin{itemize}
%\item The \MacroNI{HAD\_LIST} is consistent on all potential central managers. 
%\item The \MacroNI{HAD\_CONNECTION\_TIMEOUT} is consistent 
%on all potential central managers.
%\item The \MacroNI{COLLECTOR\_HOST} entries on all machines
%correspond to the \MacroNI{HAD\_LIST} entries,
%and are correctly ordered.
%\item Configuration variable \MacroNI{DAEMON\_LIST} contains \Expr{HAD}, \Expr{COLLECTOR} and \Expr{NEGOTIATOR}.
%\item Configuration variables \MacroNI{HAD\_ARGS} and \MacroNI{HAD\_LIST}
%contain consistent definitions of corresponding ports numbers for 
%high availability.
%\item Configuration variables \MacroNI{HOSTALLOW\_NEGOTIATOR} and
%\MacroNI{HOSTALLOW\_ADMINISTRATOR} contain all machines
%specified in the \MacroNI{COLLECTOR\_HOST} definition
%for an appropriate machine. 
%\end{itemize}
%
