%%%%%%%%%%%%%%%%%%%%%%%%%%%%%%%%%%%%%%%%%%%%%%%%%%%%%%
\section{Welcome to Condor}  
%
% .... or alternatively called the 'warm fuzzies' section
% <smirk>  
% 
%
% Warning: much of what you are about to read was very 
% hastily written by a very tired Todd.... Good Luck.  
%%%%%%%%%%%%%%%%%%%%%%%%%%%%%%%%%%%%%%%%%%%%%%%%%%%%%

\label{sec:usermanual}
\index{Condor!user manual|(}
\index{user manual|(}
Presenting Condor \VersionNotice! Condor is developed by
the Condor Team at the University of Wisconsin-Madison (UW-Madison), and
was first installed as a production system in the UW-Madison Computer
Sciences department nearly 10 years ago. This Condor pool has since
served as a major source of computing cycles to UW faculty and students.
For many, it has revolutionized the role computing plays in their
research. An increase of one, and sometimes even two, orders of
magnitude in the computing throughput of a research organization can
have a profound impact on its size, complexity, and scope. Over the
years, the Condor Team has established collaborations with scientists
from around the world and has provided them with access to surplus
cycles (one of whom has consumed 100 CPU years!). Today, our
department's pool consists of more than 700 desktop Unix workstations.
On a typical day, our pool delivers more than 500 CPU days to UW
researchers. Additional Condor pools have been established over the
years across our campus and the world. Groups of researchers, engineers,
and scientists have used Condor to establish compute pools ranging in
size from a handful to hundreds of workstations. We hope that Condor
will help revolutionize your compute environment as well.


%%%%%%%%%%%%%%%%%%%%%%%%%%%%%%%%%%%%%%%%%%%%%%%%%%%%%%%
\section{What does Condor do?}
%%%%%%%%%%%%%%%%%%%%%%%%%%%%%%%%%%%%%%%%%%%%%%%%%%%%%%%


In a nutshell, Condor is a specialized batch system 
\index{batch system}
for managing compute-intensive jobs.
Like most batch systems, Condor provides a
queueing mechanism, scheduling policy, priority scheme, and resource
classifications.  Users submit their compute jobs to Condor, Condor puts
the jobs in a queue, runs them, and then informs the user as to the
result.

Batch systems normally operate only with dedicated machines.  Often 
termed compute servers, these dedicated machines are typically owned by
one organization and dedicated to the sole purpose of running compute
jobs.  Condor can schedule jobs on dedicated machines.  But unlike traditional 
batch systems, Condor is also designed to effectively 
utilize non-dedicated machines to run jobs.  By being told to only
run compute jobs on machines which are currently not being used (no keyboard
activity, no load average, no active telnet users, etc), Condor can
effectively harness otherwise idle machines throughout a pool of machines.
This is important because often times the amount of
compute power represented by the aggregate total of all the non-dedicated 
desktop workstations sitting on people's desks throughout the
organization is far greater than the compute power of a dedicated
central resource.

Condor has several unique capabilities at its disposal which are geared 
towards effectively utilizing non-dedicated resources that are not owned or
managed by a centralized resource. These include transparent process
checkpoint and migration, remote system calls, and ClassAds.
Read section~\ref{sec:what-is-condor} for a general 
discussion of these features before reading any further.


%%%%%%%%%%%%%%%%%%%%%%%%%%%%%%%%%%%%%%%%%%%%%%%%%%%%%%%%
\section{Condor Matchmaking with ClassAds}
%%%%%%%%%%%%%%%%%%%%%%%%%%%%%%%%%%%%%%%%%%%%%%%%%%%%%%%%

Before you learn about how to submit a job, it is important to
understand how Condor allocates resources. 
\index{Condor!resource allocation}
Understanding the
unique framework by which Condor matches submitted jobs with machines is
the key to getting the most from Condor's scheduling algorithm. 

Condor simplifies job submission by acting as a matchmaker of ClassAds.
Condor's ClassAds
\index{ClassAd}
are analogous to the classified advertising section of the
newspaper. Sellers advertise specifics about what they have to sell,
hoping to attract a buyer. Buyers may advertise specifics about what
they wish to purchase. Both buyers and sellers list constraints that
need to be satisfied.
For instance, a buyer has a maximum spending limit, 
and a seller requires a minimum purchase price.
Furthermore, both want to rank requests to their own advantage.
Certainly a seller would rank
one offer of \$50 dollars higher than a different
offer of \$25.
In Condor, users submitting
jobs can be thought of as buyers of compute resources and machine owners
are sellers. 

All machines in a Condor pool advertise their attributes,
\index{ClassAd!attributes}
such as
available RAM memory, CPU type and speed, virtual memory size, current
load average, along with other static and dynamic properties.
This machine ClassAd
\index{ClassAd!machine}
also advertises under what conditions it is
willing to run a Condor job and what type of job it would prefer. These
policy attributes can reflect the individual terms and preferences by
which all the different owners have graciously allowed their machine to
be part of the Condor pool. 
You may
advertise that your machine is only willing to run jobs at night
and when there is no keyboard activity on your machine.
In addition, you may
advertise a preference (rank) for running jobs submitted by you
or one of your co-workers. 

Likewise, when submitting a job, you specify a ClassAd with
your requirements and preferences.
The ClassAd
\index{ClassAd!job}
includes the
type of machine you  wish to use. For instance, perhaps you are
looking for the fastest floating point performance available.
You want Condor to rank available machines
based upon floating point performance. Or, perhaps you
care only that the machine has a minimum of 128 Mbytes of RAM.
Or, perhaps you will
take any machine you can get! These job attributes and requirements
are bundled up into a job ClassAd.

Condor plays the role of a matchmaker by continuously reading
all the job ClassAds and all the machine ClassAds, 
matching and ranking job ads with machine ads.
Condor makes certain that all
requirements in both ClassAds are satisfied. 

%%%%%
\subsection{Inspecting Machine ClassAds with \condor{status}}
%%%%%

\index{Condor commands!condor\_status}
Once Condor is installed,
you will get a feel for what
a machine ClassAd does by trying
the \Condor{status} command.
Try the \Condor{status} command to get
a summary of information from
ClassAds about the resources available in your pool.
Type \Condor{status} and hit enter to see a summary 
similar to the following:
\begin{center}
\begin{verbatim}
Name       Arch     OpSys        State      Activity   LoadAv Mem  ActvtyTime

adriana.cs INTEL    SOLARIS251   Claimed    Busy       1.000  64    0+01:10:00
alfred.cs. INTEL    SOLARIS251   Claimed    Busy       1.000  64    0+00:40:00
amul.cs.wi SUN4u    SOLARIS251   Owner      Idle       1.000  128   0+06:20:04
anfrom.cs. SUN4x    SOLARIS251   Claimed    Busy       1.000  32    0+05:16:22
anthrax.cs INTEL    SOLARIS251   Claimed    Busy       0.285  64    0+00:00:00
astro.cs.w INTEL    SOLARIS251   Claimed    Busy       0.949  64    0+05:30:00
aura.cs.wi SUN4u    SOLARIS251   Owner      Idle       1.043  128   0+14:40:15
\end{verbatim}
\Dots 
\end{center}


The \Condor{status} command has options that summarize machine ads 
in a variety of ways.
For example,
\begin{description}
\item[\Condor{status -available}] shows only machines which are
willing to run jobs now. 
\item[\Condor{status -run}] shows only machines
which are currently running jobs.  
\item[\Condor{status -l}] lists the machine ClassAds for all machines
in the pool.
\end{description}

Refer to the \Condor{status} command 
reference page located on page~\pageref{man-condor-status}
for a complete description of the \Condor{status} command.

Figure~\ref{fig:CondorStatusL} shows the complete machine ClassAd
\index{ClassAd!machine example}
\index{machine ClassAd}
for a single workstation: alfred.cs.wisc.edu. Some of the listed
attributes are used by
Condor for scheduling. Other attributes are for information purposes.
An important point is that \emph{any} of the attributes in a
machine ad can be utilized at job submission time as part of a request
or preference on what machine to use. Additional attributes
can be easily added. For example, your site administrator can
add a physical location attribute to your machine ClassAds.

%
% figures for this section
%
% condor_status -l alfred
%
\begin{center}
\begin{figure}
\CondorVerySmall
\begin{verbatim}
MyType = "Machine"
TargetType = "Job"
Name = "alfred.cs.wisc.edu"
Machine = "alfred.cs.wisc.edu"
StartdIpAddr = "<128.105.83.11:32780>"
Arch = "INTEL"
OpSys = "SOLARIS251"
UidDomain = "cs.wisc.edu"
FileSystemDomain = "cs.wisc.edu"
State = "Unclaimed"
EnteredCurrentState = 892191963
Activity = "Idle"
EnteredCurrentActivity = 892191062
VirtualMemory = 185264
Disk = 35259
KFlops = 19992
Mips = 201
LoadAvg = 0.019531
CondorLoadAvg = 0.000000
KeyboardIdle = 5124
ConsoleIdle = 27592
Cpus = 1
Memory = 64
AFSCell = "cs.wisc.edu"
START = LoadAvg - CondorLoadAvg <= 0.300000 && KeyboardIdle > 15 * 60
Requirements = TRUE
Rank = Owner == "johndoe" || Owner == "friendofjohn" 
CurrentRank =  - 1.000000
LastHeardFrom = 892191963
\end{verbatim}
\normalsize
\caption{\label{fig:CondorStatusL}Sample output from \Condor{status -l alfred}}
\end{figure}
\end{center}


%%%%%%%%%%%%%%%%%%%%%%%%%%%%%%%%%%%%%%%%%%%%%%%%%%%%%%%%%%%%%
\section{Road-map for running jobs with Condor}
%%%%%%%%%%%%%%%%%%%%%%%%%%%%%%%%%%%%%%%%%%%%%%%%%%%%%%%%%%%%%

\index{job!preparation}
The road to using Condor effectively is a short one.  The basics
are quickly and easily learned.

Here are all the steps needed to run a job using Condor.
\begin{description}

\item[Code Preparation.]
A job run under Condor must be able to 
run as a background batch job.
\index{job!batch ready}
Condor runs the program unattended and in the background. 
A program that runs in the background will not be able
to do interactive input and output.
Condor can redirect console output (stdout and stderr)
and keyboard input (stdin)
to and from files for you.
Create any needed files that contain
the proper keystrokes needed for program input.
Make certain the program will run correctly with the files.

\item[The Condor Universe.]
Condor has four 
runtime environments (called a \Term{universe}) from which to choose.
Of the four, two are likely choices when learning
to submit a job to Condor: the standard universe and the vanilla universe.
The standard universe allows a job running under Condor to
handle system calls by returning them to the machine where the
job was submitted.
The standard universe also provides the mechanisms necessary
to take a checkpoint and migrate a partially completed job,
should the machine on which the job is executing become
unavailable.
To use the standard universe, it is necessary to
relink the program with the Condor library using the
\Condor{compile} command.
The manual page for \Condor{submit} on page~\pageref{man-condor-submit} has details.

The vanilla universe provides a way to run jobs that cannot be
relinked.
It depends on a shared file system for access to input and output
files,
and there is no way to take a checkpoint or migrate a job executed
under the vanilla universe.

Choose a universe under which to run the Condor program,
and re-link the program if necessary.

\item[Submit description file.]
Controlling the details of a job submission is a
submit description file.
The file contains information
about the job such as what executable to run, the
files to use for keyboard and screen data,
the platform type required to run the program, and
where to send e-mail when the job completes.
You can also tell Condor how many times to run a program;
it is simple to run the same program
multiple times with multiple data sets.

Write a submit description file to go with the job, using
the examples provided in section~\ref{sec:sample-submit-files}
for guidance.

\item[Submit the Job.]Submit the program to Condor with
the \Condor{submit} command.
\index{Condor commands!condor\_submit}

\end{description}

Once submitted, Condor does the rest toward running
the job.
Monitor the job's progress with the \Condor{q}
\index{Condor commands!condor\_q}
and \Condor{status} commands.
\index{Condor commands!condor\_status}
You may modify the order in which Condor will run your jobs with
\Condor{prio}. If desired, Condor can even inform you in a log file 
every time your job is checkpointed and/or migrated to a different machine. 

When your program completes, Condor will tell you
(by e-mail, if preferred) the exit status of your program and various
statistics about its performances, including time used and I/O performed.
If you are using a log file for the job(which is recommended) the exit
status will be recorded in the log file.
You can remove a job from the
queue prematurely with \Condor{rm}. 
\index{Condor commands!condor\_rm}


%%%%%%%%%%%%%%%%%%%%%%%%%%%%%%%%%%%%%%%%%%%%%%%%
\subsection{\label{sec:Choosing-Universe}
Choosing a Condor Universe}
%%%%%%%%%%%%%%%%%%%%%%%%%%%%%%%%%%%%%%%%%%%%%%%%

A \Term{universe} in Condor
\index{universe}
\index{Condor!universe}
defines an execution environment. 
Condor \VersionNotice\ supports four different
universes for user jobs:
\begin{itemize}
	\item Standard
	\item Vanilla
	\item PVM
%	\item MPI
	\item Globus
\end{itemize}

The \AdAttr{Universe} attribute is specified in the submit description file.
If a universe is not specified, the default is \Expr{standard}.

\index{universe!standard}
The \Expr{standard} universe provides migration and reliability, but has some
restrictions on the programs that can be run. 
\index{universe!vanilla}
The \Expr{vanilla} universe provides fewer services, but has very few
restrictions.
\index{universe!PVM}
The \Expr{PVM} universe is for programs written to the Parallel Virtual
Machine interface.  See section~\ref{sec:PVM} for more about PVM and Condor.
%\index{universe!MPI}
%The \Expr{MPI} universe is for programs written to the MPICH interface.
%See section~\ref{sec:MPI} for more about MPI and Condor.
\index{universe!Globus}
The Globus universe allows users to submit Globus jobs through the
Condor interface.  See \Url{http://www.globus.org} for more about Globus.

%%%%%%%%%%%%%%%%%%%%%%%%%%%%%%%%%%%%%%%%%%%%%%%%%%%%%%%%%%%%%%%%%%%%%%
\subsubsection{\label{sec:standard-universe}Standard Universe}
%%%%%%%%%%%%%%%%%%%%%%%%%%%%%%%%%%%%%%%%%%%%%%%%%%%%%%%%%%%%%%%%%%%%%%

\index{universe!standard}
In the standard universe, Condor provides \Term{checkpointing} and
\Term{remote system calls}.  These features make a job more reliable
and allow it uniform access to resources from anywhere in the pool.
To prepare a program as a standard universe job, it must be relinked
with \Condor{compile}.  Most programs can be prepared as a standard
universe job, but there are a few restrictions.

\index{checkpoint}
\index{checkpoint image}
Condor checkpoints a job at regular intervals.
A \Term{checkpoint image} is essentially a snapshot of the current
state of a job. 
If a job must be migrated from one machine to another,
Condor makes a checkpoint image, copies the image to the new machine,
and restarts the job continuing the job from where it left off.
If a machine should
crash or fail while it is running a job, Condor can restart the job on
a new machine using the most recent checkpoint image.
In this way, jobs
can run for months or years even in the face of occasional computer failures.

\index{remote system call}
\index{shadow}
Remote system calls make a job perceive that it is executing on its home
machine, even though the job may execute on many different machines over its
lifetime.
When a job runs on a remote machine, a second process, called
a \Condor{shadow} runs on the machine where the job was submitted.
\index{condor\_shadow}
\index{agents!condor\_shadow}
\index{Condor daemon!condor\_shadow}
\index{remote system call!condor\_shadow}
When the job attempts a system call, the \Condor{shadow} performs
the system call instead and sends the results to the remote
machine.
For example, if a job attempts to open a file that is
stored on the submitting machine,
the \Condor{shadow} will find the file,
and send the data to the machine where
the job is running.

To convert your program into a standard universe job, you must use
\Condor{compile} to relink it with the Condor libraries.
Put \Condor{compile} in front of your usual link command.
You do not need to modify the program's source code,
but you do need access to the unlinked object files.
A commercial program that is packaged as a single executable file cannot be
converted into a standard universe job.

For example, if you would have linked the job by executing:
\begin{verbatim}
% cc main.o tools.o -o program
\end{verbatim}

Then, relink the job for Condor with:
\begin{verbatim}
% condor_compile cc main.o tools.o -o program
\end{verbatim}

There are a few restrictions on standard universe jobs:


\begin{enumerate}

\index{Unix!fork}
\index{Unix!exec}
\index{Unix!system}
\item Multi-process jobs are not allowed.  This includes system calls such as
\Syscall{fork}, \Syscall{exec}, and \Syscall{system}.

\index{Unix!pipe}
\index{Unix!semaphore}
\index{Unix!shared memory}
\item Interprocess communication is not allowed.  This includes pipes, semaphores, and shared memory.

\index{Unix!socket}
\index{network}
\item Network communication must be brief.  A job \emph{may} make network
connections using system calls such as \Syscall{socket}, but a network
connection left open for long periods will delay checkpointing and migration.

\index{signal}
\index{signal!SIGUSR2}
\index{signal!SIGTSTP}
\item Sending or receiving the SIGUSR2 or SIGTSTP signals is not allowed.
Condor reserves these signals for its own use.  Sending or receiving all
other signals \emph{is} allowed.

\index{Unix!alarm}
\index{Unix!timer}
\index{Unix!sleep}
\item Alarms, timers, and sleeping are not allowed.  This includes system
calls such as \Syscall{alarm}, \Syscall{getitimer}, and \Syscall{sleep}.

\index{thread!kernel-level}
\index{thread!user-level}
\item Multiple kernel-level threads are not allowed.  However,
multiple user-level threads \emph{are} allowed.

\index{file!memory-mapped}
\index{Unix!mmap}
\item Memory mapped files are not allowed.  This includes system calls such
as \Syscall{mmap} and \Syscall{munmap}.

\index{file!locking}
\index{Unix!flock}
\index{Unix!lockf}
\item File locks are allowed, but not retained between checkpoints.

\index{file!read only}
\index{file!write only}
\item All files must be opened read-only or write-only.  A file opened
for both reading and writing will cause trouble if a job must be rolled back
to an old checkpoint image.  For compatibility reasons, a file opened
for both reading and writing will result in a warning but not an error.

\item A fair amount of disk space must be available on the submitting machine
for storing a job's checkpoint images.  A checkpoint image is approximately
equal to the virtual memory consumed by a job while it runs.  If disk space
is short, a special \Term{checkpoint server} can be designated for storing
all the checkpoint images for a pool.

\index{linking!dynamic}
\index{linking!static}
\item On Linux, the job must be statically linked. 
\Condor{compile} does this by default.

\index{Unix!large files} 
\item Reading to or writing from files larger than 2 GBytes is only supported
when the submit side \Condor{shadow} and the standard universe user job
application itself are both 64-bit executables.

\end{enumerate}






%%%%%%%%%%%%
\subsubsection{Vanilla Universe}
%%%%%%%%%%%%

\index{universe!vanilla}
The vanilla universe in Condor is intended
for programs which cannot
be successfully re-linked.
Shell scripts are another case where the vanilla universe
is useful.
Unfortunately, jobs run under the vanilla universe cannot checkpoint or use
remote system calls. 
This has unfortunate consequences for a job that is partially
completed 
when the remote machine running a job must be returned
to its owner.
Condor has only two choices.  It can suspend the job, hoping to
complete it at a later time,
or it can give up and restart the job \emph{from the beginning} 
on another machine in the pool.

\Notice{
Under Unix, jobs submitted as vanilla universe jobs rely on an
external mechanism for accessing data files,
such as NFS or AFS.
The job \emph{must} be able to access the data
files from any machine on which it could potentially run.
As an example,
suppose a job is submitted from blackbird.cs.wisc.edu, and the job requires
a particular data file called \File{/u/p/s/psilord/data.txt}.
If the job were to run on cardinal.cs.wisc.edu, the file
\File{/u/p/s/psilord/data.txt} must be available through either
NFS or AFS for the job to run correctly.

Condor deals with this restriction imposed by the vanilla universe by
using
the \AdAttr{FileSystemDomain} and \AdAttr{UidDomain} machine
ClassAd attributes.
These attributes reflect the reality of the pool's disk
mounting structure.
For a large pool spanning multiple
\AdAttr{UidDomain} and/or \AdAttr{FileSystemDomain}s, the job
must specify
its \AdAttr{requirements} to use the correct \AdAttr{UidDomain} and/or
\AdAttr{FileSystemDomain}s.

This mechanism is not required under Windows NT.
The vanilla universe does \emph{not} require
a shared file system due to the Condor File Transfer mechanism. Please see
chapter~\ref{condor-nt} for more details about Condor NT.
}

%%%%%%%%%%%%
\subsubsection{PVM}
%%%%%%%%%%%%

\index{universe!PVM}
The PVM universe allows programs written for the Parallel Virtual Machine
interface to be used within the opportunistic Condor environment.
Please see section~\ref{sec:PVM} for more details.

% Commented out until implementation/documentation is complete:
%%%%%%%%%%%%
%\subsubsection{MPI}
%%%%%%%%%%%%
%\index{universe!MPI}
%The MPI universe allows programs written to the MPICH
%interface to be used within the opportunistic Condor environment.
%Please see section~\ref{sec:MPI} for more details.

%%%%%%%%%%%%
\subsubsection{Globus Universe}
%%%%%%%%%%%%

\index{universe!Globus}
The Globus universe in Condor is intended to provide the standard
Condor interface to users who wish to start Globus system jobs
from Condor. Each job queued in the job submission file is translated
into a Globus RSL string and used as the arguments to the \Prog{globusrun}
program. The manual page for \Condor{submit}
on page~\pageref{man-condor-submit}
has detailed descriptions of
the Globus-related attributes.

%%%%%%%%%%%%%%%%%%%%%%%%%%%%%%%%%%%%%%%%%%%%%%%%%%%%%%%%%%%%%%
\section{Submitting a Job to Condor}
%%%%%%%%%%%%%%%%%%%%%%%%%%%%%%%%%%%%%%%%%%%%%%%%%%%%%%%%%%%%%%

\index{job!submitting}
A job is submitted for execution to Condor using the
\Condor{submit} command.
\index{Condor commands!condor\_submit}
\Condor{submit} takes as an argument the name of a
file called a submit description file.
\index{submit description file}
\index{file!submit description}
This file contains commands and keywords to direct the queuing of jobs.
In the submit description file, Condor finds everything it needs
to know about the job.  Items such as the name of the executable to run,
the initial working directory, and command-line arguments to the
program all go into
the submit description file.  \Condor{submit} creates a job
ClassAd based upon the information,
and Condor
works toward running the job.

The contents of a submit file
\index{submit description file!contents of}
can save time for Condor users.
It is easy to submit multiple runs of a program to
Condor. To run the same program 500 times on 500
different input data sets, arrange your data files
accordingly so that each run reads its own input, and each run
writes its own output.
Each individual run may have its own initial
working directory, stdin, stdout, stderr, command-line arguments, and
shell environment.
A program that directly opens its own
files will read the file names to use either from stdin
or from the command line. 
A program that opens a static filename every time
will need to use a separate subdirectory for the output of each run.

The \Condor{submit} manual page 
is on page~\pageref{man-condor-submit} and
contains a complete and full description of how to use \Condor{submit}.

%%%%%%%%%%%%%%%%%%%%
\subsection{\label{sec:sample-submit-files}Sample submit description files}  
%%%%%%%%%%%%%%%%%%%%

In addition to the examples of submit description files given
in the 
\Condor{submit} manual page, here are a few more.
\index{submit description file!examples|(}

\subsubsection{Example 1} 

Example 1 is the simplest submit description
file possible. It queues up one copy of the program \Prog{foo}(which had been
created by \Condor{compile}) for execution
by Condor.
Since no platform is specified, Condor will use its default,
which is to run the job on a machine which has the
same architecture and operating system as the machine from which it was
submitted. 
No 
\AdAttr{input},
\AdAttr{output}, and
\AdAttr{error}
commands are given in the submit
description file, so the
files \File{stdin}, \File{stdout}, and \File{stderr} will all refer to 
\File{/dev/null}.
The program may produce output by explicitly opening a file and writing to
it.
A log file, \File{foo.log}, will also be produced that contains events
the job had during its lifetime inside of Condor.
When the job finishes, its exit conditions will be noted in the log file.
It is recommended that you always have a log file so you know what
happened to your jobs.
\begin{verbatim}
  ####################                                                    
  # 
  # Example 1                                                            
  # Simple condor job description file                                    
  #                                                                       
  ####################                                                    
                                                                          
  Executable     = foo                                                    
  Log            = foo.log                                                    
  Queue    
\end{verbatim}

\subsubsection{Example 2}

Example 2 queues two copies of the program \Prog{mathematica}. The
first copy will run in directory \File{run\_1}, and the second will run in
directory \File{run\_2}. For both queued copies, 
\File{stdin} will be \File{test.data},
\File{stdout} will be \File{loop.out}, and
\File{stderr} will be \File{loop.error}.
There will be two sets of files written,
as the files are each written to their own directories.
This is a convenient way to organize data if you
have a large group of Condor jobs to run. The example file 
shows program submission of
\Prog{mathematica} as a vanilla universe job.
This may be necessary if the source
and/or object code to program \Prog{mathematica} is not available.
\begin{verbatim}
  ####################     
  #                       
  # Example 2: demonstrate use of multiple     
  # directories for data organization.      
  #                                        
  ####################                    
                                         
  Executable     = mathematica          
  Universe = vanilla                   
  input   = test.data                
  output  = loop.out                
  error   = loop.error             
  Log     = loop.log                                                    
                                  
  Initialdir     = run_1         
  Queue                         
                               
  Initialdir     = run_2      
  Queue                     
\end{verbatim}

\subsubsection{Example 3}

The submit description file for Example 3 queues 150
\index{running multiple programs}
runs of program \Prog{foo} which has been compiled and linked for
Silicon Graphics workstations running IRIX 6.5. 
This job requires Condor to run the program on machines which have
greater than 32 megabytes of physical memory, and expresses a
preference to run the program on machines with more than 64 megabytes,
if such machines are available.  It also advises Condor that it will
use up to 28 megabytes of memory when running.
Each of the 150 runs of the program is given its own process number,
starting with process number 0.
So, files 
\File{stdin}, \File{stdout}, and \File{stderr} will
refer to \File{in.0}, \File{out.0}, and \File{err.0} for the first run
of the program,
\File{in.1}, \File{out.1},
and \File{err.1} for the second run of the program, and so forth.
A log file containing entries
about when and where Condor runs, checkpoints, and migrates processes for
the 150 queued programs
will be written into file \File{foo.log}.
\begin{verbatim}
  ####################                    
  #
  # Example 3: Show off some fancy features including
  # use of pre-defined macros and logging.
  #
  ####################                                                    

  Executable     = foo                                                    
  Requirements   = Memory >= 32 && OpSys == "IRIX65" && Arch =="SGI"     
  Rank		 = Memory >= 64
  Image_Size     = 28 Meg                                                 

  Error   = err.$(Process)                                                
  Input   = in.$(Process)                                                 
  Output  = out.$(Process)                                                
  Log = foo.log

  Queue 150
\end{verbatim}

\index{submit description file!examples|)}

%%%%%%%%%%%%%%%%%
\subsection{About Requirements and Rank}
%%%%%%%%%%%%%%%%%

The 
\AdAttr{requirements} and \AdAttr{rank} commands in the submit description file
are powerful and flexible. 
\index{requirements attribute}
\index{rank attribute}
\index{ClassAd attribute!requirements}
\index{ClassAd attribute!rank}
Using them effectively requires care, and this section presents
those details.

Both \AdAttr{requirements} and \AdAttr{rank} need to be specified 
as valid Condor ClassAd expressions, however, default values are set by the
\Condor{submit} program if these aren't defined in the submit description file.
From the \Condor{submit} manual page and the above examples, you see
that writing ClassAd expressions is intuitive, especially if you
are familiar with the programming language C.  There are some
pretty nifty expressions you can write with ClassAds.
A complete description of ClassAds and their expressions
can be found in section~\ref{classad-reference} on 
page~\pageref{classad-reference}.

All of the commands in the submit description file are case insensitive, 
\emph{except} for the ClassAd attribute string values.
ClassAds attribute names are
case insensitive, but ClassAd string
values are always \emph{case sensitive}.
The correct specification for an architecture is
\begin{verbatim}
        requirements = arch == "ALPHA"
\end{verbatim}
so an accidental specification of
\begin{verbatim}
        requirements = arch == "alpha"
\end{verbatim}
will not work due to the incorrect case.

The allowed
ClassAd attributes are those 
that appear in a machine or a job ClassAd.
To see all of the machine ClassAd attributes for all machines in
the Condor pool, run \Condor{status -l}.  
\index{Condor commands!condor\_status}
The \Arg{-l} argument to
\Condor{status} means to display all the complete machine ClassAds.
The job ClassAds, if there jobs in the queue, can be seen
with the \Condor{q -l} command.
This
will show you all the available attributes you can play with.

To help you out with what these attributes all signify,
descriptions follow for the attributes which will be common to every
machine ClassAd. Remember that because ClassAds are flexible, the
machine ads in your pool may include additional attributes specific
to your site's installation and policies. 
\subsubsection{\label{user-man-machad}ClassAd Machine Attributes}
\begin{description}
%
\index{ClassAd!machine attributes}
\index{ClassAd machine attribute!Activity}
\item[\AdAttr{Activity}:] String which describes Condor job activity on the machine.
Can have one of the following values:
	\begin{description}
	\item[\AdStr{Idle}:] There is no job activity
	\item[\AdStr{Busy}:] A job is busy running
	\item[\AdStr{Suspended}:] A job is currently suspended
	\item[\AdStr{Vacating}:] A job is currently checkpointing
	\item[\AdStr{Killing}:] A job is currently being killed
	\item[\AdStr{Benchmarking}:] The startd is running benchmarks
	\end{description}
%
\index{ClassAd machine attribute!Arch}
\item[\AdAttr{Arch}:] String with the architecture of the machine.  Typically
one of the following: 
	\begin{description}
	\item[\AdStr{INTEL}:] Intel x86 CPU (Pentium, Xeon, etc).
	\item[\AdStr{IA64}:] Intel 64-bit CPU
	\item[\AdStr{ALPHA}:] Digital Alpha CPU
	\item[\AdStr{SGI}:] Silicon Graphics MIPS CPU
	\item[\AdStr{SUN4u}:] Sun UltraSparc CPU
	\item[\AdStr{SUN4x}:] A Sun Sparc CPU other than an UltraSparc, i.e.
sun4m or sun4c CPU found in older Sparc workstations such as the Sparc~10, 
Sparc~20, IPC, IPX, etc.
	\item[\AdStr{PPC}:] Power Macintosh
	\item[\AdStr{HPPA1}:] Hewlett Packard PA-RISC 1.x CPU (i.e. PA-RISC    
                      7000 series CPU) based workstation
	\item[\AdStr{HPPA2}:] Hewlett Packard PA-RISC 2.x CPU (i.e. PA-RISC    
                      8000 series CPU) based workstation
	\end{description}
%
\index{ClassAd machine attribute!CheckpointPlatform}
\item[\AdAttr{CheckpointPlatform}:] A string which opaquely encodes various
aspects about a machine's operating system, hardware, and kernel
attributes.
It is used to identify systems where previously taken checkpoints for
the standard universe may resume.
%
\index{ClassAd machine attribute!ClockDay}
\item[\AdAttr{ClockDay}:] The day of the week, where 0 = Sunday, 1 = Monday, \Dots, 6 = Saturday. 
%
\index{ClassAd machine attribute!ClockMin}
\item[\AdAttr{ClockMin}:] The number of minutes passed since midnight.
%
\index{ClassAd machine attribute!CondorLoadAvg}
\item[\AdAttr{CondorLoadAvg}:] The portion of the load average generated by Condor (either
from remote jobs or running benchmarks).
%
\index{ClassAd machine attribute!ConsoleIdle}
\item[\AdAttr{ConsoleIdle}:] The number of seconds since activity on the system
console keyboard or console mouse has last been detected.
%
\index{ClassAd machine attribute!Cpus}
\item[\AdAttr{Cpus}:] Number of CPUs in this machine, i.e. 1 = single CPU machine, 2 = dual
CPUs, etc.
%
\index{ClassAd machine attribute!CurrentRank}
\item[\AdAttr{CurrentRank}:] A float which represents this machine
owner's affinity
for running the Condor job which it is currently hosting.  If not
currently hosting a Condor job, \AdAttr{CurrentRank} is 0.0.
When a machine is claimed,
the attribute's value is computed by evaluating the machine's
\AdAttr{Rank} expression with respect to the current job's ClassAd.
%
\index{ClassAd machine attribute!Disk}
\item[\AdAttr{Disk}:] The amount of disk space on this machine available for
the job in Kbytes ( e.g. 23000 = 23 megabytes ).  Specifically, this
is the amount of disk space available in the directory specified in
the Condor configuration files by the \Macro{EXECUTE} macro, minus any
space reserved with the \Macro{RESERVED\_DISK} macro.
%
\index{ClassAd machine attribute!EnteredCurrentActivity}
\item[\AdAttr{EnteredCurrentActivity}:] Time at which the machine
entered the current Activity (see \AdAttr{Activity} entry above).  On
all platforms (including NT), this is measured in the number of
integer seconds since the Unix epoch (00:00:00 UTC, Jan 1, 1970).
%
\index{ClassAd machine attribute!FileSystemDomain}
\item[\AdAttr{FileSystemDomain}:] A ``domain'' name configured by the
Condor administrator which describes a cluster of machines which all
access the same, uniformly-mounted, networked file systems usually via
NFS or AFS.  This is useful for Vanilla universe jobs which require
remote file access.
%
\index{ClassAd machine attribute!KeyboardIdle}
\item[\AdAttr{KeyboardIdle}:] The number of seconds since activity on any
keyboard or mouse associated with this machine has last been detected.
Unlike \AdAttr{ConsoleIdle}, \AdAttr{KeyboardIdle} also takes activity 
on pseudo-terminals into
account (i.e. virtual ``keyboard'' activity from telnet and rlogin
sessions as well).  Note that \AdAttr{KeyboardIdle} will always be equal to or
less than \AdAttr{ConsoleIdle}.
%
\index{ClassAd machine attribute!KFlops}
\item[\AdAttr{KFlops}:] Relative floating point performance as determined via a
Linpack benchmark.
%
\index{ClassAd machine attribute!LastHeardFrom}
\item[\AdAttr{LastHeardFrom}:] Time when the Condor central manager last
received a status update from this machine.  
Expressed as 
the number of integer seconds since the Unix epoch (00:00:00 UTC, Jan 1, 1970).
Note: This attribute is only inserted by the central manager once it
receives the ClassAd.
It is not present in the \Condor{startd} copy of the ClassAd.
Therefore, you could not use this attribute in defining \Condor{startd}
expressions (and you would not want to).
%
\index{ClassAd machine attribute!LoadAvg}
\item[\AdAttr{LoadAvg}:] A floating point number with the machine's current load
average.
%
\index{ClassAd machine attribute!Machine}
\item[\AdAttr{Machine}:] A string with the machine's fully qualified hostname.
%
\index{ClassAd machine attribute!Memory}
\item[\AdAttr{Memory}:] The amount of RAM in megabytes.
%
\index{ClassAd machine attribute!Mips}
\item[\AdAttr{Mips}:] Relative integer performance as determined via a Dhrystone
benchmark.
%
\index{ClassAd machine attribute!MyType}
\item[\AdAttr{MyType}:] The ClassAd type; always set to the literal string \AdStr{Machine}.
%
\index{ClassAd machine attribute!Name}
\item[\AdAttr{Name}:] The name of this resource; typically the same value as
the \AdAttr{Machine} attribute, but could be customized by the site
administrator.
On SMP machines, the \Condor{startd} will divide the CPUs up into separate
virtual machines, each with with a unique name.
These names will be of the form ``vm\#@full.hostname'', for example,
``vm1@vulture.cs.wisc.edu'', which signifies virtual machine 1 from
vulture.cs.wisc.edu. 
%
\index{ClassAd machine attribute!OpSys}
\item[\AdAttr{OpSys}:] String describing the operating system running on this
machine.  For Condor \VersionNotice\ typically one of the following:
	\begin{description}
	\item[\AdStr{HPUX10}:] for HPUX 10.20
	\item[\AdStr{HPUX11}:] for HPUX B.11.00
	\item[\AdStr{IRIX6}:] for IRIX 6.2, 6.3, or 6.4
	\item[\AdStr{IRIX65}:] for IRIX 6.5
	\item[\AdStr{IRIX62}:] for IRIX 6.2
	\item[\AdStr{LINUX}:] for LINUX 2.0.x, LINUX 2.2.x,
	LINUX 2.4.x, or LINUX 2.6.x kernel systems
	\item[\AdStr{OSF1}:] for Digital Unix 4.x
	\item[\AdStr{SOLARIS25}:] for Solaris 2.4 or 5.5
	\item[\AdStr{SOLARIS251}:] for Solaris 2.5.1 or 5.5.1
	\item[\AdStr{SOLARIS26}:] for Solaris 2.6 or 5.6
	\item[\AdStr{SOLARIS27}:] for Solaris 2.7 or 5.7
	\item[\AdStr{SOLARIS28}:] for Solaris 2.8 or 5.8
	\item[\AdStr{SOLARIS29}:] for Solaris 2.9 or 5.9
	\item[\AdStr{WINNT50}:] for Windows 2000
	\item[\AdStr{WINNT51}:] for Windows XP
	\item[\AdStr{WINNT52}:] for Windows Server 2003
	\item[\AdStr{OSX}:] for Darwin
	\item[\AdStr{OSX10\_2}:] for Darwin 6.4
	\end{description}
%
\index{ClassAd machine attribute!Requirements}
\item[\AdAttr{Requirements}:] A boolean, which when evaluated within the context
of the machine ClassAd and a job ClassAd, must evaluate to
TRUE before Condor will allow the job to use this machine.
%
\index{ClassAd machine attribute!MaxJobRetirementTime}
\item[\AdAttr{MaxJobRetirementTime}:] An expression giving the
maximum time in seconds that the startd will wait for the job to
finish before kicking it off if it needs to do so.  This is evaluated
in the context of the job ClassAd, so it may refer to job attributes
as well as machine attributes.
%
\index{ClassAd machine attribute!StartdIpAddr}
\item[\AdAttr{StartdIpAddr}:] String with the IP and port address of the
\Condor{startd} daemon which is publishing this machine ClassAd.
%
\index{ClassAd machine attribute!State}
\item[\AdAttr{State}:] String which publishes the machine's Condor state.
Can be:
	\begin{description}
	\item[\AdStr{Owner}:] The machine owner is using the machine, and
it is unavailable to Condor.
	\item[\AdStr{Unclaimed}:] The machine is available to run Condor jobs,
but a good match is either not available or not 
yet found.
	\item[\AdStr{Matched}:] The Condor central manager has found a good
match for this resource, but a Condor scheduler has not yet claimed it.
	\item[\AdStr{Claimed}:] The machine is claimed by a remote
\Condor{schedd} and is probably running a job.
	\item[\AdStr{Preempting}:] A Condor job is being preempted (possibly
via checkpointing) in order to clear the machine for either a higher
priority job or because the machine owner wants the machine back.
	\end{description}   % of State
%
\index{ClassAd machine attribute!TargetType}
\item[\AdAttr{TargetType}:] Describes what type of ClassAd to match with.
Always set to the string literal \AdStr{Job}, because machine ClassAds
always want to be matched with jobs, and vice-versa.
%
\index{ClassAd machine attribute!UidDomain}
\item[\AdAttr{UidDomain}:] a domain name configured by the Condor 
administrator which describes a cluster of machines which all have 
the same \File{passwd} file entries, and therefore all have the same logins.
%
\index{ClassAd machine attribute!VirtualMachineID}
\item[\AdAttr{VirtualMachineID}:] For SMP machines, the integer
that identifies the VM.
The value will be \verb@X@ for the VM with 
\begin{verbatim}
name="vmX@full.hostname"
\end{verbatim}
For non-SMP machines with one virtual machine, the value will be 1.
%
\index{ClassAd machine attribute!VirtualMemory}
\item[\AdAttr{VirtualMemory}:] The amount of currently available virtual memory 
(swap space) expressed in Kbytes.

\end{description}

In addition, there are a few attributes that are automatically
inserted into the machine ClassAd whenever a resource is in the
Claimed state:

\begin{description}

\index{ClassAd machine attribute (in Claimed State)!ClientMachine}
\item[\AdAttr{ClientMachine}:] The hostname of the machine that has
claimed this resource

\index{ClassAd machine attribute (in Claimed State)!RemoteOwner}
\item[\AdAttr{RemoteOwner}:] The name of the user who originally
claimed this resource.

\index{ClassAd machine attribute (in Claimed State)!RemoteUser}
\item[\AdAttr{RemoteUser}:] The name of the user who is currently
using this resource.
In general, this will always be the same as the \AdAttr{RemoteOwner},
but in some cases, a resource can be claimed by one entity that hands
off the resource to another entity which uses it.
In that case, \AdAttr{RemoteUser} would hold the name of the entity
currently using the resource, while \AdAttr{RemoteOwner} would hold
the name of the entity that claimed the resource.

\index{ClassAd machine attribute (in Claimed State)!TotalClaimRunTime}
\item[\AdAttr{TotalClaimRunTime}:] A running total of the amount of
time (in seconds) that all jobs (under the same claim) ran
(have spent in the Claimed/Busy state).


\index{ClassAd machine attribute (in Claimed State)!TotalClaimSuspendTime}
\item[\AdAttr{TotalClaimSuspendTime}:] A running total of the amount of
time (in seconds) that all jobs (under the same claim) have been
suspended (in the Claimed/Suspended state).

\index{ClassAd machine attribute (in Claimed State)!TotalJobRunTime}
\item[\AdAttr{TotalJobRunTime}:] A running total of the amount of
time (in seconds) that a single job ran
(has spent in the Claimed/Busy state).

\index{ClassAd machine attribute (in Claimed State)!TotalJobSuspendTime}
\item[\AdAttr{TotalJobSuspendTime}:] A running total of the amount of
time (in seconds) that a single job has been suspended
(in the Claimed/Suspended state).

\end{description}

There are a few attributes that are only inserted into the
machine ClassAd if a job is currently executing.  
If the resource is claimed but no job are running, none of these
attributes will be defined.

\begin{description}

\index{ClassAd machine attribute (when running)!JobId}
\item[\AdAttr{JobId}:] The job's identifier (for example,
\verb@152.3@), as seen from \Condor{q}
on the submitting machine.

\index{ClassAd machine attribute (when running)!JobStart}
\item[\AdAttr{JobStart}:] The time stamp in integer seconds of when the job began
executing, since the Unix epoch (00:00:00 UTC, Jan 1, 1970).  For idle
machines, the value is UNDEFINED.

\index{ClassAd machine attribute (when running)!LastPeriodicCheckpoint}
\item[\AdAttr{LastPeriodicCheckpoint}:] If the job has performed a
periodic checkpoint, this attribute will be defined and will hold the
time stamp of when the last periodic checkpoint was begun.
If the job has yet to perform a periodic checkpoint, or cannot
checkpoint at all, the \AdAttr{LastPeriodicCheckpoint} attribute will
not be defined.

\end{description}

Finally, the single attribute, 
\Attr{CurrentTime}, is defined by the ClassAd
environment.
\begin{description}
\index{ClassAd attribute!CurrentTime}
\item[\AdAttr{CurrentTime}:] Evaluates to the 
the number of integer seconds since the Unix epoch (00:00:00 UTC, Jan 1, 1970).
\end{description}

\subsubsection{\label{user-man-machad}ClassAd Job Attributes}
\begin{description}

\index{ClassAd!job attributes}

%%% ClassAd attribute: CkptArch
\index{ClassAd job attribute!CkptArch}
\item[\AdAttr{CkptArch}] : String describing the architecture of the machine
where this job last checkpointed.  If the job has never checkpointed,
this attribute is UNDEFINED.

%%% ClassAd attribute: CkptOpSys
\index{ClassAd job attribute!CkptOpSys}
\item[\AdAttr{CkptOpSys}] : String describing the operating system of
the machine where this job last checkpointed.  If the job has never
checkpointed, this attribute is UNDEFINED.

%%% ClassAd attribute: ClusterId
\index{ClassAd job attribute!ClusterId}
\item[\AdAttr{ClusterId}] : Integer cluster identifier for this job.
A ``cluster'' is a group of jobs that were submitted together.  Each
job has its own unique job identifier within the cluster, but shares a
common cluster identifier.

%%% ClassAd attribute: Cmd
\index{ClassAd job attribute!Cmd}
\item[\AdAttr{Cmd}] : The path to and the file name of the job to be executed.

%%% ClassAd attribute: CompletionDate
\index{ClassAd job attribute!CompletionDate}
\item[\AdAttr{CompletionDate}] : The time when the job completed,
or the value 0 if the job has not yet completed.
Measured in the
number of seconds since the epoch (00:00:00 UTC, Jan 1, 1970).

%%% ClassAd attribute: CumulativeSuspensionTime
\index{ClassAd job attribute!CumulativeSuspensionTime}
\item[\AdAttr{CumulativeSuspensionTime}] : A running total of the number of
seconds the job has spent in suspension for the life of the job.

%%% ClassAd attribute: EnteredCurrentStatus
\index{ClassAd job attribute!EnteredCurrentStatus}
\item[\AdAttr{EnteredCurrentStatus}] : An integer containing the
epoch time of when the job entered into its current status
So for example, if the job is on hold, the ClassAd expression
\begin{verbatim}
    CurrentTime - EnteredCurrentStatus
\end{verbatim}
will equal the number of seconds that the job has been on hold.

%%% ClassAd attribute: ExecutableSize
\index{ClassAd job attribute!ExecutableSize}
\item[\AdAttr{ExecutableSize}] : Size of the executable in kbytes.

%%% ClassAd attribute: ExitBySignal
\index{ClassAd job attribute!ExitBySignal}
\item[\AdAttr{ExitBySignal}] : An attribute that is True
When a user job exits via a signal and false otherwise.
It is available for use 
in all universes except the globus universe.

%%% ClassAd attribute: ExitCode
\index{ClassAd job attribute!ExitCode}
\item[\AdAttr{ExitCode}] : When a user job exits by means other than a signal,
this is the exit return code of the user job.
It is available for use 
in all universes except the globus universe.

%%% ClassAd attribute: ExitSignal
\index{ClassAd job attribute!ExitSignal}
\item[\AdAttr{ExitSignal}] : When a user job exits by means of an unhandled 
signal, this attribute takes on the numeric value of the signal.
It is available for use 
in all universes except the globus universe.

%%% ClassAd attribute: ExitStatus
\index{ClassAd job attribute!ExitStatus}
\item[\AdAttr{ExitStatus}] : The way that Condor previously dealt with
a job's exit status.
This attribute should no longer be used.
It is not always accurate in
heterogeneous pools, or if the job exited with a signal.
Instead, see the attributes: \AdAttr{ExitBySignal},
\AdAttr{ExitCode}, and
\AdAttr{ExitSignal}.

%%% ClassAd attribute: HoldKillSig
\index{ClassAd job attribute!HoldKillSig}
\item[\AdAttr{HoldKillSig}] :   Currently only for scheduler universe jobs,
a string containing a name of
a signal to be sent to the job if the job is put on hold.

%%% ClassAd attribute: HoldReason
\index{ClassAd job attribute!HoldReason}
\item[\AdAttr{HoldReason}] :   A string containing a human-readable
message about why a job is on hold.
This is the message that will be displayed in response to
the command \verb@condor\_q -hold@.
It can be used to determine if a job should be released or not.

%%% ClassAd attribute: ImageSize
\index{ClassAd job attribute!ImageSize}
\item[\AdAttr{ImageSize}] : Estimate of the memory image size of the
job in kbytes.  The initial estimate may be specified in the job
submit file.  Otherwise, the initial value is equal to the size of the
executable.  When the job checkpoints, the \AdAttr{ImageSize}
attribute is set to the size of the checkpoint file (since the
checkpoint file contains the job's memory image).
A vanilla universe job's \AdAttr{ImageSize} is recomputed
internally every 15 seconds.

%%% ClassAd attribute: JobPrio
\index{ClassAd job attribute!JobPrio}
\item[\AdAttr{JobPrio}] : Integer priority for this job, set by
\Condor{submit} or \Condor{prio}.  The default value is 0.  The higher
the number, the worse the priority.

%%% ClassAd attribute: JobStartDate
\index{ClassAd job attribute!JobStartDate}
\item[\AdAttr{JobStartDate}] : Time at which the job first began
running.  Measured in the
number of seconds since the epoch (00:00:00 UTC, Jan 1, 1970).

%%% ClassAd attribute: JobStatus
\index{ClassAd job attribute!JobStatus}
\item[\AdAttr{JobStatus}] : Integer which indicates the current
status of the job, where 1 = Idle, 2 = Running, 3 = Removed, 4 =
Completed, and 5 = Held.
\index{job!state}

%%% ClassAd attribute: JobUniverse
\index{ClassAd job attribute!JobUniverse}
\item[\AdAttr{JobUniverse}] : Integer which indicates the job
universe, where 1 = Standard, 4 = PVM, 5 = Vanilla, 7 = Scheduler,
8 = MPI, 9 = Globus, and 10 = Java.
\index{job!universe}
\index{universe!job attribute definitions}

%%% ClassAd attribute: LastCkptServer
\index{ClassAd job attribute!LastCkptServer}
\item[\AdAttr{LastCkptServer}] : Hostname of the last checkpoint
server used by this job.  When a pool is using multiple checkpoint
servers, this tells the job where to find its checkpoint file.

%%% ClassAd attribute: LastCkptTime
\index{ClassAd job attribute!LastCkptTime}
\item[\AdAttr{LastCkptTime}] : Time at which the job last performed a
successful checkpoint.  Measured in the number of seconds since the
epoch (00:00:00 UTC, Jan 1, 1970).

%%% ClassAd attribute: LastMatchTime
\index{ClassAd job attribute!LastMatchTime}
\item[\AdAttr{LastMatchTime}] : An integer containing the epoch time
when the job was last successfully matched with a resource (gatekeeper) Ad.

%%% ClassAd attribute: LastRejMatchReason
\index{ClassAd job attribute!LastRejMatchReason}
\item[\AdAttr{LastRejMatchReason}] :  If, at any point in the past,
this job failed to match with a resource ad,
this attribute will contain a string with a
human-readable message about why the match failed.

%%% ClassAd attribute: LastRejMatchTime
\index{ClassAd job attribute!LastRejMatchTime}
\item[\AdAttr{LastRejMatchTime}] :  An integer containing the epoch
time when Condor-G last tried to find a match for the job,
but failed to do so.

%%% ClassAd attribute: LastSuspensionTime
\index{ClassAd job attribute!LastSuspensionTime}
\item[\AdAttr{LastSuspensionTime}] : Time at which the job last performed a
successful suspension.  Measured in the number of seconds since the
epoch (00:00:00 UTC, Jan 1, 1970).

%%% ClassAd attribute: LastVacateTime
\index{ClassAd job attribute!LastVacateTime}
\item[\AdAttr{LastVacateTime}] : Time at which the job was last
evicted from a remote workstation.  Measured in the number of seconds
since the epoch (00:00:00 UTC, Jan 1, 1970).

%%% ClassAd attribute: NiceUser
\index{ClassAd job attribute!NiceUser}
\item[\AdAttr{NiceUser}] : Boolean value which indicates whether
this is a nice-user job.

%%% ClassAd attribute: NumCkpts
\index{ClassAd job attribute!NumCkpts}
\item[\AdAttr{NumCkpts}] : A count of the number of checkpoints
written by this job during its lifetime.

%%% ClassAd attribute: NumGlobusSubmits
\index{ClassAd job attribute!NumGlobusSubmits}
\item[\AdAttr{NumGlobusSubmits}] :  An integer that is incremented each
time the \Condor{gridmanager} receives confirmation of a successful job
submission into Globus.

%%% ClassAd attribute: NumJobMatches
\index{ClassAd job attribute!NumJobMatches}
\item[\AdAttr{NumJobMatches}] : An integer that is incremented by the
\Condor{schedd} each time the job is matched with a resource ad by the
negotiator.

%%% ClassAd attribute: NumRestarts
\index{ClassAd job attribute!NumRestarts}
\item[\AdAttr{NumRestarts}] : A count of the number of restarts from a
checkpoint attempted by this job during its lifetime.

%%% ClassAd attribute: NumSystemHolds
\index{ClassAd job attribute!NumSystemHolds}
\item[\AdAttr{NumSystemHolds}] :  An integer that is incremented each time
Condor-G places a job on hold due to some sort of error condition.  This
counter is useful, since Condor-G will always place a job on hold when it
gives up on some error condition.  Note that if the user places the job
on hold using the \Condor{hold} command, this attribute is not incremented.

%%% ClassAd attribute: Owner
\index{ClassAd job attribute!Owner}
\item[\AdAttr{Owner}] : String describing the user who submitted this
job.

%%% ClassAd attribute: ProcId
\index{ClassAd job attribute!ProcId}
\item[\AdAttr{ProcId}] : Integer process identifier for this job.  In
a cluster of many jobs, each job will have the same ClusterId but will
have a unique ProcId.

%%% ClassAd attribute: QDate
\index{ClassAd job attribute!QDate}
\item[\AdAttr{QDate}] : Time at which the job was submitted to the job
queue.  Measured in the
number of seconds since the epoch (00:00:00 UTC, Jan 1, 1970).

%%% ClassAd attribute: ReleaseReason
\index{ClassAd job attribute!ReleaseReason}
\item[\AdAttr{ReleaseReason}] :    A string containing a human-readable
message about why the job was released from hold.

%%% ClassAd attribute: RemoteIwd
\index{ClassAd job attribute!RemoteIwd}
\item[\AdAttr{RemoteIwd}] : The path to the directory in which
a job is to be executed on a remote machine.

%%% ClassAd attribute: RemoteSysCpu
\index{ClassAd job attribute!RemoteSysCpu}
\item[\AdAttr{RemoteSysCpu}] : The total number of seconds
of system CPU time (the time spent at system calls) the job used
on remote machines.

%%% ClassAd attribute: RemoteUserCpu
\index{ClassAd job attribute!RemoteUserCpu}
\item[\AdAttr{RemoteUserCpu}] : The total number of seconds
of user CPU time the job used on remote machines.

%%% ClassAd attribute: RemoteWallClockTime
\index{ClassAd job attribute!RemoteWallClockTime}
\item[\AdAttr{RemoteWallClockTime}] : Cumulative number of seconds
the job has been allocated a machine.
This also includes time spent in suspension (if any),
so the total real time spent running is 
\begin{verbatim}
RemoteWallClockTime - CumulativeSuspensionTime
\end{verbatim}
Note that this number does not get reset to
zero when a job is forced to migrate from one machine to another.

%%% ClassAd attribute: RemoveKillSig
\index{ClassAd job attribute!RemoveKillSig}
\item[\AdAttr{RemoveKillSig}] :   Currently only for scheduler universe jobs,
a string containing a name of
a signal to be sent to the job if the job is removed.

%%% ClassAd attribute: StreamErr
\index{ClassAd job attribute!StreamErr}
\item[\AdAttr{StreamErr}] :  
An attribute utilized only for globus universe jobs.
The default value is \Arg{True}.
If \Arg{True}, and \Attr{TransferErr} is \Arg{True}, then 
standard error is streamed back to the submit machine, instead
of doing the transfer (as a whole) after the job completes.
If \Arg{False}, then
standard error is transfered back to the submit machine
(as a whole) after the job completes.
If \Attr{TransferErr} is \Arg{False}, then this job attribute is ignored.

%%% ClassAd attribute: StreamOut
\index{ClassAd job attribute!StreamOut}
\item[\AdAttr{StreamOut}] :  
An attribute utilized only for globus universe jobs.
The default value is \Arg{True}.
If \Arg{True}, and \Attr{TransferOut} is \Arg{True}, then 
job output is streamed back to the submit machine, instead
of doing the transfer (as a whole) after the job completes.
If \Arg{False}, then
job output is transferred back to the submit machine
(as a whole) after the job completes.
If \Attr{TransferOut} is \Arg{False}, then this job attribute is ignored.

%%% ClassAd attribute: TotalSuspensions
\index{ClassAd job attribute!TotalSuspensions}
\item[\AdAttr{TotalSuspensions}] : A count of the number of times this job
has been suspended during its lifetime.

%%% ClassAd attribute: TransferErr
\index{ClassAd job attribute!TransferErr}
\item[\AdAttr{TransferErr}] :  
An attribute utilized only for globus universe jobs.
The default value is \Arg{True}.
If \Arg{True}, then the error output from the job
is transferred from the remote machine back to the submit machine.
The name of the file after transfer is the file referred to
by job attribute \Attr{Err}.
If \Arg{False}, no transfer takes place (remote to submit machine),
and the name of the file is the file referred to
by job attribute \Attr{Err}.

%%% ClassAd attribute: TransferExecutable
\index{ClassAd job attribute!TransferExecutable}
\item[\AdAttr{TransferExecutable}] :  
An attribute utilized only for globus universe jobs.
The default value is \Arg{True}.
If \Arg{True}, then the job executable is transferred from the submit
machine to the remote machine.
The name of the file (on the submit machine)
that is transferred is given by the
job attribute \Attr{Cmd}.
If \Arg{False}, no transfer takes place, and
the name of the file used (on the remote machine) will be as
given in the job attribute \Attr{Cmd}.

%%% ClassAd attribute: TransferIn
\index{ClassAd job attribute!TransferIn}
\item[\AdAttr{TransferIn}] :  
An attribute utilized only for globus universe jobs.
The default value is \Arg{True}.
If \Arg{True}, then the job input is transferred from the submit
machine to the remote machine.
The name of the file that is transferred is given by the
job attribute \Attr{In}.
If \Arg{False}, then the job's input is taken from a file on the
remote machine (pre-staged), and 
the name of the file is given by the job attribute \Attr{In}.

%%% ClassAd attribute: TransferOut
\index{ClassAd job attribute!TransferOut}
\item[\AdAttr{TransferOut}] :  
An attribute utilized only for globus universe jobs.
The default value is \Arg{True}.
If \Arg{True}, then the output from the job
is transferred from the remote machine back to the submit machine.
The name of the file after transfer is the file referred to
by job attribute \Attr{Out}.
If \Arg{False}, no transfer takes place (remote to submit machine),
and the name of the file is the file referred to
by job attribute \Attr{Out}.

\end{description}




%%%%%%%%%%%% 
\subsection{Heterogeneous Submit: Execution on Differing Architectures} 
%%%%%%%%%%%%

\index{job!heterogeneous submit}
\index{running a job!on a different architecture}
\index{heterogeneous pool!submitting a job to}
If executables are available for the different platforms of machines
in the Condor pool,
Condor can be allowed the choice of a larger number of machines
when allocating a machine for a job.
Modifications to the submit description file allow this choice
of platforms.

A simplified example is a cross submission.
An executable is available for one platform, but
the submission is done from a different platform.
Given the correct executable, the \AdAttr{requirements} command in
the submit description file specifies the target architecture.
For example, an executable compiled for a Sun 4, submitted
from an Intel architecture running Linux would add the 
\AdAttr{requirement}
\begin{verbatim}
  requirements = Arch == "SUN4x" && OpSys == "SOLARIS251"
\end{verbatim}
Without this \AdAttr{requirement}, \Condor{submit}
will assume that the program is to be executed on
a machine with the same platform as the machine where the job
is submitted.

Cross submission works for both
\Expr{standard} and \Expr{vanilla} universes.
The burden is on the user to both obtain and specify
the correct executable for the target architecture.
To list the architecture and operating systems of the machines
in a pool, run \Condor{status}.

%%%%%%%%%%%% 
\subsection{Vanilla Universe Example for Execution on Differing Architectures} 
%%%%%%%%%%%%

A more complex example of a heterogeneous submission
occurs when a job may be executed on
many different architectures to gain full
use of a diverse architecture and operating system pool.
If the executables are available for the different architectures,
then a modification to the submit description file
will allow Condor to choose an executable after an
available machine is chosen.

A special-purpose MachineAd substitution macro can be used in
the \AdAttr{executable}, \AdAttr{environment},  and \AdAttr{arguments}
attributes in the submit description file.
The macro has the form
\begin{verbatim}
  $$(MachineAdAttribute)
\end{verbatim}
Note that this macro is ignored in all other submit description attributes.
The \$\$() informs Condor to substitute the requested 
\AdAttr{MachineAdAttribute} 
from the machine where the job will be executed.

An example of the heterogeneous job submission
has executables available for three platforms:
LINUX Intel, Solaris26 Intel, and Irix 6.5 SGI machines.
This example uses \Prog{povray}
to render images using a popular free rendering engine.

The substitution macro chooses a specific executable after
a platform for running the job is chosen.
These executables must therefore be named based on the
machine attributes that describe a platform.
The executables named \begin{verbatim}
  povray.LINUX.INTEL
  povray.SOLARIS26.INTEL
  povray.IRIX65.SGI
\end{verbatim}
will work correctly for the macro
\begin{verbatim}
  povray.$$(OpSys).$$(Arch)
\end{verbatim}

The executables or links to executables with this name
are placed into the initial working directory so that they may be
found by Condor. 
A submit description file that queues three jobs for this example:

\begin{verbatim}
  ####################
  #
  # Example of heterogeneous submission
  #
  ####################

  universe     = vanilla
  Executable   = povray.$$(OpSys).$$(Arch)
  Log          = povray.log
  Output       = povray.out.$(Process)
  Error        = povray.err.$(Process)

  Requirements = (Arch == "INTEL" && OpSys == "LINUX") || \
                 (Arch == "INTEL" && OpSys =="SOLARIS26") || \
                 (Arch == "SGI" && OpSys == "IRIX65")

  Arguments    = +W1024 +H768 +Iimage1.pov
  Queue 

  Arguments    = +W1024 +H768 +Iimage2.pov
  Queue 

  Arguments    = +W1024 +H768 +Iimage3.pov
  Queue 
\end{verbatim}

These jobs are submitted to the vanilla universe
to assure that once a job is started on a specific platform,
it will finish running on that platform.
Switching platforms in the middle of job execution cannot
work correctly.

There are two common errors made with the substitution macro.
The first is the use of a non-existent \AdAttr{MachineAdAttribute}.
If the specified \AdAttr{MachineAdAttribute} does not
exist in the machine's ClassAd, then Condor will place
the job in the machine state of hold until the problem is resolved.

The second common error occurs due to an incomplete job set up.
For example, the submit description file given above specifies
three available executables.
If one is missing, Condor report back that an
executable is missing when it happens to match the
job with a resource that requires the missing binary.

%%%%%%%%%%%% 
\subsection{Standard Universe Example for Execution on Differing Architectures} 
%%%%%%%%%%%%

Jobs submitted to the standard universe may produce checkpoints.
A checkpoint can then be used to start up and continue execution
of a partially completed job.
For a partially completed job, the checkpoint and the job are specific
to a platform.
If migrated to a different machine, correct execution requires that
the platform must remain the same.

In previous versions of Condor, the author of the heterogeneous
submission file would need to write extra policy expressions in the
\AdAttr{requirements} expression to force Condor to choose the
same type of platform when continuing a checkpointed job.
However, since it is needed in the common case, this
additional policy is now automatically added
to the \AdAttr{requirements} expression.
The additional expression is added
provided the user does not use
\AdAttr{CkptArch} in the \AdAttr{requirements} expression.
Condor will remain backwards compatible for those users who have explicitly
specified \AdAttr{CkptRequirements}--implying use of \AdAttr{CkptArch},
in their \AdAttr{requirements} expression.

The expression added when the attribute \AdAttr{CkptArch} is not specified 
will default to

\begin{verbatim}
  # Added by Condor
  CkptRequirements = ((CkptArch == Arch) || (CkptArch =?= UNDEFINED)) && \
                      ((CkptOpSys == OpSys) || (CkptOpSys =?= UNDEFINED))

  Requirements = (<user specified policy>) && $(CkptRequirements)
\end{verbatim}

% DONE TO HERE
The behavior of the \AdAttr{CkptRequirements} expressions and its addition to
\AdAttr{requirements} is as follows.
The \AdAttr{CkptRequirements} expression guarantees correct operation
in the two possible cases for a job.
In the first case, the job has not produced a checkpoint.
The ClassAd attributes \Attr{CkptArch} and \Attr{CkptOpSys}
will be undefined, and therefore the meta operator (\verb@=?=@)
evaluates to true.
In the second case, the job has produced a checkpoint.
The Machine ClassAd is restricted to require further execution
only on a machine of the same platform.
The attributes \Attr{CkptArch} and \Attr{CkptOpSys}
will be defined, ensuring that the platform chosen for further
execution will be the same as the one used just before the
checkpoint.

Note that this restriction of platforms also applies to platforms where
the executables are binary compatible.

The complete submit description file for this example:

\begin{verbatim}
  ####################
  #
  # Example of heterogeneous submission
  #
  ####################

  universe     = standard
  Executable   = povray.$$(OpSys).$$(Arch)
  Log          = povray.log
  Output       = povray.out.$(Process)
  Error        = povray.err.$(Process)

  # Condor automatically adds the correct expressions to insure that the
  # checkpointed jobs will restart on the correct platform types.
  Requirements = ( (Arch == "INTEL" && OpSys == "LINUX") || \
                 (Arch == "INTEL" && OpSys =="SOLARIS26") || \
                 (Arch == "SGI" && OpSys == "IRIX65") )

  Arguments    = +W1024 +H768 +Iimage1.pov
  Queue 

  Arguments    = +W1024 +H768 +Iimage2.pov
  Queue 

  Arguments    = +W1024 +H768 +Iimage3.pov
  Queue 
\end{verbatim}

%%%%%%%%%%%%%%%%%%%%%%%%%%%%%%%%%%%%%%%%%%
\section{Managing a Condor Job}
This section provides a brief summary of what can be done once jobs
are submitted. The basic mechanisms for monitoring a job are
introduced, but the commands are discussed briefly.
You are encouraged to
look at the man pages of the commands referred to (located in
% Karen changed this by adding sec: to both lines
Chapter~\ref{sec:command-reference} beginning on
page~\pageref{sec:command-reference}) for more information. 

When jobs are submitted, Condor will attempt to find resources
to run the jobs. 
A list of all those with jobs submitted
may be obtained through \Condor{status}
\index{Condor commands!condor\_status}
with the 
\Arg{-submitters} option. 
An example of this would yield output similar to:
\footnotesize
\begin{verbatim}
%  condor_status -submitters

Name                 Machine      Running IdleJobs HeldJobs

ballard@cs.wisc.edu  bluebird.c         0       11        0
nice-user.condor@cs. cardinal.c         6      504        0
wright@cs.wisc.edu   finch.cs.w         1        1        0
jbasney@cs.wisc.edu  perdita.cs         0        0        5

                           RunningJobs           IdleJobs           HeldJobs

 ballard@cs.wisc.edu                 0                 11                  0
 jbasney@cs.wisc.edu                 0                  0                  5
nice-user.condor@cs.                 6                504                  0
  wright@cs.wisc.edu                 1                  1                  0

               Total                 7                516                  5
\end{verbatim}
\normalsize

%%%%%%%%%%%%%%%%%%%%%%%%%%%%%%%%%%%%%%%%%%%%%%%%%%%%%%%%%%%%%%%%%%%%%%
\subsection{Checking on the progress of jobs}
%%%%%%%%%%%%%%%%%%%%%%%%%%%%%%%%%%%%%%%%%%%%%%%%%%%%%%%%%%%%%%%%%%%%%%
At any time, you can check on the status of your jobs with the \Condor{q}
command.
\index{Condor commands!condor\_q}
This command displays the status of all queued jobs.
An example of the output from \Condor{q} is
\footnotesize
\begin{verbatim}
%  condor_q

-- Submitter: submit.chtc.wisc.edu : <128.104.55.9:32772> : submit.chtc.wisc.edu
 ID      OWNER            SUBMITTED     RUN_TIME ST PRI SIZE CMD               
711197.0   aragorn         1/15 19:18   0+04:29:33 H  0   0.0  script.sh         
894381.0   frodo           3/16 09:06  82+17:08:51 R  0   439.5 elk elk.in        
894386.0   frodo           3/16 09:06  82+20:21:28 R  0   219.7 elk elk.in        
894388.0   frodo           3/16 09:06  81+17:22:10 R  0   439.5 elk elk.in        
1086870.0   gollum          4/27 09:07   0+00:10:14 I  0   7.3  condor_dagman     
1086874.0   gollum          4/27 09:08   0+00:00:01 H  0   0.0  RunDC.bat         
1297254.0   legolas         5/31 18:05  14+17:40:01 R  0   7.3  condor_dagman     
1297255.0   legolas         5/31 18:05  14+17:39:55 R  0   7.3  condor_dagman     
1297256.0   legolas         5/31 18:05  14+17:39:55 R  0   7.3  condor_dagman     
1297259.0   legolas         5/31 18:05  14+17:39:55 R  0   7.3  condor_dagman     
1297261.0   legolas         5/31 18:05  14+17:39:55 R  0   7.3  condor_dagman     
1302278.0   legolas         6/4  12:22   1+00:05:37 I  0   390.6 mdrun_1.sh        
1304740.0   legolas         6/5  00:14   1+00:03:43 I  0   390.6 mdrun_1.sh        
1304967.0   legolas         6/5  05:08   0+00:00:00 I  0   0.0  mdrun_1.sh        

14 jobs; 4 idle, 8 running, 2 held

\end{verbatim} 
\normalsize
This output contains many columns of information about the
queued jobs.
\index{status!of queued jobs}
\index{job!state}
The \verb@ST@ column (for status) shows the status of
current jobs in the queue:
\begin{description} 
  \item{\verb@R@}:  The job is currently running.
  \item{\verb@I@}:  The job is idle.  It is not running right
now, because it is waiting for a machine to become available.
  \item{\verb@H@}:  The job is the hold state. In the hold state,
the job will not be scheduled to
run until it is released. See the \Condor{hold}
manual page located on page~\pageref{man-condor-hold}
and the \Condor{release}
manual page located on page~\pageref{man-condor-release}.
\end{description} 
The \verb@RUN_TIME@ time reported for a job is the time that has been
committed to the job.

Another useful method of tracking the progress of jobs is through the
user log.  If you have specified a \AdAttr{log} command in 
your submit file, the progress of the job may be followed by viewing the
log file.  Various events such as execution commencement, checkpoint, eviction 
and termination are logged in the file.
Also logged is the time at which the event occurred.

When a job begins to run, Condor starts up a \Condor{shadow} process
\index{condor\_shadow}
\index{remote system call!condor\_shadow}
on the submit machine.  The shadow process is the mechanism by which the
remotely executing jobs can access the environment from which it was
submitted, such as input and output files.  

It is normal for a machine which has submitted hundreds of jobs to have 
hundreds of \Condor{shadow} processes running on the machine.
Since the text segments of all these processes is the same,
the load on the submit machine is usually not significant.
If there is degraded performance, limit 
the number of jobs that can run simultaneously by reducing the 
\Macro{MAX\_JOBS\_RUNNING} configuration variable.

You can also find all the machines that are running your job through the
\Condor{status} command.
\index{Condor commands!condor\_status}
For example, to find all the machines that are
running jobs submitted by \Expr{breach@cs.wisc.edu}, type:
\footnotesize
\begin{verbatim}
%  condor_status -constraint 'RemoteUser == "breach@cs.wisc.edu"'

Name       Arch     OpSys        State      Activity   LoadAv Mem  ActvtyTime

alfred.cs. INTEL    LINUX        Claimed    Busy       0.980  64    0+07:10:02
biron.cs.w INTEL    LINUX        Claimed    Busy       1.000  128   0+01:10:00
cambridge. INTEL    LINUX        Claimed    Busy       0.988  64    0+00:15:00
falcons.cs INTEL    LINUX        Claimed    Busy       0.996  32    0+02:05:03
happy.cs.w INTEL    LINUX        Claimed    Busy       0.988  128   0+03:05:00
istat03.st INTEL    LINUX        Claimed    Busy       0.883  64    0+06:45:01
istat04.st INTEL    LINUX        Claimed    Busy       0.988  64    0+00:10:00
istat09.st INTEL    LINUX        Claimed    Busy       0.301  64    0+03:45:00
...
\end{verbatim}
\normalsize
To find all the machines that are running any job at all, type:
\footnotesize
\begin{verbatim}
%  condor_status -run

Name       Arch     OpSys        LoadAv RemoteUser           ClientMachine  

adriana.cs INTEL    LINUX        0.980  hepcon@cs.wisc.edu   chevre.cs.wisc.
alfred.cs. INTEL    LINUX        0.980  breach@cs.wisc.edu   neufchatel.cs.w
amul.cs.wi X86_64   LINUX        1.000  nice-user.condor@cs. chevre.cs.wisc.
anfrom.cs. X86_64   LINUX        1.023  ashoks@jules.ncsa.ui jules.ncsa.uiuc
anthrax.cs INTEL    LINUX        0.285  hepcon@cs.wisc.edu   chevre.cs.wisc.
astro.cs.w INTEL    LINUX        1.000  nice-user.condor@cs. chevre.cs.wisc.
aura.cs.wi X86_64   WINDOWS      0.996  nice-user.condor@cs. chevre.cs.wisc.
balder.cs. INTEL    WINDOWS      1.000  nice-user.condor@cs. chevre.cs.wisc.
bamba.cs.w INTEL    LINUX        1.574  dmarino@cs.wisc.edu  riola.cs.wisc.e
bardolph.c INTEL    LINUX        1.000  nice-user.condor@cs. chevre.cs.wisc.
...
\end{verbatim}
\normalsize

%%%%%%%%%%%%%%%%%%%%%%%%%%%%%%%%%%%%%%%%%%%%%%%%%%%%%%%%%%%%%%%%%%%%%%
\subsection{Removing a job from the queue}
%%%%%%%%%%%%%%%%%%%%%%%%%%%%%%%%%%%%%%%%%%%%%%%%%%%%%%%%%%%%%%%%%%%%%%
A job can be removed from the queue at any time by using the \Condor{rm}
\index{Condor commands!condor\_rm}
command.  If the job that is being removed is currently running, the job
is killed without a checkpoint, and its queue entry is removed.  
The following example shows the queue of jobs before and after
a job is removed.
\footnotesize
\begin{verbatim}
%  condor_q

-- Submitter: froth.cs.wisc.edu : <128.105.73.44:33847> : froth.cs.wisc.edu
 ID      OWNER            SUBMITTED    CPU_USAGE ST PRI SIZE CMD               
 125.0   jbasney         4/10 15:35   0+00:00:00 I  -10 1.2  hello.remote      
 132.0   raman           4/11 16:57   0+00:00:00 R  0   1.4  hello             

2 jobs; 1 idle, 1 running, 0 held

%  condor_rm 132.0
Job 132.0 removed.

%  condor_q

-- Submitter: froth.cs.wisc.edu : <128.105.73.44:33847> : froth.cs.wisc.edu
 ID      OWNER            SUBMITTED    CPU_USAGE ST PRI SIZE CMD               
 125.0   jbasney         4/10 15:35   0+00:00:00 I  -10 1.2  hello.remote      

1 jobs; 1 idle, 0 running, 0 held
\end{verbatim}
\normalsize

%%%%%%%%%%%%%%%%%%%%%%%%%%%%%%%%%%%%%%%%%%%%%%%%%%%%%%%%%%%%%%%%%%%%%%
\subsection{Placing a job on hold}
%%%%%%%%%%%%%%%%%%%%%%%%%%%%%%%%%%%%%%%%%%%%%%%%%%%%%%%%%%%%%%%%%%%%%%
\index{Condor commands!condor\_hold}
\index{Condor commands!condor\_release}
\index{job!state}
A job in the queue may be placed on hold by running the command
\Condor{hold}.
A job in the hold state remains in the hold state until later released
for execution by the command \Condor{release}.

Use of the \Condor{hold} command causes a hard kill signal to be sent
to a currently running job (one in the running state).
For a standard universe job, this means that no checkpoint is
generated before the job stops running and enters the hold state.
When released, this standard universe job continues its execution
using the most recent checkpoint available.

Jobs in universes other than the standard universe that are running
when placed on hold will start over from the beginning when 
released.

The manual page for \Condor{hold}
on page~\pageref{man-condor-hold}
and the manual page for \Condor{release}
on page~\pageref{man-condor-release}
contain usage details.

%%%%%%%%%%%%%%%%%%%%%%%%%%%%%%%%%%%%%%%%%%%%%%%%%%%%%%%%%%%%%%%%%%%%%%
\subsection{\label{sec:job-prio}Changing the priority of jobs}
%%%%%%%%%%%%%%%%%%%%%%%%%%%%%%%%%%%%%%%%%%%%%%%%%%%%%%%%%%%%%%%%%%%%%%

\index{job!priority}
\index{priority!of a job}
In addition to the priorities assigned to each user, Condor also provides
each user with the capability of assigning priorities to each submitted job.
These job priorities are local to each queue and can be any integer value, with
higher values meaning better priority.

The default priority of a job is 0, but can be changed using the \Condor{prio}
command.
\index{Condor commands!condor\_prio}
For example, to change the priority of a job to -15,
\footnotesize
\begin{verbatim}
%  condor_q raman

-- Submitter: froth.cs.wisc.edu : <128.105.73.44:33847> : froth.cs.wisc.edu
 ID      OWNER            SUBMITTED    CPU_USAGE ST PRI SIZE CMD               
 126.0   raman           4/11 15:06   0+00:00:00 I  0   0.3  hello             

1 jobs; 1 idle, 0 running, 0 held

%  condor_prio -p -15 126.0

%  condor_q raman

-- Submitter: froth.cs.wisc.edu : <128.105.73.44:33847> : froth.cs.wisc.edu
 ID      OWNER            SUBMITTED    CPU_USAGE ST PRI SIZE CMD               
 126.0   raman           4/11 15:06   0+00:00:00 I  -15 0.3  hello             

1 jobs; 1 idle, 0 running, 0 held
\end{verbatim}
\normalsize

It is important to note that these \emph{job} priorities are completely 
different from the \emph{user} priorities assigned by Condor.  Job priorities
do not impact user priorities.  They are only a mechanism for the user to
identify the relative importance of jobs among all the jobs submitted by the
user to that specific queue.

%%%%%%%%%%%%%%%%%%%%%%%%%%%%%%%%%%%%%%%%%%%%%%%%%%%%%%%%%%%%%%%%%%%%%%
\subsection{\label{sec:job-not-running}Why is the job not running?}
%%%%%%%%%%%%%%%%%%%%%%%%%%%%%%%%%%%%%%%%%%%%%%%%%%%%%%%%%%%%%%%%%%%%%%
\index{job!analysis}
\index{job!not running}
Users occasionally find that their jobs do not run.
There are many possible reasons why a specific job is not running.
The following prose attempts to identify some of the potential issues
behind why a job is not running.

At the most basic level, the user knows the status of a job by
using \Condor{q} to see that the job is not running.
By far, the most common reason (to the novice Condor job submitter)
why the job is not running is that Condor has not yet 
been through its periodic negotiation cycle,
in which queued jobs are assigned to machines within the pool 
and begin their execution.
This periodic event occurs by default once every 5 minutes,
implying that the user ought to wait a few minutes before
searching for reasons why the job is not running.

Further inquiries are dependent on whether the job has 
never run at all, or has run for at least a little bit.

For jobs that have never run,
\index{Condor commands!condor\_q}
many problems can be diagnosed by using the \Opt{-analyze}
option of the \Condor{q} command.
For example, a job (assigned the cluster.process value of
121.000) submitted to the local pool at UW-Madison
is not running.
Running \Condor{q}'s analyzer provided the following information:

\footnotesize
\begin{verbatim}
% condor_q -pool -analyze 121.000
-- Submitter: puffin.cs.wisc.edu : <128.105.185.14:34203> : puffin.cs.wisc.edu
---
121.000:  Run analysis summary.  Of 1592 machines,
   1382 are rejected by your job's requirements
     25 reject your job because of their own requirements
    185 match but are serving users with a better priority in the pool
      0 match but reject the job for unknown reasons
      0 match but will not currently preempt their existing job
      0 match but are currently offline
      0 are available to run your job

The Requirements expression for your job is:
( ( target.Arch == "X86_64" || target.Arch == "INTEL" ) &&
( target.Group == "TestPool" ) ) && ( target.OpSys == "LINUX" ) &&
( target.Disk >= DiskUsage ) && ( ( target.Memory * 1024 ) >= ImageSize ) &&
( TARGET.FileSystemDomain == MY.FileSystemDomain )

    Condition                         Machines Matched    Suggestion
    ---------                         ----------------    ----------
1   ( target.Group == "TestPool" )    274                  
2   ( TARGET.FileSystemDomain == "cs.wisc.edu" )1258                 
3   ( target.OpSys == "LINUX" )       1453                 
4   ( target.Arch == "X86_64" || target.Arch == "INTEL" )
                                      1573                 
5   ( target.Disk >= 100000 )         1589                 
6   ( ( 1024 * target.Memory ) >= 100000 )1592                 

The following attributes are missing from the job ClassAd:

CheckpointPlatform

\end{verbatim}
\normalsize

This example also shows that the job does not run because the job
does not have a high enough priority to cause any of 185 other running jobs
to be preempted.

While the analyzer can diagnose most common problems, there are some situations
that it cannot reliably detect due to the instantaneous and local nature of the
information it uses to detect the problem.  Thus, it may be that the analyzer
reports that resources are available to service the request, but the job still 
has not run.  In most of these situations, the delay is transient, and the
job will run following the next negotiation cycle.

A second class of problems represents jobs that do or did run,
for at least a short while, but are no longer running.
The first issue is identifying whether the job is in this category.
The \Condor{q} command is not enough; it only tells the
current state of the job.
The needed information will be in the \SubmitCmd{log} file 
or the \SubmitCmd{error} file, as defined in the submit description file
for the job.
If these files are not defined, then there is little hope of
determining if the job ran at all.
For a job that ran, even for the briefest amount of time,
the \SubmitCmd{log} file will contain an event of type 1,
which will contain the string
\verb@Job executing on host@.

A job may run for a short time, before failing due to a file permission
problem.
The log file used by the \Condor{shadow} daemon will contain more information
if this is the problem.
This log file is associated with the machine on which the job was submitted.
The location and name of this log file may be discovered on the
submitting machine, using the command
\footnotesize
\begin{verbatim}
%  condor_config_val SHADOW_LOG
\end{verbatim}
\normalsize

Memory and swap space problems may be identified by looking at the log
file used by the \Condor{schedd} daemon.
The location and name of this log file may be discovered on the
submitting machine, using the command
\footnotesize
\begin{verbatim}
%  condor_config_val SCHEDD_LOG
\end{verbatim}
\normalsize
A swap space problem will show in the log with the following message:
\footnotesize
\begin{verbatim}
2/3 17:46:53 Swap space estimate reached! No more jobs can be run!
12/3 17:46:53     Solution: get more swap space, or set RESERVED_SWAP = 0
12/3 17:46:53     0 jobs matched, 1 jobs idle
\end{verbatim}
\normalsize
As an explanation,
Condor computes the total swap space on the submit machine.
It then tries to limit the total number of jobs it
will spawn based on an estimate of the size of the \Condor{shadow}
daemon's memory footprint and a configurable amount of swap space
that should be reserved.
This is done to avoid the
situation within a very large pool
in which all the jobs are submitted from a single host.
The huge number of \Condor{shadow} processes would
overwhelm the submit machine,
and it would run out of swap space and thrash.

Things can go wrong if a machine has a lot of physical memory and
little or no swap space.
Condor does not consider the physical memory size,
so the situation occurs where Condor thinks
it has no swap space to work with,
and it will not run the submitted jobs.

To see how much swap space Condor thinks a given machine has, use
the output of a \Condor{status} command of the following form:

\footnotesize
\begin{verbatim}
% condor_status -schedd [hostname] -long | grep VirtualMemory
\end{verbatim}
\normalsize
If the value listed is 0, then this is what is confusing Condor.
There are two ways to fix the problem:

\begin{enumerate}
\item Configure the machine with some real swap space.

\item Disable this check within Condor.
Define the amount of reserved swap space for the submit machine to 0.
Set \Macro{RESERVED\_SWAP} to 0 in the configuration file:

\begin{verbatim}
RESERVED_SWAP = 0
\end{verbatim}

and then send a \Condor{restart} to the submit machine.
\end{enumerate}



%%%%%%%%%%%%%%%%%%%%%%%%%%%%%%%%%%%%%%%%%%%%%%%%%%%%%%%%%%%%%%%%%%%%%%
\subsection{\label{sec:job-log-events}In the Log File}
%%%%%%%%%%%%%%%%%%%%%%%%%%%%%%%%%%%%%%%%%%%%%%%%%%%%%%%%%%%%%%%%%%%%%%
\index{job!log events}
\index{log files!event descriptions}
In a job's log file are a listing of events in
chronological order that occurred during the life of the job.
The formatting of the events is always the same, 
so that they may be machine readable.
Four fields are always present,
and they will most often be followed by other fields that give further
information that is specific to the type of event.

The first field in an event is the numeric value assigned as the
event type in a 3-digit format.
The second field identifies the job which generated the event. 
Within parentheses are the ClassAd job attributes of
\AdAttr{ClusterId} value, 
\AdAttr{ProcId} value, 
and the node number  for parallel universe jobs or a set of zeros
(for jobs run under all other universes),
separated by periods.
The third field is the date and time of the event logging.  
The fourth field is a string that briefly describes the event.
Fields that follow the fourth field give further information for the specific
event type.

These are all of the events that can show up in a job log file:

\noindent\Bold{Event Number:} 000 \\
\Bold{Event Name:} Job submitted \\
\Bold{Event Description:} This event occurs when a user submits a job.
It is the first event you will see for a job, and it should only occur
once. 

\noindent\Bold{Event Number:} 001 \\
\Bold{Event Name:} Job executing \\
\Bold{Event Description:} This shows up when a job is running.
It might occur more than once.

\noindent\Bold{Event Number:} 002 \\
\Bold{Event Name:} Error in executable \\
\Bold{Event Description:} The job could not be run because the
executable was bad.

\noindent\Bold{Event Number:} 003 \\
\Bold{Event Name:} Job was checkpointed \\
\Bold{Event Description:} The job's complete state was written to a checkpoint
file.  
This might happen without the job being removed from a machine,
because the checkpointing can happen periodically. 

\noindent\Bold{Event Number:} 004 \\
\Bold{Event Name:} Job evicted from machine \\
\Bold{Event Description:} A job was removed from a machine before it finished,
usually for a policy reason. Perhaps an interactive user has claimed
the computer, or perhaps another job is higher priority.

\noindent\Bold{Event Number:} 005 \\
\Bold{Event Name:} Job terminated \\
\Bold{Event Description:} The job has completed.

\noindent\Bold{Event Number:} 006 \\
\Bold{Event Name:} Image size of job updated \\
\Bold{Event Description:} An informational event, 
to update the amount of memory that the job is using while running. 
It does not reflect the state of the job.

\noindent\Bold{Event Number:} 007 \\
\Bold{Event Name:} Shadow exception \\
\Bold{Event Description:} 
The \Condor{shadow}, a program on the submit computer that watches
over the job and performs some services for the job, failed for some
catastrophic reason. The job will leave the machine and go back into
the queue.

\noindent\Bold{Event Number:} 008 \\
\Bold{Event Name:} Generic log event \\
\Bold{Event Description:} Not used.

\noindent\Bold{Event Number:} 009 \\
\Bold{Event Name:} Job aborted \\
\Bold{Event Description:} The user canceled the job.

\noindent\Bold{Event Number:} 010 \\
\Bold{Event Name:} Job was suspended \\
\Bold{Event Description:} The job is still on the computer, but it is no longer
executing. 
This is usually for a policy reason, such as an interactive user using
the computer. 

\noindent\Bold{Event Number:} 011 \\
\Bold{Event Name:} Job was unsuspended \\
\Bold{Event Description:} The job has resumed execution, after being
suspended earlier. 

\noindent\Bold{Event Number:} 012 \\
\Bold{Event Name:} Job was held \\
\Bold{Event Description:} The job has transitioned to the hold state.
This might happen if the user applies the \Condor{hold} command to the job.

\noindent\Bold{Event Number:} 013 \\
\Bold{Event Name:} Job was released \\
\Bold{Event Description:} The job was in the hold state and is to be re-run.

\noindent\Bold{Event Number:} 014 \\
\Bold{Event Name:} Parallel node executed \\
\Bold{Event Description:} A parallel universe program is running on a node.

\noindent\Bold{Event Number:} 015 \\
\Bold{Event Name:} Parallel node terminated \\
\Bold{Event Description:} A parallel universe program has completed on a node.

\noindent\Bold{Event Number:} 016 \\
\Bold{Event Name:} POST script terminated \\
\Bold{Event Description:} A node in a DAGMan work flow has a script
that should be run after a job. 
The script is run on the submit host. 
This event signals that the post script has completed.

\noindent\Bold{Event Number:} 017 \\
\Bold{Event Name:} Job submitted to Globus \\
\Bold{Event Description:} A grid job has been delegated to Globus
(version 2, 3, or 4).
This event is no longer used.

\noindent\Bold{Event Number:} 018 \\
\Bold{Event Name:} Globus submit failed \\
\Bold{Event Description:} The attempt to delegate a job to Globus
failed. 

\noindent\Bold{Event Number:} 019 \\
\Bold{Event Name:} Globus resource up \\
\Bold{Event Description:} The Globus resource that a job wants to run
on was unavailable, but is now available.
This event is no longer used.

\noindent\Bold{Event Number:} 020 \\
\Bold{Event Name:} Detected Down Globus Resource \\
\Bold{Event Description:} The Globus resource that a job wants to run
on has become unavailable. 
This event is no longer used.

\noindent\Bold{Event Number:} 021 \\
\Bold{Event Name:} Remote error \\
\Bold{Event Description:} The \Condor{starter} (which monitors the job
on the execution machine) has failed.

\noindent\Bold{Event Number:} 022 \\
\Bold{Event Name:} Remote system call socket lost \\
\Bold{Event Description:} The \Condor{shadow} and \Condor{starter}
(which communicate while the job runs) have lost contact.

\noindent\Bold{Event Number:} 023 \\
\Bold{Event Name:} Remote system call socket reestablished \\
\Bold{Event Description:} The \Condor{shadow} and \Condor{starter}
(which communicate while the job runs) have been able to resume
contact before the job lease expired.

\noindent\Bold{Event Number:} 024 \\
\Bold{Event Name:} Remote system call reconnect failure \\
\Bold{Event Description:} The \Condor{shadow} and \Condor{starter}
(which communicate while the job runs) were unable to resume
contact before the job lease expired.

\noindent\Bold{Event Number:} 025 \\
\Bold{Event Name:} Grid Resource Back Up \\
\Bold{Event Description:} A grid resource that was previously
unavailable is now available.

\noindent\Bold{Event Number:} 026 \\
\Bold{Event Name:} Detected Down Grid Resource \\
\Bold{Event Description:} The grid resource that a job is to
run on is unavailable.

\noindent\Bold{Event Number:} 027 \\
\Bold{Event Name:} Job submitted to grid resource \\
\Bold{Event Description:} A job has been submitted,
and is under the auspices of the grid resource.

\noindent\Bold{Event Number:} 028 \\
\Bold{Event Name:} Job ad information event triggered. \\
\Bold{Event Description:} Extra job ClassAd attributes are noted. This event is
written as a supplement to other events when the configuration
parameter \Macro{EVENT\_LOG\_JOB\_AD\_INFORMATION\_ATTRS} is set.

\noindent\Bold{Event Number:} 029 \\
\Bold{Event Name:} The job's remote status is unknown \\
\Bold{Event Description:} No updates of the job's remote status
have been received for 15 minutes.

\noindent\Bold{Event Number:} 030 \\
\Bold{Event Name:} The job's remote status is known again \\
\Bold{Event Description:} An update has been received for a job whose
remote status was previous logged as unknown.

\noindent\Bold{Event Number:} 031 \\
\Bold{Event Name:} Job stage in \\
\Bold{Event Description:} Definition not yet written.

\noindent\Bold{Event Number:} 032 \\
\Bold{Event Name:} Job stage out \\
\Bold{Event Description:} Definition not yet written.

\noindent\Bold{Event Number:} 033 \\
\Bold{Event Name:} Attribute update \\
\Bold{Event Description:} Definition not yet written.

%%%%%%%%%%%%%%%%%%%%%%%%%%%%%%%%%%%%%%%%%%%%%%%%%%%%%%%%%%%%%%%%%%%%%e
\subsection{\label{sec:job-completion}Job Completion}
%%%%%%%%%%%%%%%%%%%%%%%%%%%%%%%%%%%%%%%%%%%%%%%%%%%%%%%%%%%%%%%%%%%%%%
\index{job!completion}

When your Condor job completes(either through normal means or abnormal
termination by signal), Condor will remove it from the job queue (i.e.,
it will no longer appear in the output of \Condor{q}) and insert it into
the job history file.  You can examine the job history file with the
\Condor{history} command. If you specified a log file in your submit
description file, then the job exit status will be recorded there as well.

By default, Condor will send you an email message
when your job completes.  You can modify this behavior with the
\Condor{submit} ``notification'' command.
The message will include the exit status of your job (i.e., the
argument your job passed to the exit system call when it completed) or
notification that your job was killed by a signal.  It will also
include the following statistics (as appropriate) about your job:

\begin{description}

\item[Submitted at:] when the job was submitted with \Condor{submit}

\item[Completed at:] when the job completed

\item[Real Time:] elapsed time between when the job was submitted and
when it completed (days hours:minutes:seconds)

\item[Run Time:] total time the job was running (i.e., real time minus
queuing time)

\item[Committed Time:] total run time that contributed to job
completion (i.e., run time minus the run time that was lost because
the job was evicted without performing a checkpoint)

\item[Remote User Time:] total amount of committed time the job spent
executing in user mode

\item[Remote System Time:] total amount of committed time the job spent
executing in system mode 

\item[Total Remote Time:] total committed CPU time for the job

\item[Local User Time:] total amount of time this job's
\Condor{shadow} (remote system call server) spent executing in user
mode

\item[Local System Time:] total amount of time this job's
\Condor{shadow} spent executing in system mode

\item[Total Local Time:] total CPU usage for this job's \Condor{shadow}

\item[Leveraging Factor:] the ratio of total remote time to total
system time (a factor below 1.0 indicates that the job ran
inefficiently, spending more CPU time performing remote system calls
than actually executing on the remote machine)

\item[Virtual Image Size:] memory size of the job, computed when the
job checkpoints

\item[Checkpoints written:] number of successful checkpoints performed
by the job

\item[Checkpoint restarts:] number of times the job successfully
restarted from a checkpoint

\item[Network:] total network usage by the job for checkpointing and
remote system calls

\item[Buffer Configuration:] configuration of remote system call I/O
buffers

\item[Total I/O:] total file I/O detected by the remote system call
library

\item[I/O by File:] I/O statistics per file produced by the remote
system call library

\item[Remote System Calls:] listing of all remote system calls
performed (both Condor-specific and Unix system calls) with a count of
the number of times each was performed

\end{description}

%%%%%%%%%%%%%%%%%%%%%%%%%%%%%%%%%%%%%%%%%%

%%%%%%%%%%%%%%%%%%%%%%%%%%%%%%%%%%%%%%%%
\section{Priorities in Condor}
%%%%%%%%%%%%%%%%%%%%%%%%%%%%%%%%%%%%%%%%

Condor has two independent priority controls: \Term{job}
priorities and \Term{user} priorities.  

\subsection{Job Priority}

\index{job!priority}
\index{priority!of a job}
Job priorities allow the assignment of a priority level to
each submitted Condor job in order to
control order of execution.
To set a job priority, use the \Condor{prio} command
\index{Condor commands!condor\_prio}
--- see the example in section~\ref{sec:job-prio}, or the
command reference page on page~\pageref{man-condor-prio}.
Job priorities do not impact user priorities in any fashion.
Job priorities range from -20 to +20,
with -20 being the worst and with +20 being the best.

%%%%%%%%%%%%%%%%%%%%%%%%%%%%%%%%%%%%%%%%%%%%%%%%%%%%%%%%%%%%%%%%%%%%%%
\subsection{\label{sec:user-priority-explained}User priority}
%%%%%%%%%%%%%%%%%%%%%%%%%%%%%%%%%%%%%%%%%%%%%%%%%%%%%%%%%%%%%%%%%%%%%%

\index{user!priority}
\index{priority!of a user}
Machines are allocated to users based upon a user's priority.
A lower numerical value for user priority means higher priority,
so a user with priority 5 will get more resources than
a user with priority 50.
User priorities in Condor can be examined with the \Condor{userprio}
command (see page~\pageref{man-condor-userprio}).
\index{Condor commands!condor\_userprio}
Condor administrators can set and change individual user priorities
with the same utility.

Condor continuously calculates the share of available machines that each
user should be allocated.    This share is inversely related to the ratio
between user priorities.
For example, a user with a priority of 10 will get twice as many
machines as a user with a priority of 20.
The priority of each individual user changes according to
the number of resources the individual is using.
Each user starts out with the best possible priority: 0.5.
If the number of machines a user currently has is greater than 
the user priority,
the user priority will worsen by numerically increasing over time.
If the number of machines is less then the priority,
the priority will improve by numerically decreasing over time. 
The long-term result is fair-share access across all users.
The speed at which Condor adjusts the priorities is
controlled with the configuration macro \Macro{PRIORITY\_HALFLIFE},
an exponential half-life value.
The default is one day.
If a user that has user priority of 100 and is
utilizing 100 machines removes all his/her jobs,
one day later that user's
priority will be 50, and two days later the priority will be 25.

Condor enforces that each user gets his/her fair share of machines
according to user priority both when allocating machines which become
available and by priority preemption of currently allocated machines.
For instance, if a low priority user is utilizing all available machines
and suddenly a higher priority user submits jobs, Condor will
immediately checkpoint and vacate jobs belonging to the lower priority
user. This will free up machines that Condor will then give over to the
higher priority user. Condor will not starve the lower priority user; it
will preempt only enough jobs so that the higher priority user's fair
share can be realized (based upon the ratio between user priorities). To
prevent thrashing of the system due to priority preemption, the Condor 
site administrator can define a \Macro{PREEMPTION\_REQUIREMENTS} expression in Condor's configuration.
The default expression that ships with Condor is configured to only preempt 
lower priority jobs that have run
for at least one hour. So in the previous example, in the worse case it
could take up to a maximum of one hour until the higher priority user
receives his fair share of machines. 

User priorities are keyed on ``username@domain'', for example
``johndoe@cs.wisc.edu''. The domain name to use, if any, is configured by
the Condor site administrator.  Thus, user priority and therefore resource
allocation is not impacted by which machine the user submits from or
even if the user submits jobs from multiple machines.

\index{nice job}
\index{priority!nice job}
An extra feature is the ability to submit a job as
a ``nice'' job (see page~\pageref{man-condor-submit-nice}).
Nice jobs artificially boost the user priority 
by one million just for the nice job.
This effectively means that nice jobs will only run on
machines that no other Condor job (that is, non-niced job) wants.
In a similar fashion, a Condor administrator could set
the user priority of any specific Condor user very high.
If done, for example, with a guest account,
the guest could only use cycles not wanted by other users of the system.

%%%%%%%%%%%%%%%%%%%%%%%%%%%%%%%%%%%%%%%%%%%%%%%%%%%%%%%%%%%%%%%%%%%%%%
%%%%%%%%%%%%%%%%%%%%%%%%%%%%%%%%%%%%%%%%%%%%%%%%%%%%%%%%%%%%%%%%%%%%%%
\section{\label{sec:PVM}Parallel Applications in Condor: Condor-PVM}
%%%%%%%%%%%%%%%%%%%%%%%%%%%%%%%%%%%%%%%%%%%%%%%%%%%%%%%%%%%%%%%%%%%%%%

\newcommand{\func}[1]{\texttt{#1}}

Condor has a PVM submit Universe which allows the user to submit PVM jobs to
the Condor pool.  In this section, we will first
discuss the differences between running under normal PVM and running PVM under the Condor
environment.  Then we give some hints on how to write good PVM
programs to suit the Condor environment via an example program.  In the
end, we illustrate how to submit PVM jobs to Condor by examining a
sample Condor submit-description file which submits a PVM job.

Note that Condor-PVM is an optional Condor module.  To check and see if
it has been installed at your site, enter the following command:
\begin{verbatim}
        ls -l `condor_config_val PVMD`
\end{verbatim}
(notice the use of backticks in the above command).  If this shows the
file ``condor\_pvmd'' on your system, Condor-PVM is installed.  If not,
ask your site administrator to download Condor-PVM from
\Url{http://www.cs.wisc.edu/condor/condor-pvm} and install it.

%%%%%%%%%%%%%%%%%%%%%%%%%%%%%%%%%%%%%%%%%%%%%%%%%%%%%%%%%%%%%%%%%%%%%%
\subsection{What does Condor-PVM do?}
%%%%%%%%%%%%%%%%%%%%%%%%%%%%%%%%%%%%%%%%%%%%%%%%%%%%%%%%%%%%%%%%%%%%%%

Condor-PVM provides a framework to run parallel applications written to
PVM in Condor's opportunistic environment.  This means that you no
longer need a
set of dedicated machines to run PVM applications; Condor can be used to dynamically 
construct PVM virtual machines out of non-dedicated desktop machines on your network
which would have otherwise been idle.   In Condor-PVM, Condor acts as the
resource manager for the PVM daemon.  Whenever your PVM program asks
for nodes (machines), the request is re-mapped to Condor.  Condor then
finds a machine in the Condor pool via the usual mechanisms, and adds it
to the PVM virtual machine.  If a machine needs to leave the pool, your
PVM program is notified of that as well via the normal PVM mechanisms.

%%%%%%%%%%%%%%%%%%%%%%%%%%%%%%%%%%%%%%%%%%%%%%%%%%%%%%%%%%%%%%%%%%%%%%
\subsection{The Master-Worker Paradigm}
%%%%%%%%%%%%%%%%%%%%%%%%%%%%%%%%%%%%%%%%%%%%%%%%%%%%%%%%%%%%%%%%%%%%%%

There are several different parallel programming paradigms.  One of the
more common is the \Term{master-worker} (or \Term{pool of tasks})
arrangement.  In a master-worker program model, one node acts as the
controlling master for the parallel application and sends pieces work out to worker nodes.  The
worker node does some computation, and sends the result back to the
master node.  The master has a pool of work that needs to be
done, and simply assigns the next piece of work out to the next worker
that becomes available.  

Not all parallel programming paradigms lend themselves to an
opportunistic environment. In such an environment, any of the nodes
could be preempted and therefore disappear at any moment. The
master-worker model, on the other hand, is a model that can work well.
The idea is the master needs to keep track of which piece of work it
sends to each worker. If the master node is then informed that a worker
has disappeared, it puts the piece of work it assigned to that worker
back into the pool of tasks, and sends it out again to the next
available worker. If the master notices that the number of workers has
dropped below an acceptable level, it could request for more workers
(via \func{pvm\_addhosts()}). Or perhaps perhaps the master will request
a replacement node every single time it is notified that a worker has
gone away. The point is that in this paradigm, the number of workers is
not important (although more is better!) and changes in the size of
the virtual machine can be handled naturally.

Condor-PVM is designed to run PVM applications which follow the
master-worker paradigm.  Condor runs the master application on the
machine where the job was submitted and will not preempt it.  Workers
are pulled in from the Condor pool as they become available.

%%%%%%%%%%%%%%%%%%%%%%%%%%%%%%%%%%%%%%%%%%%%%%%%%%%%%%%%%%%%%%%%%%%%%%
\subsection{Binary Compatibility}
%%%%%%%%%%%%%%%%%%%%%%%%%%%%%%%%%%%%%%%%%%%%%%%%%%%%%%%%%%%%%%%%%%%%%%

Condor-PVM does not define a new API (application program interface);
programs can simply use the existing resource management PVM calls such
as \func{pvm\_addhosts()} and \func{pvm\_notify()}.  Because of this, some
master-worker PVM applications are ready to run under Condor-PVM with no
changes at all.  Regardless of using Condor-PVM or not, it is good
master-worker design to handle the case of a worker node disappearing,
and therefore many programmers have already constructed their master program
with all the necessary logic for fault-tolerance purposes.  

In fact, regular PVM and Condor-PVM are \underline{binary compatible}
with each other.  The same binary which runs under regular PVM will run
under Condor, and vice-versa.  There is no need to re-link for Condor-PVM.
This permits easy application development
(develop your PVM application interactively with the regular PVM console, XPVM,
etc) as well as binary sharing between Condor and some dedicated MPP systems.

%%%%%%%%%%%%%%%%%%%%%%%%%%%%%%%%%%%%%%%%%%%%%%%%%%%%%%%%%%%%%%%%%%%%%%
\subsection{Runtime differences between Condor-PVM and regular PVM}
%%%%%%%%%%%%%%%%%%%%%%%%%%%%%%%%%%%%%%%%%%%%%%%%%%%%%%%%%%%%%%%%%%%%%%

This release of the Condor-PVM is based on PVM 3.3.11.  The vast majority of the PVM
library functions under Condor maintain the same semantics as in
PVM 3.3.11, including messaging operations, group operations, and 
pvm\_catchout().

We summarize the changes and new features of PVM under running in the
Condor environment in the following list:

\begin{itemize}

\item Concept of machine class.  Under Condor-PVM, machines of
  different architectures attributes belong to different machine classes.  Machine
  classes are are numbered 0, 1, \Dots, etc.  A machine class can be
  specified by the user in the submit-description file when the job
  is submitted to Condor.

\item \func{pvm\_addhosts()}.  When the application
  needs to add a host machine, it should call \func{pvm\_addhosts()}
  with the first argument as a string that specifies the machine
  class.  For example, to specify class 0, a pointer to string ``0''
  should be used as the first argument.  Condor will find a machine
  that satisfies the requirements of class 0 and adds it to the PVM
  virtual machine.

  Furthermore, \func{pvm\_addhosts()} no longer blocks under Condor.  It
  will return immediately, before the hosts are actually added to the virtual
  machine.  After all, in a non-dedicated environment the amount of time it takes until
  a machine becomes available is not bound. The user should simply call 
  \func{pvm\_notify()} before calling
  \func{pvm\_addhosts()}, so that when a host is added later, the user
  will be notified via Condor in the usual PVM 
  fashion (via a PvmHostAdd notification message).
    
\item \func{pvm\_notify()}.  Under Condor, we added two additional 
  possible notification requests, \func{PvmHostSuspend} and
  \func{PvmHostResume}, to the function \func{pvm\_notify()}.  When a
  host is suspended (or resumed) by Condor, if the user has called
  \func{pvm\_notify()} with that host tid and with
  \func{PvmHostSuspend} (or \func{PvmHostResume}) as arguments, then
  the application will receive a notification for the corresponding
  event.

\item \func{pvm\_spawn()}.  If  the flag in \func{pvm\_spawn()} is 
  PvmTaskArch, then the string specifying the desired architecture
  class should be used.  Typically, if you are using only one class of
  machine in your virtual machine, specify ``0'' as the desired architecture.

  Furthermore, under Condor we currently only allow one
  PVM task to be spawned per node, since Condor's typical setup at most 
  sites will suspend or vacate
  a job if the load on its machine is higher than a specified
  threshold.

\end{itemize}

%%%%%%%%%%%%%%%%%%%%%%%%%%%%%%%%%%%%%%%%%%%%%%%%%%%%%%%%%%%%%%%%%%%%%%
\subsection{A Sample PVM program for Condor-PVM}
%%%%%%%%%%%%%%%%%%%%%%%%%%%%%%%%%%%%%%%%%%%%%%%%%%%%%%%%%%%%%%%%%%%%%%

Normal PVM applications assume dedicated machines.  However, when running a
PVM application under Condor, since Condor's environment is an
opportunistic environment, machines can be suspended and even removed
from the PVM virtual machine during the life-time of the PVM
application.  

Here, we include an extensively commented skeleton of a sample PVM
program \Prog{master\_sum.c}, which, we hope, will help you to
write PVM code that is better suited for a non-dedicated opportunistic
environment like Condor.

\CondorVerySmall
\begin{verbatim}
/* 
 * master_sum.c
 *
 * master program to perform parallel addition - takes a number n 
 * as input and returns the result of the sum 0..(n-1).  Addition 
 * is performed in parallel by k tasks, where k is also taken as 
 * input.  The numbers 0..(n-1) are stored in an array, and each 
 * worker adds a portion of the array, and returns the sum to the 
 * master.  The Master adds these sums and prints final sum.  
 *
 * To make the program fault-tolerant, the master has to monitor 
 * the tasks that exited without sending the result back.  The 
 * master creates some new tasks to do the work of those tasks 
 * that have exited. 
 */

#define NOTIFY_NUM 5  /* number of items to notify */

#define HOSTDELETE 12
#define HOSTSUSPEND 13
#define HOSTRESUME 14
#define TASKEXIT 15
#define HOSTADD 16
    
/* send the pertask and start number to the worker task i */
void send_data_to_worker(int i, int *tid, int *num, int pertask, 
            FILE *fp, int round)
{
     int status;
     int start_val;
     
     /* send the round number */
     pvm_initsend(PvmDataDefault); /* XDR format */
     pvm_pkint(&round, 1, 1);    /* number of numbers to add */
     status = pvm_send(tid[i], ROUND_TAG);

     pvm_initsend(PvmDataDefault); /* XDR format */
     pvm_pkint(&pertask, 1, 1);    /* number of numbers to add */
     status = pvm_send(tid[i], NUM_NUM_TAG);

     pvm_initsend(PvmDataDefault); /* XDR format */
     start_val = i * pertask; /* initial number for this task */
     pvm_pkint(&start_val, 1, 1);     /* the initial number */
     status = pvm_send(tid[i], START_NUM_TAG);   

     fprintf(fp, "Round %d: Send data %d to worker task %d, ``
           ``tid =%x. status %d \n", round, start_val, i, tid[i], status);
}

/* 
 * to see if more hosts are needed 
 * 1 = yes; 0 = no 
 */
int need_more_hosts(int i)
{
     int nhost, narch;
     char *hosts="0";  /* any host in arch class 0 */
     struct pvmhostinfo *hostp = (struct pvmhostinfo *) 
                     calloc (1, sizeof(struct pvmhostinfo));

     /* get the current configuration */
     pvm_config(&nhost, &narch, &hostp);
     
     if (nhost > i)
        return 0;
     else 
        return 1;
}

/* 
 * Add a new host until success, assuming that request for 
 * PvmAddHost notification has already been sent 
 */
void add_a_host(FILE *fp)
{
     int done = 0;
     int buf_id;
     int success = 0;
     int tid;
     int msg_len, msg_tag, msg_src;
     char *hosts="0";  /* any host in arch class 0 */
     int infos[1];

     while (done != 1) {
        /* 
        * add one host - no specific machine named 
        * add host will asynchronously, so we need
        * to receive the notification before go on.
        */
        pvm_addhosts(&hosts,1 , infos);
      
        /* receive hostadd notification from anyone */
        buf_id = pvm_recv(-1, HOSTADD);
      
        if (buf_id < 0) {
            fprintf(fp, "Error with buf_id = %d\n", buf_id);
            done = 0;
            continue;
        }
        done = 1;
     
        pvm_bufinfo(buf_id, &msg_len , &msg_tag, &msg_src);
        pvm_upkint(&tid, 1, 1);

        pvm_notify(PvmHostDelete, HOSTDELETE, 1, &tid);

        fprintf(fp, "Received HOSTADD: ");
        fprintf(fp, "Host %x added from %x\n", tid, msg_src);
        fflush(fp);
    }
}

/* 
 * Spawn a worker task until success.  
 * Return its tid, and the tid of its host. 
 */
void spawn_a_worker(int i, int* tid, int * host_tid, FILE *fp)
{
     int numt = 0;
     int status;

     while (numt == 0){
          /* spawn a worker on a host belonging to arch class 0 */
          numt = pvm_spawn ("worker_sum", NULL, PvmTaskArch, "0", 1, &tid[i]);

          fprintf(fp, "master spawned %d task tid[%d] = %x\n",numt,i,tid[i]);
          fflush(fp);
         
          /* if the spawn is successful */
          if (numt == 1) {
               /* notify when the task exits */
               status = pvm_notify(PvmTaskExit, TASKEXIT, 1, &tid[i]);
               
               fprintf(fp, "Notify status for exit = %d\n", status);
               
               if (pvm_pstat(tid[i]) != PvmOk) numt = 0;
          }
          
          if (numt != 1) {
               fprintf(fp, "!! Failed to spawn task[%d]\n", i);
               
               /* 
                * currently Condor-pvm allows only one task running on 
                * a host
                */
                while (need_more_hosts(i) == 1)
                    add_a_host(fp);
          }
     }
}


main()
{
    int n;                  /* will add <n> numbers n .. n-1 */
    int ntasks;             /* need <ntask> workers to do the addition. */
    int pertask;            /* numbers to add per task */
    int tid[MAX_TASKS];     /* tids of tasks */ 
    int deltid[MAX_TASKS];  /* tids monitored for deletion */
    int sum[MAX_TASKS];     /* hold the reported sum */
    int num[MAX_TASKS];     /* the initial numbers the workers should add */
    int host_tid[MAX_TASKS];/* the tids of the host that the *
                             * tasks <0..ntasks> are running on*/
    
    int i, numt, nhost, narch, status;
    int result;
    int mytid;    /* task id of master */
    int mypid;    /* process id of master */
    int buf_id;   /* id of recv buffer */
    int msg_leg, msg_tag, msg_src, msg_len;
    int int_val;  

    int infos[MAX_TASKS];
    char * hosts[MAX_TASKS];
    struct pvmhostinfo *hostp = (struct pvmhostinfo *) 
                    calloc (MAX_TASKS, sizeof(struct pvmhostinfo));

    FILE *fp;
    char outfile_name[100];

    char *codes[NOTIFY_NUM] = {"HostDelete", "HostSuspend", 
            "HostResume", "TaskExit", "HostAdd"};
    
    int count;   /* the number of times that while loops */
    int round_val;
    int correct = 0;
    int wrong = 0;

    mypid = getpid();

    sprintf(outfile_name, "out_sum.%d", mypid);
    fp = fopen(outfile_name, "w"); 

    /* redirect all children tasks' stdout to fp */
    pvm_catchout(stderr);  

    if (pvm_parent() == PvmNoParent){
        fprintf(fp, "I have no parent!\n");
        fflush(fp);
    }

    /* will add <n> numbers 0..(n-1) */
    fprintf(fp, "How many numbers? ");
    fflush(fp);
    scanf("%d", &n);
    fprintf(fp, "%d\n", n);
    fflush(fp);

    /* will spawn ntasks workers to perform addition */
    fprintf(fp, "How many tasks? ");
    fflush(fp);
    scanf("%d", &ntasks);
    fprintf(fp, "%d\n\n", ntasks);
    fflush(fp);

    /* will iterate count loops */
    fprintf(fp, "How many loops? ");
    fflush(fp);
    scanf("%d", &count);
    fprintf(fp, "%d\n", count);
    fflush(fp);

    /* set the hosts to be in arch class 0 */
    for (i = 0; i< ntasks; i++) hosts[i] = "0";

    /* numbers to be added by each worker */
    pertask = n/ntasks;

    /* get the master's TID */
    mytid = pvm_mytid();
    fprintf(fp, "mytid = %x; mypid = %d\n", mytid, mypid);

    /* get the current configuration */
    pvm_config(&nhost, &narch, &hostp);

    fprintf(fp, "current number of hosts = %d\n", nhost);
    fflush(fp);

    /* 
     * notify request for host addition, with tag HOSTADD, 
     * no tids to monitor.  
     *
     * -1 turns the notification request on;
     * 0 turns it off;
     * a positive integer n will generate at most n 
     * notifications.
     */     
    pvm_notify(PvmHostAdd, HOSTADD, -1, NULL);

    /* add more hosts - no specific machine named */
    i = ntasks - nhost;
    if (i > 0) {
        status = pvm_addhosts(hosts, i , infos);
      
        fprintf(fp, "master: addhost status = %d\n", status);
        fflush(fp);
    }
     
    /* if not enough hosts, loop and call pvm_addhosts */
    for (i = nhost; i < ntasks; i++) {
        /* receive notification from anyone, with HostAdd tag */
        buf_id = pvm_recv(-1, HOSTADD);

        if (buf_id < 0) {
           fprintf(fp, "Error with buf_id = %d\n", buf_id);
        } else {
           fprintf(fp, "Success with buf_id = %d\n", buf_id);
        }

        pvm_bufinfo(buf_id, &msg_len , &msg_tag, &msg_src);
        if (msg_tag==HOSTADD) {
            pvm_upkint(&int_val, 1, 1);

            fprintf(fp, "Received HOSTADD: ");
            fprintf(fp, "Host %x added from %x\n", int_val, msg_src);
           fflush(fp);
        } else {
           fprintf(fp, "Received unexpected message with tag: %d\n", msg_tag);
        }
    }

    /* get current configuration */
    pvm_config(&nhost, &narch, &hostp);

    /* notify all exceptional conditions about the hosts*/
    status = pvm_notify(PvmHostDelete, HOSTDELETE, ntasks, deltid);
    fprintf(fp, "Notify status for delete = %d\n", status);
     
    status = pvm_notify(PvmHostSuspend, HOSTSUSPEND, ntasks, deltid);
    fprintf(fp, "Notify status for suspend = %d\n", status);
     
    status = pvm_notify(PvmHostResume, HOSTRESUME, ntasks, deltid);
    fprintf(fp, "Notify status for resume = %d\n", status);

    /* spawn <ntasks> */
    for (i = 0; i < ntasks ; i++) {
        /* spawn the i-th task, with notifications. */
        spawn_a_worker(i, tid, host_tid, fp);
    }

    /* add the result <count> times */
    while (count > 0) {
        /* 
         * if array length was not perfectly divisible by ntasks, 
         *    some numbers are remaining. Add these yourself 
         */
        result = 0;
        for ( i = ntasks * pertask ; i < n ; i++)
           result += i;
     
        /* initialize the sum array with -1 */
        for (i = 0; i< ntasks; i++) 
            sum[i] = -1;
 
        /* send array partitions to each task */
        for (i = 0; i < ntasks ; i++) {
           send_data_to_worker(i, tid, num, pertask, fp, count);
        }

        /* 
        * Wait for results.  If a task exited without 
        * sending back the result, start another task to do
        * its job. 
        */
        for (i = 0; i< ntasks; ) {   
            buf_id = pvm_recv(-1, -1);
            pvm_bufinfo(buf_id, &msg_len , &msg_tag, &msg_src);
            fprintf(fp, "Receive: task %x returns mesg tag %d, ``
                ``buf_id = %d\n", msg_src, msg_tag, buf_id);
            fflush(fp);
           
            /* is a result returned by a worker */
            if(msg_tag == RESULT_TAG)  {
                int j;
                
                pvm_upkint(&round_val, 1, 1);
                fprintf(fp, "  round_val = %d\n", round_val);
                fflush(fp);
             
                if (round_val != count) continue;

                pvm_upkint(&int_val, 1, 1);
                for (j=0; (j<ntasks) && (tid[j] != msg_src); j++)
                    ;
                fprintf(fp, "  Data from task %d, tid = %x : %d\n", 
                    j, msg_src, int_val);
                fflush(fp);
                
                if (sum[j] == -1) {
                    sum[j] = int_val; /* store the sum */
                    i++;
                }
           } else if (msg_tag == TASKEXIT) {
                /* A task has exited. */
                /* Find out which task has exited. */ 
                int which_tid, j;          
                pvm_upkint(&which_tid, 1, 1);
                for (j=0; (j<ntasks) && (tid[j] != which_tid); j++)
                 ;
                fprintf(fp, "  from tid %x : task %d, tid =  %x, ``
                    ``exited.\n", 
                    msg_src, j, which_tid);
                fflush(fp);
                /* 
                 * If a task exited before sending back the message,
                 * create another task to do the same job.
                 */
                if (j < ntasks && sum[j] == -1) {
                     /* spawn the j-th task */
                     spawn_a_worker(j, tid, host_tid, fp);
                     
                     /* send unfinished work to the new task */
                     send_data_to_worker(j, tid, num, pertask, fp, count);
                }
            } else if (msg_tag == HOSTDELETE) {
                /* 
                * If a host has been deleted, check to see if 
                * the tasks running on it has been finished.  
                * If not, should create  new worker tasks to do 
                * the work on some other  hosts.
                */
                int which_tid, j;
                    
                /* get which host has been suspended/deleted */
                pvm_upkint(&which_tid, 1, 1);
                    
                fprintf(fp, "  from tid %x : %x %s\n", msg_src, which_tid, 
                    codes[msg_tag - HOSTDELETE]);
                fflush(fp);
                    
                /* 
                 * If the task on that host has not finished its
                 * work, then create new task to do the work.
                 */
                for (j = 0; j < ntasks; j++) {
                     if (host_tid[j] == which_tid && sum[j] == -1) {
                          fprintf(fp, "host_tid[%d] = %x, ``
                            ``need new task\n",
                              j, host_tid[j]);
                          fflush(fp);
                          
                          /* spawn the i-th task, with notifications. */
                          spawn_a_worker(j, tid, host_tid, fp);
                          
                          /* send the unfinished work to the new task */
                          send_data_to_worker(j,tid,num,pertask,fp,count);
                     }
                }
            } else {
                /* print out some other notifications or messages */
                int which_tid;
                pvm_upkint(&which_tid, 1, 1);
        
                fprintf(fp, "  from tid %x : %x %s\n", msg_src,
                        which_tid,   codes[msg_tag - HOSTDELETE]);
                fflush(fp);
            }
        }        
      
        /* add up the sum */
        for (i=0; i<ntasks; i++)
           result += sum[i];
          
        fprintf(fp, "Sum from  0 to %d is %d\n", n-1 , result);
        fflush(fp);
          
        /* check correctness */
        if (result == (n-1)*n/2) {
           correct++;
           fprintf(fp, "*** Result Correct! ***\n");
        } else {
           wrong++;
           fprintf(fp, "*** Result WRONG! ***\n");
        }

        fflush(fp);
        count--;
    }
     
     fprintf(fp, "correct = %d; wrong = %d\n", correct, wrong);
     fflush(fp);

     pvm_exit();
     exit(0);
}

\end{verbatim}
\normalsize

%%%%%%%%%%%%%%%%%%%%%%%%%%%%%%%%%%%%%%%%%%%%%%%%%%%%%%%%%%%%%%%%%%%%%%
\subsection{\label{sec:PVM-Submit}Sample PVM submit file}
%%%%%%%%%%%%%%%%%%%%%%%%%%%%%%%%%%%%%%%%%%%%%%%%%%%%%%%%%%%%%%%%%%%%%%

Like submitting jobs in any other universe,
to submit a PVM job, the user needs to specify the requirements and
options in the submit-desciption file and run \Condor{submit}.
Figure~\ref{fig:pvm_submit} on page~\pageref{fig:pvm_submit} is an
example of a submit-description file for a PVM job.  
This job has a master PVM program called \func{master\_pvm}.

\begin{figure}[hbt]
\CondorSmall
\begin{verbatim}
###########################################################
# sample_submit
# Sample submit file for PVM jobs. 
###########################################################

# The job is a PVM universe job.
universe = PVM  

# The executable of the master PVM program is ``master_pvm''.
executable = master_pvm

In = "in_sum"
Out = "stdout_sum"
Err = "err_sum"

###################  Architecture class 0  ##################

Requirements = (Arch == "INTEL") && (OpSys == "SOLARIS251") 

# We want at least 2 machines in class 0 before starting the 
# program.  We can use up to 4 machines.
machine_count = 2..4  
queue

###################  Architecture class 1  ##################

Requirements = (Arch == "SUN4x") && (OpSys == "SOLARIS251") 

# We need at least 1 machine in class 1 before starting the 
# executable.  We can use up to 3 to start with.
machine_count = 1..3
queue

###############################################################
# note: the program will not be started until the least 
#       requirements in all classes are satisfied.
###############################################################
\end{verbatim}
\normalsize

\label{fig:pvm_submit}
\caption{A sample submit file for PVM jobs.}
\end{figure}

In this sample submit file, the command \func{universe = PVM}
specifies that the jobs should be submitted into PVM universe.

The command \func{executable = master\_pvm} tells Condor that the PVM
master program is \func{master\_sum}.  This program will be started on
the submitting machine.  The workers should be spawned by this master
program during execution.

This submit file also tells Condor that the PVM virtual machine is
consisted of two different classes of machine architectures.  Class
0 contains machines with INTEL architecture running SOLARIS251; class
1 contains machines with SUN4x (SPARC) architecture running SOLARIS251.

By using \func{machine\_count = <min>..<max>}, the submit file tells
Condor that before the PVM program, there should be at least \verb@<min>@
number of machines of the current class.  It also asks Condor to give
it as many as \verb@<max>@ machines.  During the execution of the program,
the application can get more machines of each of the class by calling
\func{pvm\_addhosts()} with a string specifying the desired architecture
class.  (See the sample program in this section for details.)

The \func{queue} command should be inserted after the specifications of
each class.



%%%%%%%%%%%%%%%%%%%%%%%%%%%%%%%%%%%%%%%%%%%%%%%%%%%%%%%%%%%%%%%%%%%%%%

% Commented out by psilord until the new world order in MPI is 
% redocumented.
%%%%%%%%%%%%%%%%%%%%%%%%%%%%%%%%%%%%%%%%%%%%%%%%%%%%%%%%%%%%%%%%%%%%%%
%%%%%%%%%%%%%%%%%%%%%%%%%%%%%%%%%%%%%%%%%%%%%%%%%%%%%%%%%%%%%%%%%%%%%%%%
\section{\label{sec:MPI}MPI Applications}
%%%%%%%%%%%%%%%%%%%%%%%%%%%%%%%%%%%%%%%%%%%%%%%%%%%%%%%%%%%%%%%%%%%%%%
\index{MPI|(}
MPI stands for Message Passing Interface.
It provides an environment under which parallel programs
may synchronize, 
by providing communication support.
Running the MPI-based parallel programs within Condor 
eases the programmer's effort.
Condor dedicates machines for running the programs,
and it does so using the same interface used when submitting
non-MPI jobs.

The MPI universe in Condor currently supports MPICH versions 1.2.2, 1.2.3, 
and 1.2.4 using the ch\_p4 device. 
The MPI universe does not support MPICH version 1.2.5.
These supported implementations are
offered by Argonne National Labs
without charge by download.
See the web page at
\URL{http://www-unix.mcs.anl.gov/mpi/mpich/}
for details and availability.
Programs to be submitted for execution under Condor will have
been compiled using \Prog{mpicc}.
No further compilation or linking is necessary to run jobs
under Condor.

The Parallel universe~\ref{sec:Parallel} is now the preferred
way to run MPI jobs. 
Support for the MPI universe will be removed from Condor at a future date.
%%%%%%%%%%%%%%%%%%%%%%%%%%%%%%%%%%%%%%%%%%%%%%%%%%%%%%%%%%%%%%%%%%%
\subsection{\label{sec:MPI-setup}MPI Details of Set Up}

Administratively, Condor must be configured such that resources
(machines) running MPI jobs are dedicated.
\index{scheduling!dedicated}
Dedicated machines never vacate their running condor jobs
should the machine's interactive owner return.  Once the dedicated 
scheduler claims a dedicated machine for use, it will try to use
that machine to satisfy the requirements of the queue of MPI jobs.

Since Condor is not ordinarily used in this manner (Condor uses
opportunistic scheduling),
machines that are to be used as dedicated resources
must be configured as such.
Section~\ref{sec:Config-Dedicated-Jobs} of
Administrator's Manual describes the necessary
configuration and provides detailed examples.

To simplify the dedicated scheduling of resources,
a single machine becomes the scheduler of dedicated resources.
This leads to a further restriction that jobs submitted
to execute under the MPI universe (with dedicated machines)
must be
submitted from the machine running as the dedicated scheduler.

%%%%%%%%%%%%%%%%%%%%%%%%%%%%%%%%%%%%%%%%%%%%%%%%%%%%%%%%%%%%%%%%%%%
\subsection{\label{sec:MPI-submit}MPI Job Submission}

Once the programs are written and compiled, and Condor resources
are correctly configured, jobs may be submitted.
Each Condor job requires a submit description file.
The simplest submit description file for an MPI job:

\begin{verbatim}
#############################################
##   submit description file for mpi_program
#############################################
universe = MPI
executable = mpi_program
machine_count = 4
queue 
\end{verbatim}

This job specifies the \Attr{universe} as \Attr{mpi},
letting Condor know that dedicated resources will be required.
The \Attr{machine\_count} command identifies the number
of machines required by the job.
The four machines that run the program will default to
be of the same architecture
and operating system as the machine on which the job is submitted,
since a platform is not specified as a requirement.

The simplest example does not specify an input or output,
meaning that the computation completed is useless,
since both input comes from and the output goes to \File{/dev/null}.
A more complex example of a submit description file
utilizes other features.
\begin{verbatim}
######################################
## MPI example submit description file
######################################
universe = MPI
executable = simplempi
log = logfile
input = infile.$(NODE)
output = outfile.$(NODE)
error = errfile.$(NODE)
machine_count = 4
queue
\end{verbatim}

The specification of the input, output, and error files
utilize a predefined macro that is only relevant to
mpi universe jobs.
\index{macro!predefined}
See the \Condor{submit} manual page on
page~\pageref{man-condor-submit} 
for further description of predefined macros.
The \MacroU{NODE} macro is given a unique value as
programs are assigned to machines.
This value is what the MPICH version ch\_p4 implementation
terms the rank of a program.
Note that this term is unrelated and independent of the
Condor term rank.
The \MacroUNI{NODE} value is fixed for the entire length
of the job.
It can therefore be used to identify individual aspects
of the computation.
In this example, it is used to give unique names to input
and output files.

If your site does NOT have a shared file system across all the nodes
where your MPI computation will execute, you can use Condor's file
transfer mechanism.
You can find out more details about these settings by reading the
\Condor{submit} man page or section~\ref{sec:file-transfer} on
page~\pageref{sec:file-transfer}. 
Assuming your job only reads input from STDIN, here is an example
submit file for a site without a shared file system:

\begin{verbatim}
######################################
## MPI example submit description file
## without using a shared file system
######################################
universe = MPI
executable = simplempi
log = logfile
input = infile.$(NODE)
output = outfile.$(NODE)
error = errfile.$(NODE)
machine_count = 4
should_transfer_files = yes
when_to_transfer_output = on_exit
queue
\end{verbatim}

Consider the following C program that uses this example submit
description file.

\begin{verbatim}
/**************
 * simplempi.c
 **************/
#include <stdio.h>
#include "mpi.h"

int main(argc,argv)
    int argc;
    char *argv[];
{
    int myid;
    char line[128];

    MPI_Init(&argc,&argv);
    MPI_Comm_rank(MPI_COMM_WORLD,&myid);

    fprintf ( stdout, "Printing to stdout...%d\n", myid );
    fprintf ( stderr, "Printing to stderr...%d\n", myid );
    fgets ( line, 128, stdin );
    fprintf ( stdout, "From stdin: %s", line );

    MPI_Finalize();
    return 0;
}
\end{verbatim}

Here is a makefile that works with the example.
It would build the MPI executable, using the MPICH
version ch\_p4 implementation.
\begin{verbatim}
###################################################################
## This is a very basic Makefile                                 ##
###################################################################

# the location of the MPICH compiler
CC          = /usr/local/bin/mpicc
CLINKER     = $(CC)

CFLAGS    = -g
EXECS     = simplempi

all: $(EXECS)

simplempi: simplempi.o
        $(CLINKER) -o simplempi simplempi.o -lm

.c.o:
        $(CC) $(CFLAGS) -c $*.c
\end{verbatim}

The submission to Condor requires exactly four machines,
and queues four programs.
Each of these programs requires an input file (correctly
named) and produces an output file.

If input file for \MacroUNI{NODE} = 0 (called \File{infile.0}) contains
\begin{verbatim}
Hello number zero.
\end{verbatim}
and
the input file for \MacroUNI{NODE} = 1 (called \File{infile.1}) contains
\begin{verbatim}
Hello number one.
\end{verbatim}
then after the job is submitted to Condor,
there will be 
eight files created:  
\File{errfile.[0-3]} and \File{outfile.[0-3]}.
\File{outfile.0} will contain
\begin{verbatim}
Printing to stdout...0
From stdin: Hello number zero.
\end{verbatim}
and \File{errfile.0} will contain
\begin{verbatim}
Printing to stderr...0
\end{verbatim}

Different nodes for an MPI job can have different machine requirements.
For example, often the first node, sometimes called the head node, needs
to run on a specific machine.  This can be also useful for debugging.
Condor accomodates this by supporting 
multiple \Attr{queue} statements in the submit file, much like with
the other universes.  For example:

\begin{verbatim}
######################################
## MPI example submit description file
## with multiple procs
######################################
universe = MPI
executable = simplempi
log = logfile
input = infile.$(NODE)
output = outfile.$(NODE)
error = errfile.$(NODE)
machine_count = 1
should_transfer_files = yes
when_to_transfer_output = on_exit
requirements = ( machine == "machine1")
queue

requirements = ( machine =!= "machine1")
machine_count = 3
queue
\end{verbatim}

The dedicated scheduler will allocate four machines (nodes) total in 
two procs for this job.  The first proc has one node, (rank 0 
in MPI terms) and will run on the machine named machine1.  The 
other three nodes, in the second proc, will run on other machines.  
Like in the other condor universes, the second requirements command 
overwrites the first, but the other commands are inherited from the 
first proc.

When submitting jobs with multiple requirements, it is
best to write the requirements to be mutually exclusive,
or to have the most selective requirement first in the submit file.
This is because the scheduler tries to match jobs to machine in
submit file order.  If the requirements are not mutually exclusive,
it can happen that the scheduler may unable to schedule the job, even
if all needed resources are available.
\index{MPI|)}

%%%%%%%%%%%%%%%%%%%%%%%%%%%%%%%%%%%%%%%%%%%%%%%%%%%%%%%%%%%%%%%%%%%%%%

%%%%%%%%%%%%%%%%%%%%%%%%%%%%%%%%%%%%%%%%%%%%%%%%%%%%%%%%%%%%%%%%%%%%%%
\section{\label{sec:Glidein}Extending your Condor pool with Glidein}
\index{universe!Globus}
\index{Globus}
\index{\Condor{glidein}}

Condor works together with Globus software to provide the capability
of submitting Condor jobs to remote computer systems.
Globus software provides mechanisms to access and utilize
remote resources.

\Condor{glidein} is a program that can be used to add Globus resources
to a Condor pool on a temporary basis.
During this period, these resources are visible 
to users of the pool, but only the user
that added the resources is allowed to use them.
The machine in the Condor pool is referred to herein as the
local node, while the resource added to the local Condor pool
is referred to as the remote node.

These requirements are general to using any Globus resource:
\begin{enumerate}

\item An X.509 certificate issued by a Globus certificate authority.

\item Access to a Globus resource.
You must be a valid Globus user and be mapped to a valid login account by
the site's Globus administrator on every Globus resource that will be
added to the local Condor pool using \Condor{glidein}.
More information can be found at \Url{http://www.globus.org}

\item The environment variables \Env{HOME} and either
\Env{GLOBUS\_INSTALL\_PATH} or \Env{GLOBUS\_DEPLOY\_PATH}
must be set.

\end{enumerate}


\subsection{\Condor{glidein} Requirements}
In order to use \Condor{glidein} to add a Globus resource to the
local Condor pool,
there are several requirements beyond the general Globus requirements
given above.

\begin{enumerate}
\item Use Globus v1.1 or better.

\item Have \Prog{gsincftp} installed. This program is an ftp
  client modified to use Globus X.509 authentication.
  More information ca be found at
  \Url{http://www.globus.org/datagrid/deliverables/gsiftp-tools.html}.

\item Be an authorized user of the local Condor pool.

\item The local Condor pool configuration file(s) must 
  give \Macro{HOSTALLOW\_WRITE} permission
  to every resource that will be added using \Condor{glidein}. 
  Wildcards are permitted in this specification.
  An example is of adding every machine at
  cs.wisc.edu by adding *.cs.wisc.edu to the
  \Macro{HOSTALLOW\_WRITE} list.
  Recall that the changes take effect when all machines
  in the local pool are sent a reconfigure command.

\item The local Condor pool's configuration file(s) must
  set \Macro{GLOBUSRUN} to be the path of \Prog{globusrun}
  and \Macro{SHADOW\_GLOBUS} to be the path of the \Condor{shadow.globus}.

\item Included in the \Env{PATH} must be the common user programs
  directory \File{/bin}, globus tools, and the Condor user program
  directory.

\end{enumerate}

\subsection{What \Condor{glidein} Does}

\Condor{glidein} first checks that there is a valid proxy
and that the necessary files are available to \Condor{glidein}.

\Condor{glidein} then contacts the Globus resource and checks for the
presence of the necessary configuration files and Condor executables.
If the executables are not present for the machine architecture,
operating system version, and Condor version required, a
server running at UW is contacted to transfer the needed executables.
To gain access to the server, send email to \Email{condor-admin@cs.wisc.edu}
that includes the name of your X.509 certificate.

When the files are correctly in place,
Condor daemons are started.
\Condor{glidein} does this by creating a submit description file for
\Condor{submit}, which runs the \Condor{master} under the Globus
universe.
This implies that execution of the \Condor{master} is started on the Globus
resource.
The Condor daemons exit gracefully when no jobs run on the daemons for a
configurable period of time. The default length of time is 20 minutes.

The Condor executables on the Globus resource contact the local pool and
attempt to join the pool.  The \Expr{START}
expression for the \Condor{startd} daemon requires that the username
of the person running \Condor{glidein} matches the username of the jobs
submitted through Condor.

After a short length of time,
the Globus resource can be seen in the local Condor pool,
as with this example.

\begin{verbatim}
% condor_status | grep denal
7591386@denal LINUX       INTEL  Unclaimed  Idle       3.700  24064  0+00:06:35
\end{verbatim}

Once the Globus resource has been added to the local Condor
pool with \Condor{glidein},
job(s) may be submitted.
To force a job to run on the Globus resource,
specify that Globus resource as a machine requirement
in the submit description file. 
Here is an example from within the submit description file
that forces submission to the Globus resource denali.mcs.anl.gov:
\begin{verbatim}
      requirements = ( machine == "denali.mcs.anl.gov" ) \
         && FileSystemDomain != "" \
         && Arch != "" && OpSys != ""
\end{verbatim}
This example requires that the job run only on denali.mcs.anl.gov,
and it prevents Condor from inserting the filesystem domain,
architecture, and operating system attributes as requirements
in the matchmaking process.
Condor must be told not to use the submission machine's
attributes in those cases
where the Globus resource's attributes
do not match the submission machine's attributes.

%%%%%%%%%%%%%%%%%%%%%%%%%%%%%%%%%%%%%%%%%%%%%%%%%%%%%%%%%%%%%%%%%%%%%%

%%%%%%%%%%%%%%%%%%%%%%%%%%%%%%%%%%%%%%%%%%%%%%%%%%%%%%%%%%%%%%%%%%%%%%
%%%%%%%%%%%%%%%%%%%%%%%%%%%%%%%%%%%%%%%
\section{\label{sec:DAGMan}DAGMan Applications}
%%%%%%%%%%%%%%%%%%%%%%%%%%%%%%%%%%%%%%%
\index{DAGMan|(}
\index{directed acyclic graph}
\index{Directed Acyclic Graph Manager (DAGMan)}
\index{condor\_submit\_dag}
\index{job!dependencies}

A directed acyclic graph (DAG) can be used to represent a set of programs
where the input, output, or execution of one or more programs
is dependent on one or more other programs.
The programs are nodes (vertices) in the graph,
and the edges (arcs) identify the dependencies.
Condor finds machines for the execution of programs, but it
does not schedule programs (jobs) based on dependencies.
The Directed Acyclic Graph Manager (DAGMan) is a meta-scheduler for Condor
jobs. 
DAGMan submits jobs to Condor in an order represented by
a DAG and processes the results.
An input file defined prior to submission describes the DAG, and
a Condor submit description file for each program in the DAG
is used by Condor.

Each node (program) in the DAG specifies a Condor submit description file.
As DAGMan submits jobs to Condor, it monitors the Condor log file(s) to 
to enforce the ordering required for the DAG.

The DAG itself is defined by the contents of a DAGMan input file.
DAGMan is responsible for scheduling, recovery, and reporting
for the set of programs submitted to Condor.

The following sections specify the use of DAGMan.

%%%%%%%%%%%%%%%%%%%%%%%%%%%%%%%%%%%%%%%
\subsection{Input File describing the DAG}
%%%%%%%%%%%%%%%%%%%%%%%%%%%%%%%%%%%%%%%

\index{DAGMan!input file example}
The input file used by DAGMan specifies five items:
\begin{enumerate}
\item
A list of the programs in the DAG. This serves to name each program
and specify each program's Condor submit description file.
\item
Processing that takes place before submission of
any program in the DAG to Condor or after Condor has completed execution
of any program in the DAG.
\item
Description of the dependencies in the DAG.
\item
Number of times to retry if a node within the DAG fails.
\item
The definition of macros associated with a node.
\end{enumerate}

Comments may be placed in the input file that describes the DAG.
The pound character (\verb@#@) as the first character on a
line identifies the line as a comment.
Comments do not span lines.

An example input file for DAGMan is

\begin{verbatim}
	# Filename: diamond.dag
	#
	Job  A  A.condor 
	Job  B  B.condor 
	Job  C  C.condor	
	Job  D  D.condor
	Script PRE  A top_pre.csh
	Script PRE  B mid_pre.perl  $JOB
	Script POST B mid_post.perl $JOB $RETURN
	Script PRE  C mid_pre.perl  $JOB
	Script POST C mid_post.perl $JOB $RETURN
	Script PRE  D bot_pre.csh
	PARENT A CHILD B C
	PARENT B C CHILD D
	Retry  C 3
\end{verbatim}

This input file describes the DAG shown in 
Figure~\ref{fig:dagman-diamond}.

\begin{figure}[hbt]
\centering
\includegraphics{user-man/dagman-diamond.eps}
\caption{\label{fig:dagman-diamond}Diamond DAG}
\end{figure}



\index{DAGMan!Job Entry (names node of DAG)}
The first section of the input file lists
all the programs that appear in the DAG.
Each program is described by a single line called a Job Entry.
The syntax used for each Job Entry is

\Arg{JOB} \Arg{JobName} \Arg{SubmitDescriptionFileName} \oArg{DONE}

A Job Entry maps a \Arg{JobName} to a Condor submit description file.
The \Arg{JobName} uniquely identifies nodes within the
DAGMan input file and within output messages.

The keyword \Arg{JOB} and the \Arg{JobName} are not case sensitive.
A \Arg{JobName} of \Arg{joba} is equivalent to \Arg{JobA}.
The \Arg{SubmitDescriptionFileName} is case sensitive, since
the UNIX file system is case sensitive.
The \Arg{JobName} can be any string that contains no white space.

The optional \Arg{DONE} identifies a job as being already
completed.
This is useful in situations where the user wishes to verify results,
but does not need all programs within the dependency graph to be executed.
The \Arg{DONE} feature is also utilized when
an error occurs causing the DAG to not be completed.
DAGMan generates a Rescue DAG, a DAGMan input file that can be
used to restart and complete a DAG without re-executing
completed programs.

The second type of item in a DAGMan input file enumerates
processing that is done either before a program within
the DAG is submitted to Condor for execution
or after
a program within
the DAG completes its execution.
\index{DAGMan!PRE script}
Processing done before a program is submitted to Condor is
called a \Arg{PRE} script.
Processing done after a program successfully completes
its execution under Condor is
\index{DAGMan!POST script}
called a \Arg{POST} script.
A node in the DAG is comprised of the program together with
\Arg{PRE} and/or \Arg{POST} scripts.
The dependencies in the DAG are enforced based on nodes.

Syntax for \Arg{PRE} and \Arg{POST} script lines within the input file

\Arg{SCRIPT} \Arg{PRE} \Arg{JobName} \Arg{ExecutableName} \oArg{arguments}

\Arg{SCRIPT} \Arg{POST}  \Arg{JobName} \Arg{ExecutableName} \oArg{arguments}

The \Arg{SCRIPT} keyword identifies the type of line within
the DAG input file.
The \Arg{PRE} or \Arg{POST} keyword
specifies the relative timing of when the script is to be run.
The \Arg{JobName} specifies the node to which the script is attached.
The \Arg{ExecutableName}
specifies the script to be executed, and it
may be followed by any command line arguments to that script.
The \Arg{ExecutableName} and optional \Arg{arguments} have their
case preserved.

Scripts are optional for each job, and
any scripts are executed on the machine
to which the DAG is submitted.

The PRE and POST scripts are commonly used
when files must be placed into a staging area for the job to use,
and files are cleaned up or removed once the job is finished running.
An example using PRE/POST scripts involves staging files
that are stored on tape.
The PRE script reads compressed input files from the tape drive,
and it uncompresses them, placing the input files in the current directory.
The program within the DAG node is submitted to Condor,
and it reads these input files.
The program produces output files.
The POST script compresses the output files, writes them out to
the tape, and then deletes the staged input and output files.

DAGMan takes note of the exit value of the
scripts as well as the program.
If the PRE script fails (exit value != 0), then neither the program nor
the POST script runs, and the node is marked as failed.

If the PRE script succeeds, the program is submitted to Condor. 
If the program fails and there is no POST script,
the DAG node is marked as failed.
An exit value not equal to 0 indicates program failure.
It is therefore important that the program returns the exit
value 0 to indicate the program did not fail.

If the program fails and there is a POST script,
node failure is determined by the exit value of the POST script.
A failing value from the POST script marks the node as failed.
A succeeding value from the POST script (even with a failed
program) marks the node as successful.
Therefore, the POST script may need to consider the return
value from the program.

By default, the POST script is run regardless of the program's
return value.  To prevent POST scripts from running after failed jobs,
pass the \Arg{-NoPostFail} argument to \Condor{submit\_dag}.

A node not marked as failed at any point is successful.

Two variables are available to ease script writing.
The \Env{\$JOB} variable evaluates to \Arg{JobName}.
For POST scripts, the \Env{\$RETURN} variable evaluates to the return value of the program.
The variables may be
placed anywhere within the arguments.

As an example, suppose the \Arg{PRE} script expands a compressed file named
\File{\Arg{JobName}.gz}.
The \Arg{SCRIPT} entry for jobs A, B, and C are

\begin{verbatim}
	SCRIPT PRE  A  pre.csh $JOB .gz
	SCRIPT PRE  B  pre.csh $JOB .gz
	SCRIPT PRE  C  pre.csh $JOB .gz
\end{verbatim}

The script \File{pre.csh} may use these arguments

\begin{verbatim}
	#!/bin/csh
	gunzip $argv[1]$argv[2]
\end{verbatim}

% $ % this comment just has a dollar sign so that emacs will not think
%	  we're inside of a math section and will draw things more nicely

The third type of item in the DAG input file describes the
dependencies within the DAG.
\index{DAGMan!describing dependencies}
Nodes are parents and/or children within the DAG.
A parent node must be completed successfully before
any child node may be started.
A child node is started once
all its parents have successfully completed.

The syntax of a dependency line within the DAG input file:

\Arg{PARENT} \Arg{ParentJobName\Dots} \Arg{CHILD} \Arg{ChildJobName\Dots}

The \Arg{PARENT} keyword is followed by one or more
\Arg{ParentJobName}s.
The \Arg{CHILD} keyword is followed by one or more
\Arg{ChildJobName}s.
Each child job depends on every parent job on the line.
A single line in the input file can specify the dependencies from one or more
parents to one or more children.
As an example, the line
\begin{verbatim}
PARENT p1 p2 CHILD c1 c2
\end{verbatim}
produces four dependencies:
\begin{enumerate}
\item{\verb@p1@ to \verb@c1@}
\item{\verb@p1@ to \verb@c2@}
\item{\verb@p2@ to \verb@c1@}
\item{\verb@p2@ to \verb@c2@}
\end{enumerate}

The fourth type of item in the DAG input file provides a
way (optional) to retry failed nodes.
The syntax for retry is

\Arg{RETRY} \Arg{JobName} \Arg{NumberOfRetries}

where the \Arg{JobName} is the same as the name given in
a Job Entry line, and \Arg{NumberOfRetries} is an integer,
the number of times to retry the node after failure.
The default number of retries for any node is 0,
the same as not having a retry line in the file.

In the event of retry, all parts of a node within the DAG
are redone, following the same rules regarding node failure
as given above.
The PRE script is executed first,
followed by submitting the program to Condor upon success of
the PRE script.
Failure of the node is then determined by the return value of
the program, the existence and return value of a POST script.

The fifth type of item in the DAG input file provides a
method of defining a macro to be placed into the submit description
file.
These macros are defined on a per-node basis, using the
following format.

\Arg{VARS} \Arg{JobName} \Arg{macroname="string"\Dots}

The definition of the macro is available to use within the
submit description file, as the submission for the node's
executable is of the form
\begin{verbatim}
condor_submit SubmitDescriptionFileName -a "+macroname=\"string\""
\end{verbatim}
Job submission with the "+" attribute not only adds the attribute
to the ClassAd of the job,
but also allows dereferencing of the attribute through the macro
usage in the submit description file.
% the string must be enclosed in double quotes. To add a double quote inside
% the string, you can escape it with \. To add the '\' itself, you can use \\.

There may be more than one macro defined for each \Arg{JobName}.
The space character delimits the list of macros.

Correct syntax requires that the \verb@string@ must be
enclosed in double quotes.
To use a double quote inside \verb@string@,
escape it with the backslash character (\verb@\@).
To add the backslash character itself, use two backslashes (\verb@\\@).

% for version 6.5.3, the VARS are not propogated into rescue dags.


%%%%%%%%%%%%%%%%%%%%%%%%%%%%%%%%%%%%%%%
\subsection{Condor Submit Description File}
%%%%%%%%%%%%%%%%%%%%%%%%%%%%%%%%%%%%%%%

\index{DAGMan!submit description file with}
Each node in a DAG may be a unique executable, and each may have a unique
Condor submit description file.
Each program may be submitted to a different universe
within Condor, for example standard,
vanilla, or DAGMan.

One limitation exists:
each Condor submit description file must submit only one job.
There may not be multiple \verb@queue@ commands, or DAGMan will fail.
This requirement exists to enforce the requirements of a well-defined DAG.
If each node of the DAG could cause the submission of multiple
Condor jobs, then it would violate the definition of a DAG.

DAGMan no longer requires that all jobs specify the same log file.
However, if the DAG contains a very large number of jobs, each
specifying its own log file, performance may suffer.  Therefore,
if the DAG contains a large number of jobs, it is best to have
all of the jobs use the same log file.
DAGMan enforces the dependencies within a DAG
using the events recorded in the
log file(s) produced by job submission to Condor.

Here is a simple input file for a 
modified version of diamond-shaped DAG example.

\begin{verbatim}
	# Filename: diamond.dag
	#
	Job  A  diamond_job.condor 
	Job  B  diamond_job.condor 
	Job  C  diamond_job.condor	
	Job  D  diamond_job.condor
	PARENT A CHILD B C
	PARENT B C CHILD D
\end{verbatim}

A single Condor submit description file goes with all the nodes
in this DAG:

\index{DAGMan!example submit description file}
\begin{verbatim}
	# Filename: diamond_job.condor
	#
	executable   = /path/diamond.exe
	output       = diamond.out.$(cluster)
	error        = diamond.err.$(cluster)
	log          = diamond_condor.log
	universe     = vanilla
	notification = NEVER
	queue
\end{verbatim}

This example uses the same Condor submit description file
for all the jobs in the DAG.
This implies that each node within the DAG runs the
same program.
The \MacroU{cluster} macro
is used to produce unique file names for each program's output.
Each job is submitted separately, into its own cluster,
so this provides unique names for the output files.

The notification is set to \verb@NEVER@ in this example.
This tells Condor not to send e-mail about the completion of a program
submitted to Condor.
For DAGs with many nodes, this is recommended
to reduce or eliminate excessive numbers of e-mails.

A separate example shows an intended use of a \Arg{VARS} entry
in the DAG.
It can be used to dramatically reduce the number of submit description
files needed for a DAG.
In the case where the submit description file for each node
varies only in file naming, the use of a substitution macro
within the submit description file allows the use of 
a single submit description file.
Note that the node output log file currently cannot be
specified using a macro passed from the DAG.

The example uses a single submit description file in the DAG input
file, and uses the \Arg{Vars} entry to name output files.

\begin{verbatim}
	# submit description file called:  theonefile.sub
	executable   = progX
	output       = $(outfilename)
	error        = error.$(outfilename)
	universe     = standard
	queue
\end{verbatim}

The relevant portion of the DAG input file appears as 
\begin{verbatim}
JOB A theonefile.sub
JOB B theonefile.sub
JOB C theonefile.sub

VARS A outfilename="A"
VARS B outfilename="B"
VARS C outfilename="C"
\end{verbatim}

For a DAG like this one with thousands of nodes,
being able to write and maintain a single submit description file 
and a single, yet more complex, DAG input file is preferable.

%%%%%%%%%%%%%%%%%%%%%%%%%%%%%%%%%%%%%%%
\subsection{\label{dagman:submitdag}Job Submission}
%%%%%%%%%%%%%%%%%%%%%%%%%%%%%%%%%%%%%%%

A DAG is submitted using the program \Condor{submit\_dag}.
See the manual
page~\pageref{man-condor-submit}
for complete details.
A simple submission has the syntax

\Condor{submit\_dag} \Arg{DAGInputFileName}

\index{DAGMan!job submission}
The example may be submitted with

\begin{verbatim}
condor_submit_dag diamond.dag
\end{verbatim}
In order to guarantee recoverability, the DAGMan program itself
is run as a Condor job.
As such, it needs a submit description file.
\Condor{submit\_dag} produces the needed file,
naming it by appending the \Arg{DAGInputFileName} with
\File{.condor.sub}.
This submit description file may be edited if the DAG is
submitted with

\begin{verbatim}
condor_submit_dag -no_submit diamond.dag
\end{verbatim}
causing \Condor{submit\_dag} to generate the submit description file,
but not submit DAGMan to Condor.
To submit the DAG, once the submit description file is edited,
use

\begin{verbatim}
condor_submit diamond.dag.condor.sub
\end{verbatim}

An optional argument to \Condor{submit\_dag}, \Arg{-maxjobs}, 
is used to specify the maximum number of Condor jobs that DAGMan may
submit to Condor at one time.
It is commonly used when 
there is a limited amount of input file staging capacity.
As a specific example, consider a case where each job will
require 4 Mbytes of input files,
and the jobs will run in a directory with a volume of 100 Mbytes
of free space.
Using the argument \Arg{-maxjobs 25} guarantees that a maximum
of 25 jobs, using a maximum of 100 Mbytes of space,
will be submitted to Condor at one time.

% -maxscripts has been replaced with -maxpre and -maxpost
% Similarly, the \Arg{maxscripts} argument is used to specify the
% maximum number of PRE and POST scripts running at one time.
While the \Arg{-maxjobs} argument is used to limit the number
of Condor jobs submitted at one time,
it may be desirable to limit the number of scripts running
at one time.
The optional \Arg{-maxpre} argument limits the number of PRE
scripts that may be running at one time,
while the optional \Arg{-maxpost} argument limits the number of POST
scripts that may be running at one time.

%%%%%%%%%%%%%%%%%%%%%%%%%%%%%%%%%%%%%%%
\subsection{Job Monitoring}
%%%%%%%%%%%%%%%%%%%%%%%%%%%%%%%%%%%%%%%

After submission, the progress of the DAG can be monitored
by looking at the log file(s),
observing the e-mail that program submission to Condor causes,
or by using \Condor{q} \Arg{-dag}.

%%%%%%%%%%%%%%%%%%%%%%%%%%%%%%%%%%%%%%%
\subsection{Job Failure and Job Removal}
%%%%%%%%%%%%%%%%%%%%%%%%%%%%%%%%%%%%%%%

\Condor{submit\_dag} attempts to check the DAG input file.
If a problem is detected,
\Condor{submit\_dag} prints out an error message and aborts.

DAGMan normally generates a list of job log files to
monitor by examining all of the job submission files.  If
that will not work (some job submission files will be generated
by \Arg{PRE} scripts, for example), you can specify a single
common log file with the \Arg{-log} option.
An example of this submission:
\begin{verbatim}
condor_submit_dag -log diamond_condor.log
\end{verbatim}
This option tells \Condor{submit\_dag} use the given file as
the log file, if no log files are specified in submit files.

To remove an entire DAG, consisting of DAGMan plus
any jobs submitted to Condor,
remove the DAGMan job running under Condor.
\Condor{q} will list the job number.
Use the job number to remove the job, for example

\footnotesize
\begin{verbatim}

% condor_q
-- Submitter: turunmaa.cs.wisc.edu : <128.105.175.125:36165> : turunmaa.cs.wisc.edu
 ID      OWNER            SUBMITTED     RUN_TIME ST PRI SIZE CMD
    9.0   smoler         10/12 11:47   0+00:01:32 R  0   8.7  condor_dagman -f -
    11.0   smoler         10/12 11:48   0+00:00:00 I  0   3.6  B.out
    12.0   smoler         10/12 11:48   0+00:00:00 I  0   3.6  C.out

         3 jobs; 2 idle, 1 running, 0 held

% condor_rm 9.0
\end{verbatim}
\normalsize

Before the DAGMan job stops running, it uses \Condor{rm}
to remove any Condor jobs within the DAG that are running.

In the case where a
machine is scheduled to go down,
DAGMan will clean up memory and exit.
However, it will leave any submitted jobs
in Condor's queue.

%%%%%%%%%%%%%%%%%%%%%%%%%%%%%%%%%%%%%%%
\subsection{Job Recovery:  The Rescue DAG}
%%%%%%%%%%%%%%%%%%%%%%%%%%%%%%%%%%%%%%%

\index{DAGMan!rescue DAG}
DAGMan can help with the resubmission of uncompleted
portions of a DAG when one or more nodes resulted in failure.
If any node in the DAG fails,
the remainder of the DAG is continued until no more forward
progress can be made based on the DAG's dependencies.
At this point, DAGMan produces a file
called a Rescue DAG.

The Rescue DAG is a DAG input file,
functionally the same as the original DAG file.
It additionally contains indication of
successfully completed nodes using the \Arg{DONE}
option in the input description file.
If the DAG is resubmitted using this Rescue DAG input file,
the nodes marked as completed will not be re-executed.

The Rescue DAG is automatically generated by DAGMan when a node
within the DAG fails.
The file is named using the \Arg{DAGInputFileName}, and appending
the suffix \File{.rescue} to it.
Statistics about the failed DAG execution are presented as
comments at the beginning of the Rescue DAG input file.

If the Rescue DAG file is generated before all retries
of a node are completed, 
then the Rescue DAG file will also contain Retry entries.
The number of retries will be set to the appropriate remaining
number of retries. 


%%%%%%%%%%%%%%%%%%%%%%%%%%%%%%%%%%%%%%%
\subsection{DAGMan Configuration}
%%%%%%%%%%%%%%%%%%%%%%%%%%%%%%%%%%%%%%%
\index{DAGMan!configuration}

\index{DAGMan!DAGMAN\_MAX\_SUBMITS\_PER\_INTERVAL}
\Macro{DAGMAN\_MAX\_SUBMITS\_PER\_INTERVAL} is an integer that
controls how many individual jobs \Condor{dagman} will submit in a row
before servicing other requests (such as a \Condor{rm}).  It cannot be
set to less than 1, or more than 1000; if undefined, it defaults to 5.

\index{DAGMan!DAGMAN\_MAX\_SUBMIT\_ATTEMPTS}
\index{DAGMan!submit retry}
\Macro{DAGMAN\_MAX\_SUBMIT\_ATTEMPTS} is an integer that controls how
many times in a row \Condor{dagman} will attempt to execute
\Condor{submit} for a given job before giving up.  Note that
consecutive attempts use an exponential backoff, starting with 1
second.  DAGMAN\_MAX\_SUBMIT\_ATTEMPTS cannot be set to less than 1,
or more than 16 (which would result in \Condor{dagman} trying for
approximately 36 hours before giving up); if left undefined, it
defaults to 6 (approximately two minutes before giving up).

\index{DAGMan!DAGMAN\_SUBMIT\_DELAY}
\Macro{DAGMAN\_SUBMIT\_DELAY} is an integer that controls the number
of seconds \Condor{dagman} will sleep before submitting consecutive
jobs.  It can be increased to help reduce the load on the schedd.  It
cannot be set to less than 0 or more than 60; if undefined, it
defaults to 0.

\index{DAGMan!DAGMAN\_STARTUP\_CYCLE\_DETECT}
\Macro{DAGMAN\_STARTUP\_CYCLE\_DETECT} is a boolean that controls
whether \Condor{dagman} checks for cycles in the DAG structure at
startup, in addition to its runtime cycle detection.  The startup
check can be expensive for large DAGs, so if undefined, it defaults to
False.


%%%%%%%%%%%%%%%%%%%%%%%%%%%%%%%%%%%%%%%
\subsection{Visualizing DAGs with \Prog{dot}}
%%%%%%%%%%%%%%%%%%%%%%%%%%%%%%%%%%%%%%%
\index{DAGMan!dot}
\index{dot}
\index{DAGMan!visualizing DAGs}

It can be helpful to see a picture of a DAG.
DAGMan can assist you in visualizing a DAG by creating
the input files used by the AT\&T Research Labs 
\Prog{graphviz} package. 
\Prog{dot} is a program within this package,
available from \URL{http://www.research.att.com/sw/tools/graphviz},
and it is used to draw pictures of DAGs. 

DAGMan produces one or more dot files as the result of
an extra line
in a DAGMan input file. 
The line appears as
%For example, to produce a single dot
%file that shows the state of your DAG before any jobs are running, add
%the following line:
\begin{verbatim}
	DOT dag.dot
\end{verbatim}

This creates a file called \File{dag.dot}.
which contains
a specification of the DAG before any programs within the DAG
are submitted to Condor.
The \File{dag.dot} file is used to create a visualization
of the DAG by using this file as input to \Prog{dot}.
This example creates a Postscript file, with a visualization of the DAG:

\begin{verbatim}
    dot -Tps dag.dot -o dag.ps
\end{verbatim}

Within the DAGMan input file,
the DOT command can take several optional parameters:

\begin{itemize}

\item \Opt{UPDATE}  This will update the dot file every time a
significant update happens. 

\item \Opt{DONT-UPDATE} Creates a single dot file, when
the DAGMan begins executing. This is the default if the parameter
\Opt{UPDATE} is not used.

\item \Opt{OVERWRITE} Overwrites the dot file each time it
is created. This is the default, unless \Opt{DONT-OVERWRITE}
is specified.

\item \Opt{DONT-OVERWRITE} Used to create multiple dot files, instead
of overwriting the single one specified.
To create file names,
DAGMan uses the name of the file concatenated a period and an
integer. For example, the DAGMan input file line
\begin{verbatim}
	DOT dag.dot DONT-OVERWRITE
\end{verbatim}
causes files
\File{dag.dot.0},
\File{dag.dot.1},
\File{dag.dot.2},
etc. to be created.
This option is
most useful combined with the \Opt{UPDATE} option to
visualize the history of the DAG after it has finished executing. 

\item \OptArg{INCLUDE}{path-to-filename} Includes the contents
of a file given by \File{path-to-filename} in the file produced by the
\Opt{DOT} command.
The include file contents are always placed after the line of
the form
\verb@label=@.
This may be useful if further editing of the created files would
be necessary,
perhaps because you are automatically visualizing the DAG as it
progresses. 

\end{itemize}

If conflicting parameters are used in a DOT command, the last one
listed is used.
%%%%%%%%%%%%%%%%%%%%%%%%%%%%%%%%%%%%%%%
\subsection{\label{sec:DAGsinDAGs}Advanced Usage: A DAG within a DAG}
%%%%%%%%%%%%%%%%%%%%%%%%%%%%%%%%%%%%%%%
\index{DAGMan!DAGs within DAGs}

The organization and dependencies of the jobs within a DAG
are the keys to its utility.
There are cases when a DAG is easier to visualize and 
construct hierarchically,
as when a node within a DAG is also a DAG.
Condor DAGMan handles this situation with grace.

Since more than one DAG is being discussed, 
terminology is introduced to clarify which DAG is which. 
Reuse the example DAG as given in 
Figure~\ref{fig:dagman-diamond}.
Assume that node B of this diamond-shaped DAG
will itself be a DAG.
The DAG of node B is called the inner DAG,
and the diamond-shaped DAG is called the outer DAG.

To make DAGs within DAGs,
the essential element is getting the name of the submit description
file for the inner DAG correct within the outer DAG's input
file.

Work on the inner DAG first.
The goal is to generate a Condor submit description file for this inner DAG.
Here is a very simple DAG input file used as an example of the inner DAG.
\begin{verbatim}
	# Filename: inner.dag
	#
	Job  X  X.submit
	Job  Y  Y.submit
	Job  Z  Z.submit
	PARENT X CHILD Y
	PARENT Y CHILD Z
\end{verbatim}

Use \Condor{submit\_dag} to create a submit description file for this
inner dag:
\begin{verbatim}
   condor_submit_dag -no_submit inner.dag
\end{verbatim}
The resulting file will be named \File{inner.dag.condor.sub}.
This file will be needed in the DAG input file of the outer DAG.
The naming of the file is the name of the DAG input file
(\File{inner.dag}) with the suffix \File{.condor.sub}.

A simple example of a DAG input file for the outer DAG is
\begin{verbatim}
	# Filename: diamond.dag
	#
	Job  A  A.submit 
	Job  B  inner.dag.condor.sub
	Job  C  C.submit	
	Job  D  D.submit
	PARENT A CHILD B C
	PARENT B C CHILD D
\end{verbatim}

The outer DAG is then submitted as before, with
\begin{verbatim}
   condor_submit_dag diamond.dag
\end{verbatim}

More than one level of nested DAGs is supported.

One item to get right:
to locate the log files used in ordering the DAG,
DAGMan either needs a completely flat directory structure
(\emph{all} files for outer and inner DAGS within the same directory)
or
it needs full pathnames to all log files.
\index{DAGMan|)}

%%%%%%%%%%%%%%%%%%%%%%%%%%%%%%%%%%%%%%%%%%%%%%%%%%%%%%%%%%%%%%%%%%%%%%

%%%%%%%%%%%%%%%%%%%%%%%%%%%%%%%%%%%%%%%%%%%%%%%%%%%%%%%%%%%%%%%%%%%%%%

\section{Job Monitor}
\index{Job monitor}
\index{viewing!log files}

The Condor Job Monitor is a Java application designed to allow users to view user log files. 

To view a user log file, select it using the open file command in the File menu.  After the file is parsed, it will be visually represented.  Each horizontal line represents an individual job.  The x-axis
is time.  Whether a job is running at a particular time is represented by its color at that time -- white for running, black for idle.  For example, a job which appears predominantly white has made
efficient progress, whereas a job which appears predominantly black has received an inordinately small proportion of computational time. 


\subsection{\label{sec:transition-states}Transition States}

A transistion state is the state of a job at any time.  It is called a "transistion" because it is defined by the two events which bookmark it.  There are two basic transistion states: running and idle. 
An idle job typically is a job which has just been submitted into the Condor pool and is waiting to be matched with an appropriate machine or a job which has vacated from a machine and has been
returned to the pool.  A running job, by contrast, is a job which is making active progress. 

Advanced users may want a visual distinction between two types of running transistions: "goodput" or "badput".  Goodput is the transistion state preceding an eventual job completion or
checkpoint.  Badput is the transistion state preceding a non-checkpointed eviction event.  Note that "badput" is potentially a misleading nonmenclature; a job which is not checkpointed by the
Condor program may checkpoint itself or make progress in some other way.  To view these two transistion as distinct transistions, select the appropriate option from the "View" menu. 


\subsection{\label{sec:events}Events}

There are two basic kinds of events: checkpoint events and error events.   Plus advanced users can ask to see more events. 


\subsection{\label{sec:job-selector}Zooming}

To view any arbitrary selection of jobs in a job file, use the job selector tool.  Jobs appear visually by order of appearence within the actual text log file.  For example, the log file might contain jobs
775.1, 775.2, 775.3, 775.4, and 775.5, which appear in that order.  A user who wishes to see only jobs 775.2 and 775.5 can select only these two jobs in the job selector tool and click the "Ok" or
"Apply" button.  The job selector supports double clicking; double
click on any single job to see it drawn in isolation. 

\subsection{\label{sec:zooming}Zooming}

To view a small area of the log file, zoom in on the area which you would like to see in greater detail. You can zoom in, out and do a full zoom. A full zoom redraws the log file in its entirety. For
example, if you have zoomed in very close and would like to go all the way back out, you could do so with a succession of zoom outs or with one full zoom. 

There is a difference between using the menu driven zooming and the mouse driven zooming. The menu driven zooming will recenter itself around the current center, whereas mouse driven
zooming will recenter itself (as much as possible) around the mouse click. To help you refind the clicked area, a box will flash after the zoom. This is called the "zoom finder" and it can be turned
off in the zoom menu if you prefer. 

\subsection{\label{sec:k-m-shortcuts}Keyboard and Mouse Shortcuts}

\begin{enumerate}
\item The Keyboard shortcuts: 

\begin{itemize}
\item Arrows - an approximate ten percent scrollbar movement
\item PageUp and PageDown - an approximate one hundred percent scrollbar movemnet 
\item Control + Left or Right - approximate one hundred percent scrollbar movement 
\item End and Home - scrollbar movement to the vertical extreme 
\item Others - as seen beside menu items
\end{itemize}

\item The mouse shortcuts: 

\begin{itemize}
\item Control + Left click - zoom in 
\item Control + Right click - zoom out
\item Shift + left click - re-center
\end{itemize}
\end{enumerate}
 


%%%%%%%%%%%%%%%%%%%%%%%%%%%%%%%%%%%%%%%%%%%%%%%%%%%%%%%%%%%%%%%%%%%%%%

%%%%%%%%%%%%%%%%%%%%%%%%%%%%%%%%%%%%%%%%%%%%%%%%%%%%%%%%%%%%%%%%%%%%%%
\section{\label{sec:Vacate-Explained}
About How Condor Jobs Vacate Machines}
%%%%%%%%%%%%%%%%%%%%%%%%%%%%%%%%%%%%%%%%%%%%%%%%%%%%%%%%%%%%%%%%%%%%%%

\index{vacate}
When Condor needs a job to vacate a machine for whatever reason, it
sends the job an asynchronous signal specified in the \AdAttr{KillSig}
attribute of the job's ClassAd.
The value of this attribute can be specified by
the user at submit time by placing the \Opt{kill\_sig} option in the
Condor submit description file.  

If a program wanted to do some special work when required
to vacate a machine, the program may set up a
signal handler to use a trappable signal as an indication
to clean up.
When submitting this job, this clean up signal is specified to be used with
\Opt{kill\_sig}.
Note that the clean up work needs to be quick.
If the job takes too long to go away, Condor
follows up with a SIGKILL signal which immediately terminates the
process.

\index{Condor commands!condor\_compile}
A job that is linked using \Condor{compile}
and is subsequently submitted into the standard universe, 
will checkpoint and exit upon receipt of a SIGTSTP signal.
Thus, SIGTSTP is
the default value for \AdAttr{KillSig} when submitting to the standard
universe.
The user's code may still checkpoint itself at any time
by calling one of the following functions exported by the Condor libraries:
\begin{description}
\item[\func{ckpt()}] Performs a checkpoint and then returns.
\item[\func{ckpt\_and\_exit()}] Checkpoints and exits; Condor will then
restart the process again later, potentially on a different machine.
\end{description}

For jobs submitted into the vanilla universe, the default value for
\AdAttr{KillSig} is SIGTERM,
the usual method to nicely terminate a Unix program.

%%%%%%%%%%%%%%%%%%%%%%%%%%%%%%%%%%%%%%%%
\section{Special Environment Considerations}
%%%%%%%%%%%%%%%%%%%%%%%%%%%%%%%%%%%%%%%%

\subsection{AFS}

\index{file system!AFS}
\index{AFS!interaction with}
The Condor daemons do not run authenticated to AFS; they do not possess
AFS tokens.
Therefore, no child process of Condor will be AFS authenticated.
The implication of this is that you must set file permissions so
that your job can access any necessary files residing on an AFS volume
without relying on having your AFS permissions.

If a job you submit to Condor needs to access files residing in AFS,
you have the following choices:
\begin{enumerate}
\item Copy the needed files from AFS to either a local hard disk where 
Condor can access them using remote system calls (if
this is a standard universe job), or copy them to an NFS volume.
\item If you must keep the files on AFS, then set a host ACL
(using the AFS \Prog{fs setacl} command) on the subdirectory to
serve as the current working directory for the job.
If a standard universe job, then the host ACL needs
to give read/write permission to any process on the submit machine.
If vanilla universe job, then you need to set the ACL such that any host 
in the pool can access the files without being authenticated.
If you do not know how to use an AFS host ACL, ask the person at your 
site responsible for the AFS configuration.
\end{enumerate}

The Condor Team hopes to improve upon how Condor deals with AFS 
authentication in a subsequent release.

Please see section~\ref{sec:Condor-AFS-Users} on
page~\pageref{sec:Condor-AFS-Users} in the Administrators Manual for
further discussion of this problem.

\subsection{NFS Automounter}

\index{file system!NFS}
\index{NFS!interaction with}
If your current working directory when you run \Condor{submit}
\index{Condor commands!condor\_submit}
is accessed via an NFS automounter, Condor may have problems if the
automounter later decides to unmount the volume before your job has
completed.
This is because \Condor{submit} likely has stored the
dynamic mount point as the job's initial current working directory, and
this mount point could become automatically unmounted by the
automounter.

There is a simple work around: When submitting your job, use the 
\Arg{initialdir} command in your submit description file to point to
the stable access point.
For example,
suppose the NFS automounter is configured to mount a volume at mount point
\File{/a/myserver.company.com/vol1/johndoe}
whenever the directory \File{/home/johndoe} is accessed.
Adding the following line to the
submit description file solves the problem.
\begin{verbatim}
        initialdir = /home/johndoe
\end{verbatim}

\subsection{Using Globus software with Condor}
\index{universe!Globus}
\index{Globus!Interaction considerations}
Use of the Globus project software \Url{http://www.globus.org} with
Condor affects these issues:

\begin{enumerate}
\item[GSS Authentication] Is an option only in special versions of Condor,
available by request only, due to cryptographic software export controls
and Condor distribution policy.
Sites running the Condor software distributed with GSS-Authentication can 
set up their own Certification Authority (CA) by running the 
\Prog{create\_ca} script. 
Once the CA is set up, the \Condor{ca} script is used to generate 
certificates for the Condor daemons (e.g., \Condor{schedd}) and to sign
user and daemon certificates. Users can generate certificate requests
and other needed files with the \Condor{cert} program.
An X.509 certificate directory pointed to by the submit description file 
variable \Arg{x509CertDir} indicates a client program which can use GSS
authentication as a possible authentication method. Alternately, the
environment variables \Env{X509\_CERT\_DIR}, \Env{X509\_USER\_CERT},
\Env{X509\_USER\_KEY} can be used to override the default filenames and 
locations.
\Note the AUTHENTICATION\_METHOD configuration value list must contain 
the value 'GSS' for GSS authentication to be attempted.

\item[Submitting to the Globus Universe] \index{Globus!Globus universe}
\index{universe!Globus}
requires Globus version 1.1, as well as a valid Globus X.509 certificate. 
The default location for the necessary files is \$HOME/.globus, but they 
can be overridden by setting the X509\_* variables in your environment 
or the submit description file.
\Note AFS issues apply here, so you may have to copy your certificate,
trusted certificates directory, private key, and proxy to a local
file system disk.

\item[\Condor{glidein}] \index {Globus!\Condor{glidein}} requires a valid 
Globus X.509 certificate, and the PATH to the \Prog{globusrun} program 
must be in your environment.
\Note to allow a globus resource to join your Condor pool, your 
administrator must add the hostname(s) to the HOSTALLOW\_WRITE and
HOSTALLOW\_READ configuration values.
\end{enumerate}

\subsection{Condor Daemons That Do Not Run as root}

\index{Unix daemon!running as root}
\index{daemon!running as root}
Condor is normally installed such that the Condor daemons have root
permission.
This allows Condor to run the \condor{shadow} 
\index{Condor daemon!condor\_shadow}
\index{remote system call!condor\_shadow}
process and
your job with your UID and file access rights.
When Condor
is started as root, your Condor jobs can access whatever files you can.

However, it is possible that whomever installed Condor 
did not have root access, or
decided not to run the daemons as root.
That is unfortunate,
since Condor is designed to be run as the Unix user root.
To see if Condor is
running as root on a specific machine, enter the command
\begin{verbatim}
        condor\_status -master -l <machine-name>
\end{verbatim}

where \verb@machine-name@ is the name of the specified machine.
This command displays a \condor{master} ClassAd; if the
attribute \AdAttr{RealUid} equals zero,
then the Condor daemons are indeed
running with root access.  If the
\AdAttr{RealUid} attribute is not zero, then the Condor daemons do not have
root access.

\Note The Unix program \Prog{ps}
is \emph{not} an effective
method of determining if Condor is running with root access.
When using \Prog{ps},
it may often appear that the daemons are
running as the condor user instead of root.
However, note that the \Prog{ps},
command shows the current \emph{effective} owner of the
process, not the \emph{real} owner.  (See the \Cmd{getuid}{2} and
\Cmd{geteuid}{2} Unix man pages for details.)  In Unix, a process
running under the real UID of root may switch its effective UID.
(See the \Cmd{seteuid}{2} man page.)
For security reasons, the daemons
only set the effective uid to root when absolutely necessary
(to perform a privileged operation).

If they are not running with root access, you need to make any/all files
and/or directories that your job will touch readable and/or writable by
the UID (user id) specified by the RealUid attribute.
Often this may
mean using the Unix command \verb@chmod 777@
on the directory where you submit your Condor job.

%%%%%%%%%%%%%%%%%%%%%%%%%%%%%%%%%%%%%%%%
\section{Potential Problems}
%%%%%%%%%%%%%%%%%%%%%%%%%%%%%%%%%%%%%%%%

\subsection{Renaming of argv[0]}

\index{argv[0]!Condor use of}
When Condor starts up your job, it renames argv[0] (which usually
contains the name of the program) to \condor{exec}.
This is
convenient when examining a machine's processes with the Unix
command \Prog{ps}; the process
is easily identified as a Condor job.  

Unfortunately, some programs read argv[0] expecting their own program
name and get confused if they find something unexpected like
\condor{exec}.

\index{Condor!user manual|)}
\index{user manual|)}
