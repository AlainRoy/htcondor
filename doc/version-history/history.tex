%%%%%%%%%%%%%%%%%%%%%%%%%%%%%%%%%%%%%%%%%%%%%%%%%%%%%%%%%%%%%%%%%%%%%%
\section{\label{sec:History-Intro}Introduction to Condor Versions}
%%%%%%%%%%%%%%%%%%%%%%%%%%%%%%%%%%%%%%%%%%%%%%%%%%%%%%%%%%%%%%%%%%%%%%

This chapter provides descriptions of what features have been added or
bugs fixed for each version of Condor.
The first section describes the Condor version numbering scheme, what
the numbers mean, and what the different \Term{release series} are.
The rest of the sections each describe a specific release series, and
all the Condor versions found in that series.

%%%%%%%%%%%%%%%%%%%%%%%%%%%%%%%%%%%%%%%%%%%%%%%%%%%%%%%%%%%%%%%%%%%%%%
\subsection{\label{sec:Version-Number-Scheme}
Condor Version Number Scheme}
%%%%%%%%%%%%%%%%%%%%%%%%%%%%%%%%%%%%%%%%%%%%%%%%%%%%%%%%%%%%%%%%%%%%%%

Starting with version 6.0.1, Condor adopted a new, hopefully easy to
understand version numbering scheme.
It reflects the fact that Condor is both a production system and a
research project.
The numbering scheme was primarily taken from the Linux kernel's
version numbering, so if you are familiar with that, it should seem
quite natural.

There will usually be two Condor versions available at any given time,
the \Term{stable} version, and the \Term{development} version.
Gone are the days of ``patch level 3'', ``beta2'', or any other random
words in the version string.
All versions of Condor now have exactly three numbers, separated by
``.''   

\begin{itemize}

\item The first number represents the major version number, and will
change very infrequently.

\item \emph{The thing that determines whether a version of Condor is
\Term{stable} or \Term{development} is the second digit.
Even numbers represent stable versions, while odd numbers represent
development versions.}

\item The final digit represents the minor version number, which
defines a particular version in a given release series.

\end{itemize}


%%%%%%%%%%%%%%%%%%%%%%%%%%%%%%%%%%%%%%%%%%%%%%%%%%%%%%%%%%%%%%%%%%%%%%
\subsection{\label{sec:Stable-Series}The Stable Release Series}
%%%%%%%%%%%%%%%%%%%%%%%%%%%%%%%%%%%%%%%%%%%%%%%%%%%%%%%%%%%%%%%%%%%%%%

People expecting the stable, production Condor system should download
the stable version, denoted with an even number in the second digit of
the version string.
Most people are encouraged to use this version.  
We will only offer our paid support for versions of Condor from the
stable release series.

\emph{On the stable series, new minor version releases will only
be made for bug fixes and to support new platforms.}
No new features will be added to the stable series.
People are encouraged to install new stable versions of Condor when
they appear, since they probably fix bugs you care about.
Hopefully, there will not be many minor version releases for any given
stable series.


%%%%%%%%%%%%%%%%%%%%%%%%%%%%%%%%%%%%%%%%%%%%%%%%%%%%%%%%%%%%%%%%%%%%%%
\subsection{\label{sec:Developement-Series}
The Development Release Series}
%%%%%%%%%%%%%%%%%%%%%%%%%%%%%%%%%%%%%%%%%%%%%%%%%%%%%%%%%%%%%%%%%%%%%%

Only people who are interested in the latest research, new features
that haven't been fully tested, etc, should download the development
version, denoted with an odd number in the second digit of the version
string.  
We will make a best effort to ensure that the development series will
work, but we make no guarantees.

On the development series, new minor version releases will probably
happen frequently.
People should not feel compelled to install new minor versions unless
they know they want features or bug fixes from the newer development
version.

\emph{Most sites will probably never want to install a development
version of Condor for any reason.}
Only if you know what you are doing (and like pain), or were
explicitly instructed to do so by someone on the Condor Team, should
you install a development version at your site.

After the feature set of the development series is satisfactory to the
Condor Team, we will put a code freeze in place, and from that point
forward, only bug fixes will be made to that development series.
When we have fully tested this version, we will release a new stable
series, resetting the minor version number, and start work on a new
development release from there.

%%%%%%%%%%%%%%%%%%%%%%%%%%%%%%%%%%%%%%%%%%%%%%%%%%%%%%%%%%%%%%%%%%%%%%
% The rest of this file just inputs other files which contain sections
% describing each release series in detail.
%%%%%%%%%%%%%%%%%%%%%%%%%%%%%%%%%%%%%%%%%%%%%%%%%%%%%%%%%%%%%%%%%%%%%%

% upgrade instructions are in the Pool Management section
%%%%%%%%%%%%%%%%%%%%%%%%%%%%%%%%%%%%%%%%%%%%%%%%%%%%%%%%%%%%%%%%%%%%%%
\section{\label{sec:gotchas}Upgrading from the 7.6 series to the 7.8 series of Condor}
%%%%%%%%%%%%%%%%%%%%%%%%%%%%%%%%%%%%%%%%%%%%%%%%%%%%%%%%%%%%%%%%%%%%%%

\index{upgrading!items to be aware of}
While upgrading from the 7.6 series of Condor to the 7.8 series will
bring many new features and improvements introduced in the 7.7 series of
Condor, it will also introduce changes that administrators of sites
running from an older Condor version should be aware of when
planning an upgrade.
Here is a list of items that administrators should be aware of.

\begin{itemize}

\item  Placeholder.  First item goes here.

\end{itemize}


%%%      PLEASE RUN A SPELL CHECKER BEFORE COMMITTING YOUR CHANGES!
%%%      PLEASE RUN A SPELL CHECKER BEFORE COMMITTING YOUR CHANGES!
%%%      PLEASE RUN A SPELL CHECKER BEFORE COMMITTING YOUR CHANGES!
%%%      PLEASE RUN A SPELL CHECKER BEFORE COMMITTING YOUR CHANGES!
%%%      PLEASE RUN A SPELL CHECKER BEFORE COMMITTING YOUR CHANGES!

%%%%%%%%%%%%%%%%%%%%%%%%%%%%%%%%%%%%%%%%%%%%%%%%%%%%%%%%%%%%%%%%%%%%%%
\section{\label{sec:History-7-9}Development Release Series 7.9}
%%%%%%%%%%%%%%%%%%%%%%%%%%%%%%%%%%%%%%%%%%%%%%%%%%%%%%%%%%%%%%%%%%%%%%

This is the development release series of Condor.
The details of each version are described below.

%%%%%%%%%%%%%%%%%%%%%%%%%%%%%%%%%%%%%%%%%%%%%%%%%%%%%%%%%%%%%%%%%%%%%%
\subsection*{\label{sec:New-7-9-0}Version 7.9.0}
%%%%%%%%%%%%%%%%%%%%%%%%%%%%%%%%%%%%%%%%%%%%%%%%%%%%%%%%%%%%%%%%%%%%%%

\noindent Release Notes:

\begin{itemize}

\item Condor version 7.9.0 not yet released.
%\item Condor version 7.9.0 released on Month Date, 2012.

\end{itemize}


\noindent New Features:

\begin{itemize}

\item Added -autoformat option to \Condor{status}, behavior is the same as -autoformat option
added to \Condor{q} in 7.7.6.
\Ticket{2941}

\end{itemize}

\noindent Configuration Variable and ClassAd Attribute Additions and Changes:

\begin{itemize}

\item New configuration variables were added to to the \Condor{schedd} to
define statistics that count subsets of jobs. These variables
have the form \Macro{SCHEDD\_COLLECT\_STATS\_BY\_XXXX} and should contain
a ClassAd expression that evaluates to a string.
XXXX_value_ will be prefixed to the names of attributes in the \Condor{schedd} ad
such as \Attr{XXXX_string_JobsStarted} which would be the count of jobs that have
started that evaluate to "value".  An optional variable of the form
\Macro{SCHEDD\_EXPIRE\_STATS\_BY\_XXXX} can be used to set an expiration time in seconds
for each set of statistics.
\Ticket{2862}

\end{itemize}

\noindent Bugs Fixed:

\begin{itemize}

\item Condor will avoid selecting down (disabled) network interfaces.  Previously Condor could select a down interface over an up (active) interface.
\Ticket{2893}

\end{itemize}

\noindent Known Bugs:

\begin{itemize}

\item None.

\end{itemize}

\noindent Additions and Changes to the Manual:

\begin{itemize}

\item None.

\end{itemize}



%%%      PLEASE RUN A SPELL CHECKER BEFORE COMMITTING YOUR CHANGES!
%%%      PLEASE RUN A SPELL CHECKER BEFORE COMMITTING YOUR CHANGES!
%%%      PLEASE RUN A SPELL CHECKER BEFORE COMMITTING YOUR CHANGES!
%%%      PLEASE RUN A SPELL CHECKER BEFORE COMMITTING YOUR CHANGES!
%%%      PLEASE RUN A SPELL CHECKER BEFORE COMMITTING YOUR CHANGES!

%%%%%%%%%%%%%%%%%%%%%%%%%%%%%%%%%%%%%%%%%%%%%%%%%%%%%%%%%%%%%%%%%%%%%%
\section{\label{sec:History-7-8}Stable Release Series 7.8}
%%%%%%%%%%%%%%%%%%%%%%%%%%%%%%%%%%%%%%%%%%%%%%%%%%%%%%%%%%%%%%%%%%%%%%

This is a stable release series of Condor.
As usual, only bug fixes (and potentially, ports to new platforms)
will be provided in future 7.8.x releases.
New features will be added in the 7.9.x development series.

The details of each version are described below.

%%%%%%%%%%%%%%%%%%%%%%%%%%%%%%%%%%%%%%%%%%%%%%%%%%%%%%%%%%%%%%%%%%%%%%
\subsection*{\label{sec:New-7-8-0}Version 7.8.0}
%%%%%%%%%%%%%%%%%%%%%%%%%%%%%%%%%%%%%%%%%%%%%%%%%%%%%%%%%%%%%%%%%%%%%%

\noindent Release Notes:

\begin{itemize}

\item Condor version 7.8.0 not yet released.
%\item Condor version 7.8.0 released on Month Date, 2012.

\end{itemize}


\noindent New Features:

\begin{itemize}

\item None.

\end{itemize}

\noindent Configuration Variable and ClassAd Attribute Additions and Changes:

\begin{itemize}

\item None.

\end{itemize}

\noindent Bugs Fixed:

\begin{itemize}

\item None.

\end{itemize}

\noindent Known Bugs:

\begin{itemize}

\item None.

\end{itemize}

\noindent Additions and Changes to the Manual:

\begin{itemize}

\item None.

\end{itemize}



%%%      PLEASE RUN A SPELL CHECKER BEFORE COMMITTING YOUR CHANGES!
%%%      PLEASE RUN A SPELL CHECKER BEFORE COMMITTING YOUR CHANGES!
%%%      PLEASE RUN A SPELL CHECKER BEFORE COMMITTING YOUR CHANGES!
%%%      PLEASE RUN A SPELL CHECKER BEFORE COMMITTING YOUR CHANGES!
%%%      PLEASE RUN A SPELL CHECKER BEFORE COMMITTING YOUR CHANGES!

%%%%%%%%%%%%%%%%%%%%%%%%%%%%%%%%%%%%%%%%%%%%%%%%%%%%%%%%%%%%%%%%%%%%%%
\section{\label{sec:History-7-7}Development Release Series 7.7}
%%%%%%%%%%%%%%%%%%%%%%%%%%%%%%%%%%%%%%%%%%%%%%%%%%%%%%%%%%%%%%%%%%%%%%

This is the development release series of Condor.
The details of each version are described below.

%%%%%%%%%%%%%%%%%%%%%%%%%%%%%%%%%%%%%%%%%%%%%%%%%%%%%%%%%%%%%%%%%%%%%%
\subsection*{\label{sec:New-7-7-6}Version 7.7.6}
%%%%%%%%%%%%%%%%%%%%%%%%%%%%%%%%%%%%%%%%%%%%%%%%%%%%%%%%%%%%%%%%%%%%%%

\noindent Release Notes:

\begin{itemize}

\item Condor version 7.7.6 not yet released.
%\item Condor version 7.7.6 released on Month Date, 2012.

\item In the Condor directory defined by \MacroUNI{SBIN},
\File{condor\_vm\_vmware.pl} was
renamed to \File{condor\_vm\_vmware} and \File{grid\_monitor.sh} was
renamed to \File{grid\_monitor}.
This makes Condor more compliant with Linux native packaging rules.
Symbolic links to the old locations are included to ease upgrading.
\Ticket{2940}

\end{itemize}


\noindent New Features:

\begin{itemize}

\item The values of \SubmitCmd{request\_memory}, \SubmitCmd{request\_disk} and
\SubmitCmd{request\_cpus} submit description file commands will now be
automatically included in the job \Attr{Requirements} expression by
\Condor{submit}.  This is part of several changes
in code and policy intended to make partitionable slots easier to deploy
and use.  The requested values for memory, disk and cpus, as well as the
amount of these resources that a job actually uses are now printed in the
user log when the job exits.
\Ticket{2843}

\item The new \SubmitCmd{keep\_claim\_idle} submit description
file command requests that the \Condor{schedd} keep a claim for a defined
number of seconds after the job exits.
The job ClassAd attribute \Attr{KeepClaimIdle} was introduced in
Condor version 7.7.1 to implement this functionality.
See the definition of this command at 
section ~\ref{condor-submit-keep-claim-idle}.
\Ticket{2094}

\item Changed the default for \Condor{history} to print out
items in reverse chronological order.  
The new \Opt{-forwards} option enables the previous behavior of 
printing historical jobs in chronological order.
\Ticket{2808}

\item Enhanced the \Condor{negotiator} to provide the name of 
concurrency limits that cause negotiation to fail, so that 
\Condor{q} -analyze can provide more informative failure information.
\Ticket{2878}

\item Concurrency limit defaults may now be declared for named groups
using \Macro{CONCURRENCY\_LIMIT\_DEFAULT\_<group>} so that any
concurrency limit with a name of the form <group>.<name> will get its
default limit from \Macro{CONCURRENCY\_LIMIT\_DEFAULT\_<group>}.
\Ticket{2863}

\item Condor binaries will now look for the Condor configuration file in
\File{\$(HOME)/.condor/condor\_config}, in addition to the locations where
they already look.
Within the ordered search,
\File{\$(HOME)/.condor/condor\_config} is checked immediately after the 
\Env{CONDOR\_CONFIG} environment variable.
\Ticket{2657}

\item The \Condor{hdfs} daemon is now available with the source code,
and is no longer distributed as part of the Condor binaries.
See documentation in section ~\ref{sec:Condor-HDFS}.
\Ticket{2797}

\item Several of the Condor programs used to be given by a single executable
hard linked to multiple file names. 
Now, symbolic links are used; this fixes problems with Debian installations.
\Ticket{2140}

\item New ClassAd functions \Procedure{pow}, \Procedure{quantize},
\Procedure{splitUserName}, and \Procedure{splitSlotName} are available.
See section ~\ref{sec:classadFunctions} for definitions of these functions.
\Ticket{2856}
\Ticket{2891}

\item New format tags \%v and \%V have been added for use by the
\Condor{status} \Opt{-format} option.
These tags request that the value of the expression or attribute be printed 
using a format appropriate to its type.
When using the \%V format tag, string values appear as they would in
the output of \condor{q -long} or \condor{submit -long}.
\Ticket{2857}

\item \Condor{ssh\_to\_job} now provides support for X11 forwarding
via the new \Opt{-X} option.

\end{itemize}

\noindent Configuration Variable and ClassAd Attribute Additions and Changes:

\begin{itemize}

\item The new machine ClassAd attributes \Attr{RemoteGroup}, 
\Attr{RemoteNegotiatingGroup}, and \Attr{RemoteAutoregroup},
and the new job ClassAd attributes \Attr{SubmitterGroup}, 
\Attr{SubmitterNegotiatingGroup}, and \Attr{SubmitterAutoregroup}
enhance support for preemption policies with accounting group awareness.
\Ticket{2885}

\item The new configuration variable
\Macro{NEGOTIATOR\_READ\_CONFIG\_BEFORE\_CYCLE} is a boolean which causes the
\Condor{negotiator} to re-read the configuration prior to each
negotiation cycle when set to \Expr{True}.
\Ticket{2851}

\item The new configuration variable \Macro{MASTER\_NEW\_BINARY\_RESTART} 
specifies how the \Condor{master} will restart,
when it notices that the \Condor{master} binary has changed. 
Valid values are \Expr{GRACEFUL}, \Expr{PEACEFUL} and \Expr{NEVER}. 
The default value is \Expr{GRACEFUL}.
\Ticket{2779}

\item The configuration variable \MacroNI{WANT\_HOLD} now takes effect
whether or not \MacroNI{WANT\_VACATE} is \Expr{True}.  Previously,
it only took effect if \MacroNI{WANT\_VACATE} was \Expr{True}.
\Ticket{2855}

\item The new configuration variables \Macro{MEMORY\_USAGE\_METRIC} and
\Macro{MEMORY\_USAGE\_METRIC\_VM} specify the value that the 
\Condor{starter} will
set into the \Attr{MemoryUsage} attribute for a job.  It is expected that
this will be a ClassAd expression that defines the job memory usage in terms
of other job attributes.
\Ticket{2843}

\item The configuration variable \Macro{DAGMAN\_SUBMIT\_DELAY} can now be any
non negative integer.  It was formerly limited to values between 0 and 60,
inclusive.
\Ticket{2864}

\item New configuration variables have been added, 
such that the \Condor{schedd} may
define statistics that count subsets of jobs. 
These variables
have the form \Macro{SCHEDD\_COLLECT\_STATS\_FOR\_<name>} and 
are defined by a boolean ClassAd expression.
<name> will be prefixed to the names of attributes in the \Condor{schedd} 
ClassAd, such as \Attr{physicsJobsStarted}
where \MacroNI{SCHEDD\_COLLECT\_STATS\_FOR\_physics} evaluates to \Expr{True},
and this attribute would be the count of jobs that have started.
\Ticket{2862}

\item Several OpSys related attributes were added or updated to assist with selection of execute resources.
\begin{description}
\item [\AdAttr{OpSysAndVer}:] A string containing the value of the \Attr{OpSysName} attribute with the \Attr{OpSysMajorVersion} attribute appended.
\item [\AdAttr{OpSysLegacy}:] A string that holds the long-standing values for the \Attr{OpSys} attribute.
\item [\AdAttr{OpSysLongName}:] A string containing a full description of the operating system.
\item [\AdAttr{OpSysMajorVersion}:] An integer value representing the major version of the operating system.
\item [\AdAttr{OpSysName}:] A string containing a terse description of the operating system.
\item [\AdAttr{OpSysShortName}:] A string containing a short description of the operating system.
\item [\AdAttr{OpSysVer}:] An integer value representing the operating system version number.
\end{description}
\Ticket{2366}

\end{itemize}

\noindent Bugs Fixed:

\begin{itemize}

\item Fixed a bug in \Condor{vm-gahp} that caused 64-bit guest OSes that
need network access to fail on start-up when run under VMware.
\Ticket{2922}

\item Submit command \Macro{remote\_initialdir} now works for pbs and lsf
grid universe jobs.
\Ticket{2913}

\item Fixed the path to \Prog{sftp\_server} on Mac OS X and Debian
platforms.
\Ticket{2789}

\item Fixed a rare problem that caused a 20 second timeout to occur in
the \Condor{collector} when authenticating.
\Ticket{2817}

\item Fixed a rare bug in which the \Condor{schedd} would sometimes not reuse
an existing claim to run a new job when an existing job exited.  
This would result in the \Condor{schedd} daemon
waiting for a new negotiation cycle to make a new match,
and thus producing a small performance penalty due to the
wasted time during the interval between negotiation cycles.  
This bug was actually fixed in Condor version 7.7.5.
\Ticket{2802}

\item Fixed a bug in \Condor{q}, such that it no longer emits a parse
error when it times out attempting to talk to the \Condor{schedd} daemon.
\Ticket{2854}

\item The shared library \File{libcondor\_utils} now includes the Condor
version in its name. This will reduce the chance of a Condor binary
using the wrong version of the library, which can result in a crash or
other bad behavior.
\Ticket{2613}

\item There was a bug on GRACEFUL and PEACEFUL shutdown, 
as the daemons were stopped in a random order. 
This resulted in the checkpoint server 
sometimes being shut down before the \Condor{startd}.  
The \Condor{startd} is now always shut down first on GRACEFUL or PEACEFUL 
shutdown, 
with other daemons being shut down only after the \Condor{startd} has exited.
\Ticket{2779}

\item Under some circumstances, 
a job in the removed ("X") state may have ignored the \Opt{-forcex} option 
to \Condor{rm}.
The \Condor{schedd} is now more aggressive about removing such jobs 
from the queue.
\Ticket{2809}

\item Fixed the copying of scaling factors on ClassAd literal values.
\Ticket{2839}

\item When a job is killed and put on hold because of
  \MacroNI{WANT\_HOLD}, the maximum vacate time is now enforced.  If
  it takes longer than the maximum vacate time for the job to be
  gracefully killed, the job is hard-killed.  Previously, no upper
  limit was enforced.
\Ticket{2855}

\item When selecting an IPv4 network interface to use Condor would erroneously prefer private networks over public networks in some cases.  This has been fixed, Condor again prefers public networks over private networks.
\Ticket{2853}

\item The \condor{gridmanager} is much better at sending commit signals
to the GRAM job-manager in a timely manner. As a result, the occurrence of
GRAM errors 111 and 130 should be greatly reduced.
\Ticket{2859}

\item Fixed a bug that caused \condor{submit} to warn about
\MacroNI{dag\_status} and \MacroNI{failed\_count} not being used in the
submit files of most DAG node jobs (DAGMan now automatically defines
these macros for all node jobs).  This bug was introduced in 7.7.5.
\Ticket{2814}

\end{itemize}

\noindent Known Bugs:

\begin{itemize}

\item None.

\end{itemize}

\noindent Additions and Changes to the Manual:

\begin{itemize}

\item None.

\end{itemize}

%%%%%%%%%%%%%%%%%%%%%%%%%%%%%%%%%%%%%%%%%%%%%%%%%%%%%%%%%%%%%%%%%%%%%%
\subsection*{\label{sec:New-7-7-5}Version 7.7.5}
%%%%%%%%%%%%%%%%%%%%%%%%%%%%%%%%%%%%%%%%%%%%%%%%%%%%%%%%%%%%%%%%%%%%%%

\noindent Release Notes:

\begin{itemize}

\item Condor version 7.7.5 released on February 28, 2012.
This release contains all features and bug fixes from Condor version 7.6.6. 

\item Support for the gt4 grid type (that is, Web Services GRAM) in the grid
universe has been removed.
\Ticket{2782}

\end{itemize}


\noindent New Features:

\begin{itemize}

\item Condor now has experimental support for IPv6.  
This functionality is disabled by default.  
This support has a variety of limitations, 
including a lack of support for security, DNS, and mixed IPv4/IPv6 networks.  
For information on enabling IPv6 support in the 7.7 series of Condor, 
see \URL{https://condor-wiki.cs.wisc.edu/index.cgi/wiki?p=HowToEnableIpvSix}.
\Ticket{9}

\item Default values for the submit commands
\SubmitCmd{should\_transfer\_files} and \SubmitCmd{when\_to\_transfer\_output}
were introduced in Condor version 7.7.3,
but the manual did not reflect this change.
Across platforms, default values are now
\begin{verbatim}
  should_transfer_files = IF_NEEDED
  when_to_transfer_output = ON_EXIT
\end{verbatim}
See section ~\ref{sec:file-transfer-if-when} for details.
\Ticket{2281}
\Ticket{2273}

\item The performance for claiming a partitionable slot in a 
\Condor{startd} is greatly improved.  
This feature is implemented in both the \Condor{schedd} and \Condor{startd},
so both sides must be updated to at least Condor version 7.7.5 
to see the benefit.  
To disable this feature, set configuration variable
\Macro{CLAIM\_PARTITIONABLE\_LEFTOVERS} to \Expr{False} 
on either the submit or execute machines.  
The default value for this variable is \Expr{True}.
\Ticket{2790}

\item On Linux platforms, the \Condor{starter} can now optionally measure the
PSS (Proportional Set Size) of each Condor job,
if the configuration variable \Macro{USE\_PSS} is \Expr{True}.
Previously, this measurement was unconditionally on,
which can cause performance problems in the \Condor{procd} when running
many short lived jobs.
\Ticket{2710}

\item On Linux systems, the \Condor{starter} now has an ability to run a 
job under a chroot directory.  
If the configuration variable \Macro{NAMED\_CHROOT} is set to a list
of directories on an execute machine, 
the job has attribute \Attr{RequestedChroot} defined,
and the value of \Attr{RequestedChroot} matches an entry 
in the list defined by \MacroNI{NAMED\_CHROOT},
then the \Condor{starter} calls \Procedure{chroot} with that directory 
as an argument.  
Note that it is up to the administrator to provide a full environment 
for the job to run in.
\Ticket{2698}

\item On Linux platforms which support a bind type of file system mount 
(which are generally RHEL 5 systems and more recent platforms), 
the administrator can configure the \Condor{startd} 
to provide per-job file system mounts.  
One use might be to provide each job its own view of \File{/tmp} 
and \File{/var/tmp}, which are private to that Condor job,
and cleared when the job exits.  
This is implemented with the new \Macro{MOUNT\_UNDER\_SCRATCH} 
configuration variable, which describes which directories to bind mount.
\Ticket{2015}

\item Added the new \Opt{-expand} option to \Condor{config\_val}.
If both \Opt{-dump} and \Opt{-expand} options are specified,
all configuration variables are expanded before they are printed out.
\Ticket{2687}

\item The \Opt{-sort} option for \Condor{status} has been generalized to
accept expressions instead of just simple named attributes.
\Ticket{2661}

\item A new command \Condor{drain} may be used to control the draining
of an execute machine.  While a machine is draining, no new jobs may
start.  Once draining is complete, it enters the Drained/Idle state.
For more details, see page~\pageref{man-condor-drain}.
\Ticket{2330}

\item A new daemon \Condor{defrag} has been added to automate a simple
policy for draining machines.  For more details, see
page~\pageref{sec:Config-defrag}.
\Ticket{2330}

\item  \Condor{q} \Opt{-run} now displays the value of the job ClassAd
attribute \Attr{EC2RemoteVirtualMachineName} instead of
\Expr{[????????????????]},
under the HOST(S) column for grid type ec2 jobs.
\Ticket{2599}

\item Condor can now submit jobs to Grid Engine via the new sge grid type.
See section ~\ref{sec:SGE} for details.
\Ticket{1984}

\item Improved logging in more cases when Condor daemons run out of memory.
\Ticket{2559}

\item Improved verbose logging when \Dflag{MACHINE} is enabled in
\MacroNI{NEGOTIATOR\_DEBUG}.  Previously, it logged whether each
candidate machine matched or did not match with each job.  Now, it
additionally logs whether the match was subsequently rejected for
other reasons, such as insufficient priority, rank, or fair share
allocation.

\item Condor will now send email,
if the submit command \Expr{notification = Error} is set and
the job is placed on hold because of a failure, and not by user request.
Previously, email would be sent only if the job was terminated via signal.
\Ticket{1976}

\item A new feature in DAGMan implements a second way to suspend a running DAG.
See section ~\ref{sec:DagSuspend} for details.
\Ticket{2213}

\item The default settings for \Condor{dagman} have changed.
Now, if a node has children, then \Condor{dagman} uses the
\Attr{KeepClaimIdle} attribute, 
introduced in Condor version 7.7.1, to hold onto a claim.
This is a slight optimization, 
as it avoids waiting for a negotiation cycle.
The amount of time is controlled by the
\Macro{DAGMAN\_HOLD\_CLAIM\_TIME} configuration variable.
\Ticket{2673}

\item Improved the output of \Condor{q} \Opt{-dag},
to show the DAG structure as a tree,
with children indented below their parents.
\Ticket{1281}

\item The new FINAL node feature in DAGMan allows the specification
of a special DAG node, 
which is always run at the end of the workflow,
whether the DAG ended successfully or not.
See section~\ref{sec:DAGFinalNode} for details.
\Ticket{1482}

\item Improved the output of \Condor{userprio} to better support hierarchical
groups. 
The first column of the output no longer truncates long user or group names.
User names are shown indented under group names,
when hierarchical groups are in use.
New columns were added to show group quota information.
A new \Opt{-most} option was added to show 
the most useful fields,
since \Opt{-all} now produces a very wide display.
\Ticket{2680}

\end{itemize}

\noindent Configuration Variable and ClassAd Attribute Additions and Changes:

\begin{itemize}

\item The new configuration variable \Macro{JOB\_QUEUE\_LOG} 
specifies an alternative path and file name for the \File{job\_queue.log} file.
The default value is \File{\MacroUNI{SPOOL}/job\_queue.log}.
This alternative location can be
useful if there is a solid state drive which is big enough to hold the
frequently written to \File{job\_queue.log},
but not big enough to hold the whole contents of the spool directory.
\Ticket{2598}

\item The new configuration variable \Macro{DAGMAN\_HOLD\_CLAIM\_TIME}
specifies the amount of time in seconds that the \Condor{schedd} 
will hold a claim idle for a DAGMan job, 
using the \AdAttr{KeepClaimIdle} attribute in the job ClassAd.
\Ticket{2673}

\item The job ClassAd attributes
\AdAttr{ResidentSetSize} and \AdAttr{ProportionalSetSizeKb} now
report the maximum observed memory usage.  
Previously, they reported the most recently observed memory usage.  
This change makes these attributes similar to \AdAttr{ImageSize}, 
which also reports the maximum observed value.  
Previously, \AdAttr{ResidentSetSize} was
usually reported as 0 in the job history for completed jobs, because
when the job was finished, the final observation of memory usage
was 0.
\Ticket{2725}

\item The job ClassAd attribute \Attr{ResidentSetSize} is now rounded 
by default,
using the new default configuration setting
\Expr{SCHEDD\_ROUND\_ATTR\_ResidentSetSize = 25\%}.
\Ticket{2729}

\item The configuration variable \Macro{PROCD\_LOG} now defaults to
\File{\$(LOG)/ProcLog}.  Previously, there was no default value,
so the \Condor{procd} did not log by default.
\Ticket{2775}

\item The meaning of the \Attr{VirtualMemory} attribute of the \Condor{startd} 
has been changed for Linux platforms.
Previously, it was the amount of paging space configured for the system.
So, if a machine with a lot of memory had no paging space, 
the \Attr{VirtualMemory} attribute would report zero.
Now, the \Attr{VirtualMemory} attribute on Linux platforms 
is the sum of paging space and physical memory, 
which more accurately represents the virtual memory size of the machine.
\Ticket{2763}

\item The submit command \SubmitCmd{globus\_xml} is no longer
recognized. Therefore, the following configuration variables are no longer
recognized:
\begin{itemize}
  \item \Expr{GRIDFTP\_SERVER}
\index{GRIDFTP\_SERVER configuration variable no longer exists as of Condor version 7.7.5@\texttt{GRIDFTP\_SERVER} configuration variable no longer exists as of Condor version 7.7.5}
  \item \Expr{GRIDFTP\_SERVER\_WRAPPER}
\index{GRIDFTP\_SERVER\_WRAPPER configuration variable no longer exists as of Condor version 7.7.5@\texttt{GRIDFTP\_SERVER\_WRAPPER} configuration variable no longer exists as of Condor version 7.7.5}
  \item \Expr{GRIDFTP\_URL\_BASE}
\index{GRIDFTP\_URL\_BASE configuration variable no longer exists as of Condor version 7.7.5@\texttt{GRIDFTP\_URL\_BASE} configuration variable no longer exists as of Condor version 7.7.5}
  \item \Expr{GT4\_GAHP}
\index{GT4\_GAHP configuration variable no longer exists as of Condor version 7.7.5@\texttt{GT4\_GAHP} configuration variable no longer exists as of Condor version 7.7.5}
  \item \Expr{GT4\_LOCATION}
\index{GT4\_LOCATION configuration variable no longer exists as of Condor version 7.7.5@\texttt{GT4\_LOCATION} configuration variable no longer exists as of Condor version 7.7.5}
  \item \Expr{GT42\_GAHP}
\index{GT42\_GAHP configuration variable no longer exists as of Condor version 7.7.5@\texttt{GT42\_GAHP} configuration variable no longer exists as of Condor version 7.7.5}
  \item \Expr{GT42\_LOCATION}
\index{GT42\_LOCATION configuration variable no longer exists as of Condor version 7.7.5@\texttt{GT42\_LOCATION} configuration variable no longer exists as of Condor version 7.7.5}
  \item \Expr{GRIDMANAGER\_MAX\_WS\_DESTROYS\_PER\_RESOURCE}
\index{GRIDMANAGER\_MAX\_WS\_DESTROYS\_PER\_RESOURCE configuration variable no longer exists as of Condor version 7.7.5@\texttt{GRIDMANAGER\_MAX\_WS\_DESTROYS\_PER\_RESOURCE} configuration variable no longer exists as of Condor version 7.7.5}
\end{itemize}
\Ticket{2782}

\item The new configuration variable
\Macro{GRIDMANAGER\_PROXY\_REFRESH\_TIME} controls when the
\Condor{gridmanager} forwards a refreshed proxy to the remote GRAM server.
The lifetime remaining on the proxy on the remote server (in seconds) must
fall below this value before the \Condor{gridmanager} will forward a
refreshed proxy. 
The default value is 21600 seconds (6 hours).
Previously, this value was not configurable.
\Ticket{2792}

\end{itemize}

\noindent Bugs Fixed:

\begin{itemize}

\item Fixed a bug in which \Condor{submit} allowed the specification of
\SubmitCmd{ec2\_secret\_access\_key} and \SubmitCmd{ec2\_access\_key\_id}
to be directories instead of files.
\Condor{submit} now generates an error in these cases.
\Ticket{2619}

\item Communication errors were not always correctly handled when
fetching results of a query when using the \Opt{-stream} option to
\Condor{q}.  This problem was introduced in Condor version 7.7.0.
\Ticket{2601}

\item Fixed Condor's CronTab (Crondor, section~\ref{sec:CronTab})
scheduling of jobs,
as they did not correctly take into account
shifts in time caused by daylight savings time transitions.
\Ticket{2620}

\item Previously, \Condor{ssh\_to\_job} sessions inherited the \Condor{starter}
environment.  Now, this only happens when
\MacroNI{JOB\_INHERITS\_STARTER\_ENVIRONMENT} is \Expr{True}.
\Ticket{2621}

\item On Linux platforms, the memory usage was ignored for job sub-processes
that were created via \Procedure{fork} without calling \Procedure{exec}.
This problem affected \Attr{ImageSize} and \Attr{ResidentSetSize},
but not \Attr{ProportionalSetSize}.

\item Fixed a rare condition that could cause a job to remain in the
running state indefinitely when the job was removed or put on hold
and there was a communication failure between the \Condor{shadow}
and the \Condor{starter}.  
This problem was introduced in Condor version 7.7.2.
\Ticket{2591}

\item Fixed a bug in the \Condor{gridmanager} that could cause crashes
and prevent the attribute \Attr{x509UserProxyEmail} from being set properly for
jobs forwarded via Condor-C.
\Ticket{2655}

\item Fixed the output of \Condor{q} \Opt{-dag},
such that children of a non-existent DAG node would not be mistakenly 
shown as belonging to another instance of \Condor{dagman}.
This can happen, for example, when a \Condor{dagman} process dies while
its children are still running.
\Ticket{2463}

\item Fixed a bug in \Condor{dagman} that caused a DAG to fail if node
job user log files were actually symbolic links.  
This problem was introduced in the Condor 7.7 development series.
\Ticket{2704}

\item Fixed a bug in the collection of Statistics attributes,
introduced in Condor version 7.7.2.
Condor did not count completed scheduler universe jobs in reported statistics.
\Ticket{2731}

\item Fixed a rare bug in which the \Condor{c-gahp} process could get
into an infinite loop on start up,
if more than one \Condor{c-gahp} was running under different users,
and the names of the users only differed in their last character.
\Ticket{2749}

\end{itemize}

\noindent Known Bugs:

\begin{itemize}

\item None.

\end{itemize}

\noindent Additions and Changes to the Manual:

\begin{itemize}

\item Condor's ability to use cgroup-based process tracking,
available since Condor version 7.7.0,
has now been documented in section~\ref{sec:CGroupTracking}.
\Ticket{1831}
\Ticket{2120}

\item Submitter ClassAd attributes are now documented in the unnumbered
appendix on page~\pageref{sec:Submitter-ClassAd-Attributes}.

\end{itemize}


%%%%%%%%%%%%%%%%%%%%%%%%%%%%%%%%%%%%%%%%%%%%%%%%%%%%%%%%%%%%%%%%%%%%%%
\subsection*{\label{sec:New-7-7-4}Version 7.7.4}
%%%%%%%%%%%%%%%%%%%%%%%%%%%%%%%%%%%%%%%%%%%%%%%%%%%%%%%%%%%%%%%%%%%%%%

\noindent Release Notes:

\begin{itemize}

\item Condor version 7.7.4 released on December 21, 2011.
This release contains all features and bug fixes from Condor version 7.6.5 
as are currently documented (section~\ref{sec:New-7-6-5}) in this manual. 

\end{itemize}


\noindent New Features:

\begin{itemize}

\item Condor version 7.7.4 has all of the features and fixes of 7.7.3, it 
includes work toward running on a pure IPv6 network.  This is disabled by
default.  There is an severe bug where enabling IPv6 in a multi-computer pool
may cause the \Condor{starter} to crash.  For 
more information on enabling IPv6 support in the 7.7 series of Condor, see \URL{https://condor-wiki.cs.wisc.edu/index.cgi/wiki?p=HowToEnableIpvSix}.
\Ticket{9}

\end{itemize}

\noindent Configuration Variable and ClassAd Attribute Additions and Changes:

\begin{itemize}

\item None.

\end{itemize}

\noindent Bugs Fixed:

\begin{itemize}

\item None.

\end{itemize}

\noindent Known Bugs:

\begin{itemize}

\item When IPv6 is enabled and you have multiple computers in your pool, the \Condor{starter} may crash.

\end{itemize}

\noindent Additions and Changes to the Manual:

\begin{itemize}

\item None.

\end{itemize}


%%%%%%%%%%%%%%%%%%%%%%%%%%%%%%%%%%%%%%%%%%%%%%%%%%%%%%%%%%%%%%%%%%%%%%
\subsection*{\label{sec:New-7-7-3}Version 7.7.3}
%%%%%%%%%%%%%%%%%%%%%%%%%%%%%%%%%%%%%%%%%%%%%%%%%%%%%%%%%%%%%%%%%%%%%%

\noindent Release Notes:

\begin{itemize}

\item Condor version 7.7.3 not yet released.
%\item Condor version 7.7.3 released on Month Date, 2011.

\item On Linux and Mac OS X, the Condor binaries now dynamically link with
\File{libcondor\_utils}, 
a shared library that contains all Condor code that is
used by multiple binaries. 
This library is not meant to be linked with user applications.
\Ticket{2132}

\item \emph{Condor now dynamically links with the ClassAds, Globus and VOMS
libraries on Mac OS X.}
A copy of these libraries is included with Condor.
\Ticket{2482}

\end{itemize}


\noindent New Features:

\begin{itemize}

\item In Condor version 7.7.2, multiple Condor installations led to the
possibility for some installations to use the wrong version of the ClassAds 
library.
This should no longer be an issue, 
as the binaries now use \Env{RUNPATH} instead of \Env{RPATH}, 
allowing use of the \Env{LD\_LIBRARY\_PATH} environment variable 
to set where to look for the shared libraries.
\Ticket{2539}

\item The Amazon SOAP interface is no longer present or supported in Condor.
The EC2 REST interface is favored and supported in Condor
using a \SubmitCmd{grid\_resource} of \SubmitCmd{ec2}.
\Ticket{2523}

\item The new \Condor{gather\_info} tool introduced in 
Condor version 7.5.6 has been updated and enhanced.
It collects data about a Condor installation, and, if desired, 
about a specific job. 
This information is useful to Condor developers to help 
debug problems in a pool or with a job.
\Ticket{1664}
\Ticket{2372}

\item The \Condor{userprio} tool supports two new command line options.
The \Opt{-grouporder} flag displays submitter entries 
for accounting groups at top of the list,
 in breadth-first order by group hierarchy.
The \Opt{-grouprollup} flag reports accounting statistics for groups 
as summed at a level within the group hierarchy.
\Ticket{1926}

\item The \Condor{collector} now avoids the performance problems caused
previously when clients initiated communication with the \Condor{collector},
but then delayed sending input.
\Ticket{2506}

\item When using versions of \Prog{glexec} that create a copy of the proxy 
for use by the job, 
Condor now ensures that this copy of the proxy is cleaned up
when the job is done.
\Ticket{2501}

\item The \Condor{startd} now logs a clear message, if it rejects a job
because no valid \Condor{starter} daemons were detected.
\Ticket{2470}

\item The new submit command \SubmitCmd{want\_graceful\_removal}
may be used to specify that a job being removed or put on hold should
be shut down gracefully, rather than being immediately hard-killed.
This allows the job to perform some final actions such as cleaning
up or saving state.  The usual policies governing the Preempting/Vacating
state apply in this case.  

This new submit command replaces a different mechanism that was added 
in Condor version 7.5.2 to achieve some of the same effects.  
The version 7.5.2 mechanism applied to vanilla jobs under Linux;
if the job set \SubmitCmd{remove\_kill\_sig} or \SubmitCmd{kill\_sig},
the hard-kill signal that Condor would normally send to end the job was
replaced with the signal specified by the user.  

With the new submit command, the version 7.5.2 mechanism is no longer used.
The soft-kill signal may still be customized using
\SubmitCmd{kill\_sig}, so a similar effect can be achieved by setting
\Expr{want\_graceful\_removal=True} and setting \SubmitCmd{kill\_sig}
to an alternative signal, if desired.  The new mechanism works on all
platforms and works for all universes in which the job is managed by
the \Condor{startd}; as such the new mechanism is not supported
in the grid, local, or scheduler universes.

In addition, the new submit command \SubmitCmd{job\_max\_vacate\_time}
replaces the \SubmitCmd{kill\_sig\_timeout} command.
\SubmitCmd{job\_max\_vacate\_time}
adjusts the time given to an evicted job for gracefully shutting down.
\Ticket{2536}

\item The \Condor{master} now logs a more informative error message
when it fails to start a daemon.
\Ticket{2580}

\item The \Condor{schedd} daemon now logs a more informative error message
when it rejects job ClassAd updates from the \Condor{shadow} due to
authorization problems.
\Ticket{2581}

\end{itemize}

\noindent Configuration Variable and ClassAd Attribute Additions and Changes:

\begin{itemize}

\item The new configuration variable \Macro{MachineMaxVacateTime} is
now used to express the maximum time in seconds that the machine is
willing to wait for a job to gracefully shut down.  
The default is 600 seconds (10 minutes).  
The boolean \MacroNI{KILL} expression was
previously used to terminate the graceful shutdown of jobs.  
It should normally be set to \Expr{False} now.  If desired, it may be
used to abort the graceful shutdown of the job earlier than
\MacroNI{MachineMaxVacateTime}.
\Ticket{2536}

\item The new configuration variable \Macro{NEGOTIATOR\_SLOT\_CONSTRAINT} 
defines an expression which constrains which ClassAds are fetched
by the \Condor{negotiator} from the \Condor{collector}
for the negotiation cycle. 
\Ticket{2277}

\item The new configuration variable 
\Macro{NEGOTIATOR\_SLOT\_POOLSIZE\_CONSTRAINT} 
replaces \Macro{GROUP\_DYNAMIC\_MACH\_CONSTRAINT}.
\MacroNI{GROUP\_DYNAMIC\_MACH\_CONSTRAINT} may still be used,
but a warning is written to the log,
identifying that the configuration needs to be updated to use the new name.
The pool size resulting from applying this constraint is used
to determine quotas for both dynamic quotas in hierarchical groups,
and when there are no groups.
\Ticket{2277}

\item The configuration variable \Macro{NEGOTIATOR\_STARTD\_CONSTRAINT\_REMOVE} 
was introduced in Condor version 7.7.1.
It has now been removed, as its functionality 
was made obsolete by \MacroNI{NEGOTIATOR\_SLOT\_CONSTRAINT}.
\Ticket{2277}

\item The configuration variables \Macro{IGNORE\_NFS\_LOCK\_ERRORS}
and \Macro{BIND\_ALL\_INTERFACES} no longer support the undocumented use of
'Y' or 'y' to mean \Expr{True}.

\end{itemize}

\noindent Bugs Fixed:

\begin{itemize}

\item Fixed a bug from Condor version 7.7.1
that caused submit description file commands using a substitution macro,
\$\$(),
to not work correctly when a \Condor{shadow} daemon is recycled,
as it is when the configuration variable \Macro{SHADOW\_WORKLIFE}
is set to a non-zero value.
\Ticket{2552}

\item When the \Condor{procd}'s named command pipe is removed, 
or when the inode of the pipe has been changed while the daemon is running, 
the \Condor{procd} will now exit.
Its previous behavior had the \Condor{procd} continue to execute 
in a useless mode of operation, unable to receive any communication.
\Ticket{2500}

\item For Mac OS X platforms, 
improper detection of a non existent process led to lines such as
\begin{verbatim}
ProcAPI sanity failure on pid 1317, age = -1901476270
\end{verbatim}
appearing in the \Condor{master} daemon log.
This should no longer be the case.
\Ticket{2594}

\item Fixed a bug introduced with hierarchical group quotas that
failed to correctly initialize table entries.
The fix adds logic to the accounting mechanism in the
\Condor{negotiator} daemon,
such that initialization occurs correctly 
when starting up and upon reconfiguration.
\Ticket{2509}

\item When \Condor{advertise} is used with the \Opt{-tcp} option, this
used to cause the following log message to appear in the \Condor{collector}
log:
\begin{verbatim}
DaemonCore: Can't receive command request from IP (perhaps a timeout?)
\end{verbatim}
\Ticket{2483}

\item Fixed a bug introduced in Condor version 7.7.0,
in which the setting of \MacroNI{NETWORK\_INTERFACE} did not have any effect.
\Ticket{2513}

\item \Prog{glexec} now also works when Condor is running as root.
\Ticket{2503}

\item The \Condor{master} daemon now successfully advertises itself in 
a Personal Condor installation,
when the \Condor{collector} is configured to use port 0
and to operate through a shared port.
\Ticket{2555}

\item Since Condor version 7.7.1, 
the configuration variable \Macro{WANT\_HOLD} did not work,
unless \Macro{WANT\_HOLD\_SUBCODE} was set to a non-zero value.
\Ticket{2565}

\item Since Condor version 7.7.2, there was a rare condition that could cause
a job to be removed from the queue,
if the job was put on hold while it was running.
In such cases, there was also a spurious
unsuspend event logged in the job's user log.
\Ticket{2577}

\item Fixed a bug introduced in Condor version 7.7.2 by the change 
of \Attr{OpSys} to \AdStr{WINDOWS}.
Submit description files that used old syntax for the 
\SubmitCmd{environment} command
were using Unix syntax rather than Windows syntax.
\Ticket{2607}

\item Fixed the linking of Kerberos libraries on RHEL 3. 
The bug could cause
the Condor binaries to fail on some systems with the error:
\begin{verbatim}
relocation error: /usr/kerberos/lib/libgssapi_krb5.so.2: 
undefined symbol: krb5int_enc_arcfour
\end{verbatim}
\Ticket{2627}

\end{itemize}

\noindent Known Bugs:

\begin{itemize}

\item None.

\end{itemize}

\noindent Additions and Changes to the Manual:

\begin{itemize}

\item None.

\end{itemize}


%%%%%%%%%%%%%%%%%%%%%%%%%%%%%%%%%%%%%%%%%%%%%%%%%%%%%%%%%%%%%%%%%%%%%%
\subsection*{\label{sec:New-7-7-2}Version 7.7.2}
%%%%%%%%%%%%%%%%%%%%%%%%%%%%%%%%%%%%%%%%%%%%%%%%%%%%%%%%%%%%%%%%%%%%%%

\noindent Release Notes:

\begin{itemize}

\item Condor version 7.7.2 released on October 11, 2011.
This release contains all features and bug fixes from Condor version 7.6.4
as are currently documented (section~\ref{sec:New-7-6-4}) in this manual. 

\item
\emph{Condor now dynamically links with the ClassAds, Globus and VOMS libraries on
linux.}
A copy of these libraries is included with Condor, under
\File{lib/condor/} in the tarball releases and under
\File{/usr/lib/condor/} or \File{/usr/lib64/condor/} in the native package
releases.
\Ticket{2389}
\Ticket{2390}

\end{itemize}


\noindent New Features:

\begin{itemize}

\item Condor's standard universe now supports reading from and writing to
files that are larger than 2 GBytes,
when the standard universe application and
the \Condor{shadow} daemon are both 64-bit executables.
\Ticket{2337}

\item There is command line support to both suspend and continue jobs. 
The new tools \Condor{suspend} and \Condor{continue} will 
suspend and continue running jobs.
\Ticket{2368}

\item The EC2 GAHP now supports X.509 for connecting to and authenticating
with EC2 services.  See section~\ref{sec:Amazon-submit} for details
on using the X.509 protocol.
\Ticket{2084}

\item Previously, the dedicated scheduler attempted to change the
\Attr{Scheduler} attribute on all parallel job processes in a durable fashion,
resulting in an \Procedure{fsync} for each process.
This has been changed to be not durable, 
thereby improving the scalability by reducing the 
number of \Procedure{fsync} calls without impacting correctness. 
\Ticket{2367}

\item In PrivSep mode, when an error is encountered when trying to
switch to the user account chosen for running the job, 
the error message has been improved to make debugging easier.  
Now, the error message distinguishes between safety check failures 
for the UID, tracking group ID, primary group ID, and supplementary group IDs.
\Ticket{2364}

\item The name of the user used to execute the job is now logged in
the \Condor{starter} log, except when using \Prog{glexec}.
\Ticket{2268}

\item \Condor{dagman} now defaults to writing a partial DAG file
for a Rescue DAG,
as opposed to a full DAG file.
The Rescue DAG file is parsed in combination with the original DAG file, 
meaning that any
changes to the original DAG input file take effect when running a Rescue DAG.
\Ticket{2165}

\item The behavior of DAGMan is changed, such that, by default, 
POST scripts will be run regardless of the return value from 
the PRE script of the same node as described in section~\ref{dagman:SCRIPT}.  
The previous behavior of not running the POST script can be restored by
either adding the \Opt{-AlwaysRunPost} option to the \Condor{submit\_dag}
command line, 
or by setting the new configuration variable
\Macro{DAGMAN\_ALWAYS\_RUN\_POST} to \Expr{False}, 
as defined at~\ref{param:DAGmanAlwaysRunPost}.
\Ticket{2057}

\item DAGMan will now copy PRIORITY values from the DAG input file to 
the \Attr{JobPrio} attribute in the job ClassAd.  
Furthermore, the PRIORITY values are propagated to child nodes and SUBDAGs, 
so that child nodes always have priority at least that
of the maximum of the priorities of its parents.  
This has been a cause of confusion for DAGMan users.
\Ticket{2167}

% moved to 7.7.2 
% gittrac #659 
%\item Filip Krikava supplied a patch that limits the number of 
%file descriptors that DAGMan has open at a time.
%The reason for creating this capability is that
%DAGMan tends to fail on wide DAGs, where many jobs are independent,
%rather than being linear, where jobs have many dependencies.

\item A matchmaking optimization has significantly improved the speed 
of matching,
when there are machines with many slots.
\Ticket{2403}

\item When the \Condor{schedd} is starting up and it encounters corruption
in its job transaction log, the error message in the log file now reports
the offset within the file at which the error occurred.
\Ticket{2450}

\end{itemize}

\noindent Configuration Variable and ClassAd Attribute Additions and Changes:

\begin{itemize}

\item The new job ClassAd attribute \Attr{PreserveRelativeExecutable}, 
when \Expr{True} prevents the \Condor{starter} from 
prepending \Attr{Iwd} to the command executable \Attr{Cmd},
when \Attr{Cmd} is a relative path name and \Attr{TransferExecutable} 
is \Expr{False}.
\Ticket{2460}

\item Attributes have been added to all daemons to publish statistics 
about the the number of timers, signals, socket, and pipe messages 
that have been handled, as well as the amount of time spent handling them.	Statistics attributes for DaemonCore
have names that begin with \Expr{DC} or \Expr{RecentDC}.
\Ticket{2354}

\item The default value of \Attr{OpSys} on Windows machines has been changed
to \AdStr{WINDOWS}, and a new attribute \Attr{OpSysVer} has been added 
that contains the version number of the operating system.  
This behavior is controlled by a new configuration variable
\Macro{ENABLE\_VERSIONED\_OPSYS} which defaults to \Expr{False} on Windows 
and to \Expr{True} on other platforms.  
The new machine ClassAd attribute \Attr{OpSys\_And\_Ver} will always contain 
the versioned operating system.  
Note that this change could cause problems with mixed pools,
because Condor version 7.7.2 \Condor{submit} may add \Expr{OpSys="WINDOWS"}, 
but machines running Condor versions prior to 7.7.2 will be publishing 
a versioned \Attr{OpSys} value,
unless there is an override in the configuration.
\Ticket{2366}

\item Configuration variable \Macro{COLLECTOR\_ADDRESS\_FILE} is now set 
in the example configuration,
similar to \MacroNI{MASTER\_ADDRESS\_FILE}.
This configuration variable is required when \Macro{COLLECTOR\_HOST} 
has the port set to 0, which means to select any available port.
In other environments, it should have no visible impact.
\Ticket{2375}

% gittrac #2197
\item Attributes have been added to the \Condor{schedd} 
to publish aggregate statistics
about jobs that are running and have completed, as well as counts of various
failures. 
% Next sentence is made into a comment, as there is no documentation
%     to look at.
% See section ??? for details.
\Ticket{2197}

\item The new configuration variable \Macro{DAGMAN\_WRITE\_PARTIAL\_RESCUE}
enables the new feature of writing a partial DAG file, instead of a full
DAG input file, as a Rescue DAG.  
See section~\ref{param:DAGManWritePartialRescue} for a definition.
Also, the configuration variable
\Macro{DAGMAN\_OLD\_RESCUE} no longer exists,
as it is incompatible with the implementation of partial Rescue DAGs.
\Ticket{2165}

\end{itemize}

\noindent Bugs Fixed:

\begin{itemize}

\item Fixed a bug introduced in Condor version 7.7.1, 
in the standard universe,
where the \Syscall{getdirentries} call failed during remote I/O situations.
\Ticket{2467}

\item Fixed a bug in the \Condor{startd} that was preventing dynamic slots
from being properly instantiated from partitionable slots.
\Ticket{2507}

\item Fixed a bug introduced in Condor version 7.7.0, 
in which the \Condor{startd} may erroneously report 
\Expr{Can't find hostname of client machine.}
In cases where Condor was unable to identify the host name, 
the \Attr{ClientMachine}
attribute in the machine ClassAd would have gone unset.
\Ticket{2382}

\item Fixed a bug existing since April 2001,
in which on start up of the \Condor{schedd}, with parallel universe jobs, 
the job queue sanity checking code would change the \Attr{Scheduler}
attribute on jobs,
only to have the attribute changed later by the dedicated scheduler.
\Ticket{2367}

\item Machine ClassAds with the \Attr{Offline} attribute set to \Expr{True},
and  with neither \Attr{MyType} nor \Attr{TargetType} 
attributes defined caused
the \Condor{collector} to fail to start when it was next restarted.
\Ticket{2417}

\item Fixed a file descriptor leak in the EC2 GAHP,
which would cause grid-type ec2 jobs to become held. 
The \Attr{HoldReason} for most such jobs would be 
\Expr{Unable to read from accesskey file.}
\Ticket{2447}

\item Fixed a bug that could cause a job's standard output and error to
be written to the wrong location when \SubmitCmd{should\_transfer\_files} was
set to \Expr{IF\_NEEDED},
and the job runs on the machine where file transfer is not needed.
If the standard output or error file names contained any path information,
the output would be written to \File{\_condor\_stdout} or
\File{\_condor\_stderr} in the job's initial working directory.
\Ticket{1811}

\item Fixed a bug introduced in Condor version 7.7.1
that could cause the \Condor{schedd} daemon to crash after
failing to expand a \verb@$$@ macro in the job ClassAd.
\Ticket{2491}

\end{itemize}

\noindent Known Bugs:

\begin{itemize}

\item In Condor version 7.7.2, 
the Condor daemons on Linux platforms rely on shared libraries.  
A bug in Condor version 7.7.1 and all previous versions of Condor
prevents a 7.7.1 \Condor{master} from starting 7.7.2 or later daemons.
This also means that a 7.7.1 \Condor{master} cannot upgrade itself to 
version 7.7.2.  
If a 7.7.1 \Condor{master} binary is replaced with 
a 7.7.2 \Condor{master} binary, 
Condor will shut off, and need to be restarted by hand.

\end{itemize}

\noindent Additions and Changes to the Manual:

\begin{itemize}

\item None.

\end{itemize}


%%%%%%%%%%%%%%%%%%%%%%%%%%%%%%%%%%%%%%%%%%%%%%%%%%%%%%%%%%%%%%%%%%%%%%
\subsection*{\label{sec:New-7-7-1}Version 7.7.1}
%%%%%%%%%%%%%%%%%%%%%%%%%%%%%%%%%%%%%%%%%%%%%%%%%%%%%%%%%%%%%%%%%%%%%%

\noindent Release Notes:

\begin{itemize}

%\item Condor version 7.7.1 not yet released.
\item Condor version 7.7.1 released on September 12, 2011.
This developer release contains all bug fixes from Condor version 7.6.3.

\end{itemize}


\noindent New Features:

\begin{itemize}

\item
\emph{Condor now dynamically links with the OpenSSL and Kerberos security
libraries, and Condor will use the operating system's version of these
libraries,  when they are available.} 
The tarball release of Condor on Linux platforms includes 
a copy of these libraries.  
If the operating system's version is incompatible with Condor, 
Condor will use its own copy instead.
Condor's copy of these libraries is located under \File{lib/condor/}.
To prevent Condor from considering using them, delete these libraries.
\Ticket{1874}

\item 
The ClassAd language now has an \Procedure{unparse} function.  
It converts an expression into a string, 
which is handy with the new \Procedure{eval} function.
\Ticket{1613}

\item
The new job ClassAd attribute \Attr{KeepClaimIdle} is defined with an integer
number of seconds in the job submit description file, as the example:
\begin{verbatim}
  +KeepClaimIdle = 300
\end{verbatim}
If set, then when the job exits, 
if there are no other jobs immediately ready to run for this user, 
the \Condor{schedd} daemon,
instead of relinquishing the claim back to the \Condor{negotiator}, 
will keep the claim for the specified number of seconds.  
This is useful if another job will be arriving soon, 
which can happen with linear DAGs.  
The \Condor{startd} slot
will go to the Claimed Idle state for at least that many seconds until
either a new job arrives or the timeout occurs.
See page~\pageref{sec:Job-ClassAd-Attributes},
the unnumbered Appendix A for a complete definition of this
job ClassAd attribute.
\Ticket{2094}

% gittrac #2122
\item The new \Arg{PRE\_SKIP} key word in DAGMan changes the
behavior of DAG node execution such that the node's job and POST script
may be skipped based on the exit value of the PRE script.
See section ~\ref{dagman:SCRIPT} for details.
\Ticket{2122}

% uncomment item, if it appears in 7.7.1
% gittrac #659 
%\item Filip Krikava supplied a patch that limits the number of 
%file descriptors that DAGMan has open at a time.
%The reason for creating this capability is that
%DAGMan tends to fail on wide DAGs, where many jobs are independent,
%rather than being linear, where jobs have many dependencies.

\end{itemize}

\noindent Configuration Variable and ClassAd Attribute Additions and Changes:

\begin{itemize}

\item The new configuration variable 
\Macro{NEGOTIATOR\_STARTD\_CONSTRAINT\_REMOVE} defaults to \Expr{False}.
When \Expr{True}, any ClassAds not satisfying the expression 
in \MacroNI{GROUP\_DYNAMIC\_MACH\_CONSTRAINT} are removed from the
list of \Condor{startd} ClassAds considered for negotiation.
\Ticket{2232}

\item The new configuration variable
\Macro{NEGOTIATOR\_UPDATE\_AFTER\_CYCLE} defaults to \Expr{False}.
When \Expr{True}, it forces the \Condor{negotiator} daemon
to update the negotiator ClassAd in the \Condor{collector} daemon
at the end of every negotiation cycle.  
This is handy for monitoring and debugging activities.
\Ticket{2373}

\end{itemize}

\noindent Bugs Fixed:

\begin{itemize}

\item Expressions for periodic policies such as 
\MacroNI{PERIODIC\_HOLD} and \MacroNI{PERIODIC\_RELEASE} 
could inadvertently cause a claim to be released,
 if the \Condor{shadow} exited before waiting for final update from the 
\Condor{starter}. 
\Ticket{2329}

\item \Condor{submit} previously could incorrectly detect references
in the requirements expression to special attributes such as
\Attr{Memory} when the name of the attribute happened to appear in a
string literal or as part of the name of some other attribute.  
The detection of references to various special attributes influences the
automatic requirements which are appended to the job requirements.
\Ticket{2350}

\item In rare cases, CCB requests could cause the server to hang for
20 seconds while waiting for all of the request to arrive.
\Ticket{2360}

\end{itemize}

\noindent Known Bugs:

\begin{itemize}

\item None.

\end{itemize}

\noindent Additions and Changes to the Manual:

\begin{itemize}

\item None.

\end{itemize}


%%%%%%%%%%%%%%%%%%%%%%%%%%%%%%%%%%%%%%%%%%%%%%%%%%%%%%%%%%%%%%%%%%%%%%
\subsection*{\label{sec:New-7-7-0}Version 7.7.0}
%%%%%%%%%%%%%%%%%%%%%%%%%%%%%%%%%%%%%%%%%%%%%%%%%%%%%%%%%%%%%%%%%%%%%%

\noindent Release Notes:

\begin{itemize}

\item Condor version 7.7.0 released on July 29, 2011.
This developer release contains all bug fixes from Condor version 7.6.2.

\end{itemize}


\noindent New Features:

\begin{itemize}

\item A full port of Condor is available for RedHat Enterprise Linux 6
on the x86\_64 processor.
A full port includes support for the standard universe.

\item The matchmaking attributes \Attr{SubmitterUserResourcesInUse}
and \Attr{RemoteUserResourcesInUse} are now biased by slot weights.

% gittrac #1971
\item \Condor{submit} now accepts the new command line option \Opt{-addr},
naming the IP address of the \Condor{schedd} to submit to.

\item The \Condor{vm\_gahp} now is dynamically linked to libvirt.  
We believe this makes it more portable.

\item Programs \Condor{reconfig\_schedd} and \Condor{master\_off}
are no longer part of the distribution.
These programs were replaced many years ago by the more general
\Condor{reconfig} and \Condor{off} commands.

\item On Windows platforms, improved the ability of the \Condor{starter}
and \Condor{shadow} daemons to clean up the execute directory,
if jobs have changed the ACLs or permissions on files they have created.

\item \Condor{submit} now sets a default value for job ClassAd attribute
\Attr{RequestMemory}.

\item The submission performance of cream grid jobs has been substantially
improved by batching submit requests.

\item \Condor{q} \Opt{-better} now has cleaner output, and informs
the user when negotiation has not yet occurred.

\item Implemented many improvements to the Condor \Prog{init} scripts.

\item Deltacloud support has been updated to deltacloud version 0.8.

% gittrac #1960
\item As of Condor version 7.6.0,
vm universe submit description files no longer support
automatic creation of cdrom images from text input file.
Users must now explicitly create ISO images and transfer them
with the job.

\item \Condor{q} now supports the new option \Opt{-stream-results}.
  When this option is specified, \Condor{q} displays results as they
  are fetched from the job queue, rather than buffering up the query
  results before displaying anything.

% gittrac #1871 
% gittrac #2295
\item The new submit description file command \SubmitCmd{stack\_size} 
  applies to Linux jobs that are not running in the standard universe. 
  It sets the allocation of stack space to be other than the default
  value, which is unlimited.
  It also advertises the job ClassAd attribute \AdAttr{StackSize}.

% gittrac #1550
\item The new ClassAd function \Code{stringListsIntersect} evaluates to 
  \Expr{True} if two strings of delimited elements have any matching elements,
  and it evaluates to \Expr{False} otherwise.

% gittrac #1821
\item The grid universe now supports the \SubmitCmd{ec2} resource type,
  which uses the EC2 Query (REST) API to start virtual machines on cloud
  resources.

% gittrac #2090 
\item The behavior of DAGMan has changed, 
such that if multiple definitions of a VARS macroname 
for a specific node within a DAG input exist,
a warning is written to the log, of the format
\begin{verbatim}
Warning: VAR <macroname> is already defined in job <JobName>
Discovered at file "<DAG input file name>", line <line number>
\end{verbatim}
See section ~\ref{dagman:VARS} for details.

% gittrac #2297
\item The version number for ClassAds now matches the Condor version number. 

% gittrac #2259
\item When \Prog{glexec} fails to execute a job,
diagnostic error messages produced by \Prog{glexec} used to be discarded.
These error messages are now displayed in the log of the \Condor{starter} 
and in the job's hold reason. 

% gittrac #2185
\item New submit description file commands
\SubmitCmd{periodic\_hold\_reason}, \SubmitCmd{periodic\_hold\_subcode},
\SubmitCmd{on\_exit\_hold\_reason}, and \SubmitCmd{on\_exit\_hold\_subcode}
permit the job to set a hold reason string and subcode number.
Similarly, the system job policy can specify the reason and subcode 
using \Macro{SYSTEM\_PERIODIC\_HOLD\_REASON} and 
\Macro{SYSTEM\_PERIODIC\_HOLD\_SUBCODE}.
In addition, the \Condor{hold} command now accepts a \Opt{-subcode} option,
which is used to set the job attribute \Attr{HoldReasonSubCode}. 

\item If the \Condor{shadow} cannot write to the user log, 
the job is now put on hold.

\end{itemize}


\noindent Configuration Variable and ClassAd Attribute Additions and Changes:

\begin{itemize}

\item The new configuration variable \Macro{NEGOTIATOR\_UPDATE\_AFTER\_CYCLE}
defaults to \Expr{False}.
If set to \Expr{True}, it will force the \Condor{negotiator} daemon
to publish an update ClassAd to the \Condor{collector} at the end of 
every negotiation cycle. 
This is useful if monitoring cycle-based statistics.

\item The configuration variables for security 
\Macro{DENY\_CLIENT} and \Macro{HOSTDENY\_CLIENT}
now also look for the prefixes \Expr{TOOL} and \Expr{SUBMIT}.
 
% gittrac #1249
\item \Macro{CONDOR\_VIEW\_HOST} is now a comma and/or white space separated
list of hosts, in order to support more than one CondorView host.

\item For a job with an X.509 proxy credential, the new job ClassAd
attribute \AdAttr{X509UserProxyEmail} is the email address extracted
from the proxy.

% gittrac 2067
\item On Linux execute machines with kernel version more recent than 2.6.27,
the proportional set size (PSS) in Kbytes summed across all
processes in the job is now reported in the attribute
\AdAttr{ProportionalSetSizeKb}.  If the execute machine does not
support monitoring of PSS or PSS has not yet been measured, this
attribute will be undefined.  PSS differs from \AdAttr{ImageSize} in
how memory shared between processes is accounted.  The PSS for one
process is the sum of that process' memory pages divided by the
number of processes sharing each of the pages.  \AdAttr{ImageSize} is
the same, except there is no division by the number of processes
sharing the pages.

% gittrac #1755
\item The new configuration variable \Macro{DAGMAN\_USE\_STRICT} 
turns warnings into errors, as defined in section~\ref{param:DAGManUseStrict}.

% gittrac #2006
\item The \Condor{schedd} now publishes performance-related statistics.
  Page~\pageref{sec:Scheduler-ClassAd-Attributes} in Appendix A contains
  definitions for these new attributes:
  \begin{itemize}
    \item \Attr{DetectedMemory}
    \item \Attr{DetectedCpus}
    \item \Attr{UpdateInterval}
    \item \Attr{WindowedStatWidth}
    \item \Attr{ExitCode<N>}
    \item \Attr{ExitCodeCumulative<N>}
    \item \Attr{JobsSubmitted}
    \item \Attr{JobsSubmittedCumulative}
    \item \Attr{JobsStarted}
    \item \Attr{JobsStartedCumulative}
    \item \Attr{JobsCompleted}
    \item \Attr{JobsCompletedCumulative}
    \item \Attr{JobsExited}
    \item \Attr{JobsExitedCumulative}
    \item \Attr{ShadowExceptions}
    \item \Attr{ShadowExceptionsCumulative}
    \item \Attr{JobSubmissionRate}
    \item \Attr{JobStartRate}
    \item \Attr{JobCompletionRate}
    \item \Attr{MeanTimeToStart}
    \item \Attr{MeanTimeToStartCumulative}
    \item \Attr{MeanRunningTime}
    \item \Attr{MeanRunningTimeCumulative}
    \item \Attr{SumTimeToStartCumulative}
    \item \Attr{SumRunningTimeCumulative}
  \end{itemize}

% gittrac #1930
\item For Windows platforms, the \Condor{startd} now publishes the 
ClassAd attribute \Attr{DotNetVersions},
containing a comma separated list of installed .NET versions.

\end{itemize}

\noindent Bugs Fixed:

\begin{itemize}

\item Fixed a bug in which the \Condor{startd} daemon can get stuck in a
loop trying to execute an invalid, 
that is non-existent, Daemon ClassAd Hook job.

\item Fixed bug that would cause the \Condor{startd} daemon to incorrectly
report Benchmarking activity instead of Idle activity,
when there is a problem launching the benchmarking programs.

\item On Windows only, fixed a rare bug that could cause
a sporadic access violation when a Condor daemon spawned another process.

\item Fixed a bug introduced in Condor version 7.5.5,
which caused the \Condor{schedd} to die managing parallel jobs.

% commented out, as this bug fix is listed in the 7.6.1 version history.
% \item Fixed bug present throughout ClassAds,
% where expressions expecting a floating point value returned an error,
% if they got a boolean value.  This is common in \MacroNI{RANK} expressions.

\item The \Condor{startd} daemon now looks up the \Condor{kbdd} daemon address
on every update.  
This fixed problems if the \Condor{kbdd} daemon is restarted 
during the \Condor{startd} lifespan.

\item Fixed bug in \Condor{hold} that happened if the hold
reason contained a double quote character.

\item Fixed a bug introduced in Condor version 7.5.6 that
caused any Daemon ClassAd hook job with non-empty value for
\MacroNI{STARTD\_CRON\_<JobName>\_ARGS},
\MacroNI{SCHEDD\_CRON\_<JobName>\_ARGS}
or \MacroNI{BENCHMARKS\_<JobName>\_ARGS} to fail.
Also, the specification of 
\MacroNI{STARTD\_CRON\_<JobName>\_ENV},
\MacroNI{SCHEDD\_CRON\_<JobName>\_ENV},
or \MacroNI{BENCHMARKS\_<JobName>\_ENV} for these jobs was ignored.

\item Fixed bug in the RPM \Prog{init} script. 
A status request would always report Condor as inactive, 
and a shutdown request would not report failure if there was a
timeout shutting down Condor.

\item File transfer plug-ins now have a correctly set environment.

\item Fixed a problem with detecting IBM Java Virtual Machines whose
version strings have embedded newline characters.

\item \Condor{q} \Opt{-analyze} now works with ClassAd built-in functions.

\item Fixed bug in \Condor{q} \Opt{-run}, such that it displays
the host name correctly for local and scheduler universe jobs.

\item Standalone checkpointing now works with compressed checkpoints again.
This had been broken in Condor version 7.5.4.

%gittrac 1962
\item On Windows, \Prog{net stop condor} would sometimes cause the
\Condor{master} daemon to crash.  This is now fixed.

% gittrac #1928
\item \AdAttr{JobUniverse} was effectively a required attribute for
  jobs created via the Fetch Work hook,
  due to the need to set the \MacroNI{IS\_VALID\_CHECKPOINT\_PLATFORM}
  expression, such that it would not evaluate to \Expr{Undefined}.
  Now the default \MacroNI{IS\_VALID\_CHECKPOINT\_PLATFORM} expression
  evaluates to \Expr{True} when \AdAttr{JobUniverse} is not defined.

% gittrac #1943
\item When there are multiple cpus but only one slot, the slot name no
longer begins with \Expr{slot1@}.

% gittrac #1805 
\item The tool \Condor{advertise} seemed to be trying too hard to resolve
host names. This was fixed to only do the minimally necessary 
number of look ups.

\end{itemize}

\noindent Known Bugs:

\begin{itemize}

\item None.

\end{itemize}

\noindent Additions and Changes to the Manual:

\begin{itemize}

\item None.

\end{itemize}


%%%      PLEASE RUN A SPELL CHECKER BEFORE COMMITTING YOUR CHANGES!
%%%      PLEASE RUN A SPELL CHECKER BEFORE COMMITTING YOUR CHANGES!
%%%      PLEASE RUN A SPELL CHECKER BEFORE COMMITTING YOUR CHANGES!
%%%      PLEASE RUN A SPELL CHECKER BEFORE COMMITTING YOUR CHANGES!
%%%      PLEASE RUN A SPELL CHECKER BEFORE COMMITTING YOUR CHANGES!

%%%%%%%%%%%%%%%%%%%%%%%%%%%%%%%%%%%%%%%%%%%%%%%%%%%%%%%%%%%%%%%%%%%%%%
\section{\label{sec:History-7-6}Stable Release Series 7.6}
%%%%%%%%%%%%%%%%%%%%%%%%%%%%%%%%%%%%%%%%%%%%%%%%%%%%%%%%%%%%%%%%%%%%%%

This is a stable release series of Condor.
As usual, only bug fixes (and potentially, ports to new platforms)
will be provided in future 7.6.x releases.
New features will be added in the 7.7.x development series.

The details of each version are described below.

%%%%%%%%%%%%%%%%%%%%%%%%%%%%%%%%%%%%%%%%%%%%%%%%%%%%%%%%%%%%%%%%%%%%%%
\subsection*{\label{sec:New-7-6-2}Version 7.6.2}
%%%%%%%%%%%%%%%%%%%%%%%%%%%%%%%%%%%%%%%%%%%%%%%%%%%%%%%%%%%%%%%%%%%%%%

\noindent Release Notes:

\begin{itemize}

\item Condor version 7.6.2 released on July 19, 2011.

\end{itemize}


\noindent New Features:

\begin{itemize}

\item Improved how \Condor{dagman} deals with certain parse errors:
missing node name or submit description file in JOB lines.
Also, \Condor{dagman}
now prints DAG input file lines as they are parsed, 
if the debug verbosity setting is 6 or above,
as set with the \Condor{submit\_dag} command line option \Opt{-debug}.

\end{itemize}

\noindent Configuration Variable and ClassAd Attribute Additions and Changes:

\begin{itemize}

\item None.

\end{itemize}

\noindent Bugs Fixed:

\begin{itemize}

% gittrac #2275 
\item Fixed a bug in the \Condor{negotiator} that impacted the processing 
of machine \MacroNI{RANK} such that \Condor{startd} \MacroNI{RANK}
preemption only occurred if the preempting user had sufficient user priority 
to claim another machine. 

% gittrac #2235 
\item \Condor{ssh\_to\_job} did not work on systems using the 
dash shell for \Prog{/bin/sh}.

% gittrac #2263 
\item \Condor{ssh\_to\_job} now works with jobs that are run via 
\Prog{glexec}. Previously, it did not. 

% gittrac #1642 
\item When \Prog{glexec} was configured with \Expr{linger=on},
the \Condor{starter} would become unresponsive for the duration of the job. 
For jobs longer than the value set by configuration variable
\MacroNI{NOT\_RESPONDING\_TIMEOUT},
this caused the job to be aborted. 
This also prevented job resource usage monitoring from working 
while the job was running.

% gittrac #2262 
\item Fixed a bug in hierarchical group quotas that caused 
a warning to be logged, despite correct implementation.

% gittrac #2261 
\item \Condor{preen} now properly respects the convention that
the \Opt{-debug} option causes \Procedure{dprintf} logging to \Code{stderr}. 

% gittrac #2253 
% gittrac #2294 
\item Fixed a problem introduced in Condor version 7.5.5 
that could cause the \Condor{schedd} to crash when a job was removed 
during negotiation or when an idle parallel universe job left the queue. 

% gittrac #2247 
\item Fixed a problem that sometimes caused the \Condor{procd} to die.
The chain of events for this fixed bug were that
the \Condor{startd} killed the \Condor{starter} due to unresponsiveness,
and the \Condor{procd} would die.
Then \Condor{startd} logged the message
\Expr{ProcD has failed} and the \Condor{startd} exited. 

% gittrac #2233 
\item Fixed a problem introduced in Condor version 7.6.1 
that caused the \Condor{shadow} to crash without successfully putting the job 
on hold when the user log could not be opened for writing. 

% gittrac #2210 
\item \Condor{history} no longer crashes when given a constraint expression 
longer than 512 characters. 

% gittrac #2248 
\item PBS and LSF grid jobs that arrive in a queue via Condor-C
or remote submission again work properly. 

% gittrac #2210 
\item Fix a bug that can cause the \Condor{gridmanager} to crash 
when a CREAM job ClassAd is missing the \Attr{X509UserProxy} attribute. 

% gittrac #2202 
\item Fix a bug that caused CREAM jobs to have incomplete input files,
if the \Condor{gridmanager} crashed during transfer of those input files. 

% gittrac #2201 
\item Fix a bug in \Condor{submit} that caused grid jobs intended for 
CREAM services whose names contain extra dashes to become held. 

\item Fixed a bug in which \Condor{submit} would try, 
but fail to open the Deltacloud password file,
when the file name was dependent on an expression specified with \Expr{\$\$()}.

% gittrac #2173 
\item If the \Attr{Owner} attribute was not set in the ClassAd associated
with a cluster of jobs,
shared spooled executables were not correctly cleaned up.

% gittrac #2238 
\item Fixed a bug for 64-bit versions of Windows in which the
user \Login{condor-reuse-slot<N>} showed up on the login screen.

% gittrac #2288 
\item Fixed a bug introduced in Condor version 7.5.5,
which changed the default value of the configuration variable
\Macro{INVALID\_LOG\_FILES} from the empty set to a file called \File{core}.
This resulted in core files being removed unexpectedly by \Condor{preen},
and that complicated debugging of Condor.
Previous behavior has been restored.

% gittrac #2278 
\item Fixed a bug existing since Condor version 7.5.5 on Windows platforms.
The installer installed Java jar files in the correct \verb|$(BIN)| directory,
while the value for the configuration variable 
\MacroNI{JAVA\_CLASSPATH\_DEFAULT} utilized the obsolete \verb|$(LIB)|
directory.
The installer now correctly sets \MacroNI{JAVA\_CLASSPATH\_DEFAULT} 
to the \verb|$(BIN)| directory.

% gittrac #2308
\item Fixed a problem causing Condor to fail to start when
privsep was enabled and the environment had any variables
containing newlines.

\end{itemize}

\noindent Known Bugs:

\begin{itemize}

\item For Condor versions 7.6.2, 7.6.1, and 7.6.0,
a bug causes parallel universe jobs to be preempted upon 
expiration of the job lease, 
which has a default value of 20 minutes, 
essentially meaning that no parallel universe job that takes
longer than 20 minutes can ever finish.
The work around for this bug is to place the following
configuration variable in the configuration of the submit machine:
\begin{verbatim}
  STARTD_SENDS_ALIVES = FALSE
\end{verbatim}
A \Condor{reconfig} is required, 
after which the preempted parallel universe jobs will then be
able to run to completion.

\end{itemize}

\noindent Additions and Changes to the Manual:

\begin{itemize}

\item None.

\end{itemize}


%%%%%%%%%%%%%%%%%%%%%%%%%%%%%%%%%%%%%%%%%%%%%%%%%%%%%%%%%%%%%%%%%%%%%%
\subsection*{\label{sec:New-7-6-1}Version 7.6.1}
%%%%%%%%%%%%%%%%%%%%%%%%%%%%%%%%%%%%%%%%%%%%%%%%%%%%%%%%%%%%%%%%%%%%%%

\noindent Release Notes:

\begin{itemize}

\item Condor version 7.6.1 released on June 3, 2011.

\end{itemize}


\noindent New Features:

\begin{itemize}

\item None.

\end{itemize}

\noindent Configuration Variable and ClassAd Attribute Additions and Changes:

\begin{itemize}

\item None.

\end{itemize}

\noindent Bugs Fixed:

\begin{itemize}

% gittrac #2170 
\item A bug introduced in Condor version 7.5.5 caused the \Condor{schedd}
to die when its attempt to claim a slot for a parallel universe job 
was rejected by the \Condor{startd}. 

% gittrac #2059
\item \Condor{q} \Opt{-analyze} failed to provide detailed analysis of
the job's requirements expression when the expression contained ClassAd
function calls in some cases. 

% gittrac #2192
\item Fixed an incorrect exit code from \Condor{q} 
when invoked with the \Opt{-name} option and using Quill.

%gittrac #2013
\item Fixed a segmentation fault bug introduced in Condor version 7.5.5,
in the checkpoint and restart of jobs using compressed checkpoint images
under the standard universe.
By default, Condor will not compress checkpoints under the standard universe.
Jobs which do not compress their checkpoints were immune to this bug.  
Compressed checkpoints are only available in 32-bit versions of Condor.
Generation of checkpoints in 64-bit versions of Condor are unaffected.

% gittrac #2069
\item In Condor version 7.6.0, the \Condor{schedd} would create a 
spool directory for every job. The corrected and previous behavior 
has now been restored, 
which is to create a spool directory only when needed.

%gittrac #2086
\item Fixed a bug introduced in Condor version 7.5.2,
that caused the \Condor{negotiator} daemon to crash
if any machine ClassAds contained cyclical attribute references.

%gittrac #2101
\item Fixed a bug that caused usage by \SubmitCmd{nice\_user} jobs to
be charged to the user directly rather than `nice-user.\emph{user}'.
This bug was introduced in the 7.5 series.

%gittrac #2081
\item Fixed bugs in the RPM init script that could cause some 
shutdown failures to be unreported, 
and they could cause status requests,
such as \Expr{service condor status},
to always report Condor as inactive.

\item Fixed a bug in the \Condor{gridmanager} that could cause a crash 
when a grid type \SubmitCmd{amazon} job was missing required attributes.

%gittrac #2105
\item Fixed bug in the \Condor{shadow}, in which it would treat 
the closed socket to the execute machine as an error,
after both it had asked for the claim to be deactivated and the 
\Condor{schedd} daemon was busy.  
As a result, a busy \Condor{schedd} could result in the job being re-run.

%gittrac #2109
\item The matchmaking attributes 
\Attr{SubmitterUserResourcesInUse} and \Attr{RemoteUserResourcesInUse} 
no longer ignore \Attr{SlotWeight}, if used by the \Condor{negotiator}.

%gittrac #2102
\item On Windows, the \Condor{kbdd} daemon was missing changes to the
port on which the \Condor{startd} was listening.
This resulted in failure of the \Condor{kbdd} to send updates in 
keyboard and mouse activity,
further causing the failure of policy implementation that relied upon 
knowledge of the activity.

%gittrac #2163
\item Fixed a bug present throughout ClassAds,
in which expressions expecting a floating point value returned an error,
if the expression actually evaluated to a boolean.
This is most common in machine \MacroNI{RANK} expressions.

%gittrac #2172
\item Fixed a bug in the \Condor{negotiator} daemon,
which caused a crash if the \Condor{negotiator} was reconfigured 
during a negotiation cycle, 
but only if hierarchical group quotas were in use.

%gittrac #2162
\item Fixed a bug in which when submitting a job into the \Condor{schedd}
remotely, or with spooling, 
the file transfer plug-ins activated on the submit machine 
and pulled down all the specified URLs in the transfer list 
to the spool directory. 
This behavior has been changed so that URLs are only downloaded 
when the job is actually running with a \Condor{starter} above it. 
This is usually on an execute node, but could also be in the local universe. 

%gittrac #2178
\item Removed the requirement that the Condor GAHP needs DAEMON-level 
authorization access to the \Condor{gridmanager}. 

%gittrac #2181
\item On Windows platforms only, 
fixed a bug that could cause a sporadic access violation 
when a Condor daemon spawned another process.

%gittrac #2191
\item Fixed a bug that would cause the \Condor{startd} to 
incorrectly report \Expr{Benchmarking} as its activity, instead of \Expr{Idle}
when there was a problem launching the benchmarking programs. 

%gittrac #2193
\item Fixed a bug in which the \Condor{startd} can get stuck in a loop,
trying to execute an invalid, non-existent Daemon ClassAd Hook job. 

%gittrac #2088
\item Fixed a bug in which the dedicated scheduler did not correctly 
deactivate claims,
tending to result in jobs that appear to move back and forth between
the \Expr{Idle} and \Expr{Running} states,
with the \Condor{shadow} daemon exiting each time with status 108.

\end{itemize}

\noindent Known Bugs:

\begin{itemize}

\item None.

\end{itemize}

\noindent Additions and Changes to the Manual:

\begin{itemize}

\item None.

\end{itemize}


%%%%%%%%%%%%%%%%%%%%%%%%%%%%%%%%%%%%%%%%%%%%%%%%%%%%%%%%%%%%%%%%%%%%%%
\subsection*{\label{sec:New-7-6-0}Version 7.6.0}
%%%%%%%%%%%%%%%%%%%%%%%%%%%%%%%%%%%%%%%%%%%%%%%%%%%%%%%%%%%%%%%%%%%%%%

\noindent Release Notes:

\begin{itemize}

\item Condor version 7.6.0 released on April 19, 2011.

% gittrac #2016
\item Prior to Condor version 7.5.0, commenting out \MacroNI{PREEN} in the
  default configuration file disabled \Condor{preen}.  
  Starting in Condor version 7.5.0,
  there was an internal default value for \MacroNI{PREEN}, so if
  the configuration variable was not set in any configuration file,
  \Condor{preen} would still run.
  To disable this functionality, \MacroNI{PREEN} can be explicitly set to
  nothing.

\end{itemize}


\noindent New Features:

\begin{itemize}

\item Condor can now create and manage virtual machine instances in a
cloud service via Deltacloud. This is done via the new
\SubmitCmd{deltacloud} grid type in the grid universe.
See section ~\ref{sec:Deltacloud} for details.

% gittrac #1931
\item Improved scalability of submission of cream grid type jobs.

\end{itemize}

\noindent Configuration Variable and ClassAd Attribute Additions and Changes:

\begin{itemize}

\item The new configuration variable \Macro{DELTACLOUD\_GAHP} specifies
where the \Prog{deltacloud\_gahp} binary is located. This binary is used to
manage deltacloud grid type jobs in the grid universe.
In a normal Condor installation, the value should be
\File{\$(SBIN)/deltacloud\_gahp}.

\item Several new job ClassAd attributes have been added to support
the deltacloud grid type in the grid universe.
These attributes are taken from the submit description file:
\AdAttr{DeltacloudUsername},
\AdAttr{DeltacloudPasswordFile},
\AdAttr{DeltacloudImageId},
\AdAttr{DeltacloudRealmId},
\AdAttr{DeltacloudHardwareProfile},
\AdAttr{DeltacloudHardwareProfileCpu},
\AdAttr{DeltacloudHardwareProfileMemory},
\AdAttr{DeltacloudHardwareProfileStorage},
\AdAttr{DeltacloudKeyname}, and
\AdAttr{DeltacloudUserData}.
%\AdAttr{DeltacloudRetryTimeout},
These attributes are set by Condor when the instance runs:
\AdAttr{DeltacloudAvailableActions},
\AdAttr{DeltacloudPrivateNetworkAddresses},
\AdAttr{DeltacloudPublicNetworkAddresses}.
See section ~\ref{sec:Deltacloud} for details of running jobs under
Deltacloud, and see section ~\ref{sec:Job-ClassAd-Attributes}
for definitions of these job ClassAd attributes.

% gittrac #2024
\item The configuration variable \Macro{JAVA\_MAXHEAP\_ARGUMENT} 
  has been removed. 
  This means that Java universe jobs will now run with the JVM's 
  default maximum heap setting,
  unless specified otherwise by the administrator using the configuration
  of \Macro{JAVA\_EXTRA\_ARGUMENTS},
  or by the job via 
  \SubmitCmd{java\_vm\_args} in the submit description file
  as described in section~\ref{sec:Java}.

% gittrac #2066
\item The configuration variable \Macro{TRUST\_UID\_DOMAIN}
  was set to \Expr{True} in the default \File{condor\_config.local}
  in the rpm and Debian packages.  This is no longer the case.
  This setting will therefore use the default value \Expr{False}.

\item The configuration variable \Macro{NEGOTIATOR\_INTERVAL} was set
  to 20 in the default \File{condor\_config.local} in the rpm and
  Debian packages.  This is no longer the case.  This setting
  therefore will use the default value 60.

\end{itemize}

\noindent Bugs Fixed:

\begin{itemize}

% gittrac #1957
\item Fixed a bug in \Condor{dagman} that caused it to fail when in recovery
mode in the face of a specific combination of node job failures with
retries.

% gittrac #1991
\item Fixed a bug that resulted in the spooled user log not being
  fetched by \Condor{transfer\_data}.  Prior to Condor version 7.5.4, this
  problem affected spooled jobs submitted with an explicit list of
  output files to transfer.  In Condor version 7.5.4, this problem also
  affected spooled jobs that auto-detected output files.

% gittrac #1985
\item When a job is submitted with the \Opt{-spool} option to \Condor{submit},
the \Condor{schedd} now correctly writes the submit event to the user log 
in its spool directory. 
Previously, it would try to write the event using the user
log path given to \Condor{submit} by the user, 
which \Condor{submit} may not have access to.

% gittrac #2001
\item Fixed a file descriptor leak in the \Condor{vm-gahp}. The leak would
cause the daemon to fail if a VMware job ran for more than 16 hours on a
Linux machine.

%gittrac #2017
\item Fixed a bug in \Condor{dagman} that caused it to treat any dollar
sign in the log file name of a node job's submit description file
as an illegal \Condor{dagman} macro.
Now only the sequence of characters \Expr{\$(} delimits a macro.

\end{itemize}

\noindent Known Bugs:

\begin{itemize}

\item There are two known issues related to the automatic creation
of checkpoints with the Condor checkpointing library in 
Condor version 7.6.0.
The first is that compression of
standalone checkpoints is disabled for 32-bit binaries.
It was always disabled previously, for 64-bit binaries.
A standalone checkpoint is one that is run outside
of Condor's standard universe.  The second problem has to do with compressed
32-bit checkpoint files within the standard universe.
If a user requests a compressed 32-bit checkpoint in the standard universe,
the resulting checkpoint will not be compressed.
As with standalone checkpoints, this has never been supported
in 64-bit binaries.  We hope to fix both problems in 
Condor version 7.6.1.

\end{itemize}

\noindent Additions and Changes to the Manual:

\begin{itemize}

\item None.

\end{itemize}


% as of April 2012, Karen no longer wants to include these older
% version histories with the 7.4 and 7.5 manuals.
%%%%      PLEASE RUN A SPELL CHECKER BEFORE COMMITTING YOUR CHANGES!
%%%      PLEASE RUN A SPELL CHECKER BEFORE COMMITTING YOUR CHANGES!
%%%      PLEASE RUN A SPELL CHECKER BEFORE COMMITTING YOUR CHANGES!
%%%      PLEASE RUN A SPELL CHECKER BEFORE COMMITTING YOUR CHANGES!
%%%      PLEASE RUN A SPELL CHECKER BEFORE COMMITTING YOUR CHANGES!

%%%%%%%%%%%%%%%%%%%%%%%%%%%%%%%%%%%%%%%%%%%%%%%%%%%%%%%%%%%%%%%%%%%%%%
\section{\label{sec:History-7-5}Development Release Series 7.5}
%%%%%%%%%%%%%%%%%%%%%%%%%%%%%%%%%%%%%%%%%%%%%%%%%%%%%%%%%%%%%%%%%%%%%%

This is the development release series of Condor.
The details of each version are described below.

%%%%%%%%%%%%%%%%%%%%%%%%%%%%%%%%%%%%%%%%%%%%%%%%%%%%%%%%%%%%%%%%%%%%%%
\subsection*{\label{sec:New-7-5-6}Version 7.5.6}
%%%%%%%%%%%%%%%%%%%%%%%%%%%%%%%%%%%%%%%%%%%%%%%%%%%%%%%%%%%%%%%%%%%%%%

\noindent Release Notes:

\begin{itemize}

\item Condor version 7.5.6 released on March 21, 2011.

\item What used to be known as the \Condor{startd} and \Condor{schedd} cron
  mechanisms are now collectively called \Term{Daemon ClassAd Hooks}.
  The significant changes in this Condor version 7.5.6 release are 
  given in the New Features section.

% gittrac #1935
\item In the release directory, the subdirectory \File{lib/glite/} has
  been moved to \File{libexec/glite/}.

% gittrac #1897
\item This development series of Condor is no longer officially released 
  for the platforms PowerPC AIX, PowerPC-64 SLES 9, PowerPC MacOS 10.4, 
  Solaris 5.9 on all architectures, 
  Solaris 5.10 on all architectures, 
  Itanium IA64 RHEL 3, PS3 (PowerPC) YDL 5.0, and x86 Debian 4.

% gittrac #1924
\item Support for GCB has been removed.

% gittrac #1661
\item The default Unix Sys-V init script has been completely reworked.
  In addition to new features, this changes the following:
  \begin{itemize}
    \item The default location of the Condor configuration file is now
      \File{/etc/condor/condor\_config}.  This location can be changed by
      editing the \File{sysconfig} file or the init script itself.
    \item The default location of the Condor installation is now
      \File{/usr/}, with binaries in \File{/usr/bin} and \File{/usr/sbin}.
      These locations can also be changed by editing the \File{sysconfig} file
      or the init script itself.
  \end{itemize}

\end{itemize}


\noindent New Features:

\begin{itemize}

% gittrack 1800
\item Condor no longer relies on DNS to determine its IP address.
  Instead, it examines the list of system network devices.

% gittrac #1754
\item \Condor{dagman} now gives a warning if a node category has no
nodes assigned to it or no throttle set.

% gittrac #1855
\item \Condor{dagman} now has a \Env{\$MAX\_RETRIES} macro for PRE and
POST script arguments.
Also, \Condor{dagman} now prints a warning if an unrecognized macro is
used for a PRE or POST script argument.
See ~\pageref{dagman:SCRIPT} for details.

% gittrac #1886
\item The \Condor{schedd} is now more efficient in handling the exit of
  \Condor{shadow} processes, when there are large numbers of 
  \Condor{shadow} processes.

%gittrac #1628
\item Condor's Chirp protocol has been updated with new commands.
 The Chirp C++ client
 and \Condor{chirp} command are updated to use the new commands.
  See section ~\ref{man-condor-chirp} for details on the new commands.

\item The Daemon ClassAd Hooks mechanism is described in
section~\ref{sec:daemon-classad-hooks},
with configuration variables defined in section~\ref{sec:Config-hooks}.
The mechanism has the following new features:
  \begin{itemize}
    %gittrac #1086
    \item The \Condor{startd}'s benchmarks are no longer hard coded into
    the \Condor{startd}.  Instead, the benchmarks are now implemented
    via the Daemon ClassAd Hooks mechanism.  Two new programs are
    shipped with Condor version 7.5.6:
    \Condor{mips} and \Condor{kflops}.
    These programs are in  the \File{libexec} directory). 
    They implement the original mips and kflops benchmarks for this 
    new implementation.
    Additional benchmarks can now easily be implemented;
    the list of benchmarks is controlled
    via the new \Macro{BENCHMARKS\_JOBLIST} configuration variable.

  \item Several fixes to the the mips and kflops benchmarks should
    increase the reproducibility of their results.

  %gittrac #1837
  \item Two new job types have been implemented in the Daemon ClassAd
    Hooks mechanism.  They are called \Expr{OneShot} and \Expr{OnDemand}.
    Currently, \Expr{OnDemand} is used only by the new \Expr{BENCHMARKS} 
    mechanism.
  \end{itemize}

\item \Condor{dagman} now prints  out all boolean configuration
variable values as \Expr{True} or \Expr{False},
instead of 1 or 0 within the \File{dagman.out} file.

% gittrac #434
\item Because of the new \Macro{DAGMAN\_VERBOSITY} configuration setting
(see section~\ref{sec:DAGMan-Config-File-Entries}),
the \Opt{-debug} flag is no longer propagated from a top-level DAG to a
sub-DAG; furthermore, \Opt{-debug} is no longer set in a
\File{.condor.sub} file unless it is set on the \Condor{submit\_dag}
command line.

% gittrac #1790
\item When job ClassAd attributes are modified via \Condor{qedit}, 
the changes are now propagated to the \Condor{shadow} and \Condor{gridmanager}.
This allows a user's changes to the job ClassAd to affect the job policy 
expressions while the job is managed by these daemons.

\item Several improvements for CREAM grid jobs:
  \begin{itemize}
  % gittrac #1936
  \item CREAM commands are retried if the server closes the connection
    prematurely.
  % gittrac #940
  \item All jobs going to a CREAM server share the same lease handle.
  % gittrac #1931
  \item Multiple CREAM status requests for single jobs are now batched
    into a single command to the server.
  % gittrac #958
  \item When there are too many commands to be issued to a CREAM server
    simultaneously, new job submissions have lower priority than commands
    operating on existing jobs.
  \end{itemize}

% gittrac #1664  needs documentation?
\item The new script \Condor{gather\_info}, located in \File{bin/},
  creates reports with 
  information from a Condor pool about a specific job ID.
  It also gathers some understanding of the pool under which it runs.

%gittrac #1393
\item Added support for hierarchical accounting groups and group quotas.

%gittrac #1670
\item \Condor{q} -better-analyze now identifies jobs that have not yet been 
  considered by matchmaking, instead of characterizing them as not 
  matching \emph{for unknown reasons}.

% gittrac #1661
\item The default Unix Sys-V init script has been completely reworked.
  The new version should now work on all Unix and Linux systems.
  Major features and changes in the new script:
  \begin{itemize}
    \item Supports the use of a Linux-style \File{sysconfig} file
    \item Supports the use of a Linux-style PID file
    \item Supports the following commands:
      \begin{itemize}
         \item start
         \item stop
         \item restart
         \item try-restart
         \item reload
         \item force-reload
         \item status
      \end{itemize}
    \item The default location of the Condor configuration file is now
     \File{/etc/condor/condor\_config}.  This location can be changed by
      editing the \File{sysconfig} file or the init script itself.
    \item The default location of the Condor installation is now
      \File{/usr/}, with binaries in \File{/usr/bin} and \File{/usr/sbin}.
      These locations can be changed by editing the \File{sysconfig} file
      or the init script itself.
  \end{itemize}

\end{itemize}

\noindent Configuration Variable and ClassAd Attribute Additions and Changes:

\begin{itemize}

% gittrac #1935
\item The default value of configuration variable \Macro{GLITE\_LOCATION}
  has changed to \verb|$(LIBEXEC)/glite|. This reflects the change made in the
  layout of the Condor release files.

% gittrac 1800
\item Values for configuration variables \Macro{NETWORK\_INTERFACE} and
  \Macro{PRIVATE\_NETWORK\_INTERFACE} may now specify a network
  device name or an IP address.  The asterisk character (\verb|*|)
  may be used as a wild card in either a name or IP address.
  This makes it easier to apply the same
  configuration to a large number of machines, because the IP address
  does not have to be customized for each host.

% gittrac #1812
\item The new configuration variable
  \Macro{DELEGATE\_JOB\_GSI\_CREDENTIALS\_LIFETIME} specifies the
  maximum number of seconds for which delegated job proxies should be
  valid.  The default is one day.  A value of 0 indicates that the
  delegated proxy should be valid for as long as allowed by the
  credential used to create the proxy; this was the behavior in
  previous releases of Condor.  This configuration variable currently
  only applies to proxies delegated for non-grid jobs and Condor-C
  jobs.  It does not currently apply to globus grid jobs.  The job may
  override this configuration variable by using the
  \SubmitCmd{delegate\_job\_GSI\_credentials\_lifetime} submit description file
  command.

\item The new configuration variable
  \Macro{DELEGATE\_JOB\_GSI\_CREDENTIALS\_REFRESH} specifies a
    floating point number between 0 and 1 that indicates when
    delegated credentials with limited lifetime should be renewed, as
    a fraction of the delegated lifetime.  The default is 0.25.

\item The new configuration variable
  \Macro{SHADOW\_CHECKPROXY\_INTERVAL} specifies the number of
  seconds between tests to see if the job proxy has been updated or
  should be refreshed.  The default is 600 (10 minutes).  Previously,
  the \Condor{shadow} checked for proxy updates once per minute.

% gittrac #1698
\item Daemon ClassAd Hooks no longer support what was identified as 
  the \emph{old} syntax.  
  Due to this, variables
  \Macro{STARTD\_CRON\_JOBS} and \Macro{HAWKEYE\_JOBS} no longer exist.  
  In previous versions of Condor, the \Condor{startd} would issue a
  warning if this syntax was found, but, starting with 7.5.6, any use
  of these macros will be ignored.

% gittrac #434
\item New configuration variables \Macro{DAGMAN\_VERBOSITY},
\Macro{DAGMAN\_MAX\_PRE\_SCRIPTS}, \Macro{DAGMAN\_MAX\_POST\_SCRIPTS},
and \Macro{DAGMAN\_ALLOW\_LOG\_ERROR}
are defined in section~\ref{sec:DAGMan-Config-File-Entries}.

% gittrac #1719
\item The new configuration variable
\Macro{STARTD\_PUBLISH\_WINREG} can contain a list of Windows 
registry key names, 
whose values will be published in the \Condor{startd} daemon's ClassAd.

\item The new configuration variable
\Macro{CONDOR\_VIEW\_CLASSAD\_TYPES} is a string list that specifies
the types of the ClassAds that will be forwarded to
the location defined by \MacroNI{CONDOR\_VIEW\_HOST}. 
See the definition at section~\ref{sec:Collector-Config-File-Entries}.

% gittrac  #677
\item Added a \Opt{-local-name} command line option to
  \Condor{config\_val} to inspect the values of attributes that use
  local names.


\end{itemize}

\noindent Bugs Fixed:

\begin{itemize}

% gittrac #1892
\item Fixed a bug for parallel universe jobs,
  introduced in Condor version 7.5.5,
  where the \Condor{schedd} would crash under certain conditions when
  a parallel job was removed or exited.

% gittrac #1909
\item Fixed a memory leak in the \Condor{quill} daemon.

% gittrac #1935
\item Fixed a problem in Condor version 7.5.5 release,
  in which binaries used for the grid universe's pbs and lsf grid types 
  were not marked as executable.

% gittrac #1900
\item Fixed a bug introduced in Condor version 7.5.5
that caused running \SubmitCmd{vanilla}, \SubmitCmd{java},
and \SubmitCmd{vm} universe jobs to leave the queue when held.

% gittrac 1826
\item A bug has been fixed that caused SOAP transactions in the
  \Condor{schedd} daemon to result in a log message of the form 
\begin{verbatim}
Timer <X> not found
\end{verbatim}
  This bug is not known to have produced any other undesired behaviors.

% gittrac #1869
\item The job ClassAd attribute \AdAttr{JobLeaseDuration} is now set 
appropriately when a Condor-C job is forwarded to a remote pool.
Previously, a default value was not supplied,
causing jobs to be unnecessarily killed if the
submit and execute machines temporarily lost contact with each other.

% gittrac #1586
\item Fixed a bug that caused \Condor{dagman} to sometimes falsely
report that a cycle existed in a DAG.

% gittrac #741
\item Using the \Condor{hold} command on a Windows platform job managed
by \Condor{dagman} no longer removes the node job of the DAG.
This behavior on Windows now matches the behavior on other platforms.

% gittrac #1490
\item Using the \Condor{hold} command followed by the \Condor{remove} 
command on a job managed by \Condor{dagman}
now removes node jobs of the DAG, rather than leaving them as orphans.

\item A bug has been fixed in the \Condor{config\_val} program,
  which caused it to try to contact the \Condor{collector} before printing
  usage information, if the command line was syntactically incorrect.

% gittrac #1922
\item A bug has been fixed that caused Condor daemons to crash in
  response to DNS look up failures when receiving connections.  The
  crash occurred during authorization of the new connection.  
  This problem was introduced in Condor version 7.5.4.

% gittrac #1806
\item Fixed a bug that caused \Condor{submit} to silently ignore parts
of attribute values if an equals sign was omitted.

% gittrack 1915
\item Starting in Condor version 7.5.5, the \Condor{schedd} daemon would
  sometimes generate an error message and exit abnormally when
  shutting down.  The error message contained the following text:
\begin{verbatim}
ERROR ``Assertion ERROR on (m_ref_count == 0)''
\end{verbatim}

% gittrack 1913
\item Changes to the \Condor{negotiator} daemon's address were 
  not published to the \Condor{collector} until the \Condor{negotiator} daemon
  was reconfigured or restarted.  
  This was a problem in some situations when using \Condor{shared\_port}.

% gittrack #1945
\item A bug introduced in 7.5.5 resulted in failure to advance the
  flocking level due to lack of activity from one of the negotiators
  in the flocking list.

%gittrac #1875
\item Fixed a Windows-specific problem where the main daemon loop can 
  get into a state where it is busy waiting.

%gittrac #1915
\item Fixed \Condor{schedd} exception on shutdown caused by bad reference count.

%gittrac #677
\item Releases of Condor with versions from 7.5.0 to 7.5.5 incorrectly
  implemented the macro language used for configuration with variables
  having \Expr{LOCAL.} at the prefix. This was a regression
  from the Condor 7.4 series. It is now fixed and the functionality
  has been restored.

\end{itemize}

\noindent Known Bugs:

\begin{itemize}

% gittrac #1910
\item If a cycle exists in the set of jobs to be removed defined by 
the job ClassAd attribute \Attr{OtherJobRemoveRequirements},
removing any of the jobs in the set will cause the
\Condor{schedd} to go into an infinite loop.
\Attr{OtherJobRemoveRequirements} is defined on
page ~\pageref{attribute-OtherJobRemoveRequirements}.

% gittrac #1751
\item In a \Condor{dagman} workflow, if a splice contains nothing but
another splice, parsing the DAG will fail.  This can be worked around
by putting any non-splice job, including a DAG-level NOOP job,
into the offending splice.  
This bug has apparently existed since the splice
feature was introduced in \Condor{dagman}.

% gittrac #1947
\item If an individual Daemon ClassAd Hook manager is not \emph{named},
  the jobs under it will attempt to use incorrectly named configuration
  variables.
  For example, the following correct configuration will \emph{not} work,
  because the Daemon ClassAd Hook manager will fail to look up the job's
  executable variable, given the error in configuration variable naming:

\begin{verbatim}
STARTD_CRON_JOBLIST = TEST
...
STARTD_CRON_TEST_MODE       = periodic
STARTD_CRON_TEST_EXECUTABLE = $(LIBEXEC)/test
...
\end{verbatim}

Condor version 7.5.6 and all previous 7.x Condor versions will incorrectly 
name the variables from this example \MacroNI{STARTD\_TEST\_MODE} and
\MacroNI{STARTD\_TEST\_EXECUTABLE} instead.
If instead, the Daemon ClassAd Hook Manager had been named, 
using the no-longer-supported \MacroNI{STARTD\_CRON\_NAME},
the code works as expected.  For example:

\begin{verbatim}
STARTD_CRON_NAME = HAWKEYE
HAWKEYE_JOBLIST  = TEST
...
HAWKEYE_TEST_MODE       = periodic
HAWKEYE_TEST_EXECUTABLE = $(LIBEXEC)/test
...
\end{verbatim}

This \emph{old} behavior is, as of Condor version 7.5.6,
documented as unsupported and is going away,
primarily because it is confusing.
But, for this release, it still works.
It is believed that this same behavior exists in all 7.x releases of Condor,
but because the naming feature is used, the incorrect behavior went undetected.

This affects the \MacroNI{STARTD\_CRON} and \MacroNI{SCHEDD\_CRON}
Daemon ClassAd Hook managers, and will be fixed in Condor version 7.6.0.

\end{itemize}

\noindent Additions and Changes to the Manual:

\begin{itemize}

\item None.

\end{itemize}


%%%%%%%%%%%%%%%%%%%%%%%%%%%%%%%%%%%%%%%%%%%%%%%%%%%%%%%%%%%%%%%%%%%%%%
\subsection*{\label{sec:New-7-5-5}Version 7.5.5}
%%%%%%%%%%%%%%%%%%%%%%%%%%%%%%%%%%%%%%%%%%%%%%%%%%%%%%%%%%%%%%%%%%%%%%

\noindent Release Notes:

\begin{itemize}

\item Condor version 7.5.5 released on January 26, 2011.

\item This version of Condor uses a different layout in the spool
  directory for storing files belonging to jobs that are in the queue.
  Conversion of the spool directory is automatic when upgrading, but
  be aware that \emph{downgrading to a previous version of Condor
  requires extra effort}.  The procedure for downgrading is either
  to drain all jobs with spooled files from the queue, or to manually
  convert the spool back to the older format.  To manually convert
  back to the older format, stop Condor and back up the spool directory
  in case of problems.  Then move all subdirectories matching the form
  \verb|$(SPOOL)/<#>/<#>/cluster<#>.proc<#>.subproc<#>| into
  \verb|$(SPOOL)|.  Also do this for any files of the form
  \verb|$(SPOOL)/<#>/cluster<#>.ickpt.subproc<#>|.  Edit
  \verb|$(SPOOL)/job_queue.log| with a text editor, and change all
  references to the old paths to the new paths.  Then, remove
  \verb|$(SPOOL)/spool_version|.  Finally, start up Condor.

\item For those who compile Condor from the source code rather than
  using packages of pre-built executables, be aware that in this
  release Condor is built using \Prog{cmake} instead of \Prog{imake}.
  See the \File{README.building} file for the new instructions on how
  to build Condor.

\item This release note serves to remind users that as of Condor version 7.5.1,
  the RPMs come with native packaging.  
  Therefore, items are in different locations, as given by FHS locations,
  such as \File{/usr/bin}, \File{/usr/sbin}, \File{/etc}, and \File{/var/log}.  
  Please see section~\ref{sec:install-rpms} for installation documentation.

\item Quill is now available only within the source code distribution 
  of Condor.
  It is no longer included in the builds of Condor provided by UW,
  but it is available as a feature that can be enabled by those who compile
  Condor from the source code.
  Find the code within the \File{condor\_contrib} directory, in the
  directories \File{condor\_tt} and \File{condor\_dbmsd}. 

%\item Windows packages are not included in this release.  We expect to
%  release Windows packages in Condor version 7.5.6.

\item The AIX 5.2 packages in this release have been found to be
  incompatible with AIX 5.3.

\item We are planning to drop support for AIX.  Please contact us if
  this is a problem for you.

\item The directory structure within the Unix tar file package of Condor
  has changed.  Previously, the tar file contained a top level
  directory named \File{condor-<\emph{version}>}.  The top level
  directory is now the same as the tar file name, but without the
  \File{.tar.gz} extension.

\item On Unix platforms, the following executables used to be located in both
  the \File{sbin} and \File{bin} directories,
  but are now only located in the \File{bin}
  directory: \Prog{condor}, \Condor{checkpoint}, \Condor{reschedule}, and
  \Condor{vacate}.

\item The size of the Condor installation has increased by as much as
  60\% compared to Condor version 7.5.4.  We hope to eliminate most of this
  increase in Condor version 7.5.6.

\item Previously, packages containing debug symbols were available for
  many Unix platforms.  In this release, the debug packages contain
  full, `unstripped' executables instead of just the debug symbols.

\item The contents of the Windows zip and MSI packages of Condor have
  changed.  The \File{lib} and \File{libexec} folders no longer exist,
  and all contents previously within them are now in \File{bin}.
  \Condor{setup} and \Condor{set\_acls} have been moved from the top
  level directory into \File{bin}.

\item The Windows MSI installer for Condor version 7.5.5 requires that 
  all previous
  MSI installations of Condor be uninstalled.  Before uninstalling
  previous versions, make backup copies of configuration files.  
  Any settings that
  need to be preserved must be reapplied to the configuration of the
  new installation.

\item The following list itemizes changes included in this Condor version
  7.5.5 release that belong to Condor version 7.4.5.  That stable series
  version will not yet have been released as this development version 
  is released.
  \begin{itemize}

  % gittrac #1713
  % gittrac #1715
  \item \Condor{dagman} now prints a message in the \File{dagman.out} file
  whenever it truncates a node job user log file.
  \Condor{dagman} now prints additional diagnostic information in the
  case of certain log file errors.

  % gittrac #1750
  \item Fixed a bug in which
  a network disconnect between the submit machine and execute
  machine during the transfer of output files caused the
  \Condor{starter} daemon to immediately give up, rather than waiting
  for the \Condor{shadow} to reconnect.  This problem was introduced
  in Condor version 7.4.4.

  % gittrac #1743
  \item Fixed a bug in which
  if \Condor{ssh\_to\_job} attempted to connect to a job while the
  job's input files were being transferred, this caused the file
  transfer to fail, which resulted in the job returning to the idle
  state in the queue.

  % gittrac #1785
  \item In privsep mode, the transfer of output failed if a job's execute
  directory contained symbolic links to non-existent paths.

  \end{itemize}
\end{itemize}


\noindent New Features:

\begin{itemize}

% gittrac 1069
\item Negotiation is now handled asynchronously in the \Condor{schedd} daemon.
This means that the \Condor{schedd} remains responsive during 
negotiation and is less prone to falling behind on communication 
with \Condor{shadow} processes.

% gittrac 1707
\item Improved monitoring and avoidance of a \Term{lock convoy} problem
observed when there were more than 30,000 \Condor{shadow} processes.
At this scale,
locking the \Condor{shadow} daemon's log on each write to the log file
has been observed
on Linux platforms to sometimes result in a situation where the system does
very little productive work, and is instead consumed by rapid context
switching between the \Condor{shadow} daemons that are waiting for the lock.

% gittrac 1706
\item On Linux platforms, if the \Condor{schedd} daemon's spool directory is
  on an ext3 file system, Condor can now scale to a larger number
  of spooled jobs.  Previously, Condor created two subdirectories
  within the spool directory for each spooled job and for each running
  job.  The ext3 file system only supports 31,997 subdirectories.  This
  effectively limited the number of spooled jobs to less than 16,000.
  Now, Condor creates a hierarchy of subdirectories within
  the spool directory, to increase the limit on the number of spooled jobs
  in ext3 to 320,000,000, which is likely to be larger than other limits
  on the size of the job queue, such as memory.

%gittrac 793
\item The \Condor{shadow} daemon uses less memory than it has since 
Condor version 7.5.0.
Memory usage should now be similar to the 7.4 series.

%gittrac 1752
\item The \Condor{dagman} and \Condor{submit\_dag} command-line flag
\Opt{-DumpRescue} causes the dump of an incomplete Rescue DAG,
when the parsing of the DAG input file fails.
This may help in figuring out what went wrong.
See section~\ref{sec:DAGMan-rescue} for complete details on Rescue DAGs.

%gittrac #1077
\item \Condor{dagman} now has the capability to create the
\File{jobstate.log} file needed for the Pegasus workflow manager.
See section~\ref{sec:DAGJobstateLog} for details.

%gittrac 1745
\item \Condor{wait} can now work on jobs with logs that are only
  readable by the user running \Condor{wait}.  Previously, write
  access to the job's user log was required.

%gittrac #1641
\item Added a new value for the job ClassAd attribute \Attr{JobStatus}.
The \Expr{TRANSFERRING\_OUTPUT} status is used
when transferring a job's output files after the job has finished running.
Jobs with this status will have their \Attr{JobStatus} attribute set to 6.
The standard \Condor{q} display will show the job's status as \Expr{>}.

\end{itemize}

\noindent Configuration Variable and ClassAd Attribute Additions and Changes:

\begin{itemize}

% gittrac 1707
\item The new configuration variable \Macro{LOCK\_DEBUG\_LOG\_TO\_APPEND}
controls whether a daemon's debug lock is used when appending to the log.
When the default value of \Expr{False},
the debug lock is only used when rotating the log file.
When \Expr{True}, the debug lock is used when writing to
the log as well as when rotating the log file.
See section~\ref{param:LockDebugLogToAppend} for the complete definition.

%gittrac 1552
\item The new configuration variable
  \Macro{LOCAL\_CONFIG\_DIR\_EXCLUDE\_REGEXP} may be set to a regular
  expression that specifies file names to ignore when looking for
  configuration files within the directories specified via
  \MacroNI{LOCAL\_CONFIG\_DIR}.  
  See section~\ref{param:LocalConfigDirExcludeRegexp} for the 
  complete definition.

\end{itemize}

\noindent Bugs Fixed:

\begin{itemize}

\item In previous versions of Condor, the \Condor{starter} could not 
write the \File{.machine.ad} and \File{.job.ad} files to the \File{execute}
directory when PrivSep was enabled.  This has now been fixed, and these files
are correctly emitted in all cases.

% gittrac 1773
\item Since Condor version 7.5.2, the speed of \Condor{q} was not as high
  as earlier 7.5 and 7.4 releases,
  especially when retrieving large numbers of jobs.
  Viewing 100K jobs took about four times longer.
  This release fixes the performance,
  making it about the same as before Condor version 7.5.2.

% gittrac #1737
\item A bug introduced in Condor version 7.5.4 prevented parallel 
universe jobs with multiple \SubmitCmd{queue} statements in 
the submit description file from working with \Condor{dagman}.
This is now fixed.

% gittrac #1681
\item Improved the way Condor daemons send heartbeat messages to their parent
process.  This resolves a problem observed on busy submit machines using the
\Condor{shared\_port} daemon.  The \Condor{master} daemon sometimes incorrectly
determined that the \Condor{schedd} was hung, and would kill and restart it.

% gittrac #1688
\item When the configuration variable \Macro{NOT\_RESPONDING\_WANT\_CORE}
is \Expr{True},
the \Condor{master} daemon now follows up with a \Expr{SIGKILL},
if the child process does not exit within ten minutes of receiving
\Expr{SIGABRT}.
This addresses observed cases in
which the child process hangs while writing a core file.

% gittrac #1720
\item Host name-based authorization failed in Condor version 7.5.4
under Mac OS X 10.4,
because look ups of the host name for incoming connections failed.

% gittrac #1724
\item A bug introduced in Condor version 7.5.0 caused
the attributes \AdAttr{MyType} and \AdAttr{TargetType}
in offline ClassAds to get set to \Expr{"(unknown type)"}
when the offline ClassAd was matched to a job.

% gittrac #1715
\item \Condor{dagman} now excepts in the case of certain log file errors,
when continuing would be likely to put \Condor{dagman} into an incorrect 
internal state.

% gittrac #1762
\item Fixed a bug that caused DAG node jobs to have their coredumpsize
limit set according to the \MacroNI{CREATE\_CORE\_FILES} configuration
variable, rather than the user's coredumpsize limit.

% gittrac #1777
\item Fixed a case introduced in Condor version 7.5.4 on Windows platforms,
in which the following spurious log message was produced:
\begin{verbatim}
SharedPortEndpoint: Destructor: Problem in thread shutdown notification: 0
\end{verbatim}

% gittrac #1088
\item Since Condor version 7.4.1,
Condor-C jobs submitted without file transfer enabled could
fail with the following error in the \Condor{starter} log:
\begin{verbatim}
FileTransfer: DownloadFiles called on server side
\end{verbatim}

% gittrac #1796
\item Fixed a memory leak caused by use of the ClassAd \Expr{eval()}
  function.  This problem was introduced in Condor version 7.5.2.

% gittrac #1804
\item Fixed a bug that could cause the \Condor{negotiator} daemon to
  crash when groups are configured with
  \MacroNI{GROUP\_QUOTA\_DYNAMIC\_<group\_name>}, or when
  \MacroNI{GROUP\_QUOTA\_<group\_name>} is not defined to be something
  greater than 0.

% gittrac #1809
\item Fixed a bug that caused random characters to appear for the
  value of \Expr{AuthMethods} when logging with \Expr{D\_FULLDEBUG}
  and \Expr{D\_SECURITY} enabled.
  This problem was introduced in Condor version 7.5.3.

% gittrac #1849
\item Fixed a memory leak in the  \Condor{schedd} 
introduced in Condor version 7.5.4.

% gittrac #1866
\item Fixed a problem introduced in Condor version 7.5.4 that could cause the
  \Condor{schedd} daemon to enter an infinite loop while in the
  process of shutting down.  For the problem to happen, it was
  necessary for flocking to have been enabled.

% gittrac #1828
\item Configuration variable \MacroNI{SCHEDD\_QUERY\_WORKERS} was effectively 
  ignored when \Condor{q} authenticated itself to the \Condor{schedd}.
  The query was always
  processed in the main \Condor{schedd} process rather than in a sub-process.
  This problem has existed since before Condor version 7.0.0.

% gittrac #1844
\item Fixed a problem affecting jobs that store their output in the
  \Condor{schedd}'s spool directory.  The problem affected jobs that
  include an empty directory in their list of output files to
  transfer.  This problem was introduced in Condor version 7.5.4,
  when support for the transfer of directories was added.

% gittrac #1749
\item Fixed a problem affecting the \Condor{master} daemon since
  Condor version 7.5.3.  
  The \Condor{master} daemon would crash if it was instructed
  to shut down a daemon that was not currently running,
  but which was waiting to be restarted.

% gittrac #1650
\item Fixed a bug in \Condor{submit} that prevented the submission of
multiple \SubmitCmd{vm} universe jobs in a single submit file.

% gittrac #1761
\item Fixed a bug in the \Condor{schedd} that could cause it to temporarily
under count the number of running local and scheduler universe jobs. 
In Condor version 7.5.4, 
this under counting could cause the daemon to crash.

% gittrac #1695
\item Fixed a bug that could cause the \Condor{gridmanager} to crash if
a GAHP server did not behave as expected on start up.

% gittrac #1747
\item Improved the hold reason reported in several failure cases for 
CREAM grid jobs.

% gittrac #1771
\item The \Attr{KFlops} attribute reported by 
\begin{verbatim}
  condor_status -run -total 
\end{verbatim}
could erroneously be reported as negative.  This has been fixed.

% gittrac #1867
\item Since Condor version 7.5.4, the refreshing of the proxy for the job in the
  remote queue did not work in Condor-C.  Therefore, if the original job proxy
  expired, the job was halted and put on hold, even if the proxy had
  been renewed on the submit machine.

\end{itemize}

\noindent Known Bugs:

\begin{itemize}

% gittrac #1900
\item In Condor version 7.5.5, when a running job is put on hold, the job
  is removed from the job queue.

\end{itemize}

\noindent Additions and Changes to the Manual:

\begin{itemize}

\item None.

\end{itemize}



%%%%%%%%%%%%%%%%%%%%%%%%%%%%%%%%%%%%%%%%%%%%%%%%%%%%%%%%%%%%%%%%%%%%%%
\subsection*{\label{sec:New-7-5-4}Version 7.5.4}
%%%%%%%%%%%%%%%%%%%%%%%%%%%%%%%%%%%%%%%%%%%%%%%%%%%%%%%%%%%%%%%%%%%%%%

\noindent Release Notes:

\begin{itemize}

\item Condor version 7.5.4 released on October 20, 2010.

\item All of the bug fixes and features which are in
Condor version 7.4.4 are in this 7.5.4 release.

% gittrac #1539
\item The release now contains all header files necessary to compile
code that uses the job log reading and writing utilities contained
in libcondorapi. Some headers were missing starting in Condor 7.5.1.

\end{itemize}


\noindent New Features:

\begin{itemize}

% gittrac #1447
\item Concurrency limits now work with parallel universe jobs
scheduled by the dedicated scheduler.

% gittrac #1522
\item Transfer of directories is now supported by
  \SubmitCmd{transfer\_input\_files} and
  \SubmitCmd{transfer\_output\_files} for non-grid universes and
  Condor-C.  The auto-selection of output files, however, remains the
  same: new directories in the job's output sandbox are \emph{not}
  automatically selected as outputs to be transferred.

% gittrac #1520
\item Paths other than simple file names with no directory information
  in \SubmitCmd{transfer\_output\_files} previously did not have well
  defined behavior.  Now, paths are supported for non-grid universes
  and Condor-C.  When a path to an output file or directory is
  specified, this specifies the path to the file on the execute side.
  On the submit side, the file is placed in the job's initial working
  directory and it is named using the base name of the original path.
  For example, \File{path/to/output\_file} becomes \File{output\_file}
  in the job's initial working directory.  The name and path of the
  file that is written on the submit side may be modified by using
  \SubmitCmd{transfer\_output\_remaps}.

% gittrac #991
\item The \Condor{shared\_port} daemon is now supported on Windows platforms.

% gittrac #1257
\item Jobs can now by submitted to multiple EC2 servers via the amazon
grid type. The server's URL must be specified via the \SubmitCmd{grid\_resource}
submit description file command for each job.
See section~\ref{sec:Amazon} for details.

% gittrac #1545
\item The grid universe's amazon grid type can now be used to submit
virtual machine jobs to Eucalyptus systems via the EC2 interface.

%gittrac #1179
\item \Condor{q} now uses the queue-management API's projection feature when 
  used with \Opt{-run}, \Opt{-hold}, \Opt{-goodput}, \Opt{-cputime}, 
  \Opt{-currentrun}, and \Opt{-io} options when called with no display options
  or with \Opt{-format}. 

% gittrac #1460
\item Decreased the CPU utilization of \Condor{dagman} when it is
	submitting ready jobs into Condor.

%gittrac #1479
\item \Condor{dagman} now logs the number of queued jobs in the DAG
that are on hold,
as part of the DAG status message in the \File{dagman.out} file.

%gittrac #825
\item \Condor{dagman} now logs a note in the \File{dagman.out} file
when the \Condor{submit\_dag} and \Condor{dagman} versions differ,
even if the difference is permissible.

%gittrac #1483
\item Added the capability for \Condor{dagman} to create and periodically
rewrite a file that lists the status of all nodes within a DAG.
Alternatively, the file may be continually updated as the DAG runs.
See section~\ref{sec:DAG-node-status} for details.

%gittrac #1560
\item The \Condor{schedd} daemon now uses a better algorithm for
determining which flocking level is being negotiated.  No special
configuration is required for the new algorithm to work.  In the
past, the algorithm depended on DNS and the
configuration variables \MacroNI{NEGOTIATOR\_HOST} and
\MacroNI{FLOCK\_NEGOTIATOR\_HOSTS}.  In some networking environments,
such as that of a multi-homed central manager, it was difficult to
configure things correctly.  When wrongly configured, negotiation
would be aborted with the message, \Expr{Unknown negotiator}.  The new
algorithm is only used when the \Condor{negotiator} is version 7.5.4 or
newer.  Of course, the \Condor{schedd} daemon must still be configured to
authorize the \Condor{negotiator} daemon at the \DCPerm{NEGOTIATOR}
authorization level.

% gittrack #1496
\item \Condor{advertise} has a new option, \Opt{-multiple}, which
allows multiple ClassAds to be published.  This is more efficient than
publishing each ClassAd in a separate invocation of \Condor{advertise}.

% gittrack #1647
\item The \Condor{job\_router} is no longer restricted to routing only vanilla
universe jobs.  It also now automatically avoids recursively routing jobs.

% gittrac #1441
\item The \Condor{schedd} now writes the submit event to the user job log.
Previously, \Condor{submit} wrote the event.

% gittrac #1665
\item The \Condor{schedd} daemon now scales better when there are many
job auto clusters.

% gittrac #1173
\item The \Condor{q} command with option \Opt{-run}, \Opt{-hold}, 
\Opt{-goodput}, \Opt{-cputime}, \Opt{-currentrun} or \Opt{-io}
is now much more efficient in its communication with the \Condor{schedd}.

\end{itemize}

\noindent Configuration Variable and ClassAd Attribute Additions and Changes:

\begin{itemize}

% gittrac #1545
\item The new configuration variable \Macro{SOAP\_SSL\_SKIP\_HOST\_CHECK}
can be used to disable the standard check that a SOAP server's host name
matches the host name in its X.509 certificate. This is useful when submitting
grid type amazon jobs to Eucalyptus servers, which often have certificates
with a host name of \Expr{localhost}.

% gittrac #61
\item Added default values for \MacroNI{<SUBSYS>\_LOG} configuration variables.
  If a \MacroNI{<SUBSYS>\_LOG} configuration variable is not set in 
  files \File{condor\_config} or \File{condor\_config.local},
  it will default to \File{\$(LOG)/<SUBSYS>LOG}.

%gittrac #1385
\item The new job ClassAd attribute \AdAttr{CommittedSuspensionTime}
is a running total of the number of seconds the job has spent in
suspension during time in which the job was not evicted without a
checkpoint.  This complements the existing attribute
\AdAttr{CumulativeSuspensionTime}, which includes all time spent in
suspension, regardless of job eviction.

%gittrack #1385
\item The new job ClassAd attributes \AdAttr{CommittedSlotTime} and
\AdAttr{CumulativeSlotTime} are just like \AdAttr{CommittedTime} and
\AdAttr{RemoteWallClockTime} respectively, except the new attributes
are weighted by the \AdAttr{SlotWeight} of the machine(s) that ran
the job.

%gittrack #1385
\item The new configuration variable
\Macro{SYSTEM\_JOB\_MACHINE\_ATTRS} specifies a list of machine
attributes that should be recorded in the job ClassAd.  The default
attributes are \Attr{Cpus} and \Attr{SlotWeight}.  When there are
multiple run attempts, history of machine attributes from previous
run attempts may be kept.  The number of run attempts to store is
specified by the new configuration variable
\Macro{SYSTEM\_JOB\_MACHINE\_ATTRS\_HISTORY\_LENGTH}, which defaults
to 1.  A machine attribute named \Attr{X} will be inserted into the
job ClassAd as an attribute named \Attr{MachineAttrX0}.  The previous
value of this attribute will be named \Attr{MachineAttrX1}, the
previous to that will be named \Attr{MachineAttrX2}, and so on, up to
the specified history length.  Additional attributes to record may be
specified on a per-job basis by using the new \SubmitCmd{job\_machine\_attrs}
submit file command.  The history length may also be extended on a
per-job basis by using the new submit file command
\SubmitCmd{job\_machine\_attrs\_history\_length}.

% gittrac 1458
\item The new configuration variable
  \Macro{NEGOTIATION\_CYCLE\_STATS\_LENGTH} specifies how many
  recent negotiation cycles should be included in the history that is
  published in the \Condor{negotiator}'s ClassAd.  The default is 3.  See
  page~\pageref{param:NegotiationCycleStatsLength} for the
  definition of this configuration variable, and see
  page~\pageref{attr:LastNegotiationCycleActiveSubmitterCount<X>} for a
  list of attributes that are published.

%gittrac #1560
\item The configuration variable \Macro{FLOCK\_NEGOTIATOR\_HOSTS} is now
optional.  Previously, the \Condor{schedd} daemon refused to flock without
this setting.  When this is not set, the addresses of the flocked
\Condor{negotiator} daemons are found by querying the flocked 
\Condor{collector} daemons.
Of course, the \Condor{schedd} daemon must still be configured to
authorize the \Condor{negotiator} daemon at the \DCPerm{NEGOTIATOR}
authorization level.  Therefore, when using host-based security,
\MacroNI{FLOCK\_NEGOTIATOR\_HOSTS} may still be useful as a macro for inserting
the negotiator hosts into the relevant authorization lists.

%gittrack #1312
\item The configuration variable \MacroNI{FLOCK\_HOSTS} is no longer used.
For backward compatibility, this setting used to be treated as a default
for \MacroNI{FLOCK\_COLLECTOR\_HOSTS} and \MacroNI{FLOCK\_NEGOTIATOR\_HOSTS}.

% gittrac #1257
\item The configuration variable \MacroNI{AMAZON\_EC2\_URL} is now only used
for previously-submitted jobs when upgrading Condor to version 7.5.4 or
beyond. New grid type amazon jobs must specify which EC2 service to use
by setting the \SubmitCmd{grid\_resource} submit description file command.

%gittrack #121
\item The new job ClassAd attribute \AdAttr{NumPids} is the total number of 
 child processes a running job has.

%gittrac #1480
\item The new configuration variable \MacroNI{DAGMAN\_MAX\_JOB\_HOLDS}
specifies the maximum number of times a DAG node job is allowed to go
on hold.  See section~\ref{param:DAGManMaxJobHolds} for details.

% gittrac #1652
\item The configuration variable \Macro{STARTD\_SENDS\_ALIVES} now only
needs to be set for the \Condor{schedd} daemon. Also, the default value has
changed to \Expr{True}.

% gittrac #1593
\item The job ClassAd attributes \SubmitCmd{amazon\_user\_data} and
\SubmitCmd{amazon\_user\_data\_file} can now both be used for the same
job. When both are provided, the two blocks of data are concatenated,
with the value of the one specified by \SubmitCmd{amazon\_user\_data}
occurring first.

% gittrac #1653
\item The new configuration variable \Macro{GRAM\_VERSION\_DETECTION}
can be used to disable Condor's attempts to distinguish between \Expr{gt2}
(GRAM2) and \Expr{gt5} (GRAM5) servers.
The default value is \Expr{True}.
If set to \Expr{False}, Condor trusts the \Expr{gt2} or \Expr{gt5} value
provided in the job's \SubmitCmd{grid\_resource} attribute.

% gittrac #1390
\item The new job ClassAd attribute \AdAttr{ResidentSetSize} is an integer
measuring the amount of physical memory in use by the job on the execute
machine in kilobytes.

% gittrac #1502
\item The new job ClassAd attribute \AdAttr{X509UserProxyExpiration} is an
integer representing when the job's X.509 proxy credential will expire,
measured in the
number of seconds since the epoch (00:00:00 UTC, Jan 1, 1970).

% gittrac #1315
\item The new configuration variable \Macro{SCHEDD\_CLUSTER\_MAXIMUM\_VALUE}
is an upper bound on assigned job cluster ids. If set to
value $M$, the maximum job cluster id assigned to any job will be $M-1$.
When the maximum id is reached, assignment of cluster ids will wrap around 
back to \MacroNI{SCHEDD\_CLUSTER\_INITIAL\_VALUE}. The default value is zero,
which does not set a maximum cluster id. 

% gittrac #1348
% gittrac #1487
\item The default value of configuration variable 
\MacroNI{MAX\_ACCEPTS\_PER\_CYCLE} has been changed from 1 to 4.

%gittrac #1310
\item The configuration variable \Macro{NEW\_LOCKING}, introduced in
  Condor version 7.5.2, has been changed to
  \Macro{CREATE\_LOCKS\_ON\_LOCAL\_DISK} and now defaults to \Expr{True}.

\end{itemize}

\noindent Bugs Fixed:

\begin{itemize}

% gittrac #1766
\item Fixed a bug that occurred with x64 flavors of the Windows operating system. 
  Condor was setting the default value of \AdAttr{Arch} to \Expr{INTEL} when it 
  should have been \Expr{X86\_64}.  This was a consequence of the fact that the 
  Condor runs in the WOW64 sandbox on 64-bit Windows.  This was fixed so that
  \AdAttr{Arch} would contain the value for the native architecture rather than 
  the WOW64 sandbox architecture.

% gittrac #1500
\item Fixed a bug in the user privilege switching code in Windows that 
  caused the \Condor{shadow} daemon to except when the \Condor{schedd} 
  daemon attempted to re-use it. 

% gittrack #1667
\item Fixed the output in the \Condor{master} daemon log file to be
  clearer when an authorized user tries to use \Condor{config\_val}
  \Opt{-set} and \Macro{ENABLE\_PERSISTENT\_CONFIG} is \Expr{False}.
  The previous
  output implied that the operation succeeded when, in fact, it did not.

% gittrac #1523
\item Since Condor version 7.5.2,
  the following \Condor{job\_router} features were
  effectively non-functional: \Attr{UseSharedX509UserProxy},
  \Attr{JobShouldBeSandboxed}, and \Attr{JobFailureTest}.

% gittrack #1561
\item The submit description file command \SubmitCmd{copy\_to\_spool}
  did not work properly in Condor version 7.5.3.
  When sending the executable to the execute machine, it was
  transferred from the original path rather than from the spooled copy
  of the file.

% gittrack #1521
\item When output files were auto-selected and spooled, Condor-C and
  \Condor{transfer\_data} would copy back both the output files and
  all other contents of the job's spool directory, which typically
  included the spooled input and the user log.  
  Now, only the output files are retrieved.
  To adjust which files are retrieved, the job
  attribute \Attr{SpooledOutputFiles} can be manipulated, but this
  typically should be managed by Condor.

% gittrac 1139
\item The \Condor{master} daemon now invalidates its ClassAd,
  as represented in the \Condor{collector} daemon, before it shuts down.

% gittrac #1563
\item Fixed a bug that caused \SubmitCmd{vm} universe jobs to not run
if the VMware \File{.vmx} file contained a space.

% gittrac #1549
\item Fixed a bug introduced in Condor version 7.5.1 that caused integers 
in ClassAd expressions that had leading zeros to be read as octal (base eight).

% gittrac #1516
\item Fixed a bug introduced in Condor version 7.5.1 that did not recognize 
a semicolon as a separator of function arguments in ClassAds.

% gittrac #1544
\item Fixed a bug that caused integers larger than $\pm2^{31}$ in a ClassAd
expression to be parsed incorrectly. Now, when these integers are
encountered, the largest 32-bit integer (with matching sign) is used.

% gittrac #1537
\item Fixed a bug that caused the \Condor{gridmanager} to exit when
receiving badly-formatted error messages from the \Prog{nordugrid\_gahp}.

% gittrac #1342
% gittrac #1644
\item Fixed a problem affecting the use of version 7.5.3 \Condor{startd} and
  \Condor{master} daemons in a pool with a \Condor{collector} from before
  version 7.5.2.  On shutdown, the \Condor{startd} and the \Condor{master}
  caused all \Condor{startd} and \Condor{master} ClassAds, respectively,
  to be removed from the \Condor{collector}.

% gittrac #1590
\item Fixed a bug that caused delegation of an X.509 RFC proxy between
two Condor processes to fail.

% gittrac #1563
\item Fixed a bug in \Condor{submit} that would cause failures if a file
name containing a space was used with the submit description file commands
\SubmitCmd{append\_files}, \SubmitCmd{jar\_files} or
\SubmitCmd{vmware\_dir}.

% gittrac #890
\item Fixed a bug that could cause the \Condor{gridmanager} to lock up if
a GAHP server it was using wrote a large amount of data to its \File{stderr}.

% gittrac #1653
% gittrac #1475
\item Fixed a bug that could cause the \Condor{gridmanager} to wrongly
conclude that a \Expr{gt2} (that is, GRAM2) server was a \Expr{gt5}
(that is, GRAM5) server.
Such a conclusion can be disastrous, as Condor's mechanisms to
prevent overloading a \Expr{gt2} server are then disabled. The new
configuration variable \Macro{GRAM\_VERSION\_DETECTION} can be used 
to disable Condor's attempts to distinguish between the two.

% gittrac #1689
\item Fixed a bug introduced in Condor version 7.5.3. 
When file transfer failed for a \SubmitCmd{grid} universe job of grid type 
cream,
Condor would write a hold event to the job log,
but not actually put the job on hold.

% gittrac #1694
\item Fixed a bug in the \Condor{gridmanager} that could cause it to crash
while handling cream grid type jobs destined for different resources.

% gittrac #1481
\item Fixed a bug that prevented the \Condor{shadow} from managing
additional jobs after its first job completed when 
\Macro{SEC\_ENABLE\_MATCH\_PASSWORD\_AUTHENTICATION} was set to \Expr{True}.

% gittrac #1533
\item The timestamps in the log defined by \Macro{PROCD\_LOG}
now print the real time.

% gittrac #1580
\item Fixed how some daemons advertise themselves to the \Condor{collector}.
Now, all daemons set the attribute \AdAttr{MyType} to indicate what
type of daemon they are.

% gittrac #1630
\item \Condor{chirp} no longer crashes on a put operation,
if the remote file name is omitted.

% gittrac #1489
% gittrac #1494
\item Fixed the packaging of Hadoop File System support in Condor. This includes
updating to HDFS 0.20.2 and making the HDFS web interface work properly.

% gittrac #1717
\item Condor no longer tries to invoke \Prog{glexec} if the job's X.509 proxy
is expired.

\end{itemize}

\noindent Known Bugs:

\begin{itemize}

% gittrac #1720
\item Using host names for host-based authentication,
such as in the definitions of configuration variables 
\MacroNI{ALLOW\_*} and \MacroNI{DENY\_*},
does not work on Mac OS X 10.4.
Later versions of the OS are not affected.
As a work around, IP addresses can be used instead of host names.

\end{itemize}

\noindent Additions and Changes to the Manual:

\begin{itemize}

\item None.

\end{itemize}


%%%%%%%%%%%%%%%%%%%%%%%%%%%%%%%%%%%%%%%%%%%%%%%%%%%%%%%%%%%%%%%%%%%%%%
\subsection*{\label{sec:New-7-5-3}Version 7.5.3}
%%%%%%%%%%%%%%%%%%%%%%%%%%%%%%%%%%%%%%%%%%%%%%%%%%%%%%%%%%%%%%%%%%%%%%

\noindent Release Notes:

\begin{itemize}

\item Condor version 7.5.3 released on June 29, 2010.

\end{itemize}


\noindent New Features:

\begin{itemize}

% gittrac 1274
\item \Condor{q} \Opt{-analyze} now notices the \Opt{-l} option, and if both
are given, then the analysis prints out the list of machines
in each analysis category.

% gittrac 1302
\item The behavior of macro expansion in the configuration file has
  changed.  Previously, most macros were effectively treated as
  undefined unless explicitly assigned a value in the configuration
  file.  Only a small number of special macros had pre-defined values
  that could be referred to via macro expansion.  Examples include
  \MacroNI{FULL\_HOSTNAME} and \MacroNI{DETECTED\_MEMORY}.  Now, most
  configuration settings that have default values can be referred to
  via macro expansion.  There are a small number of exceptions where
  the default value is too complex to represent in the current
  implementation of the configuration table.  Examples include the
  security authorization settings. All such configuration settings
  will also be reported as undefined by \Condor{config\_val} unless
  they are explicitly set in the configuration file.

% gittrac 1131
\item Unauthenticated connections are now identified as
  \verb|unauthenticated@unmapped|.  Previously, unauthenticated
  connections were not assigned a name, so some authorization policies
  that needed to distinguish between authenticated and unauthenticated
  connections were not expressible.  Connections that are
  authenticated but not mapped to a name by the mapfile used to be
  given the name \verb|auth-method@unmappeduser|, where
  \emph{auth-method} is the authentication method that was used.  Such
  connections are now given the name \verb|auth-method@unmapped|.
  Connections that match \verb|*@unmapped| are now forbidden from
  doing operations that require a user id, regardless of configuration
  settings.  Such operations include job submission, job removal, and
  any other job management commands that modify jobs.

% gittrac 1131
\item There has been a change of behavior when authentication fails.
  Previously, authentication failure always resulted in the command
  being rejected, regardless of whether the ALLOW/DENY settings
  permitted unauthenticated access or not.  This is still true if either
  the client or server specifies that authentication is required.
  However, if both sides specify that authentication is not required
  (i.e. preferred or optional), then authentication failure only results
  in the command being rejected if the ALLOW/DENY settings reject
  unauthenticated access.  This change makes it possible to have some
  commands accept unauthenticated users from some network addresses
  while only allowing authenticated users from others.

\item Improved log messages when failing to authenticate requests.  At
  least the IP address of the requester is identified in all cases.

% gittrac 1357
\item The new submit file command \SubmitCmd{job\_ad\_information\_attrs}
may be used to specify attributes from the job ad that should be saved
in the user log whenever a new event is being written.  See
page~\pageref{man-condor-submit-job-ad-information-attrs} for details.

%gittrac 1391
\item Administrative commands now support the \Opt{-constraint} option, which
  accepts a ClassAd expression.  This applies to \Condor{checkpoint},
  \Condor{off}, \Condor{on}, \Condor{reconfig}, \Condor{reschedule},
  \Condor{restart}, \Condor{set\_shutdown}, and \Condor{vacate}.

%gittrac #1351
\item File transfer plugins can be used for vm universe jobs. Notably,
  \Expr{file://} URLs can be used to allow VM image files to be pre-staged
  on the execute machine. The submit description file command 
  \SubmitCmd{vmware\_dir} is now optional.
  If it is not given, then all relevant VMware image files
  must be listed in \SubmitCmd{transfer\_input\_files}, possibly as URLs.

%gittrac #1403
\item File transfers for CREAM grid universe jobs are now initiated by
  the \Condor{gridmanager}. This removes the need for a GridFTP server
  on the client machine.

%gittrac #1403
\item Improved the parallelism of file transfers for nordugrid jobs.

%gittrac #1298
\item Removed the distinction between regular and full reconfiguration
  of Condor daemons. Now, all reconfigurations are full and require the
  WRITE authorization level. \Condor{reconfig} accepts but ignores the
  \Opt{-full} command-line option.

\item The \Prog{batch\_gahp}, used for pbs and lsf grid universe jobs, has been
updated from version 1.12.2 to 1.16.0.

\item \Condor{dagman} now prints a message to the \File{dagman.out} file
when it truncates a node job user log file.

%gittrac 1410
\item \Condor{dagman} now allows node categories to include
nodes from different splices.  See section~\ref{sec:DAG-node-category}
for details.

%gittrac 1410
\item \Condor{dagman} now allows category throttles in splices to
be overridden by higher levels in the DAG splicing structure.
See section~\ref{sec:DAG-node-category} for details.

%gittrac 1158
\item Daemon logs can now be rotated several times instead of only once 
  into a single \File{.old} file. In order to do so, the newly introduced 
  configuration variable \Macro{MAX\_NUM\_<SUBSYS>\_LOG} needs to be set 
  to a value greater than 1. The file endings will be ISO timestamps, and
  the oldest rotated file will still have the ending \File{.old}.
 

\end{itemize}

\noindent Configuration Variable and ClassAd Attribute Additions and Changes:

\begin{itemize}

\item The new configuration variable \Macro{JOB\_ROUTER\_LOCK}  specifies a
  lock file used to
  ensure that multiple instances of the \Condor{job\_router} never run
  with the same values of \MacroNI{JOB\_ROUTER\_NAME}.
  Multiple instances running
  with the same name could lead to mismanagement of routed jobs.

\item The new configuration variable \Macro{ROOSTER\_MAX\_UNHIBERNATE}
  is an integer
  specifying the maximum number of machines to wake up per cycle.
  The default value of 0 means no limit.

\item The new configuration variable \Macro{ROOSTER\_UNHIBERNATE\_RANK}
  is a ClassAd
  expression specifying which machines should be woken up first in a
  given cycle.  Higher ranked machines are woken first.
  If the number of machines to be woken up is limited by
  \MacroNI{ROOSTER\_MAX\_UNHIBERNATE}, the rank may be used for
  determining which machines are woken before reaching the limit.

% gittrac 1228
\item The new configuration variable \Macro{CLASSAD\_USER\_LIBS}
  is a list of libraries
  containing additional ClassAd functions to be used during ClassAd
  evaluation.

% gittrac 1375
\item The new configuration variable \MacroNI{SHADOW\_WORKLIFE}
  specifies the number of seconds after which the \Condor{shadow} will exit,
  when the current job finishes, instead of fetching a new job to
  manage.  Having the \Condor{shadow} continue managing jobs helps
  reduce overhead and can allow the \Condor{schedd} to achieve higher
  job completion rates.  The default is 3600, one hour.  The value 0
  causes \Condor{shadow} to exit after running a single job.

%gittrac 1158  
\item The new configuration variable \Macro{MAX\_NUM\_<SUBSYS>\_LOG} 
  will determine how often the daemon log of \MacroNI{SUBSYS} will rotate.
  The default value is 1 which leads to the old behavior of a single 
  rotation into a \File{.old} file.

\end{itemize}

\noindent Bugs Fixed:

\begin{itemize}

% gittrac 1332
\item Configuration variables with a default value of 0
  that were not defined in the configuration file
  were treated as though they were undefined by \Condor{config\_val}.
  Now Condor treats this case like any other:
  the default value is displayed.

% gittrac #1203
\item Starting in Condor version 7.5.1,
  using literals with a logical operator
  in a ClassAd expression (for example, \Expr{1 || 0}) caused the expression
  to evaluate to the value \Expr{ERROR}. The previous behavior has been
  restored: zero values are treated as \Expr{False},
  and non-zero values are treated as \Expr{True}.


% gittrac 1378
\item Starting in Condor version 7.5.0,
  the \Condor{schedd} no longer supported queue
  management commands when security negotiation was disabled,
  for example, if \Expr{SEC\_DEFAULT\_NEGOTIATION = NEVER}.

% gittrac 1395
\item Fixed a bug introduced in Condor version 7.5.1.
ClassAd string literals containing
characters with negative ASCII values were not accepted.

% gittrac #1334
\item Fixed a bug introduced in Condor version 7.5.0,
which caused Condor to not renew
job leases for CREAM grid jobs in most situations.

% gittrac #1331
\item Question marks occurring in a ClassAd string are no longer preceded
by a backslash when the ClassAd is printed.

\end{itemize}

\noindent Known Bugs:

\begin{itemize}

\item None.

\end{itemize}

\noindent Additions and Changes to the Manual:

\begin{itemize}

\item None.

\end{itemize}


%%%%%%%%%%%%%%%%%%%%%%%%%%%%%%%%%%%%%%%%%%%%%%%%%%%%%%%%%%%%%%%%%%%%%%
\subsection*{\label{sec:New-7-5-2}Version 7.5.2}
%%%%%%%%%%%%%%%%%%%%%%%%%%%%%%%%%%%%%%%%%%%%%%%%%%%%%%%%%%%%%%%%%%%%%%

\noindent Release Notes:

\begin{itemize}

\item Condor version 7.5.2 released on April 26, 2010.

% gittrac 1003
\item Condor no longer supports SuSE 8 Linux on the Itanium 64 architecture.

% gittrac #600
\item The following submit description file commands are no longer recognized.
Their functionality is replaced by the command \SubmitCmd{grid\_resource}.
\begin{description}
  \item{\SubmitCmd{grid\_type}}
  \item{\SubmitCmd{globusscheduler}}
  \item{\SubmitCmd{jobmanager\_type}}
  \item{\SubmitCmd{remote\_schedd}}
  \item{\SubmitCmd{remote\_pool}}
  \item{\SubmitCmd{unicore\_u\_site}}
  \item{\SubmitCmd{unicore\_v\_site}}
\end{description}

\end{itemize}


\noindent New Features:

\begin{itemize}

% gittrac 1231
% gittrac 1287
\item The \Condor{schedd} daemon uses less disk bandwidth when logging
updates to job ClassAds from running jobs and also when removing jobs
from the queue and handling job eviction and \Condor{shadow} exceptions.
This should improve performance in situations where
disk bandwidth is a limiting factor.
Some cases of updates to the job user log
have also been optimized to be less disk intensive.

% gittrac 1288
\item The \Condor{schedd} daemon uses less CPU when scheduling
some types of job queues.  Most likely to benefit from this improvement is
a large queue of short-running, non-local, and non-scheduler universe jobs,
with at least one idle local or scheduler universe job.

% gittrac 1266
\item The \Condor{schedd} automatically grants the \Condor{startd}
  authority to renew leases on claims and to evict claims.
  Previously, this required that the \Condor{startd} be trusted for
  general \DCPerm{DAEMON}-level command access.  Now this only
  requires \DCPerm{READ}-level command access.  The specific commands
  that the \Condor{startd} sends to the \Condor{schedd} can
  effectively only operate on the claims associated with that \Condor{startd},
  so this change does not open up these operations to access by anyone
  with \DCPerm{READ} access.  It reduces the level of trust that
  the \Condor{schedd} must have in the \Condor{startd}.

% gittrac 834
\item The \Condor{procd}'s log now rotates if logging is activated. 
  The default maximum size is 10Mbytes. To change the default,
  use the configuration variable \Macro{MAX\_PROCD\_LOG}.  

%gittrac 1310
\item For Unix systems only, 
  user job log and global job event log lock files can now optionally 
  be created in a directory on a 
  local drive by setting \MacroNI{NEW\_LOCKING} to \Expr{True}. 
  See section~\ref{param:NewLocking} for 
  the details of this configuration variable.
  
%gittrac 507
\item \Condor{dagman} and \Condor{submit\_dag} now default to lazy
creation  of the \File{.condor.sub} files for nested DAGs.
\Condor{submit\_dag} no longer creates them, and \Condor{dagman}
itself creates the files as the DAG is run.
The previous "eager" behavior can
be obtained with a combination of command-line and configuration settings.
There are several advantages to the "lazy" submit file creation:
\begin{itemize}
\item The DAG file for a nested DAG does not have to exist until that node
is ready to run, so the DAG file can be dynamically created by earlier
parts of the top-level DAG (including by the PRE script of the nested
DAG node).
\item It is now possible to have nested DAGs within splices, which is not
possible with "eager" submit file creation.
\end{itemize}

\end{itemize}

\noindent Configuration Variable and ClassAd Attribute Additions and Changes:

\begin{itemize}

%gittrac 507
\item The new configuration variable
\MacroNI{DAGMAN\_GENERATE\_SUBDAG\_SUBMITS} controls whether
\Condor{dagman} itself generates the \File{.condor.sub} files for
nested DAGs, rather than relying on \Condor{submit\_dag} "eagerly"
creating them.  See section~\ref{param:DAGManGenerateSubDagSubmits} for
more information.

%gittrac 1310
\item The new configuration variable \Macro{NEW\_LOCKING} can specify that
  job user logs and the global job event log to be written to a local drive,
  avoiding locking problems with NFS.
  See section~\ref{param:NewLocking} for 
  the details of this configuration variable.
\end{itemize}

\noindent Bugs Fixed:

\begin{itemize}

% gittrac 1300
\item The \Condor{job\_router} failed to work on SLES 9 PowerPC,
AIX 5.2 PowerPC,
and YDL 5 PowerPC due to a problem in how it detected EOF in the job queue log.

% gittrac 742
\item When jobs are removed, the \Condor{schedd} sometimes did not
  quickly reschedule a different job to run on the slot to which the
  removed job had been matched.  Instead, it would take up to
  \MacroNI{SCHEDD\_INTERVAL} seconds to do so.

% gittrac #1279
% Not documenting gittrac #1280, as it was completely covered up by
% #1279.
\item Fixed a bug introduced in Condor version 7.5.1 that caused the
\Prog{gahp\_server} to crash when
first communicating with most gt2 or gt5 GRAM servers.

\end{itemize}

\noindent Known Bugs:

\begin{itemize}

\item None.

\end{itemize}

\noindent Additions and Changes to the Manual:

\begin{itemize}

\item None.

\end{itemize}


%%%%%%%%%%%%%%%%%%%%%%%%%%%%%%%%%%%%%%%%%%%%%%%%%%%%%%%%%%%%%%%%%%%%%%
\subsection*{\label{sec:New-7-5-1}Version 7.5.1}
%%%%%%%%%%%%%%%%%%%%%%%%%%%%%%%%%%%%%%%%%%%%%%%%%%%%%%%%%%%%%%%%%%%%%%

\noindent Release Notes:

\begin{itemize}

\item Condor version 7.5.1 released on March 2, 2010.

\item Some, but not all of the bug fixes and features which are in
Condor version 7.4.2, are in this 7.5.1 release.

\item The Condor release is now available as a proper RPM or Debian
package.

\item Condor now internally uses the version of New ClassAds provided
as a stand-alone library (\URL{http://www.cs.wisc.edu/condor/classad/}).
Previously, Condor 
used an older version of ClassAds that was heavily tied to the Condor 
development libraries. This change should be transparent in the 
current development series. In the next development series (7.7.x),
Condor  will begin to use features of New ClassAds that were unavailable in 
Old ClassAds. 
Section~\ref{sec:classad-newandold} details the differences.

\item HPUX 11.00 is no longer a supported platform.

\end{itemize}


\noindent New Features:

\begin{itemize}

% gittrac #1102
\item A port number defined within \Macro{CONDOR\_VIEW\_HOST} may now use 
  a shared port.

% gittrac #1104
\item The \Condor{master} no longer pauses for 3 seconds after starting
  the \Condor{collector}.  However, if the configuration variable
  \MacroNI{COLLECTOR\_ADDRESS\_FILE} defines a file, 
  the \Condor{master} will wait for that file to be created
  before starting other daemons.

% gittrac #1144
\item In the grid universe, Condor can now automatically distinguish
between GRAM2 and GRAM5 servers, that is grid types \SubmitCmd{gt2} and
\SubmitCmd{gt5}.
Users can submit jobs using a grid type of \SubmitCmd{gt2} or \SubmitCmd{gt5}
for either type of server.

% gittrac #938
\item Grid universe jobs using the CREAM grid system now batch up
common requests into larger single requests.  This
reduces network traffic, increases the number of parallel tasks
the Condor can handle at once, and reduces the load on the remote
gatekeeper.

% gittrac #1100
\index{submit commands!cream\_attributes}
\item The new submit description file command \SubmitCmd{cream\_attributes}
sets additional attribute/value pairs for the CREAM job description
that Condor creates when submitting a grid universe job 
destined for the CREAM grid system.

% gittrac #1138
\item The \Condor{q} command with option \Opt{-analyze} is now performs
the same analysis as previously occurred with the \Opt{-better-analyze} option.
Therefore, the output of \Condor{q} with the \Opt{-analyze} option
has different output than before.
The \Opt{-better-analyze} option is still recognized and behaves the same
as before, though it may be removed from a future version.

% gittrac #1169
\item Security sessions that are not used for longer than an hour are
now removed from the security session cache to limit memory usage.

% gittrac #1169
\item The number of security sessions in the cache is now advertised in
the daemon ClassAd as \Attr{MonitorSelfSecuritySessions}.

% gittrac #1078
\item \Condor{dagman} now has the capability to run DAGs containing nodes
that are declared to be NOOPs -- for these nodes, a job is never actually
submitted.  See section~\ref{dagman:NOOP} for information.

% gittrac #1128
\index{submit commands!vm\_macaddr}
\item The submit file attribute \SubmitCmd{vm\_macaddr} can now be used to set
the MAC address for vm universe jobs that use VMware. The range of valid
MAC addresses is constrained by limits imposed by VMware.

% gittrac #1173
\item The \Condor{q} command with option \Opt{-globus}
is now much more efficient in its communication with the \Condor{schedd}.

\end{itemize}

\noindent Configuration Variable and ClassAd Attribute Additions and Changes:

\begin{itemize}

% gittrac #1242
\item The new configuration variable \MacroNI{STRICT\_CLASSAD\_EVALUATION}
controls whether new or old ClassAd expression evaluation semantics are
used. In new ClassAd semantics, an unscoped attribute reference is only
looked up in the local ad. The default is False (use old ClassAd semantics).

% gittrac #221
\item The configuration variable
\MacroNI{DELEGATE\_FULL\_JOB\_GSI\_CREDENTIALS} now applies to all proxy
delegations done between Condor daemons and tools.
The value is a boolean and defaults to \Expr{False},
which means that when doing delegation Condor will now create a limited proxy
instead of a full proxy.

\item The new configuration variable
  \MacroIndex{SEC\_DEFAULT\_SESSION\_LEASE}
  \Macro{SEC\_<access-level>\_SESSION\_LEASE} specifies the maximum
  number of seconds an unused security session will be kept in a daemon's
  session cache before being removed to save memory.  The default is 3600.
  If the server and client have different configurations, the smaller
  one will be used.

\end{itemize}

\noindent Bugs Fixed:

\begin{itemize}

% gittrac #1141
\item The default value for \Macro{SEC\_DEFAULT\_SESSION\_DURATION}
  was effectively 3600 in Condor version 7.5.0.
  This produced longer than desired
  cached sessions for short-lived tools such as \Condor{status}.
  It also produced shorter than possibly desired cached sessions for
  long-lived daemons such as \Condor{startd}.  
  The default has been restored to what it was before Condor version 7.5.0,
  with the exception of \Condor{submit},
  which has been changed from 1 hour to 60 seconds.
  For command line tools, the default is 60 seconds,
  and for daemons it is 1 day.

% gittrac #1142
\item \MacroNI{SEC\_<access-level>\_SESSION\_DURATION} previously did
  not support integer expressions, but it did not detect invalid
  input, so the use of an expression could produce unexpected results.
  Now, like other integer configuration variables,
  a constant expression can be used and input is fully validated.

% gittrac #1196
\item The configuration variable \MacroNI{LOCAL\_CONFIG\_DIR} is no longer
ignored if defined in a local configuration file.

% gittrac #767
\item Removed the incorrect issuing of the following Condor version 7.5.0 
  warning to the
  \Condor{starter}'s log, even when the outdated, and no longer used
  configuration
  variable \MacroNI{EXECUTE\_LOGIN\_IS\_DEDICATED} was not defined.

\begin{verbatim}
WARNING: EXECUTE_LOGIN_IS_DEDICATED is deprecated.
Please use DEDICATED_EXECUTE_ACCOUNT_REGEXP instead.
\end{verbatim}


\end{itemize}

\noindent Known Bugs:

\begin{itemize}

\item None.

\end{itemize}

\noindent Additions and Changes to the Manual:

\begin{itemize}

\item None.

\end{itemize}


%%%%%%%%%%%%%%%%%%%%%%%%%%%%%%%%%%%%%%%%%%%%%%%%%%%%%%%%%%%%%%%%%%%%%%
\subsection*{\label{sec:New-7-5-0}Version 7.5.0}
%%%%%%%%%%%%%%%%%%%%%%%%%%%%%%%%%%%%%%%%%%%%%%%%%%%%%%%%%%%%%%%%%%%%%%

\noindent Release Notes:

\begin{itemize}

\item All bug fixes and features which are in 7.4.1 are in this 7.5.0 release.

\end{itemize}


\noindent New Features:

\begin{itemize}

% gittrac #892
\item Added the new daemon \Condor{shared\_port} for Unix platforms 
  (except for HPUX).
  It allows Condor daemons to share a
  single network port.  This makes opening access to Condor through a
  firewall easier and safer.  It also increases the scalability of a
  submit node by decreasing port usage. See
  section~\ref{sec:Config-shared-port} for more information.

% gittrac #960
\item Improved CCB's handling of rude NAT/firewalls that silently drop
TCP connections.

% gittrac #968
\item Simplified the publication of daemon addresses.
  \Attr{PublicNetworkIpAddr} and \Attr{PrivateNetworkIpAddr} have been removed.
  \Attr{MyAddress} contains both public and private addresses.  For now,
  \Attr{<Subsys>IpAddr} contains the same information.  In a future release,
  the latter may be removed.

% gittrac #975
\item Changes to \MacroNI{TCP\_FORWARDING\_HOST},
  \MacroNI{PRIVATE\_NETWORK\_ADDRESS}, and
  \MacroNI{PRIVATE\_NETWORK\_NAME} can now be made without requiring a
  full restart.  It may take up to one \Condor{collector} update interval 
  for the changes to become visible.

% gittrac #1002
\item Network compatibility with Condor prior to 6.3.3 is no longer
  supported unless \MacroNI{SEC\_CLIENT\_NEGOTIATION} is set to
  \Expr{NEVER}.  This change removes the risk of communication errors
  causing performance problems resulting from automatic fall-back to the
  old protocol.

% gittrac #930
\item For efficiency, authentication between the \Condor{shadow} and
  \Condor{schedd} daemons is now able to be cached and reused in more
  cases.  Previously, authentication for updating job information was
  only cached if read access was configured to require authentication.

\item \Condor{config\_val} will now report the default value for
  configuration variables that are not set in the configuration files.

% gittrac #939
\item The \Condor{gridmanager} now uses a single status call to obtain
the status of all CREAM grid universe jobs from the remote server.

% gittrac #955
\item The \Condor{gridmanager} will now retry CREAM commands that time out.

% gittrac #941
\item Forwarding a renewed proxy for CREAM grid universe jobs to the
remote server is now much more efficient.

\end{itemize}

\noindent Configuration Variable and ClassAd Attribute Additions and Changes:

\begin{itemize}

% gittrac #997
\item Removed the configuration variable 
  \MacroNI{COLLECTOR\_SOCKET\_CACHE\_SIZE}.
  Configuration of this parameter used to be mandatory to enable TCP updates
  to the \Condor{collector}.  Now no special configuration of the
  \Condor{collector} is required to allow TCP updates, but it is
  important to ensure that there are sufficient file descriptors for
  efficient operation.  See section~\ref{sec:tcp-collector-update} for
  more information.

% gittrac #892
\item The new configuration variable \MacroNI{USE\_SHARED\_PORT} 
  is a boolean value that specifies
  whether a Condor process should rely on the \Condor{shared\_port} daemon for
  receiving incoming connections.  Write access to
  \Macro{DAEMON\_SOCKET\_DIR} is required for this to take effect.
  The default is \Expr{False}.  If set to \Expr{True}, \MacroNI{SHARED\_PORT}
  should be added to \MacroNI{DAEMON\_LIST}.  See
  section~\ref{sec:Config-shared-port} for more information.

% gittrac #960
\item Added the new configuration variable \MacroNI{CCB\_HEARTBEAT\_INTERVAL}.
  It is the maximum
  number of seconds of silence on a daemon's connection to the CCB server
  after which it will ping the server to verify that the connection still
  works.  
  The default value is 1200 (20 minutes).
  This feature serves to both speed
  up detection of dead connections and to generate a guaranteed minimum
  frequency of activity to attempt to prevent the connection from being
  dropped.

\end{itemize}

\noindent Bugs Fixed:

\begin{itemize}

\item Fixed problem with a ClassAd debug function,
so it now properly emits debug information for ClassAd \Code{IfThenElse}
clauses.

\end{itemize}

\noindent Known Bugs:

\begin{itemize}

\item None.

\end{itemize}

\noindent Additions and Changes to the Manual:

\begin{itemize}

\item None.

\end{itemize}

%%%%      PLEASE RUN A SPELL CHECKER BEFORE COMMITTING YOUR CHANGES!
%%%      PLEASE RUN A SPELL CHECKER BEFORE COMMITTING YOUR CHANGES!
%%%      PLEASE RUN A SPELL CHECKER BEFORE COMMITTING YOUR CHANGES!
%%%      PLEASE RUN A SPELL CHECKER BEFORE COMMITTING YOUR CHANGES!
%%%      PLEASE RUN A SPELL CHECKER BEFORE COMMITTING YOUR CHANGES!

%%%%%%%%%%%%%%%%%%%%%%%%%%%%%%%%%%%%%%%%%%%%%%%%%%%%%%%%%%%%%%%%%%%%%%
\section{\label{sec:History-7-4}Stable Release Series 7.4}
%%%%%%%%%%%%%%%%%%%%%%%%%%%%%%%%%%%%%%%%%%%%%%%%%%%%%%%%%%%%%%%%%%%%%%

This is a stable release series of Condor.
As usual, only bug fixes (and potentially, ports to new platforms)
will be provided in future 7.4.x releases.
New features will be added in the 7.5.x development series.

The details of each version are described below.

%%%%%%%%%%%%%%%%%%%%%%%%%%%%%%%%%%%%%%%%%%%%%%%%%%%%%%%%%%%%%%%%%%%%%%
\subsection*{\label{sec:New-7-4-5}Version 7.4.5}
%%%%%%%%%%%%%%%%%%%%%%%%%%%%%%%%%%%%%%%%%%%%%%%%%%%%%%%%%%%%%%%%%%%%%%

\noindent Release Notes:

\begin{itemize}

\item Condor version 7.4.5 not yet released.
%\item Condor version 7.4.5 released on Month Date, 2010.

\end{itemize}


\noindent New Features:

\begin{itemize}

% gittrac #1713
\item \Condor{dagman} now prints a message in the \File{dagman.out} file
whenever it truncates a node job user log file.

% gittrac #1715
\item \Condor{dagman} now prints additional diagnostic information in the
case of certain log file errors.

\end{itemize}

\noindent Configuration Variable and ClassAd Attribute Additions and Changes:

\begin{itemize}

\item None.

\end{itemize}

\noindent Bugs Fixed:

\begin{itemize}

% gittrac #1750
\item A network disconnect between the submit machine and execute
  machine during the transfer of output files caused the
  \Condor{starter} daemon to immediately give up, rather than waiting
  for the \Condor{shadow} to reconnect.  This problem was introduced
  in Condor version 7.4.4.

% gittrac #1743
\item If \Condor{ssh\_to\_job} attempted to connect to a job while the
  job's input files were being transferred, this caused the file
  transfer to fail, which resulted in the job returning to the idle
  state in the queue.

% gittrac #1785
\item In privsep mode, the transfer of output failed if a job's execute
  directory contained symbolic links to non-existent paths.

\end{itemize}

\noindent Known Bugs:

\begin{itemize}

\item None.

\end{itemize}

\noindent Additions and Changes to the Manual:

\begin{itemize}

\item None.

\end{itemize}


%%%%%%%%%%%%%%%%%%%%%%%%%%%%%%%%%%%%%%%%%%%%%%%%%%%%%%%%%%%%%%%%%%%%%%
\subsection*{\label{sec:New-7-4-4}Version 7.4.4}
%%%%%%%%%%%%%%%%%%%%%%%%%%%%%%%%%%%%%%%%%%%%%%%%%%%%%%%%%%%%%%%%%%%%%%

\noindent Release Notes:

\begin{itemize}

\item Condor version 7.4.4 released on October 18, 2010.

% gittrac #1508
\item \Security 
This release fixes a bug in which Amazon EC2 jobs
(jobs with \SubmitCmd{universe = grid} and \SubmitCmd{grid\_resource = amazon})
that use the \SubmitCmd{amazon\_keypair\_file}
command may expose the private SSH key to other users.
The created file had insecure permissions,
allowing other users on the submit host to read the file.
Any other user who could see the file could learn about these EC2 jobs
using \Condor{q}, 
and the other user could then connect to them with the private SSH key.

To work around the bug without installing this release,
do one or both of the following:
\begin{itemize}
\item Do not use the submit description file command
\SubmitCmd{amazon\_keypair\_file}.
\item Ensure that the directory holding the private SSH key 
has suitably restrictive permissions,
such that other users cannot read files inside the directory.
\end{itemize}


% gittrac #1524
\item Condor can now be built on Mac OS X 10.6.

% gittrac #1696
\item The \Condor{master} shutdown program, which is configured via 
  the \Macro{MASTER\_SHUTDOWN\_$<$Name$>$} configuration variable,
  is now run with root (Unix) or administrator (Windows) privileges.
  The adminstrator must ensure
  that this cannot be used in such a way as to violate system integrity.

\end{itemize}


\noindent New Features:

\begin{itemize}

\item \SubmitCmd{load\_profile} is now supported by the Unix version of
\Condor{submit} when submitting jobs to Windows.  Previously, this command
was only supported by the Windows version of \Condor{submit}.

\item Added an example Mac OS X launchd configuration file for starting Condor.

\end{itemize}

\noindent Configuration Variable and ClassAd Attribute Additions and Changes:

\begin{itemize}

\item None.

\end{itemize}

\noindent Bugs Fixed:

\begin{itemize}

% gittrac 1434
\item Fixed bad behavior in \Condor{quill} where, under certain error
conditions, many copies of the \File{schedd\_sql.log} file would be
inserted into the database, filling up the disk volume used by the
database. As a consequence of this bug fix, the \verb@LogBody@ column
for each row in the \verb@Error_SqlLogs@ table will be empty. Please
consult the \Condor{quill} daemon log file for the error instead.

% gittrac 1654
\item Fixed a bug with how the \SubmitCmd{standard} universe 
remote system call \Syscall{getrlimit} functioned.
Under certain conditions with
32-bit and 64-bit \SubmitCmd{standard} universe binaries,
\Syscall{getrlimit} would erroneously report 2147483647 bytes as a limit,
when \Expr{RLIM\_INFINITY} should have been the correct response.

% gittrac 1631
\item Fixed a misleading error message issued by \Condor{run},
which stated
\begin{verbatim}
The DAGMan job was aborted by the user.
\end{verbatim}
when the job submitted by \Condor{run} was aborted by the user.
It now states 
\begin{verbatim}
The job was aborted by the user.
\end{verbatim}

% gittrac 1543
\item When the \Condor{startd} daemon is running with an execute directory on
a very large file system, with more than 32 bits worth of free blocks
on a 32-bit system, it would incorrectly report 0 free bytes.  This
has been fixed.

\item For spooled jobs, input files were sometimes transferred twice from
the submit machine to the execute machine.  This happened if the input files
were specified without any path information,
as with a file name with no directory specified.
This problem has existed since before Condor version 7.0.0.

% gittrac 457
\item The configuration variable \MacroNI{NETWORK\_INTERFACE} did not
work in some situations, because of Condor's attempts to
automatically rewrite published addresses to match the IP address of
the network interface used to make the publication.

% gittrac 961
\item Fixed a bug in which the default unit of configuration variable
\MacroNI{STARTD\_CRON\_TEST\_PERIOD}
should have been seconds, but instead was \Expr{Undefined}.

% gittrac 1485
\item Fixed a bug in which \Condor{submit} checked for bad \Condor{schedd} cron 
 arguments incorrectly within a submit description file.
 Now \Condor{submit} will detect the problem and print out an error message.

% gittrac 1565
\item With some versions of \Prog{ssh}, \Condor{ssh\_to\_job} failed if
the \Env{SHELL} environment variable was set to \Prog{/bin/csh}.

% gittrac 1567
\item Submission of \SubmitCmd{vm} universe jobs via Globus was not possible,
because the Globus Condor jobmanager explicitly set the input, output,
and error files to \File{/dev/null},
and \Condor{submit} refused any setting of these files for
\SubmitCmd{vm} universe jobs.  
Now, \File{/dev/null} is an allowed setting for the input, output,
and error files for \SubmitCmd{vm} universe jobs.

% gittrac #1564
\item Fixed a bug that caused a \SubmitCmd{vm} universe job's output files
to be incorrectly transferred back to the submit machine, 
when the submit description file command \SubmitCmd{vm\_no\_output\_vm}
was set to \Expr{false},
indicating that no files should be transferred.

% gittrac #416
\item String literals within \verb@$$([])@ expressions within a submit
description file failed to be evaluated and resulted in the job going on hold.
This problem has existed since before Condor 7.0.0.

% gittrac #106
\item \Condor{preen} was not able to clean up files in the \MacroNI{EXECUTE}
directory when in privsep mode.

% gittrac #1589
\item A problem was fixed that could cause a Condor daemon that
  connects to other daemons via CCB to permanently run out of space
  for more registered sockets until restarted.  This problem appeared
  in the logs as the following message:

\begin{verbatim}
file descriptor safety level exceeded
\end{verbatim}

% gittrac #1596
\item Fixed a problem that could cause the \Condor{collector} to crash
when receiving updated matchmaking information for offline ClassAds that do
not exist.

% gittrac #1518
\item \Condor{ssh\_to\_job} did not work when
\MacroNI{SEC\_DEFAULT\_NEGOTIATION} was set to \MacroNI{OPTIONAL}.

% gittrac #1611 #1612
\item The \SubmitCmd{vm} universe now works properly on machines that 
have Condor's Privilege Separation mechanism enabled.

% gittrac #1624
\item \Condor{submit} no longer pads a \SubmitCmd{vm} universe job's disk usage
estimation by 100MB.

% gittrac #1553
\item Fixed a bug with the \Macro{vm\_cdrom\_files} submit file
command, that caused VMware \SubmitCmd{vm} universe jobs to fail if the virtual
machine already had a CD-ROM image associated with it.

% gittrac #1465
\item Configuration variables \Macro{SOAP\_SSL\_CA\_DIR} and
\Macro{SOAP\_SSL\_CA\_FILE} are now properly used when authenticating
with Amazon EC2 servers.

% gittrac #1484
\item Fix a bug with the \Macro{<subsys>\_LOCK} configuration variable.
It could let daemons writing to the same daemon log overwrite each other's
entries and cause daemons to exit when the log is rotated.

% gittrac #1557
\item Fixed a bug that caused nordugrid jobs to fail if the
\SubmitCmd{grid\_resource} attribute included a port as part of the server
host name.

% gittrac #1672
\item Fixed a confusing error message mentioning
  \verb@LocalUserLog::logStarterError()@ when the \Condor{starter} fails to
  communicate with the \Condor{shadow} before the job has started.

% gittrac #1602
\item Fixed the event log and shadow log for standard universe jobs to 
identify the checkpoint server on which a job might have failed to store 
its checkpoint or from which it might have failed to restore its checkpoint.

\item Fixed a bug in the \Condor{gridmanager} that could cause it to crash
while handling grid-type cream jobs.

% gittrac #1699
% gittrac #1700
\item Improved the \Condor{gridmanager}'s handling of grid-type cream jobs
that are held or removed by the user. Canceling the cream job is much less
likely to fail and jobs can no longer get stuck in the cream state of
CANCELED.

% gittrac #1701
\item Fixed the web server feature controlled by \Macro{ENABLE\_WEB\_SERVER}.
Previously, all HTTP GET requests would fail on non-linux Unix machines.

\end{itemize}

\noindent Known Bugs:

\begin{itemize}

\item None.

\end{itemize}

\noindent Additions and Changes to the Manual:

\begin{itemize}

\item The Windows platform installation instructions have been updated.

\item Section~\ref{sec:file-transfer} on Condor's File Transfer Mechanism
has been revised and updated.

\item Section~\ref{classad-query-examples}, providing examples of utilizing
ClassAd expressions within the \Opt{-constraint} option of \Condor{q}
or \Condor{status} commands has been expanded to clarify both
Unix and Windows platform specifics.

\end{itemize}


%%%%%%%%%%%%%%%%%%%%%%%%%%%%%%%%%%%%%%%%%%%%%%%%%%%%%%%%%%%%%%%%%%%%%%
\subsection*{\label{sec:New-7-4-3}Version 7.4.3}
%%%%%%%%%%%%%%%%%%%%%%%%%%%%%%%%%%%%%%%%%%%%%%%%%%%%%%%%%%%%%%%%%%%%%%

\noindent Release Notes:

\begin{itemize}

\item Condor version 7.4.3 released on August 16, 2010.

\end{itemize}


\noindent New Features:

\begin{itemize}

\item None.

\end{itemize}

\noindent Configuration Variable and ClassAd Attribute Additions and Changes:

\begin{itemize}

\item The new configuration variable \Macro{ENABLE\_CHIRP} 
defaults to \Expr{True}. 
An administrator may set it to \Expr{False}, which 
disables Chirp remote file access from execute machines.

\item The new configuration variable
  \Macro{ENABLE\_ADDRESS\_REWRITING} defaults to \Expr{True}.  It may
  be set to \Expr{False} to disable Condor's dynamic algorithm for choosing
  which IP address to publish in multi-homed environments.  The dynamic
  algorithm chooses the IP address associated with the network interface
  used to make the publication, for example, the network interface used 
  to communicate with the \Condor{collector}.

% gittrac 1407
\item Configuration variable \Macro{VM\_BRIDGE\_SCRIPT} has been removed
  and is no longer valid.

% gittrac 1402 and 1407
\item The new configuration variable
  \Macro{VM\_NETWORKING\_BRIDGE\_INTERFACE} specifies the networking interface
  that Xen or KVM \SubmitCmd{vm} universe jobs will use.
  See section~\ref{param:VMNetworkingBridgeInterface} for documentation.

% gittrac #1333
\item
Allowed the configuration file entries \MacroNI{GSI\_DAEMON\_TRUSTED\_CA\_DIR}
and \MacroNI{GSI\_DAEMON\_DIRECTORY} to be set with environment variables,
as the rest of Condor configuration variables can be.

\end{itemize}

\noindent Bugs Fixed:

\begin{itemize}

\item
When using file transfer semantics,
if the job exited in such a manner so as to not produce all
output files specified in \SubmitCmd{transfer\_output\_files},
then which files were transferred was potentially incorrect.
Now, all existing files are transferred back,
and the files which are not able to be transferred back due to non-existence
appear as zero length files.
An example of when this occurred would be the job dumping core
and then being placed on hold.

% gittrac 1185
\item
Fetch work hooks to prepare are now invoked as the \Login{condor} user,
instead of as the job user.

\item
Improved the file extension detection on Windows platforms.

\item
\Condor{wait} could occasionally get stuck in an infinite loop,
if it missed the execution event of the job it is waiting for.
This is now fixed.

% gittrac 1413
\item
Fixed a bug within the \Condor{startd} cron capabilities,
that only occurred on Windows platforms.
\Condor{startd} cron scripts failed to run if an initial directory was left
unspecified.

% gittrac 1012
\item
Fixed a bug in which a job would be incorrectly placed on hold, with 
a confusing error message that appeared similar to
\footnotesize
\begin{verbatim}
Condor failed to start your job 9090.-1 because job attribute Args contains $$(VirtualMachineID).
\end{verbatim}
\normalsize
This occurred if the submit command \SubmitCmd{copy\_to\_spool} 
was \Expr{true},
the submit description file for the job contained \$\$ macros,
and \Condor{preen} ran after the job was submitted and before it started.

% gittrac 1427
\item
Added the jobs\_vertical\_history table to the list of tables that
\Condor{quill} periodically re-indexes.

\item
Fixed bug in \Condor{preen} in which it would delete \Condor{startd} daemon
history files.

% gittrac 487
\item
  Fixed a bug where if a user job using file transfer with
  \SubmitCmd{transfer\_output\_files} and \SubmitCmd{when\_to\_transfer\_output}
  is set to \Expr{ON\_EXIT\_OR\_EVICT} fails
  to produce all of the specified files and exit, as when core
  dumping due to a fault, then the stdout, stderr, and core file of the
  job were not transferred back to the submitting machine.

\item
  Fixed numerous, small, rare memory leaks.

\item 
  Fixed a bug in which processor affinity settings were ignored if
  privilege separation was enabled.

% gittrac 1329
\item Network connections for Condor file transfers were ignoring
  private network settings.  The connection from the execute node to
  the submit node always attempted to use the public network address
  of the submit machine.

% gittrac 1405
\item The configuration variable \MacroNI{TCP\_FORWARDING\_HOST} did not work
in some situations
because of Condor's attempts to automatically rewrite published addresses to
match the IP address of the network interface used to make the publication.

% gittrac 1346
\item A single job could match multiple offline slots in a single
negotiation cycle.  This problem could cause \Condor{rooster} to
wake up too many offline machines for the number of jobs available
to run on them.  The fix for this problem requires updating both
the \Condor{negotiator} and the \Condor{schedd}.

% gittrac 1349
\item Fixed a problem that caused the \Condor{startd} daemon to
crash in some cases when \MacroNI{STARTD\_SENDS\_ALIVES} was \Expr{True}.
This setting is \Expr{False} by default.

% gittrac #1337
\item Fixed a problem where the \Condor{kbdd} has a chance of
entering an infinite loop on platforms that use X-Windows.
Microsoft Windows and Mac OS X platforms were not affected.  This bug is
present in all earlier 7.4.x Condor releases.

\item The default path to \Prog{sftp-server} has been improved,
 so that \Condor{ssh\_to\_job} can use \Prog{sftp} out-of-the-box on 
RedHat Enterprise Linux 5 platforms.

% gittrac #1383
\item A \Prog{nordugrid\_gahp} binary built on RedHat Enterprise Linux 3
no longer crashes
when run on a RedHat Enterprise Linux 4 or Scientific Linux 4 machine.

% gittrac #1418
\item Fixed a bug in \Condor{rm} that caused it to misinterpret user names
that begin with a digit, such as \Expr{4abc}.
It incorrectly used them as cluster numbers. 

% gittrac #1423
\item Fixed a bug that caused the \Condor{startd} to invoke the
  ``power\_state'' plug-in as the condor user.  This caused
  hibernation to fail because power\_state requires root privileges
  to function properly.

% gittrac #1330
\item Fixed a bug that could cause the \Condor{schedd} to crash if there
were any idle scheduler universe jobs when files were staged into the
\Condor{schedd} for a new job.

% gittrac #1404
\item Fixed a bug in the \Prog{nordugrid\_gahp} that could cause it to exit
when connecting to a misconfigured LDAP server.

% gittrac #1352
\item Fixed a bug that prevented the log file defined with the configuration
variable \Macro{NEGOTIATOR\_MATCH\_LOG} from rotating.
See section~\ref{param:SubsysLevelLog} for the definition of this variable. 

% gittrac #1413
\item Fixed a bug that caused \Prog{startd\_cron} jobs to fail on Windows. 
This bug is present in all earlier 7.4.x Condor releases.

% gittrac #1551
\item The submit description file command \SubmitCmd{vm\_cdrom\_files}
now works properly with Windows execute machines. 
Previously, creation of the ISO file would fail, 
causing job execution to be aborted.

% gittrac #1423
\item Fixed a bug that caused the \Condor{startd} to invoke the
  \Prog{power\_state} plug-in as the condor user.
  This caused hibernation to fail, 
  because \Prog{power\_state} requires root privileges to function properly.

\end{itemize}

\noindent Known Bugs:

\begin{itemize}

\item None.

\end{itemize}

\noindent Additions and Changes to the Manual:

\begin{itemize}

\item Searching the PDF version of the manual for items containing 
underscore characters, such as many configuration variable names,
now works correctly.

% gittrac #1340
\item The new subsection~\ref{ClassAd:examples} provides examples of
evaluation results when using the operators \Expr{==}, \Expr{=?=},
\Expr{!=}, and \Expr{=!=}.

\item Section~\ref{sec:vmuniverse} with specifics on \SubmitCmd{vm}
universe jobs has been updated to contain more details about
both checkpoints and \SubmitCmd{vm} universe jobs in general.

\end{itemize}


%%%%%%%%%%%%%%%%%%%%%%%%%%%%%%%%%%%%%%%%%%%%%%%%%%%%%%%%%%%%%%%%%%%%%%
\subsection*{\label{sec:New-7-4-2}Version 7.4.2}
%%%%%%%%%%%%%%%%%%%%%%%%%%%%%%%%%%%%%%%%%%%%%%%%%%%%%%%%%%%%%%%%%%%%%%

\noindent Release Notes:

\begin{itemize}

\item Condor version 7.4.2 released on April 6, 2010.

\end{itemize}


\noindent New Features:

\begin{itemize}

\item None.

\end{itemize}

\noindent Configuration Variable and ClassAd Attribute Additions and Changes:

\begin{itemize}

% gittrac #1001
\item When \MacroNI{WANT\_SUSPEND} is defined and evaluates to anything
other than the value \Expr{True},
it is now utilized as if it were \Expr{False}.
If \MacroNI{WANT\_SUSPEND} is not explicitly defined,
the \Condor{startd} exits with an error message.
Previously, if \Expr{Undefined}, it was treated as an error,
which caused the \Condor{startd} to exit with an error message.

\end{itemize}

\noindent Bugs Fixed:

\begin{itemize}

% gittrac 1217
\item Fixed a bug in which the \Condor{schedd} would sometimes negotiate
  for and try to run
  more jobs than specified by \MacroNI{MAX\_RUNNING\_JOBS}.  Once the
  jobs started running, it would then kill them off to get back below
  the limit.  This was more likely to happen with slow preemption
  caused by \MacroNI{MaxJobRetirementTime} or by a large timeout
  imposed by \MacroNI{KILL}.  This problem has existed since before
  Condor 6.5.  When this problem happened, the following message
  appeared in the \Condor{schedd} log:

\begin{verbatim}
Preempting X jobs due to MAX_JOBS_RUNNING change
\end{verbatim}

% gittrac 1250
\item Fixed a problem that caused \Condor{ssh\_to\_job} to fail to connect
to a job running on a slot with multiple '@' signs in its name.  This bug
has existed since the introduction of \Condor{ssh\_to\_job} in 7.3.2.

% gittrac 116
\item In all previous versions of Condor, \Condor{status} refused to
  accept \Opt{-long}, \Opt{-xml}, and \Opt{-format} when followed by
  an argument such as \Opt{-master} that specified which type of
  daemon to look at.  The order of the arguments had to be reversed or
  it would produce a message such as the following:

\begin{verbatim}
Error:  arg 4 (-master) contradicts arg 1 (-format)
\end{verbatim}

% gittrac #1201
\item Fixed a bug which caused the \Condor{master} to crash if
\MacroNI{VIEW\_SERVER} was included in \MacroNI{DAEMON\_LIST} and
\MacroNI{CONDOR\_VIEW\_HOST} was unset.

% gittrac #1196
\item Fixed a bug that caused configuration parameter
\MacroNI{LOCAL\_CONFIG\_DIR} to be ignored if it was set in a local
configuration file, as opposed to the top-level configuration file.

% gittrac #1202
\item Fixed a bug that could cause the \Condor{schedd} to behave
incorrectly when reading an invalid job queue log on startup.

% gittrac #1215
\item Fixed a bug that could corrupt the job queue log
if the \Condor{schedd} daemon's attempt to compact it fails.

% gittrac #1256
\item Fixed a problem that in rare cases caused the \Condor{schedd} to
crash shortly after the \Condor{gridmanager} exited.
This bug has existed since before Condor version 6.8.

% gittrac #1270
\item Fixed a problem that was resulting in messages such as the following:

\footnotesize
\begin{verbatim}
ERROR: receiving new UDP message but found a long message still waiting
to be closed (consumed=0). Closing it now.
\end{verbatim}
\normalsize

\item The file extension specified to \Condor{fetch\_log} can no longer
contain a path delimiter.

% gittrac 1299
\item When in graceful shutdown mode, the \Condor{schedd} was
  sometimes starting idle scheduler universe jobs.  With a large
  enough number of scheduler universe jobs, this could lead to a cycle
  of stopping and restarting jobs until the graceful shutdown time
  expired.

% gittrac #1259
\item Fixed multiple bugs that prevented Condor from building on or
  running correctly on OpenSolaris X86/64 version 2009.06.

% gittrac #1238
\item Fixed a bug which caused the \Condor{startd} to incorrectly
  count the number of processors on some machines with
  Hyper-threading enabled.  This bug was introduced in
  Condor version 7.3.2, and exists in 7.4.0 and 7.4.1.

% gittrac #1167
\item Fixed a problem with GSI authentication in Condor that would cause
daemons to consume more and more memory over time.  The biggest source
of trouble was introduced in Condor version 7.3.2.
However, a smaller memory leak that
existed in all previous versions of Condor has also been fixed.

% gitrack #553
\item Fixed a bug where if \Condor{compile} is invoked in a manner such as:
\begin{verbatim}
  condor_compile gcc -print-prog-name=ld 
\end{verbatim}
an error would be emitted,
and \Condor{compile} would exit with a bad exit code.

% gittrac #1093
\item The sort based on \Condor{status} output accidentally changed in 
Condor version 7.3,
so that the output was based on the slot name first, then machine name.
The behavior is now restored to the original sorting: first on machine name,
then slot name.

% gittrac #728
\item If one machine running a parallel job crashed,
and job leases are enabled (which they are by default),
the job would not exit until the job lease duration expired.
As the \Condor{starter} will not get respawned,
there is no need to wait.
Many sites set long job lease durations,
to prevent jobs from being killed when the machine running
the \Condor{schedd} daemon reboots.
Now, if one node goes away, the whole computation is shut down immediately.

\item Fixed the verbosity level of some \Condor{dagman} messages written to
the \File{dagman.out} file.

% gittrac #1137
\item Fixed a bug introduced in Condor version 7.3.2 that resulted in
  messages such as the following even in cases where no problem in
  communicating with the \Condor{collector} had been encountered:

\begin{verbatim}
Collector <X> is still being avoided if an alternative succeeds.
\end{verbatim}

This problem was believed to be fixed in Condor 7.4.1, but some cases
of the problem remained in that version.

% gittrac 1160
\item Fixed a bug from Condor version 6.1.14,
that resulted in the \Condor{schedd} performing
the operation scheduled via \MacroNI{WALL\_CLOCK\_CKPT\_INTERVAL} at the
specified frequency (default time of 1 hour),
multiplied by the number of times the
\Condor{schedd} daemon had been reconfigured during its lifetime.
This could lead to degraded performance,
especially prior to Condor version 7.4.1,
when this operation was more disk-intensive.

% gittrac 1184
% gittrac 1181
\item 32-bit Linux versions of Condor running in a 64-bit environment would
sometimes not detect the existence of some processes and sometimes
wrongly detect that a tracked process belonged to root when it
actually belonged to some other user.  This could lead to failure to run
jobs or failure to properly monitor and clean up after them.  When the wrong
process ownership problem happened,
the following message appeared in the \Condor{master} and/or \Condor{procd}
logs:

\begin{verbatim}
ProcAPI: fstat failed in /proc! (errno=75)
\end{verbatim}

If \Condor{procd} failed to detect the existence of its own parent process,
it would exit with the following message in its log:

\begin{verbatim}
ERROR: master has exited
\end{verbatim}

% gittrac 1186
\item Fixed a problem in the \Condor{job\_router} daemon,
  introduced in Condor version 7.2.2,
  that could cause the daemon to crash when failing to carry out the change
  of state dictated by a job's periodic policy expressions,
  for example, the failure to put a job on hold when \AdAttr{periodic\_hold}
  becomes \Expr{True}.

% gittrac #1209
\item Fixed a bug introduced in Condor 7.3.2 that caused Grid Monitor
jobs to receive a full X.509 proxy. Now, it always receives a limited
proxy, which was the previous behavior.

% gittrac #1070
\item Fixed a bug that could cause the nordugrid\_gahp to crash.

% gittrac #742
\item Fixed a problem introduced in 7.4.0 that could cause two 
  \Condor{schedd} daemons
  with a match to the same slot to both fail to claim it, rather than
  letting the first one to claim it succeed.  This sort of situation
  can happen when the \Condor{negotiator} has a stale view of the pool,
  either because the gap between negotiation cycles is configured to
  be shorter than usual, or because updates from the \Condor{startd}
  to the \Condor{collector}
  are not reliably delivered and processed.

% gittrac #1251
\item The \Condor{kbdd} is no longer ignored by the \Condor{startd}
when the configuration variable \Macro{CONSOLE\_DEVICES} is defined.

% gittrac #92
\item When using the \Opt{-d} command line argument with a daemon,
the values of \MacroNI{LOG}, \MacroNI{SPOOL}, and \MacroNI{EXECUTE}
no longer change every time a \Condor{reconfig} command is received.

\end{itemize}

\noindent Known Bugs:

\begin{itemize}

% gittrac #1337
\item The \Condor{kbdd} has a chance of entering an infinite loop
on platforms that use X-Windows.  Microsoft Windows and Mac OS X
are not affected.  Removing KBDD from \MacroNI{DAEMON\_LIST} is a
workaround, although this impairs Condor's ability to detect
console usage.  This bug is fixed in Condor version 7.4.3.

\end{itemize}

\noindent Additions and Changes to the Manual:

\begin{itemize}

\item Descriptions of all the commands that may be placed into a
submit description file are now located within the \Condor{submit}
manual page, instead of within Chapter 2, the Users' Manual.

\item An initial, but not yet complete set of configuration variables
that require a restart when changed,
is listed in section~\ref{sec:Macros-Requiring-Restart}.
Using \Condor{reconfig} to change these variables' values is not sufficient.

\end{itemize}


%%%%%%%%%%%%%%%%%%%%%%%%%%%%%%%%%%%%%%%%%%%%%%%%%%%%%%%%%%%%%%%%%%%%%%
\subsection*{\label{sec:New-7-4-1}Version 7.4.1}
%%%%%%%%%%%%%%%%%%%%%%%%%%%%%%%%%%%%%%%%%%%%%%%%%%%%%%%%%%%%%%%%%%%%%%

\noindent Release Notes:

\begin{itemize}

% gittrac #1018
\item \Security A flaw was found that could allow a user who already is authorized to
submit jobs into Condor, to queue a job under the guise of  a different
user.  In this way, someone who has access to a Condor submission
service and is allowed to submit jobs into Condor could gain access to
another non-root or non-administrator account on the system.
This flaw was discovered during the development process; no incidents
have been reported.  Details of the problem will be made available on Feb 1st,
2010.

% gittrac #918
\item The default value of \MacroNI{JOB\_ROUTER\_NAME} has changed
  from an empty string to \verb|jobrouter| in order to address
  problems caused by the previous default.  Without special handling,
  this means that jobs being managed by \Condor{job\_router} before
  upgrading will not be adopted by the new version of
  \Condor{job\_router} if the default \MacroNI{JOB\_ROUTER\_NAME} was
  being used.  To correct this, follow the instructions given in the
  description of \MacroNI{JOB\_ROUTER\_NAME} on
  page~\pageref{JobRouterName}.

\end{itemize}


\noindent New Features:

\begin{itemize}

% gittrac #921, #999
\item Condor allows submit files to specify an \SubmitCmd{IwdFlushNFSCache}
expression,
to control whether or not Condor tries to flush the NFS cache for 
a job's initial working directory on job completion.

% gittrac #929, #943
\item The new \Opt{-attributes} option to \Condor{status}
  explicitly specifies the attributes to be listed when using the
  \Opt{-xml} or \Opt{-long} options.

\end{itemize}

\noindent Configuration Variable and ClassAd Attribute Additions and Changes:

\begin{itemize}

% gittrac #161, #935, #936
\item New VOMS attributes have been introduced into the job ad to keep them
separate from the X509UserProxySubjectName.

\item The default for \MacroNI{JOB\_ROUTER\_NAME} has changed from an
  empty string to \verb|jobrouter|.  See the release notes for more
  information about upgrading from an old version.

\item The configuration variable \Macro{TCP\_FORWARDING\_HOST}
  has existed in Condor since version 7.0.0, but was not documented.
  See section~\ref{param:TcpForwardingHost} for documentation.

% gittrac #933
\item The new configuration variable \MacroNI{STARTD\_PER\_JOB\_HISTORY\_DIR}
allows ClassAds of completed jobs to be stored in a directory separate 
from the existing one specified with \MacroNI{PER\_JOB\_HISTORY\_DIR}.

\end{itemize}

\noindent Bugs Fixed:

\begin{itemize}

% gittrac #749
\item  Condor no longer creates the job sandbox in its \MacroNI{SPOOL}
directory if it is not needed.

% gittrac #1019
\item Fixed a problem introduced in Condor version 7.4.0 that caused GSI
authentication between Condor processes to fail with using a
non-legacy format X.509 proxy.

% gittrac #1028
\item Fixed a problem with CCB under Windows platforms that has existed since
Condor version 7.3.0.  
This problem caused CCB-enabled daemons to become unresponsive
after the exit of a child process.

% gittrac #931 -- Fixed minor spelling errors, not worthy of listing.

% gittrac #923
\item Improved the handling of previously-submitted gt2 grid jobs upon
release from hold, when there is no Globus job manager for the job running
on the remote resource.

% gittrac #453
\item Fixed a problem with job leases for jobs that use a \Condor{shadow}.
Previously, while these jobs were running, lease renewals from the 
submitter would not be
noticed, and the job would be aborted when the original lease expired.

% gittrac #870
\item Fixed a bug that only allowed approximately 50 splices to be included into
a DAG input file. There is now no limit to the number of splices
one may include into a DAG input file except, of course, for the
implicit memory allocation limit of the \Condor{dagman} process.

% gittrac #909
\item Removed attempted limiting of swap space via \Prog{ulimit -v} using the
\Attr{VirtualMemory} machine ClassAd attribute in the script
\File{condor\_limits\_wrapper.sh}.

% gittrac #899
\item Fixed a bug that caused \MacroNI{ALLOW\_CONFIG} and
  \MacroNI{HOSTALLOW\_CONFIG}, as well as the corresponding
  \MacroNI{DENY} configuration variables to incorrectly handle a
  setting consisting of a single \Expr{*} or the equivalent \Expr{*/*}.  This
  also fixes a bug that caused incorrect merging of \MacroNI{ALLOW}
  and \MacroNI{HOSTALLOW} settings when one, but not both, consisted of
  a single \Expr{*} or the equivalent \Expr{*/*}.
  These bugs have existed since before Condor version 6.8.

% gittrac #905
\item Fixed a bug introduced in Condor version 7.3.0 that could cause 
Condor daemons to crash when reading malformed network addresses.

% gittrac #883
\item Removed a check for root ownership of a script specified by
the configuration variable \MacroNI{VM\_SCRIPT}.

% gittrac #884
\item Fixed a bug in writing the header of the file identified by
the configuration variable \MacroNI{EVENT\_LOG}.

% gittrac #891
\item Fixed a bug that could cause the \Condor{startd} to segfault on shutdown
when using dynamic slots.

% gittrac #871
\item Fixed a problem introduced in Condor version 7.3.2 that changed 
  the behavior of
  an undocumented method for selecting attributes to be displayed in
  \Condor{q} \Opt{-xml}.  Prior to this bug, the following command
  would produce XML output with the attributes \Attr{A} and \Attr{B},
  plus a few other attributes that were always shown.

\begin{verbatim}
condor_q -xml -format "%s" A+B
\end{verbatim}

In Condor versions 7.3.2 and 7.4.0,
this same command produced an empty XML ClassAd.
The workaround was to use multiple \Opt{-format} options, each listing
just one desired attribute, rather than a single one with an
expression of all desired attributes.  Although this is now fixed, the
more straightforward way to select attributes since Condor version 7.3.2
is to use the \Opt{-attributes} option.

% gittrac #907
\item Fixed a bug introduced in Condor version 7.3.2 that resulted in 
  messages such
  as the following even in cases where no problem in communicating
  with the \Condor{collector} had been encountered:

\begin{verbatim}
Collector <X> is still being avoided if an alternative succeeds.
\end{verbatim}

% gittrac #859
\item Fixed a bug that has been in the \Condor{startd} since before
  Condor version 6.8.  If the \Condor{startd} ever failed to send signals to the
  \Condor{starter} process, it could fail to properly clean up the
  machine ClassAd, leaving attributes from
  \MacroNI{STARTD\_JOB\_EXPRS} in the ClassAd but not making them visible
  in \Condor{status} queries.  One possible problem resulting from
  this could be matches being made by the \Condor{negotiator} that are then
  rejected by the \Condor{startd}.  Repeated messages such as the following
  would then result in the \Condor{startd} log:

\begin{verbatim}
slot1: Request to claim resource refused.
\end{verbatim}

%gittrac #908
\item Fixed a problem that resulted in the following message in the
  \Condor{startd} log:

\begin{verbatim}
Timer -1 not found
\end{verbatim}

%gittrac #937
\item Fixed a problem in which security sessions were not cached
  correctly when using CCB.  This resulted in re-authentication in
  some cases where a cached security session could have been used.

% gittrac #161, #935, #936, #1020
\item Fixed multiple problems with the handling of VOMS attributes in GSI
proxies.

% gittrac #934
\item Fixed a bug that caused \Condor{dagman} to hang when running a
DAG with POST scripts, if the global event log is turned on.

% gittrac #973
\item Improved how the private network address is published when using
  the configuration variables \MacroNI{PRIVATE\_NETWORK\_NAME} and
  \MacroNI{PRIVATE\_NETWORK\_INTERFACE}.  In some cases, this
  information was not being used, and therefore connections were made
  to the public address when they could have been made to the private
  address.

% gittrac #801
\item Fixed a bug exhibited under Windows XP,
where using \MacroNI{USE\_VISIBLE\_DESKTOP}
would cause strange behavior after a job completed.

% gittrac #713
\item CCB now works with \MacroNI{TCP\_FORWARDING\_HOST}.  Previously,
  the reverse connection was made to the private address rather than
  to the host defined by \MacroNI{TCP\_FORWARDING\_HOST}.

% gittrac #852
\item Removed a bad optimization that caused some information about job
execution to be lost during job completion or removal,
if a history file was not configured.

% gittrac #893
\item Condor now checks whether the configuration variable
\MacroNI{GRIDFTP\_URL\_BASE} is set before
submitting cream grid jobs, as that variable is required for cream jobs
to function properly. If the variable is not set, cream jobs are put on
hold with an appropriate message.

% gittrac #920
\item Fixed a bug that allowed running virtual machines to be leaked
if the \Condor{startd} crashed.

% gittrac #912
\item Fixed a bug in \Prog{cream\_gahp} which could cause crashes when
there were more than 500 cream jobs queued.

% gittrac #972
\item Improved recovery when Condor crashes during the submission of a cream
grid job. Before, affected jobs would remain in \Expr{REGISTERED} state
on the cream server, but never run.

% gittrac #954
\item Improved the \Attr{HoldReason} message when cream grid jobs are
held by the \Condor{gridmanager}.

% gittrac #895
\item When naming a resource for a cream grid job, Condor now properly
recognizes the format used by the standard cream client UI:
\File{https://foo.edu:8443/cream-pbs-cream\_queue}.

% gittrac #795
% The memory leak is not worth documenting.
\item The configuration variable \MacroNI{SOAP\_SSL\_CA\_FILE} is now 
consulted in addition to
\MacroNI{SOAP\_SSL\_CA\_DIR} when authenticating
an https proxy for Amazon EC2, when \MacroNI{AMAZON\_HTTP\_PROXY} is defined.

% gittrac #485
\item Previously, if \Condor{rm} and friends were given both a constraint
and a user name or cluster id, they would act on all jobs matching the
constraint and all jobs associated with the user or cluster. Now, this
combination of arguments results in an error.

% gittrac #1062
\item Failure to purge a cream grid universe job from the remote server
because it was previously purged no longer results in the job being held.

% gittrac #1044
\item The \Condor{gridmanager} now recognizes VOMS attributes in X.509
proxies and will handle them appropriately. For example, it recognizes
that two proxies with the same identity but different VOMS attributes may
be mapped to different accounts on a remote machine.

% gittrac #947
% Not documenting, as the parameter being removed was added for a specifc
% customer and never documented.

% gittrac #932
% Not documenting, as the bug hasn't caused any problems.

% gittrac #1043
% Not documenting, as the problem never made it into a release.

% gittrac #979
\item Fixed a bug in \Condor{dagman}, introduced in 7.3.2, that will
cause \Condor{dagman} running on Windows to hang on any DAG using
more than one log file for the node jobs.

% gittrac #967
\item Fixed a bug in \Condor{dagman}, introduced in 7.3.2, that could
cause \Condor{dagman} to fail on a DAG using node job log files on
multiple devices, if log files on different devices happened to have
the same inode number.

% gittrac #981
\item Fixed a bug that caused the \Condor{schedd} daemon to segfault when
spooling more than 9 files.

% gittrac #1011
\item Fixed a bug that caused the \Condor{startd} daemon to crash on
Debian Stable.

% gittrac #1033
\item Fixed keyboard activity detection on the Windows XP platform.

% gittrac #1068
\item Fixed a bug in the \Condor{had} daemon that caused it to not start
the controlled daemon if CCB was enabled.

\end{itemize}

\noindent Known Bugs:

\begin{itemize}

% gittrac #1337
\item The \Condor{kbdd} has a chance of entering an infinite loop
on platforms that use X-Windows.  Microsoft Windows and Mac OS X
are not affected.  Removing KBDD from \MacroNI{DAEMON\_LIST} is a
workaround, although this impairs Condor's ability to detect
console usage.  This bug is fixed in Condor version 7.4.3.

% gittrac #983
\item \Condor{dagman} may fail on Windows if the set of node job log
file names includes multiple paths that are hard links (not symbolic links)
to the same file.

% gittrac #1081
\item \Condor{dagman} PRE and POST script arguments (and the names of
the scripts themselves) cannot contain spaces.

% gittrac #1082
\item \Condor{dagman} VARS values cannot contain single quotes.

\end{itemize}

\noindent Additions and Changes to the Manual:

\begin{itemize}

% gittrac #725
\item Added documentation about how to include spaces (and other
special characters) in \Condor{dagman} VARS values.

\end{itemize}


%%%%%%%%%%%%%%%%%%%%%%%%%%%%%%%%%%%%%%%%%%%%%%%%%%%%%%%%%%%%%%%%%%%%%%
\subsection*{\label{sec:New-7-4-0}Version 7.4.0}
%%%%%%%%%%%%%%%%%%%%%%%%%%%%%%%%%%%%%%%%%%%%%%%%%%%%%%%%%%%%%%%%%%%%%%

\noindent Release Notes:

\begin{itemize}

\item The default configuration file within the release now uses
  \MacroNI{ALLOW}/\MacroNI{DENY} in place of
  \MacroNI{HOSTALLOW}/\MacroNI{HOSTDENY} for security related settings.
  We recommend making this
  same change throughout all configuration files.  That way,
  a policy that depends on the default policy should continue to
  work as it did before.  The behavior of these configuration variables
  remains unchanged.  The \MacroNI{ALLOW}/\MacroNI{DENY} lists are
  added to the \MacroNI{HOSTALLOW}/\MacroNI{HOSTDENY} lists to form the
  authorization policy.  Both styles support the same syntax.  
  This change permits an anticipated
  phasing out of the \MacroNI{HOSTALLOW}/\MacroNI{HOSTDENY}  configuration
  variables, in order to simplify configuration.

\item As of Condor version 7.3.2, \Condor{q} \Opt{-xml} output no longer 
  begins with the non-XML consisting of two blank lines followed by a line
  of the following form:

\begin{verbatim}
-- Submitter: schedd-name : <IP> : hostname
\end{verbatim}

\item All \Prog{Stork} data placement is now supported by the Stork
project at the 
LSU Center for Computation and Technology
(\URL{http://www.cct.lsu.edu/www.cct.lsu.edu}).
Please see the home page of the Stork project at
\URL{http://www.cct.lsu.edu/~kosar/stork/index.php} for details and
software.

\end{itemize}


\noindent New Features:

\begin{itemize}

\item Condor is now integrated with the Hadoop Distributed File System (HDFS). 
See documentation in section~\ref{sec:Condor-HDFS} and 
section~\ref{sec:HDFS-Config-File-Entries}.

% commit af65de7ccc1a281c2b05b8f68ac70bcfa56b2dd1
\item \Condor{q} using the options \Opt{-analyze} and \Opt{-better-analyze}
  now provide analysis for scheduler and local universe jobs.
  Specifically, the \MacroNI{START\_SCHEDULER\_UNIVERSE} and
  \MacroNI{START\_LOCAL\_UNIVERSE} expressions are checked.

% #824
\item Added the ClassAd attributes
\Attr{TotalLocalRunningJobs}, \Attr{TotalLocalIdleJobs},
\Attr{TotalSchedulerRunningJobs}, and \Attr{TotalSchedulerIdleJobs}
to the published ClassAd for the \Condor{schedd}.  This means that
\Condor{q} \Opt{-analyze} can still give helpful information about
why local or scheduler universe jobs are idle when
the configuration variables \MacroNI{START\_LOCAL\_UNIVERSE} or
\MacroNI{START\_SCHEDULER\_UNIVERSE} refer to these attributes.
These attributes were already present internally within 
the \Condor{schedd} daemon, 
just not published.

% #688
\item The \Condor{vm-gahp} now supports KVM and links with libvirt, rather 
than calling virsh command-line tools.

% #760 #771 #769 #772 #773 #775
\item Greatly improved the \Condor{gridmanager}'s scalability when handling
many grid type gt2 grid universe jobs.  Improvements include more quickly
processing updated X.509 certificates, not checking jobs for status updates if 
they have not been submitted to the remote site, and eliminating unnecessary 
updates to the \Condor{schedd} daemon.

% commit 75f6b2fe920b88717712a0d41765d16665ad7fe6
\item Latency in the submission and cleaning up of Condor-C jobs
has been improved by changing the default value of
\Macro{C\_GAHP\_CONTACT\_SCHEDD\_DELAY} from 20 to 5.

% commit 8c2d88c695d6981be3bdab7e10c5d92e9f6bb55b
\item The \Expr{eval()} ClassAd function added in Condor version 7.3.2
is now also understood by the \Condor{job\_router} and
\Condor{q} using the \Opt{-better-analyze} option.

\item The submit command \SubmitCmd{run\_as\_owner} is now supported
for Unix platforms. Previously, it was only supported on Windows platforms.

% #795
\item When setting \MacroNI{AMAZON\_HTTP\_PROXY}, a username and password
can now be given as part of the proxy URL.
The value of \MacroNI{SOAP\_SSL\_CA\_DIR} is now consulted when authenticating
an https proxy for Amazon EC2, when \MacroNI{AMAZON\_HTTP\_PROXY} is defined.

% #694
\item The \Condor{collector} daemon now advertises to itself, and will appear
in the output of \Condor{status} \Opt{-collector}.

% #775, cf02764d9d0fdd2b36ef3629f862f856ee41a717, and more
\item Optimizations in core Condor systems should provide minor speed 
improvements.

% 823
\item Updated the maximum log size to the maximum operating system's file size.

\end{itemize}

\noindent Configuration Variable and ClassAd Attribute Additions and Changes:

\begin{itemize}

% commit 0e8800c201f81eac54cba925b3d7f6d81a83aeca
\item The undocumented configuration variable 
  \Macro{TOOLS\_PROVIDE\_OLD\_MESSAGES} is no longer recognized by Condor.

% #768
\item The new configuration variable 
  \Macro{SCHEDD\_JOB\_QUEUE\_LOG\_FLUSH\_DELAY} sets an
  upper bound in seconds on how long it takes for changes to the job
  ClassAd to be visible to the Condor Job Router and to Quill.
  The default value is 5 seconds.
  Previously, there was no upper limit.  Typically, other activity in
  the job queue, such as jobs being submitted or completed would cause
  buffered data to be flushed to disk, such that the effective upper bound was
  a function of how busy the job queue was.

% commit 55525e0a338be8b2ba2d9173ce832e43d05413c3
\item The default configuration file now uses
  \MacroNI{ALLOW}/\MacroNI{DENY} in place of
  \MacroNI{HOSTALLOW}/\MacroNI{HOSTDENY}.  See the release notes above
  for more information.

% commit 7199e217f9228082a8465b85aaee18c2ebb19787
\item The default value for \Macro{MAX\_JOBS\_RUNNING} has changed.
  Previously, it was 200.  Now it is defined by an expression that depends 
  on the total amount of memory and the operating system.  The default
  expression requires 1MByte of RAM per running job, on the submit machine.
  In some environments and configurations, this is overly
  generous and can be cut by as much as 50\%.  Under Windows, the
  number of running jobs is still capped at 200.
  A 64-bit version of Windows  is recommended in order to raise the value
  above the default.
  Under Unix, the maximum default is now 10,000.  To scale higher, we
  recommend that the system ephemeral port range is extended
  such that there are at least 2.1 ports per running job.

% #767 commit 18296bfdfa92f16684a73d8d57a54d231b48dc33
\item The default value of \MacroNI{RESERVED\_SWAP} has changed to
  the value 0, which
  disables the \Condor{schedd} daemon's check for sufficient swap space
  before starting more jobs.  The new expression defined with 
  \MacroNI{MAX\_JOBS\_RUNNING} has a more appropriate memory check, so
  the configuration variable \MacroNI{RESERVED\_SWAP} will no longer
  be used in the near future.
  For cases where 
  \MacroNI{RESERVED\_SWAP} is not set to 0, the default value
  of \MacroNI{SHADOW\_SIZE\_ESTIMATE} has changed to 800 Kbytes.
  Previously, it was 200 if not set,
  but it was set to 1800 in the example configuration file.

% #767 commit c80e8a40e67ef4faa4e2b32b3671877eae1e1a19
\item The default values of \Macro{START\_LOCAL\_UNIVERSE} and
  \Macro{START\_SCHEDULER\_UNIVERSE} have changed.  Previously,
  these were set to \Expr{True}.  Now, they are set using an expression
  that allows
  up to 200 local universe and 200 scheduler universe jobs to run.

% #767 commit c4f4d911a808e1bdb18552e1cdeb0407b6344969
\item The default value of
  \Macro{GRIDMANAGER\_MAX\_SUBMITTED\_JOBS\_PER\_RESOURCE} has
  changed from 100 to 1000.

% #767 commit 9e6dfa463c71c28c8dc2c0c0c215b51d6762e811
% commit b4fd08ad1a8c69da24c371565796ef73522a61fc
\item The default value of \Macro{NEGOTIATOR\_INTERVAL}
   has changed from 300 to 60.

% #767 commit 8b91877ec8186810887402e1dd1df07b6341ade7
% Probably at least one other commit
\item The default value of \Macro{ENABLE\_GRID\_MONITOR} has been
  changed from \Expr{False} to \Expr{True}.  This variable
  was assigned to \Expr{True} in the example configuration file, so
  the change in default value now matches the value set in the example
  configuration.

% #631
\item The configuration variable \MacroNI{VM\_VERSION} has been removed,
as has the machine ClassAd attribute of the same name.
When the virtual machine version information is needed in the machine ClassAd,
the configuration variable \MacroNI{STARTD\_ATTRS} can be used to
add it.
 
% #861
\item The default configuration now uses
  \MacroNI{VM\_BRIDGE\_SCRIPT} and \MacroNI{VM\_SCRIPT} in place of
  \MacroNI{XEN\_BRIDGE\_SCRIPT} and \MacroNI{XEN\_SCRIPT} due to the
  support of KVM. 
  Submit description file commands have also been added, and they include:
  \SubmitCmd{kvm\_disk}, \SubmitCmd{kvm\_transfer\_files},
   and \SubmitCmd{kvm\_cd\_rom\_device}.

% #872
\item The configuration variables \MacroNI{XEN\_DEFAULT\_KERNEL}
  and \MacroNI{XEN\_DEFAULT\_INITRD} have been removed.
  Corresponding to this, the submit description file command
  \Expr{xen\_kernel = any} is no longer valid.

\end{itemize}

\noindent Bugs Fixed:

\begin{itemize}

\item Fixed a bug that prevented parallel universe jobs from running 
  on \Condor{startd} daemons with dynamic slots enabled.

% #706
\item Fixed a race condition bug in the \Condor{startd} which allowed
it to send Unix signals, intended for \Condor{starter} processes, as
root to non-Condor related processes.

% 735
\item A Windows platform bug has been fixed.
The bug caused a 20-second interval in which
the \Condor{shadow}, \Condor{startd}, and \Condor{starter} daemons
appeared as deadlocked. 
The bug was visible if a job ClassAd update from the \Condor{starter} caused
the job's periodic hold or remove policy to become \Expr{True}.

%gittrac #622
\item Fixed a bug that could cause \Condor{dagman} to generate an
illegal rescue DAG, if it read events incorrectly in recovery mode.
\Condor{dagman} now checks for events that violate DAG semantics
when reading events in recovery mode, and it exits without creating a
rescue DAG if it reads such an event.

% gittrac #744
\item Fixed a bug that could cause \Condor{dagman} to abort if it saw
the combination of a terminated event and an aborted event on a node with
retries.

% commit 5039a08cf00b0d0fafcc3fd8241518d1854ec3a1
\item Changed some logged warnings in \Condor{dagman} to not be
printed at the default verbosity setting.

% gittrac #825
\item The version compatibility checking between a \File{.condor.sub}
file and the \Condor{dagman} binary which is done at DAG startup
is now much more permissive.
Currently, \File{.condor.sub} files with
Condor versions of 7.1.2 and later accepted by \Condor{dagman}.

% gittrac #851
\item Fixed a bug introduced with the new \Condor{dagman} lazy log file
evaluation code in Condor version 7.3.2.
The bug sometimes caused failure when running rescue DAGs.

% #211 commit d6c0144739000523e94205a192be3cf9afe9ca9f
\item Fixed a bug originating in Condor version 7.1.4.
When a user submitted a job
with an executable that did not have execute permission enabled,
Condor was running as root, and file transfer was specified in the job,
Condor failed to automatically turn on execute permission after
transferring the file.

% commit 3bb847691bfda4f26d2f570bed1a412fb3afb439
\item Fixed a bug that appeared in Condor version 7.3.2.
The configuration variable
\MacroNI{COUNT\_HYPERTHREAD\_CPUS} was ignored and was effectively
treated as \Expr{False} in all cases.

% #761
\item Fixed a bug in which the Condor Job Router was not able
to see matchmaking diagnostic attributes such as \Attr{LastRejMatchTime}.
Therefore, when evaluating policy
expressions that referred to these attributes, they were effectively
treated as though \Expr{Undefined}.
Quill was also not able to see these attributes.

% #822
\item Fixed a bug introduced in Condor version 7.3.2 that could cause the
\Condor{gridmanager} to crash repeatedly on startup,
if the job queue
contained grid type gt2 jobs that had been previously submitted.

% #724, #774, #786
\item Fixed two bugs introduced in Condor version 7.3.2,
and related to VOMS. 
The first bug
prevented jobs with X.509 proxies from being submitted on platforms
on which Condor does not support VOMS.
The second bug prevented submission
of jobs with VOMS proxies, if the authenticity of the VOMS extensions
could not be verified.
At the same time, improved memory usage when VOMS extensions are not used.

\item Fixed a bad default in the file \File{batch\_gahp.config},
that prevented
Condor from observing job state changes for grid universe jobs
with a grid type of pbs or lsf.

% #748
\item Fixed a bug that caused Condor-C jobs to fail if
the submit description file command \SubmitCmd{transfer\_executable}
was set to \Expr{False}.

% #784
\item Fixed a bug that caused Condor-C jobs to fail if the executable
or one of the \File{stdin}, \File{stdout}, or \File{stderr} file names
contained a comma.

% #460
\item File transfer for grid type gt4 jobs requires an empty directory
within \File{/tmp}, which the \Condor{gridmanager} creates. 
If this directory is deleted, the \Condor{gridmanager} will now recreate it.

%gittrac #790
\item Fixed a bug that could cause the user job log to become
  corrupted on Windows platforms.  This bug would manifest itself only if the
  same log file was specified with different paths.  For example, the
  following submit file could have triggered this bug:
\begin{verbatim}
...
initialdir = /data/job1
log = ../JobLog
queue

initialdir = /data/job2
log = ../JobLog
queue
\end{verbatim}


% commit a26fcd9fe54cd3920fe777d5d8e0b2ffefbc3b1f
\item Fixed a memory leak introduced into Condor version 7.3.2.
The leak was in the \Condor{collector} daemon.

% commit 1663b7e183e6bf1df8152af98d9387412c2ae146
\item Fixed a bug introduced in Condor version 7.3.2
that resulted in the \Condor{negotiator} daemon
refusing to run, if the configuration variable \MacroNI{GROUP\_QUOTA}
for any group was set to 0.

% gittrac #731
\item Fixed a bug that caused the \Code{ctime} in the event log header
  to always be zero.

% #862 commit 9a432e2f3497e5dce120db5c733e79212257f6a5
\item Fixed the output of \Condor{status} when used with the command-line
  options \Opt{-java} or \Opt{-vm}.

\item Fixed a problem in the \Condor{schedd} daemon introduced in
  7.3.2.  For \Condor{schedd} daemons with lots of jobs having periodic release
  expressions, this bug could result in the \Condor{schedd} taking a long
  time while evaluating periodic expressions, causing it to become
  unresponsive to queries and other tasks.
  With a job queue of 30,000 jobs,
  a period of unresponsiveness of an hour was observed,
  whereas the evaluation of periodic expressions in this same environment
  normally takes less than 5 seconds.

\item Potential bugs and memory leaks were identified and 
fixed throughout Condor.  The Condor Team is not aware of anyone having 
encountered these bugs.

% #692 commit 8bc6bb4e06f11b2fdca28214d98c68c34c0ab9a4
\item The \Condor{starter} cleans up working directories in more
situations.  Previously during some error conditions, the working
directory within \MacroUNI{EXECUTE} might be left behind.

% #692 commit 8bc6bb4e06f11b2fdca28214d98c68c34c0ab9a4
\item If the user log cannot be accessed when a local universe
job starts, the job would fail and immediately be retried.  Now
the job is placed on hold.

% 826 
\item Fixed a bug in the \Condor{startd} in which vacating jobs would not 
respect the value of \Attr{JobLeaseDuration}.

% 802
\item Updated the detection of \Attr{HasVM} within the \Condor{startd}
 to publish an update to the \Condor{collector},
 when the configuration variable \MacroNI{VM\_RECHECK\_INTERVAL} is specified.

% commit 68f06088fa36eb0eb332a4f72a5c48ccd48b1d5a
\item Fixed a bug in which the \Condor{gridmanager} could, in rare cases,
waste a
small amount of memory and processor time checking for proxy files no longer
being used by any active jobs.

% commit bc66aa432e1f4e69d88a5b769204a4fce0648bfc
\item The setting \Macro{CREAM\_GAHP} was added to the default configuration 
file with a value of \File{\$(SBIN)/cream\_gahp}.
Existing installations desiring to 
submit jobs to CREAM should add this setting.

% #702
\item Fixed a bug where \Condor{restart} would fail on a \Condor{collector}
daemon configured for high availability with multiple \Condor{collector}
daemons.

% commit f44a68fb351e528ea5b251dd2c3cf9767b0c1fba
\item Fixed a bug in which multiple entries of output from 
the command
\Condor{status} \Opt{-negotiator}
would be on a single line.  They are now listed one per line.

% #778
\item Fixed a bug in which the command
\Condor{submit} \Opt{-dump} would crash if multiple
jobs were queued from within a single submit file.

% #742
\item Fixed a bug in which a slot whose associated job disappeared
could remain in the Claimed/Idle state until the claim lease expired.
The slot should now promptly return to the Unclaimed/Idle state.

% commit 0d5e3ad8fc85f0cd0dc58f73b503c76c0ad49bc4
\item Fixed a bug in which a \Condor{startd} using dynamic slots could
crash on shutdown or reconfiguration.



\end{itemize}

\noindent Known Bugs:

\begin{itemize}

% gittrac #1337
\item The \Condor{kbdd} has a chance of entering an infinite loop
on platforms that use X-Windows.  Microsoft Windows and Mac OS X
are not affected.  Removing KBDD from \MacroNI{DAEMON\_LIST} is a
workaround, although this impairs Condor's ability to detect
console usage.  This bug is fixed in Condor version 7.4.3.

% gittrac #161, #935, #936, #1020
\item There are multiple bugs related to using VOMS attributes.
In Condor version 7.4.0, VOMS support should be disabled by setting
the configuration variable \Expr{USE\_VOMS\_ATTRIBUTES = FALSE}.

\item A configuration variable of  \Macro{USE\_VISIBLE\_DESKTOP} set 
to \Expr{True} will corrupt the visible desktop.
  This bug is present back through Condor version 7.2.4.
This configuration variable did not work at all in 7.2 releases
prior to 7.2.4.  This bug will be fixed in Condor version 7.4.1.

% gittrac #934
\item If the global event log (see section~\ref{param:EventLog}) is
turned on, \Condor{dagman} will hang when running any DAG that has
POST scripts.

% gittrac #979
\item \Condor{dagman} will hang on Windows when running any DAG that
uses more than one log file for the node jobs.

\end{itemize}

\noindent Additions and Changes to the Manual:

\begin{itemize}

\item See section~\ref{sec:Condor-HDFS} and 
section~\ref{sec:HDFS-Config-File-Entries} for preliminary documentation of
Condor's integration with the Hadoop Distributed File System (HDFS). 

\end{itemize}


% as of April 2011, Karen no longer wants to include these older
% version histories with the 7.6 and beyond manuals.
%%%%      PLEASE RUN A SPELL CHECKER BEFORE COMMITTING YOUR CHANGES!
%%%      PLEASE RUN A SPELL CHECKER BEFORE COMMITTING YOUR CHANGES!
%%%      PLEASE RUN A SPELL CHECKER BEFORE COMMITTING YOUR CHANGES!
%%%      PLEASE RUN A SPELL CHECKER BEFORE COMMITTING YOUR CHANGES!
%%%      PLEASE RUN A SPELL CHECKER BEFORE COMMITTING YOUR CHANGES!

%%%%%%%%%%%%%%%%%%%%%%%%%%%%%%%%%%%%%%%%%%%%%%%%%%%%%%%%%%%%%%%%%%%%%%
\section{\label{sec:History-7-3}Development Release Series 7.3}
%%%%%%%%%%%%%%%%%%%%%%%%%%%%%%%%%%%%%%%%%%%%%%%%%%%%%%%%%%%%%%%%%%%%%%

This is the development release series of Condor.
The details of each version are described below.

%%%%%%%%%%%%%%%%%%%%%%%%%%%%%%%%%%%%%%%%%%%%%%%%%%%%%%%%%%%%%%%%%%%%%%
\subsection*{\label{sec:New-7-3-2}Version 7.3.2}
%%%%%%%%%%%%%%%%%%%%%%%%%%%%%%%%%%%%%%%%%%%%%%%%%%%%%%%%%%%%%%%%%%%%%%

\noindent Release Notes:

\begin{itemize}

\item The format of the output from \Condor{status} with the \Opt{-grid} option
has been changed to provide more useful information.

% gittrac #12
\item Removed the newline appended to the end of \Condor{status}
\Opt{-format} output.
Therefore, code which parses the output of this command should now
be careful when trimming the last line.

\end{itemize}

\noindent New Features:

\begin{itemize}

\item \Condor{fetchlog} may now fetch the history files of a \Condor{schedd}
daemon.  And, the history file kept by the \Condor{schedd} daemon may
now be rotated daily or monthly.

\item The \Condor{ckpt\_server} will automatically clean up stale
checkpoint files. The configuration variables which control this
behavior are described below.

% gittrac #177
\item The \Condor{ckpt\_server} (either the 32-bit or 64-bit) executable
will now communicate correctly between 32-bit and 64-bit submit nodes.
If by some chance bit width issues arise in the checkpoint protocol
(for example, with file sizes),
clear error messages are logged in the checkpoint server logs.

\item The new \Condor{ssh\_to\_job} tool allows interactive debugging of running
jobs.  See the manual page at~\pageref{man-condor-ssh-to-job} for details.

\item The \Condor{status} command is now substantially faster, 
especially with the \Opt{-format} option.

% gittrac #676
\item Grid universe grid type \Opt{gt5} has been added for submission to
the new Globus GRAM5 service. When a GRAM service is identified as
\SubmitCmd{gt5}, jobmanager throttling and the Grid Monitor are not used.
See section~\ref{sec:Using-gt5} for details.

\item Grid universe grid type \Opt{cream} has been added for submission
to the CREAM job service of \Prog{gLite}.
See section~\ref{sec:CREAM} for details.

\item When low on file descriptors for creating new network sockets,
the \Condor{schedd} daemon now avoids the unlimited stacking up of
messages that it sends periodically to the \Condor{negotiator} 
and \Condor{startd}.

% gittrac #429
\item The performance and failure handling of the Grid Monitor have been
improved.

% gittrac #356
\item For grid type \SubmitCmd{nordugrid} in the grid universe,
job status information
is now obtained using Nordugrid ARC's LDAP server, which should greatly
improve performance. Also, Condor can now tell when these jobs are running.

%gittrac #527
\item The new \Opt{-valgrind} option to \Condor{submit\_dag}
causes \Condor{submit\_dag} to generate a submit description file that
uses \emph{valgrind} on \Condor{dagman}, instead of the \Condor{dagman}
binary as its executable.

%gittrac #328
\item \Condor{dagman} now lazily evaluates and opens node job log files.
Instead of parsing all submit description files and 
immediately opening their specified log files at start up,
\Condor{dagman} now parses
the submit description files just before each job is submitted,
and has each log file open only when relevant jobs are in the queue
or executing POST scripts.
In addition, \Condor{dagman} now automatically generates a default user log
file for any node job that does not specify one.

\item Both the support and documentation for the MPI universe have been removed.
MPI applications are supported through the use of the parallel universe.

% gittrac #551
\item When the \Condor{startd} daemon's test of virtual machine software fails
(for machines configured as capable of running virtual machines),
the \Condor{startd} will periodically retry the test until it succeeds.

%gittrac #654
\item The \Prog{nordugrid\_gahp} now limits the number of connections
made to each NorduGrid ARC server and reuses connections when possible.

\item Added the ClassAd function \Code{eval()}, which takes a string
argument and evaluates the contents of the string as a ClassAd
expression.  An policy example where this is useful is described in
section~\ref{sec:Job-Suspension} on job suspension.

\item The new \Condor{q} option \Opt{-attributes} limits the
attributes which are displayed when using the \Opt{-xml} or \Opt{-long}
options.
Limiting the number of attributes also increases the efficiency of the query.

%gittrac #383
\item Condor's power management capabilities are now implemented as a
  plug-in.  In particular, the \Condor{startd} now runs an
  external program, as specified by the configuration variable
  \Macro{HIBERNATION\_PLUGIN},
  to perform the detection of available low power states and the
  switching to these low power states.

\item The new Condor daemon \Condor{rooster} has been added to wake up
hibernating machines when the expression defined by the configuration variable
\Macro{UNHIBERNATE} becomes \Expr{True}.
The configuration variables relating to \Condor{rooster}
are described in section~\ref{sec:Config-rooster}.

%gittrac #483
\item Added the ability to extract information from the user event log
  reader's state buffer to the user log reader.  This is implemented
  through a new \Code{ReadUserLogStateAccess} C++ class
  as defined in \File{read\_user\_log.h}.

\item Changes to the value of the configuration variable
\MacroNI{CERTIFICATE\_MAPFILE} or the contents
of the file to which it refers no longer require a full restart of Condor.
Instead, the command \Condor{reconfig} will cause the changes to be utilized.

\item The \Condor{master} daemon will now print the path and arguments
  to any daemons it starts if D\_FULLDEBUG is enabled.  Previously,
  there was no way to get it to display the arguments with which it
  was starting a daemon.

% gitrac #308
\item The \Condor{had} daemon now has the ability to control daemons
  other than the \Condor{negotiator}.  This is controlled via the
  \MacroNI{HAD\_CONTROLLEE} macro.

%gittrac #161
\item Condor now recognizes VOMS extensions in X.509 proxies.
The VOMS attributes are encoded in the job ClassAd attribute
\Attr{X509UserProxySubject}.

%gittrac #514
\item The \Condor{startd} can now clean up stranded virtual machines,
following a crash of Condor or its host operating system.

%gittrac #301
\item Following a crash, the \Condor{gridmanager} no longer restarts all
of the jobmanagers for gt2 jobs. This should improve recovery time.

\item Condor works better with the ClassAds categorized as generic
in the \Condor{collector} daemon.
Various daemons that register themselves with generic ClassAds
can now have tools which use the \Opt{-subsystem} option manipulate
their ClassAds properly.

\item Condor now provides a mechanism to enforce strict resource limiting for
some universes of running jobs.

\end{itemize}

\noindent Configuration Variable Additions and Changes:

\begin{itemize}

\item The new configuration variable \Macro{EMAIL\_SIGNATURE} specifies
a custom signature to be appended to e-mail sent by the Condor system.
If defined, then this custom signature replaces the
default one specified internally.

\item The new configuration variable \Macro{CKPT\_SERVER\_CLIENT\_TIMEOUT}
informs the \Condor{schedd} how long in seconds it is willing to wait
to try and talk to a \Condor{ckpt\_server} process before declaring a
\Condor{ckpt\_server} down.
See section~\ref{param:CkptServerClientTimeout} for the complete description.

\item The new configuration variable
\Macro{CKPT\_SERVER\_CLIENT\_TIMEOUT\_RETRY} informs the \Condor{schedd}
that once a \Condor{ckpt\_server} is been marked as down, how may seconds
must pass before the \Condor{schedd} will try and communicate with the
\Condor{ckpt\_server} again.
See section~\ref{param:CkptServerClientTimeoutRetry} 
for the complete description.

\item The new configuration variable
\Macro{CKPT\_SERVER\_REMOVE\_STALE\_CKPT\_INTERVAL} informs the
\Condor{ckpt\_server} to begin removal of stale checkpoints at the specified
interval in seconds.
See section~\ref{param:CkptServerRemoveStaleCkptInterval} 
for the complete description.

\item The new configuration variable
\Macro{CKPT\_SERVER\_STALE\_CKPT\_AGE\_CUTOFF} informs the
\Condor{ckpt\_server} how old a checkpoint file's access time must be
in order to be considered stale. This time is compared against the
current notion of now
when the checkpoint server checks the checkpoint image file.
See section~\ref{param:CkptServerStaleCkptAgeCutoff} 
for the complete description.

% gittrac 331
\item The new configuration variable \Macro{SlotWeight} may be used to
give a slot greater weight when calculating usage, computing fair
shares, and enforcing group quotas.  
See \ref{param:SlotWeight} for the complete description.

\item The new configuration variable \Macro{MAX\_PERIODIC\_EXPR\_INTERVAL}
  implements a ceiling on the time between evaluation of periodic expressions,
  due to the adaptive timing implied by the configuration variable
  \MacroNI{PERIODIC\_EXPR\_TIMESLICE}.
  See \ref{param:MaxPeriodicExprInterval} for the complete description.

% gittrac #475
\item The new configuration variable \Macro{GRIDMANAGER\_SELECTION\_EXPR}
can be used to control how many \Condor{gridmanager} processes will be
spawned to manage grid universe jobs. As a part of this change, removed
the configuration variable and supporting code for 
\Macro{GRIDMANAGER\_PER\_JOB} since the new configuration variable
supersedes it.
See \ref{param:GridManagerSelectionExpr} for the complete description.

% gittrac #207
\item The configuration variable
\Macro{GRIDMANAGER\_MAX\_PENDING\_SUBMITS\_PER\_RESOURCE} and the
corresponding throttle \Macro{GRIDMANAGER\_MAX\_PENDING\_SUBMITS}
have been removed.

% gittrac #429
\item The new configuration variable \Macro{GRID\_MONITOR\_DISABLE\_TIME}
controls how long the \Condor{gridmanager} will wait after encountering
an error before attempting to restart a Grid Monitor job.
See \ref{param:GridMonitorDisableTime} for the complete description.

\item The new pre-defined configuration macro \Macro{DETECTED\_MEMORY}
indicates the amount of physical memory (RAM) detected by Condor.
The value is given in Mbytes.

\item The new pre-defined configuration macro \Macro{DETECTED\_CORES}
indicates the number of CPU cores detected by Condor.

%gittrac #621
\item The new configuration variable
\Macro{DELEGATE\_FULL\_JOB\_GSI\_CREDENTIALS}
controls whether a full or limited X.509 proxy is delegated for grid type
\SubmitCmd{gt2} \SubmitCmd{grid} universe jobs.
See \ref{param:DelegateFullJobGSICredentials}
for the complete description.

%gittrack #383
\item The new configuration variable \Macro{UNHIBERNATE} is used by
the \Condor{startd} to advertise in its ClassAd a boolean expression
specifying when the machine should be woken up, 
for example by \Condor{rooster}.
See \ref{param:Unhibernate} for the complete description.

%gittrac #383
\item The new configuration variable \Macro{HIBERNATION\_PLUGIN} specifies the
  path to the plug-in which the \Condor{startd} uses both to detect
  the low power state capabilities of a machine and to switch the
  machine to a low power state.
  See \ref{param:HibernationPlugin} for the complete description.

%gittrac #383
\item The new configuration variable \Macro{HIBERNATION\_PLUGIN\_ARGS}
  specifies additional command line arguments which the
  \Condor{startd} will pass to the plug-in when invoking it to
  switch the machine to a low power state.
  See \ref{param:HibernationPluginArgs} for the complete description.

%gittrac #383
\item The new configuration variable \Macro{HIBERNATION\_OVERRIDE\_WOL} can be
  used to direct the \Condor{startd} to ignore Wake On LAN (WOL)
  capabilities of the machine's network interface, and to switch to a
  low power state even if the interface does not support WOL, or if
  WOL is disabled on it.
  See \ref{param:HibernationOverrideWOL} for the complete description.

% gittrac 586
\item The new configuration variable \Macro{DAGMAN\_USER\_LOG\_SCAN\_INTERVAL}
controls how long \Condor{dagman} waits between checking job log files
for status updates.
See \ref{param:DAGManUserLogScanInterval} for the complete description.

\item The new configuration variable \Macro{DAGMAN\_DEFAULT\_NODE\_LOG} sets
the default log file name for the new \Condor{dagman}
default node log file feature.
See \ref{param:DAGManDefaultNodeLog}
for the complete description.

\item Removed the configuration variable
\Macro{DAGMAN\_DELETE\_OLD\_LOGS}; new log file reading code makes it
obsolete.

% gittrac #308
\item The new configuration variable \Macro{HAD\_CONTROLLEE} is used
  to specify the name of the daemon which the \Condor{had} controls.
  This name should match the daemon name in the \Condor{master}'s
  \MacroNI{DAEMON\_LIST}.

\end{itemize}

\noindent Bugs Fixed:

\begin{itemize}

\item Fixed a bug in ClassAd functions where arguments which should have been
correctly coerced into strings instead evaluated to \texttt{ERROR}.

\item Fixed a confusing diagnostic message with the JobRouter, which happened
when a job was removed within 5 minutes of being submitted.

% from 7.5.2
% gittrac #581
\item Fixed a bug in which the use of dynamic slots 
(see section~\ref{sec:SMP-dynamicprovisioning})
caused the machine ClassAd attribute \Attr{SLOT<N>\_STARTD\_ATTRS}
to disappear from the ClassAd for some slots.

% gittrac #510
% Karen's rewrite of
%\item Fixed a bug in the Windows port where windows of an application running
%under Condor may never receive a paint message.
\item Fixed a Windows platform bug in which the window belonging to
a Condor job does not receive a paint message.

\item Fixed a bug causing \Condor{q} \Opt{-analyze} to crash when there was no 
\Condor{schedd} daemon ClassAd file.

\item Fixed a \Condor{procd} crash caused when the environment of 
a monitored process exceeded 1MByte in \File{/proc}.

% gittrac #708
\item Fixed a Windows platform bug which could cause the \Condor{credd} 
to crash if a requested credential is not in the password store.

% gittrac #535
\item Fixed a bug that was causing the job event log rotation lock to be
created with incorrect permissions.

% gittrac #601
\item Fixed a bug in the rotation of the job event log which could cause it
never to be rotated in the Windows port of Condor.

% gittrac #691
\item Fixed a potential race condition in the job event log initialization.

% gittrac #695
\item Fixed race condition which could cause a crash of the \Condor{collector}
and \Condor{schedd} on shutdown.

% gittrac #690
\item Fixed a bug in which the \Condor{master} would sometimes die and produce
a \File{dprintf\_failure.MASTER} file when either restarting due to new
binary timestamps or when started initially.

% This is from the unreleased 7.2.5.
\item Fixed a memory leak related to SOAP configuration variables
that occurred when Condor was reconfigured.

% This is from the unreleased 7.2.5
\item Fixed a bug in which the submit description file command
\Attr{cron\_day\_of\_week} was erroneously ignored.

% gittrac #580
\item Fixed bug in which the configuration variables
\Macro{MAX\_JOB\_QUEUE\_LOG\_ROTATIONS} and \Macro{GRIDMANAGER\_SELECTION\_EXPR}
would not work properly at start up; they only worked after a \Condor{reconfig}.

\item Fixed a bug in which SOAP operations were being incorrectly authorized
with the peer IP \Sinful{0.0.0.0}.

\item Fixed a Windows platform bug in which not all Condor daemons were trusted
by the Windows Firewall
(previously known as Internet Connection Firewall or ICF).

% Commented out by Karen until there is enough info to make the item's
% existence useful.
%\item Fixed a race condition in the parallel universe.

\item Fixed a shutdown race condition in the \Condor{master} with respect to
high availability daemons.

\item Fixed a bug in which a Condor daemon incorrectly determined it had
run out of socket descriptors.

% gitrac #552
\item Fixed a bug where the \Condor{schedd} would block for very long
periods of time while trying to connect to a down checkpoint server. Now
the \Condor{schedd} will do a blocking connect with a timeout to the
checkpoint server for a configurable number of seconds. If the connect
fails, the \Condor{schedd} will put a moratorium on connecting to the
checkpoint server until the configurable moratorium period passes. The
configuration file variables that describe this behavior are described
above.

% See gittrac #215
\item Changed the check that \Condor{dagman} does for other
\Condor{dagman} instances
running the same DAG, if it finds a lock file at startup.
Now, if \Condor{dagman} is not sure whether the other DAGMan is alive,
it continues, rather than exiting.

\item Fixed a major file descriptor leak in the Stork daemon.

\item Fixed a bug in which successful Stork transfers were marked as failed.

\item Fixed an uncommon memory leak in the user event log file reading code
when reading badly formatted events.

% gittrac 593
\item Fixed a bug in which multiple machine ClassAds in the
\Condor{collector} with the same \Attr{Name},
but different \Attr{StartdIPAddr} attribute values,
would cause the \Condor{negotiator} to exit with an error.
This is unusual and should not happen in a typical Condor installation.
The most likely cause is using \Condor{advertise}
to advertise custom ClassAds for grid matchmaking.

%gittrac #435
\item Fixed a bug that caused \Condor{dagman} to core dump if all
submit attempts failed on a DAG node having a POST script.
This bug has existed since Condor version 7.1.4.

\item Fixed a memory leak in the \Condor{schedd}, which occurred when
the configuration variable \MacroNI{NEGOTIATOR\_MATCH\_EXPRS} was used.

% gittrac #504
\item Fixed a bug in the Windows platform code that treats scripts as
  executables.
  Unknown file extensions were treated as an error,
  rather than as a Windows executable.

\item The \Condor{job\_router} now correctly sets the ClassAd attribute
\Attr{EnteredCurrentStatus} to the current time when creating a new routed job.
Previously, it copied this attribute from the original job.

\item The \Condor{job\_router} emits a more friendly log message when it
observes that the routed copy of the job was removed.

\item A fix has been made for a problem seen in 7.3.1 in which Condor daemons
using CCB to connect to other Condor daemons would sometimes consume
large amounts of CPU time for no good reason.

\item Fixed a rare failure case bug in which attempts to connect via
CCB could stay in a pending state indefinitely.

\item A Unix only bug caused Condor daemons to fail to start if
\MacroNI{MAX\_FILE\_DESCRIPTORS} was configured higher
than the current hard limit inherited by Condor.  If Condor is running
as root, this is no longer the case.

% gittrac %627
\item The \Condor{gridmanager} now advertises grid ClassAds properly when there
are multiple \Condor{collector} daemons.

%gittrack #647
\item When using \Condor{q} \Opt{-xml} and \Opt{-format} together to
limit the number of ClassAd attributes returned in the query, the XML
\verb@<classads>@ container tag was not generated.  This is fixed, but
now the preferred way to limit the returned attributes is to
use \Condor{q} option \Opt{-attributes}.

\item Fixed a bug in which the Unix \Condor{master} failed
when trying to restart itself,
if the configuration variable \MacroNI{MASTER\_LOCK} was defined,
or if the \Condor{master} was invoked with the \Opt{-t} option.
This bug has existed since the 7.0 series,
and likely has existed much longer than that.

%gittrac #575
\item Fixed a significant memory leak in the \Prog{gahp\_server}. This
leak was only present in previous Condor 7.3.x releases.

%gittrac #570
\item Fixed a bug that can cause a removed job that is held and then
released to return to idle status.

%gittrac #693
\item The Globus jar files distributed with the x86-64 RHEL 5 RPMs were
damaged, causing \SubmitCmd{gt4} grid type jobs to fail. This has been fixed.

\end{itemize}

\noindent Known Bugs:

\begin{itemize}

%gittrac #851
\item The version 7.3.2 \Condor{dagman} binary sometimes has problems running
rescue DAGs.  Probably the best work around for this problem is to use
version 7.3.1 rather than 7.3.2 \Condor{dagman} and \Condor{submit\_dag}
binaries, even if using version 7.3.2 for the rest of the Condor
installation.

\end{itemize}

\noindent Additions and Changes to the Manual:

\begin{itemize}

\item None.

\end{itemize}


%%%%%%%%%%%%%%%%%%%%%%%%%%%%%%%%%%%%%%%%%%%%%%%%%%%%%%%%%%%%%%%%%%%%%%
\subsection*{\label{sec:New-7-3-1}Version 7.3.1}
%%%%%%%%%%%%%%%%%%%%%%%%%%%%%%%%%%%%%%%%%%%%%%%%%%%%%%%%%%%%%%%%%%%%%%

\noindent Release Notes:

\begin{itemize}

\item None.

\end{itemize}


\noindent New Features:

\begin{itemize}

\item Added the STARTD\_HISTORY configuration parameter.  If set, this
is a pathname to a history file, just like the condor\_schedd maintains,
but only for jobs run on that startd.

\item Added the JavaSpecificationVersion attribute to startds which
support Java.  This allows users to request machines which support
a particular major version of Java, without specifying the exact
specific version.  So, Java versions 1.6.0\_01, 1.6.1\_02 and 1.6.2\_03
all advertise JavaSpecificationVersion of 1.6.

\item Implemented a performance increase to \Condor{dagman} which can
decrease the parsing times of DAG input files by up to 60 times.
This performance increase works for certain common DAG geometries.
This will help in submission and recovery
time for DAGs whose nodes have a very large number of dependency edges
associated with them.

\item \Condor{q} -analyze and -better-analyze now emit warnings
if the \Condor{schedd} will not run jobs when it is out of swap space or
has hit the limit imposed by the configuration variable
\MacroNI{MAX\_JOBS\_RUNNING}.

\item When matching Condor-G jobs to resources, if multiple jobs
match multiple resources, and every job has identical job rank, the
matchmaker would always fill up one particular resource first.  Now,
the resources will be matched in a round robin fashion.  This can be
overridden by setting job rank appropriately.

\item Made the \Condor{schedd} more efficient in how it stores
information about \verb@$$()@ expansions in the job ClassAd.
Also made the \Condor{schedd} more efficient in how it contacts
the \Condor{negotiator} to submit reschedule requests.

\item Improved the Job Router's heuristic for site throttle adjustment.  It
is now quicker to release the throttle when the failure rate drops
below the configured threshold.

\item Made the Job Router more efficient on startup by improving the way it
reads the job queue log file.

\item Added an accessor class to the user log reader API to allow the
  application to query about reader state, including the
  difference in the event numbers and log position of two states.  This
  can be used by the application to determine the number of events
  missed when missed events are detected.

\item Added the ability to throttle the rate at which jobs are
stopped via \Condor{rm}, \Condor{hold}, \Condor{vacate\_job},
and during a graceful shutdown of the \Condor{schedd} daemon.

\item In the configuration file, Condor now accepts expressions for
the values of configuration variables that are required to be 
numeric literals or boolean constants.  
Note that this does not imply that the
expressions may freely reference ClassAd values in places where they
could not before.  
See section~\ref{sec:Intro-to-Config-Files} for an example with
further explanation.

\end{itemize}

\noindent Configuration Variable Additions and Changes:

\begin{itemize}

\item Added the STARTD\_HISTORY configuration parameter.  If set, this
is a pathname to a history file, just like the condor\_schedd maintains,
but only for jobs run on that startd.

\item The new configuration variable \Macro{UPDATE\_OFFSET} 
  causes the \Condor{startd} to
  delay the initial (and all further) updates that it sends to the
  \Condor{collector}.  See \ref{param:UpdateOffset} for more details.

\item The new configuration variables
  \Macro{JOB\_STOP\_COUNT} and \Macro{JOB\_STOP\_DELAY}
  limit the rate at which jobs are stopped via \Condor{rm},
  \Condor{hold}, \Condor{vacate\_job}, and during a graceful shutdown of
  the \Condor{schedd} daemon.
  See \ref{param:JobStopCount} and \ref{param:JobStopDelay} 
  for full definitions.

\end{itemize}

\noindent Bugs Fixed:

\begin{itemize}

\item Fixed a problem with job removal in the local universe that 
  would cause spurious error messages to be written to the log of the
  \Condor{schedd} daemon.

\item The \Condor{schedd} was failing to send `reschedule' commands to
flocked negotiators, so unless some other schedd in the negotiator's
pool sent it a reschedule command, negotiation cycles would only
happen every \Macro{NEGOTIATOR\_INTERVAL}.

\end{itemize}

\noindent Known Bugs:

\begin{itemize}

\item When using CCB to connect to other Condor daemons, Condor 7.3.1
daemons can sometimes consume large amounts of CPU, potentially
causing performance problems.  Condor 7.3.0 did not suffer from this
problem.

\end{itemize}

\noindent Additions and Changes to the Manual:

\begin{itemize}

\item None.

\end{itemize}

%%%%%%%%%%%%%%%%%%%%%%%%%%%%%%%%%%%%%%%%%%%%%%%%%%%%%%%%%%%%%%%%%%%%%%
\subsection*{\label{sec:New-7-3-0}Version 7.3.0}
%%%%%%%%%%%%%%%%%%%%%%%%%%%%%%%%%%%%%%%%%%%%%%%%%%%%%%%%%%%%%%%%%%%%%%

\noindent Release Notes:

\begin{itemize}

\item This release is incompatible when communicating with
previous versions of Condor if CCB is enabled or if
\Macro{PRIVATE\_NETWORK\_NAME} is configured.

\item Updated the DRMAA version.
This new version is compliant with GFD.133,
the DRMAA 1.0 grid recommendation standard.
Three new functions were added to meet the specification's requirements,
and several bugs were fixed.

\end{itemize}


\noindent New Features:

\begin{itemize}

\item Added support for using any recognized script as an executable
in a submit file on Windows. For more information please see
section~\ref{sec:windows-scripts-as-executables} on
page~\pageref{sec:windows-scripts-as-executables}.

\item Improved support for private networks:
Added CCB, the Condor Connection Broker.  It is similar in
functionality to GCB, the Generic Connection Broker, but it has
several advantages, including ease of use and working on Windows as
well as Unix platforms.
GCB continues to work, but we may remove
it some time in the 7.3 development series.  The main missing feature
in CCB at the moment that prevents it from replacing GCB,
is support for connectivity from one private network to another.
CCB only works
when connecting from a public network to a private one.  For example,
jobs may be sent from a \Condor{schedd} on the public Internet to 
\Condor{startd} daemons on a
private network, if the \Condor{startd} daemons are configured
to use a CCB server that is accessible to the \Condor{schedd} daemon.
However, if the \Condor{schedd} daemon is on one private
network and the \Condor{startd} daemons are on a different private network,
CCB does not help.  For more information on CCB, see section~ \ref{sec:CCB}.

\item Added support for a CPU affinity on both Windows and Linux platforms.

\item Added support for the \Condor{q} \Opt{-better-analyze} option on Windows.

\item Added \MacroNI{WANT\_HOLD}.  When \MacroNI{PREEMPT} becomes
true, if \MacroNI{WANT\_HOLD} is true, the job is put on hold for the
reason (optionally) specified by \MacroNI{WANT\_HOLD\_REASON} and
\MacroNI{WANT\_HOLD\_SUBCODE}.  These policy expressions are evaluated
by the execute machine.  As usual, the job owner may specify
\AdAttr{periodic\_release} and/or \AdAttr{periodic\_remove}
expressions to react to specific hold states automatically.

\item Added the ClassAd function \Procedure{debug}.
See section~ \ref{sec:classadFunctions} for the details of this function.

% Commented out by Karen, as this is useless to a user taking the
% time to read a version history. More info is needed. 
%\item Log messages have been made more clear.
% Includes: Give a clear warning instead of a terse error, when lacking a COLLECTOR.

\item The \Condor{schedd} can now use MD5 check sums to avoid storing
multiple copies of the same executable in its \Macro{SPOOL} directory.
Note that this feature only affects executables sent to the
\Condor{schedd} via the \SubmitCmd{copy\_to\_spool} command within
a submit description file.

% gittrac #197
\item Reduced the number of sleeps \Condor{dagman} does to maintain log
file consistency when a DAG uses multiple user logs for node jobs.
DAGMan now does one sleep per submit cycle,
instead of one sleep for each submit.

% gittrac #166, #208
\item Added the \Opt{-import\_env} command-line flag to
\Condor{submit\_dag}.  This explicitly puts the submittor's environment
into the \File{.condor.sub} file.

\item Optimized the removal of large numbers of jobs.  
Previously, removal of tens of thousands of jobs caused the
\Condor{schedd} daemon to consume
a lot of CPU time for several minutes.

\item Reduced memory usage by the \Condor{shadow} daemon.  Since there is one
\Condor{shadow} process per running job, this helps increase the
number of running jobs that a submit machine can handle.  Under Linux 2.6,
we found that running 10,000 jobs from a single submit machine
requires about 10GBytes of system RAM.  We also found in this case that to
run more than 10,000 simultaneous jobs requires a 64-bit submit
machine.  On a 32-bit Linux platform, kernel memory is exhausted,
regardless of how much additional RAM the system has.

\item Reduced the memory usage of the \Condor{collector} daemon,
when \Expr{UPDATE\_COLLECTOR\_WITH\_TCP = True}.

\end{itemize}

\noindent Configuration Variable Additions and Changes:

\begin{itemize}

\item The new configuration variable \Macro{OPEN\_VERB\_FOR\_<EXT>\_FILES}
allows the default interpreter for scripts with an extension \textit{EXT} to
be changed.  For more information please see
section~\ref{sec:windows-scripts-as-executables} on
page~\pageref{sec:windows-scripts-as-executables}.

\item The new configuration variable \Macro{CCB\_ADDRESS}
configures a daemon to use one or more
CCB servers to allow communication with Condor components outside of
the private network.  See page~\pageref{sec:CCB}.

\item The new configuration variable \Macro{MAX\_FILE\_DESCRIPTORS}
(on Unix platforms only) specifies the
required file descriptor limit for a Condor daemon.  File descriptors
are a system resource used for open files and for network connections.
Condor daemons that make many simultaneous network connections may
require an increased number of file descriptors.  For example, see
page~\pageref{sec:CCB} for information on file descriptor requirements
of CCB.

\item The new configuration variables \Macro{ENFORCE\_CPU\_AFFINITY} and 
\Macro{SLOT<N>\_CPU\_AFFINITY} on Linux platforms allow for
Condor to lock slots to given CPUs.
Definitions for these variables are at \ref{param:EnforceCpuAffinity}.

\item The new configuration variable \Macro{DEBUG\_TIME\_FORMAT}
  allows a custom specification for the format of the time
  printed at the start of each line in a daemon's log file.
  See \ref{param:DebugTimeFormat} for the complete definition of
  this variable.

\item The new configuration variable \Macro{SHARE\_SPOOLED\_EXECUTABLES}
  is a boolean value that determines whether the \Condor{schedd} daemon will
  use MD5 check sums to avoid storing multiple copies of the same
  executable in the \MacroNI{SPOOL} directory. The default setting is
  \Expr{True}.

\item The new boolean configuration variable
  \Macro{EVENT\_LOG\_FSYNC} provides control of the behavior of
  Condor when writing events to the event log.  Previously,
  the behavior was as if this parameter were set to \Expr{False}.
  See \ref{param:EventLogFsync} for the complete definition of
  this variable.

\item The new boolean configuration variable
  \Macro{EVENT\_LOG\_LOCKING} provides control of the behavior of
  Condor when writing events to the event log.  Previously,
  the behavior was controlled by \MacroNI{ENABLE\_USERLOG\_LOCKING}.
  See \ref{param:EventLogLocking} for the complete definition of
  this variable.

\end{itemize}

\noindent Bugs Fixed:

\begin{itemize}

\item None.

\end{itemize}

\noindent Known Bugs:

\begin{itemize}

\item None.

\end{itemize}

\noindent Additions and Changes to the Manual:

\begin{itemize}

\item None.

\end{itemize}

%%%%      PLEASE RUN A SPELL CHECKER BEFORE COMMITTING YOUR CHANGES!
%%%      PLEASE RUN A SPELL CHECKER BEFORE COMMITTING YOUR CHANGES!
%%%      PLEASE RUN A SPELL CHECKER BEFORE COMMITTING YOUR CHANGES!
%%%      PLEASE RUN A SPELL CHECKER BEFORE COMMITTING YOUR CHANGES!
%%%      PLEASE RUN A SPELL CHECKER BEFORE COMMITTING YOUR CHANGES!

%%%%%%%%%%%%%%%%%%%%%%%%%%%%%%%%%%%%%%%%%%%%%%%%%%%%%%%%%%%%%%%%%%%%%%
\section{\label{sec:History-7-2}Stable Release Series 7.2}
%%%%%%%%%%%%%%%%%%%%%%%%%%%%%%%%%%%%%%%%%%%%%%%%%%%%%%%%%%%%%%%%%%%%%%

This is a stable release series of Condor.
As usual, only bug fixes (and potentially, ports to new platforms)
will be provided in future 7.2.x releases.
New features will be added in the 7.3.x development series.

The details of each version are described below.

%%%%%%%%%%%%%%%%%%%%%%%%%%%%%%%%%%%%%%%%%%%%%%%%%%%%%%%%%%%%%%%%%%%%%%
\subsection*{\label{sec:New-7-2-5}Version 7.2.5}
%%%%%%%%%%%%%%%%%%%%%%%%%%%%%%%%%%%%%%%%%%%%%%%%%%%%%%%%%%%%%%%%%%%%%%

\noindent Release Notes:

\begin{itemize}

\item \Security A flaw was found that could allow a user who already is authorized to
submit jobs into Condor, to queue a job under the guise of  a different
user.  In this way, someone who has access to a Condor submission
service and is allowed to submit jobs into Condor could gain access to
another non-root or non-administrator account on the system.
This flaw was discovered during the development process; no incidents
have been reported.

\end{itemize}


\noindent New Features:

\begin{itemize}

\item The \Condor{ckpt\_server} will automatically clean up stale
checkpoint files. The configuration file parameters which describe this
behavior are described below.

\end{itemize}

\noindent Configuration Variable Additions and Changes:

\begin{itemize}

\item The new configuration variable \Macro{CKPT\_SERVER\_CLIENT\_TIMEOUT}
informs the \Condor{schedd} how long in seconds it is willing to wait
to try and talk to a \Condor{ckpt\_server} process before declaring a
\Condor{ckpt\_server} down.
See section~\ref{param:CkptServerClientTimeout} for more information.

\item The new configuration variable
\Macro{CKPT\_SERVER\_CLIENT\_TIMEOUT\_RETRY} informs the \Condor{schedd}
that once a \Condor{ckpt\_server} is been marked as down, how may seconds
must pass before the \Condor{schedd} will try and communicate with the
\Condor{ckpt\_server} again.
See section~\ref{param:CkptServerClientTimeoutRetry} for more information.

\item The new configuration variable
\Macro{CKPT\_SERVER\_REMOVE\_STALE\_CKPT\_INTERVAL} informs the
\Condor{ckpt\_server} to begin removal of stale checkpoints at the specified
interval in seconds.
See section~\ref{param:CkptServerRemoveStaleCkptInterval} for more information.

\item The new configuration variable
\Macro{CKPT\_SERVER\_STALE\_CKPT\_AGE\_CUTOFF} informs the
\Condor{ckpt\_server} how old a checkpoint file's access time must be
in order to be considered stale. This time is compared against ``now''
when the checkpoint server checks the checkpoint image file.
See section~\ref{param:CkptServerStaleCkptAgeCutoff} for more information.

\end{itemize}

\noindent Bugs Fixed:

\begin{itemize}
%gitrac #601
\item Fixed a bug in the event log code which could cause the event log
 to never be rotated on Windows. 

%gitrac #708
\item Fixed a bug that can cause \Condor{credd} to crash on Win32. 
The \Condor{credd} on Win32 will likely crash with an ACCESS\_VIOLATION
during logging if a credential is requested that is not in the password 
store.

% gitrac #535
\item Fixed a bug that was causing the event log rotation lock to be 
created with incorrect permissions 

% gitrac #552
\item Fixed a bug where the \Condor{schedd} would block for very long
periods of time while trying to connect to a down checkpoint server. Now
the \Condor{schedd} will do a blocking connect with a timeout to the
checkpoint server for a configurable number of seconds. If the connect
fails, the \Condor{schedd} will put a moratorium on connecting to the
checkpoint server until the configurable moratorium period passes. The
configuration file variables that describe this behavior are described
above.

% See gittrac #215
\item Changed the check that \Condor{dagman} does for other 
\Condor{dagman} instances
running the same DAG, if it finds a lock file at startup.
Now, if \Condor{dagman} is not sure whether the other DAGMan is alive,
it continues, rather than exiting.

\item Fixed a major file descriptor leak in the Stork daemon.

% gittrac 593
\item Fixed bug in which multiple Machine ClassAds in the
\Condor{collector} with the same \Attr{Name},
but different \Attr{StartdIPAddr} attribute values
would cause the \Condor{negotiator} to exit with an error.
This is unusual and should not happen in a typical Condor installation.
The most likely cause is using \Condor{advertise}
to advertise custom ClassAds for grid matchmaking. 

%gittrac #435
\item Fixed a bug that caused \Condor{dagman} to core dump if all
submit attempts failed on a DAG node having a POST script.  
This bug has existed since Condor version 7.1.4.

\item Fixed a memory leak in the \Condor{schedd}, which occurred when
the configuration variable \MacroNI{NEGOTIATOR\_MATCH\_EXPRS} was used.

\end{itemize}

\noindent Known Bugs:

\begin{itemize}

\item None.

\end{itemize}

\noindent Additions and Changes to the Manual:

\begin{itemize}

\item None.

\end{itemize}


%%%%%%%%%%%%%%%%%%%%%%%%%%%%%%%%%%%%%%%%%%%%%%%%%%%%%%%%%%%%%%%%%%%%%%
\subsection*{\label{sec:New-7-2-4}Version 7.2.4}
%%%%%%%%%%%%%%%%%%%%%%%%%%%%%%%%%%%%%%%%%%%%%%%%%%%%%%%%%%%%%%%%%%%%%%

\noindent Release Notes:

\begin{itemize}

\item None.

\end{itemize}


\noindent New Features:

\begin{itemize}

\item None.

\end{itemize}

\noindent Configuration Variable Additions and Changes:

\begin{itemize}

\item None.

\end{itemize}

\noindent Bugs Fixed:

\begin{itemize}

% gittrac #177
\item Fixed a bug in the checkpoint server that caused failure of
checkpoint image storage and retrieval if the requesting submission
machine was running a 32-bit installation of Condor and the checkpoint
server was from a 64 bit installation, or vice versa. The checkpoint
image server, both the 32-bit and 64-bit installation, now handles both
protocols. It is recommended that any checkpoint server installation which
may be used in a flocking situation or other federated joining of pools
use the 64-bit binary. This is due to the possibility that there could be
a checkpoint image larger than what is representable in 32 bits. A 32-bit
checkpoint image server will now notice if this situation occurs and log
a message suggesting an upgrade to the 64-bit version.

\item Fixed a bug that caused \Condor{procd} to sometimes fail when monitoring
processes with environments larger than 1MB.

% gittrac #481
\item Fixed a bug that caused \Condor{dagman} to fail in recovery mode on
a DAG in which any nodes had been retried.

% gittrac #484
\item Xen-based virtual machines now have the correct amount of memory.
Previously, the amount of memory was too small by a factor of 1024.

%gittrack #509
\item Fixed a bug in the handling of \texttt{\$\$(VARIABLE)} submit
  file expressions.

% gittrac #510
\item Fixed a bug in the code related to \MacroNI{USE\_VISIBLE\_DESKTOP} 
  that was causing the windows created by the job behave incorrectly.

\item Fixed a bug that caused Stork to treat successful file transfers
as failed.

% gittrac #337
\item Fixed several bugs in the user log reader in the handling of
  files of size zero.

% gittrac #542
\item Fixed a problem affecting parallel universe jobs with very short
tasks.  If any of the parallel tasks exited before the first node
started, the entire job was prematurely treated as though it had
finished.  If the job ClassAd attribute \AdAttr{ParallelShutdownPolicy} was
set to \AdStr{WAIT\_FOR\_ALL}, the job was prematurely treated as though it
had finished if all stated tasks completed before the remaining tasks
started.

\end{itemize}

\noindent Known Bugs:

\begin{itemize}

\item Occasionally, Condor daemons will, for unknown reasons, bind the
  command socket to the invalid IP address of 0.0.0.0, resulting in
  the daemon crashing or otherwise malfunctioning.  The command socket
  address is always logged in the daemon's log, so the condition can
  be detected by looking for a line like in the daemon's log header
  with the address of 0.0.0.0 as in:
\begin{verbatim}
5/6 16:34:26 DaemonCore: Command Socket at <0.0.0.0:53795>
\end{verbatim}
  If you encounter this problem, please send an e-mail to
  condor-admin@cs.wisc.edu with any relevant details.

\end{itemize}

\noindent Additions and Changes to the Manual:

\begin{itemize}

\item Descriptions and definitions of all commands that may be placed within
  the submit description file have been moved from the \Condor{submit} 
  manual page to section~\ref{sec:submit-cmds}.

\item Added a description of the configuration variable
  \MacroNI{NEGOTIATOR\_MATCHLIST\_CACHING}.
  See \ref{param:NegotiatorMatchlistCaching} for the definition.

\end{itemize}


%%%%%%%%%%%%%%%%%%%%%%%%%%%%%%%%%%%%%%%%%%%%%%%%%%%%%%%%%%%%%%%%%%%%%%
\subsection*{\label{sec:New-7-2-3}Version 7.2.3}
%%%%%%%%%%%%%%%%%%%%%%%%%%%%%%%%%%%%%%%%%%%%%%%%%%%%%%%%%%%%%%%%%%%%%%

\noindent Release Notes:

\begin{itemize}

\item The header files for ClassAds are now included within the release.

\end{itemize}

\noindent New Features:

\begin{itemize}

\item Enhanced the Debian 5.0 Condor port on the x86\_64 platform to 
include support for the standard universe. 

\end{itemize}

\noindent Configuration Variable Additions and Changes:

\begin{itemize}

\item The new integer configuration variable
  \Macro{SEC\_TCP\_SESSION\_DEADLINE} specifies the
  number of seconds after which the client should give up its attempt to
  establish a security session with a daemon that it is connecting to.
  The default value is 120 seconds.

\item The new configuration variables \Macro{SCHEDD\_CLUSTER\_INITIAL\_VALUE}
  and \Macro{SCHEDD\_CLUSTER\_INCREMENT\_VALUE} are integers that 
  specify the cluster number to use for the first job submission,
  and the stride used to increment the cluster id upon successive submissions.
  See \ref{param:ScheddClusterInitialValue} and
  \ref{param:ScheddClusterIncrementValue}
  for the complete definitions of these variables.

\end{itemize}

\noindent Bugs Fixed:

\begin{itemize}

\item Fixed a memory leak in the \Condor{collector} daemon.
  The growth in memory over time was approximately 10Mbytes per day
  per 1000 slots.
  This bug was introduced in Condor version 7.2.0.

\item Fixed a problem that caused integrity checking of most UDP packets
  longer than about 40Kbytes to fail.
  This bug affected all previous versions of Condor.

\item By adding the new configuration variable
  \MacroNI{SEC\_TCP\_SESSION\_DEADLINE}, fixed a problem
  that has existed since Condor version 7.1.2. 
  The problem was that non-blocking read
  operations in the security handshake had no timeout,
  and could therefore lead to a socket remaining allocated indefinitely,
  if the other side of the connection did not respond.
  When this problem was observed,
  the following message appeared in the log written by the \Condor{schedd}
  daemon:
\begin{verbatim}
  file descriptor safety level exceeded
\end{verbatim}

\item Fixed a rarely observed bug in the event log reader code
  that could cause it to not detect missed events.

\item A bug in the Chirp java client has been fixed.
  The ChirpInputStream's \Procedure{read} method was returning
  negative values when encountering binary data.

% Gnats PR 872
\item \Condor{dagman} now rejects negative node retry values.

% Gnats PR 946
\item \Condor{dagman} no longer generates a rescue DAG if the DAG is
  aborted, but is considered successful;
  this is when ABORT-DAG-ON returns the value 0.

\item The user log event numbered 27,
  named \AdStr{Job submitted to grid resource},
  is now written for all grid universe jobs.
  Previously, it was not written for pbs, lsf, nordugrid, or unicore grid types.

% gittrac #376
\item Fixed a bug where a Condor-C job with both \MacroNI{remote\_<foo>}
and \MacroNI{remote\_remote\_<foo>} attributes would not have a
\MacroNI{remote\_<foo>} attribute when submitted to the remote
\Condor{schedd} daemon.

% gittrac #380
\item Fixed a bug in \Condor{configure} and \Condor{install} that would
  leave the configuration variable \MacroNI{CONDOR\_HOST} unset when
  configuring a central manager without using the \Arg{--central-manager}
  command-line argument.

% gittrac #395
\item Fixed a bug that could cause the \Condor{schedd} daemon
  to leak memory and file descriptors when using the
  \MacroNI{EVENT\_LOG} configuration variable.

\item Fixed a bug in the \Condor{gridmanager} that could cause it to
not send a clean up signal to the GRAM jobmanager for removed gt2 jobs.

% gittrac #426
\item Fixed a bug that caused parallel jobs to not work when
encryption was enabled.

% gittrac #261
\item Fixed a bug in the Windows installer that caused it to fail to start
Condor.

\end{itemize}

\noindent Known Bugs:

\begin{itemize}

\item None.

\end{itemize}

\noindent Additions and Changes to the Manual:

\begin{itemize}

\item None.

\end{itemize}



%%%%%%%%%%%%%%%%%%%%%%%%%%%%%%%%%%%%%%%%%%%%%%%%%%%%%%%%%%%%%%%%%%%%%%
\subsection*{\label{sec:New-7-2-2}Version 7.2.2}
%%%%%%%%%%%%%%%%%%%%%%%%%%%%%%%%%%%%%%%%%%%%%%%%%%%%%%%%%%%%%%%%%%%%%%

\noindent Release Notes:

\begin{itemize}

\item None.

\end{itemize}


\noindent New Features:

\begin{itemize}

\item Added a full port of Condor to Debian 5.0 on the x86 platform.

\item Added a clipped port of Condor to Debian 5.0 on the x86\_64 platform.

\item Added the \Opt{-DumpRescue} command-line flag to \Condor{dagman}
and \Condor{submit\_dag}.  This flag is intended mainly for testing.

\item Added support for the \Opt{-debug} option to \Condor{qedit}.

\item The Job Router now uses a time slice timer for periodic expression
  evaluation, similar to the \Condor{schedd} daemon.
  The evaluation interval is controlled by 
  the configuration variable \MacroNI{PERIODIC\_EXPR\_INTERVAL},
  and defaults to 60 seconds, the same default value used by
  the \Condor{schedd} daemon.

\item The Job Router now resets the source job, if a failure occurs when
  updating the \Condor{schedd} daemon for a periodic expression that
  evaluated to \Expr{True}.  The job's periodic expressions should be
  evaluated again some time in the future with a successful update.

\end{itemize}

\noindent Configuration Variable Additions and Changes:

\begin{itemize}

\item The new boolean configuration variable
  \Macro{EVENT\_LOG\_FSYNC} provides control of the behavior of
  Condor when writing events to the event log.  Previously,
  the behavior was as if this parameter were set to \Expr{False}.
  See \ref{param:EventLogFsync} for the complete definition of
  this variable.

\item The new boolean configuration variable
  \Macro{EVENT\_LOG\_LOCKING} provides control of the behavior of
  Condor when writing events to the event log.  Previously,
  the behavior was controlled by \MacroNI{ENABLE\_USERLOG\_LOCKING}.
  See \ref{param:EventLogLocking} for the complete definition of
  this variable.

% gittrac #314
\item The new string configuration variable \Macro{TRANSFERER}
  specifies the path to the \Condor{transferer} program which is
  invoked by the \Condor{replication} daemon to perform the actual
  transfer of the file set by \MacroNI{STATE\_FILE}.
  This is part of the high availability framework.
  Prior to Condor 7.2.2, the value of \MacroNI{TRANSFERER} was hard coded to
  \File{\MacroUNI{RELEASE\_DIR}/sbin/condor\_transferer}.  The use of
  this hard coded behavior should be considered obsolete behavior, and
  will be removed in a future version of Condor.

\item The \MacroNI{PREEMPTION\_REQUIREMENTS} and the \MacroNI{RANK}
  expression in the matchmaker can now reference many more ClassAd
  attributes than just \Attr{SubmittorPrio}.  New attributes allow
  this expression to take into account resources currently in use, as
  well as group usage and quota info.  New attributes are:
  \MacroNI{SubmitterUserResourcesInUse},
  \MacroNI{RemoteUserResourcesInUse},
  \MacroNI{RemoteGroupResourcesInUse}, \MacroNI{RemoteGroupQuota},
  \MacroNI{SubmitterGroupResourcesInUse},
  \MacroNI{SubmitterGroupQuota}.

\item Added \MacroNI{JOB\_ROUTER\_ATTRS\_TO\_COPY} configuration
  option. This is a comma separated list of attributes that the Job
  Router should copy from the routed ad to the source ad in addition
  to internally hard coded attributes that are copied.

\item Added \MacroNI{JOB\_ROUTER\_RELEASE\_ON\_HOLD}. configuration
  option that will control whether the Job Router will reset the
  source job to an untouched state if it needs to yield the job
  because the routed job went on hold.  The option defaults to
  resetting the source job.

\item The new configuration variables \Macro{PREEMPTION\_REQUIREMENTS\_STABLE}
  and \Macro{PREEMPTION\_RANK\_STABLE} identify for Condor if all
  attributes in the variables \MacroNI{PREEMPTION\_REQUIREMENTS} and
  \MacroNI{PREEMPTION\_RANK} will not change within
  a negotiation interval.

\item The new configuration variables \Macro{OFFLINE\_LOG}
  and \Macro{OFFLINE\_EXPIRE\_ADS\_AFTER} specify the location of
  persistent machine ClassAds for hibernating machines,
  as well as the lifetime of the persistent ClassAds.

\end{itemize}

\noindent Bugs Fixed:

\begin{itemize}

\item Fixed the \Condor{collector} daemon such that hibernating machines
  never time out.

\item Fixed incorrectly set ClassAd attribute values of machines
  entering a hibernation state.
  All hibernating machines are unclaimed and idle,
  they have no load, the CPU is not busy, and
  the keyboard and console appear as if they had been idle for a long time.

\item Fixed a bug where if any idle slot satisfied the
  \MacroNI{HIBERNATE} expression, Condor would put the machine into a
  sleep state irrespective of any active slots.

\item Fixed a bug on Windows that made it impossible to use
  the defined string \verb@"S5"@ for hibernation.

\item Fixed a bug in the \Condor{starter} where it would be running as
  real uid condor after job hooks are invoked which causes issues when
  accessing files.

\item Fixed a bug where some machines would send a final update ad to
  the \Condor{collector}, invalidating the persistent one that was
  previously sent (when \MacroNI{HIBERNATE} evaluates to \Expr{True}).
  This had the effect of dropping the machine out of the pool once the
  ad had grown stale.

\item Fixed a bug where any two Condor daemons on Windows were able to
  bind to the same port at the same time.

\item Fixed the behavior of the \Condor{negotiator} so that when a
  Condor-G matchmaking ad matches, the machine's ad will be shuffled
  to the end for round-robin matching to multiple gatekeepers with the
  same rank.

\item Resolved a bug in which the submit description file command
  \SubmitCmd{vm\_macaddr} was improperly parsed,
  and thus ignored, by \Condor{submit} for vm universe jobs.

\item Condor's Windows zip file distribution now includes the new
  C/C++ runtime libraries.

\item Fixed a Windows platform bug for jobs that enable streaming I/O.
  The bug caused the \Condor{starter} to crash upon invocation of the
  job.

\item Fixed a bug in which an ill-formed network packet could crash a
Condor daemon.  This would not be seen in normal Condor operation, but
sometimes port-scanning software could trigger such a crash.

\item Fixed a bug in which \Condor{q} would sometimes exit with 
  the value zero, indicating success,
  when it could not connect to a \Condor{schedd} daemon.
  It now exits with an error code.

\item Fixed two seemingly small memory leaks in Condor's SOAP
interface. A small amount of memory was lost per SOAP transaction. On
a high traffic machine, this leak would eventually render the
\Condor{schedd} daemon unresponsive.

\item Fixed a bug in the parallel universe where periodic expressions
involving the \Attr{JobStatus} attribute would not function properly.

\item Fixed a bug where Condor daemons could segmentation fault while trying
to write a core file to disk in the Unix ports.

\item Fixed a bug in which the use of dedicated execute accounts
(indicated by use of the configuration variable
\MacroNI{DEDICATED\_EXECUTE\_ACCOUNT\_REGEXP}) did not work properly
in PrivSep mode: those with the configuration variable
\MacroNI{PRIVSEP\_ENABLED} set to \Expr{True}.

\item Fixed an erroneous log message that reported that
the hook defined by \Macro{HOOK\_UPDATE\_JOB\_INFO} had run,
but would print the \MacroUNI{HOOK\_PREPARE\_JOB} path.
The correct hook ran, so this was only a logging error.
The log message is only visible at the \Expr{D\_FULLDEBUG} level.

% PR 953
\item Fixed a bug that caused \Condor{dagman} to crash if the
\File{dagman.out} file reached a size of 2 GBytes.

\item Fixed a problem affecting the \Condor{starter} when in PrivSep mode.
After the user job exited, an error was printed in the
\Condor{starter} log file complaining that it failed to \Prog{chown} the
sandbox to Condor ownership.  This error was not actually harmful,
just noisy.

\item Fixed a bug in the \Condor{master} that caused it to not have
  \MacroNI{REPLICATION} in its default list for \MacroNI{DC\_DAEMON\_LIST}.
  The example
  configuration file for HAD has been updated to match, as well.

\item Fixed the \Condor{transferer} daemon and documentation to consistently
  use the value of the configuration variable 
  \MacroNI{MAX\_TRANSFERER\_LIFETIME} in High Availability code.

% gittrac #285
\item Fixed a bug that caused \Condor{dagman} to crash,
if a splice DAG has node categories.

\item Changed splice-related \Condor{dagman} debug messages
to \emph{not} be printed at the default verbosity.
They are now mostly printed at debug level 4.
For definitions of the debug levels, see the \Condor{dagman} manual
page at section~ \ref{man-condor-dagman}.

% gittrac #313
\item Fixed a bug that caused the \Condor{replication} daemon,
  as part of the high availability framework,
  to start the \Condor{transferer} client incorrectly; the end result was
  that the \Condor{transferer} was unable to authenticate via GSI
  using host-based certificates.

\item Fixed a bug in which the ClassAd attribute \AdAttr{RemoteWallClockTime}
  could get too big after a restart of the \Condor{schedd} daemon,
  for jobs that were running at the time of the restart.

% gittrac #231
\item Fixed a bug that was causing the \Condor{startd} to log the
  error message 
\begin{verbatim}
  ioctl(SIOCETHTOOL/GWOL) failed: Operation not permitted (1)
\end{verbatim}
  when started as a Personal Condor on Linux.
  The message is now suppressed in this case.  When the message is
  printed, an additional message is logged informing the user that
  this error can be ignored, unless hibernation is being used.

% gittrac #330
\item Fixed a bug that was causing the \Condor{startd} to sometimes
  publish the network adapter's hardware address incorrectly in its
  ClassAd.

\item Fixed a case in which \Condor{history} could get into an infinite
loop when searching through a corrupted history file.

% gittrac #355
\item Fixed a bug in the user log reader code that could cause it to
  get into an inconsistent state after detecting missed events.

\item Condor version 7.2.2 and previous releases do not support 
  communication with Condor 7.3.x daemons using the new 7.3.x
  configuration variables \MacroNI{CCB\_ADDRESS} or
  \MacroNI{PRIVATE\_NETWORK\_NAME}.
  The version 7.2.2 \Condor{collector} daemon now
  recognizes when it is receiving ClassAds from such daemons,
  and it will reject them.
  In prior versions, Condor would accept the ClassAds,
  but attempts to use them led to unexpected behavior.

\end{itemize}

\noindent Known Bugs:

\begin{itemize}

\item None.

\end{itemize}

\noindent Additions and Changes to the Manual:

\begin{itemize}

\item A manual page for \Condor{power} now appears in the manual.
\Condor{power} sends a packet to a machine in a low power state,
to cause the machine to wake from that state.

\item Reorganized the user manual section that describes DAGMan.

\item Added a note about the fact that environment values specified
with the \Opt{environment} submit description file command override values from
the submittor's environment, as imported with \Opt{getenv = True}.

\item Added new information to the section on Power Management
  pertaining to the handling of hibernating machines.
  

\end{itemize}


%%%%%%%%%%%%%%%%%%%%%%%%%%%%%%%%%%%%%%%%%%%%%%%%%%%%%%%%%%%%%%%%%%%%%%
\subsection*{\label{sec:New-7-2-1}Version 7.2.1}
%%%%%%%%%%%%%%%%%%%%%%%%%%%%%%%%%%%%%%%%%%%%%%%%%%%%%%%%%%%%%%%%%%%%%%

\noindent Release Notes:

\begin{itemize}

\item This release addresses reported 7.2.0 problems with the
Windows distribution.

\end{itemize}


\noindent New Features:

\begin{itemize}

\item Condor now has a clipped port to i386 Debian 5.0 (Lenny).

\item Added standard universe support for \Prog{gfortran}.

\item Added support for standard output and standard error to be greater
than 2 Gigabytes.

\end{itemize}

\noindent Configuration Variable Additions and Changes:

\begin{itemize}

\item The configuration variable \Macro{JAVA\_MAXHEAP\_ARGUMENT} now
defaults to the value \Opt{-Xmx1024m}.  The installation process of
Condor resets this value to \Expr{UNDEFINED} in the local
configuration file, if the detected JVM is not from Sun Microsystems.

\item A new feature has been added to the \Condor{master} that makes
it possible to append to the \MacroNI{DC\_DAEMON\_LIST} configuration
variable, instead of overwriting it.  To take advantage of this, place
the plus character ('\verb@+@') as the first character in the
\MacroNI{DC\_DAEMON\_LIST} definition.  For example:
\begin{verbatim}
  DAEMON_LIST     = NEW_DAEMON
  DC_DAEMON_LIST  = +NEW_DAEMON
\end{verbatim}

% PR 959
\item The new configuration variable \Macro{DAGMAN\_COPY\_TO\_SPOOL}
controls whether the \Condor{dagman} binary gets copied to the spool
directory when a DAG is submitted.  See \ref{param:DAGManCopyToSpool}
for details.

% PR 964
\item Added \Opt{-version} and \Opt{-help} command line options to
\Condor{submit\_dag}.

\end{itemize}

\noindent Bugs Fixed:

\begin{itemize}

\item Fixed a bug in the \Condor{collector} which could cause it
to hang indefinitely while reading network input in rare conditions.

\item Fixed a bug in \Condor{chirp} for Windows which was causing it
to crash on invocation.

\item Fixed a bug in the Windows \Condor{mail} program, which was causing
it to become unresponsive when run.  If left running, the application also
increased its memory consumption.

\item Fixed a bug that could cause the \Condor{schedd} to never
evaluate periodic expressions.

\item Fixed a bug on Unix platforms where \Condor{configure} would
provide incorrect defaults for the \MacroNI{JAVA\_MAXHEAP\_ARGUMENT}
attribute in the installed configuration files. The new current
default for Sun Java JVMs is \Opt{-Xmx1024m}.

\item Fixed a bug on Unix platforms where \Condor{configure} would
imply that using the Unix user \Login{root} or UID 0 for the
\Opt{--owner} option is a good thing.  It is not, and would then complain
that it could not find user \Login{root} in the password file.

\item Fixed a bug on Unix platforms where \Condor{configure} would
emit errors about not being able to execute \Prog{ldd} when installing
Condor on the Mac OS X 10.5 platform.  \Condor{configure} now
correctly detects shared library requirements when installing the
Condor binaries on the Mac OS X 10.5 platform.

\item Fixed a bug where execute-side daemons started before the
\Condor{credd} would fail to match with Windows jobs with
\SubmitCmd{run\_as\_owner} set.  This condition persisted until the
execute-side daemons were either restarted or reconfigured.

\item Fixed a problem affecting the Job Router and Condor-C.  When jobs
spool input files, they enter a temporary hold state, which could
trigger actions by a naive periodic remove or release expression.
Periodic expressions are no longer evaluated when in this temporary
hold state, which has the hold reason \AdStr{Spooling input data files}.

\item The example init script \Prog{condor.boot.generic} erroneously claimed
that the \Condor{master} would begin sending SIGKILL to child
processes after 20 seconds if SIGQUIT (the fast shutdown) failed.  The
\Condor{master} will actually wait \MacroUNI{SHUTDOWN\_FAST\_TIMEOUT}
seconds, a value that currently defaults to 300 seconds.

\item Environment variable names are now properly treated as
case-insensitive on Windows. The most common symptom of this bug was
the inability to specify a custom \Env{PATH} environment variable
for a job from its submit description file.

\item Changed \Condor{submit} \Opt{-debug} to issue a warning when ignoring
environment variables. This occurs with \SubmitCmd{getenv = True} set
in a submit description file.

\item Fixed a long-standing memory leak in SOAP interface.
This caused the leak of a few hundred bytes of memory for each connection.
This could eventually have caused the \Condor{schedd} daemon to crash.

\item Fixed Job Router hooks so that their output is properly
propagated where appropriate.

\item Implemented a fix for the \Condor{startd} that prevents it from
crashing if the user specified the configuration variable
\MacroNI{NUM\_SLOTS\_TYPE\_N}, without also specifying \MacroNI{SLOT\_TYPE\_N}.

\item The sample configuration files now correctly set the default
universe to vanilla.  This default has been true since 7.2.0,
but was not reflected in the sample configuration files.

\item Fixed a bug that incorrectly set the value of the
job ClassAd attribute \Attr{RequestMemory} to be 1024 times its
correct size due to a mismatch in units;
the attribute \Attr{RequestMemory} is given in Mbytes, while
the attribute \Attr{ImageSize} is given in Kbytes.

\item Fixed a memory leak in \Condor{dagman} that leaked a small
amount of memory for each job submitted.

\item Fixed a bug that was causing the network mask to be advertised
as a Condor sinful string, rather than a dotted-quad.

\item Fixed a handle leak in the \Condor{procd} on Windows.

\end{itemize}

\noindent Known Bugs:

\begin{itemize}

\item None.

\end{itemize}

\noindent Additions and Changes to the Manual:

\begin{itemize}

\item Added a FAQ entry for Windows describing how machines
with miss-configured performance counters may cause the \Condor{procd}
to crash.

\item Added a manual page for the command \Condor{router\_history}.

\end{itemize}



%%%%%%%%%%%%%%%%%%%%%%%%%%%%%%%%%%%%%%%%%%%%%%%%%%%%%%%%%%%%%%%%%%%%%%
\subsection*{\label{sec:New-7-2-0}Version 7.2.0}
%%%%%%%%%%%%%%%%%%%%%%%%%%%%%%%%%%%%%%%%%%%%%%%%%%%%%%%%%%%%%%%%%%%%%%

\noindent Release Notes:

\begin{itemize}

\item A bug in some older Xen kernels can result in Condor errors
due to a broken assumption in the \Condor{procd} daemon.
See the FAQ entry at section~ \ref{sec:xen-jiffies-bug} for details.

\item A problem has been discovered when using snapshot disks with 
\SubmitCmd{vm} universe VMware jobs,
if the path that the \Condor{vm-gahp} uses to refer to the
virtual machine's VMX file contains a symbolic link.
See the FAQ entry at section~ \ref{sec:vmware-symlink-bug} for details.

\item The name of the Amazon EC2 GAHP binary has changed from
\Prog{amazon-gahp} to \Prog{amazon\_gahp}. This makes it consistent
with the naming of other Condor binaries.

\end{itemize}


\noindent New Features:

\begin{itemize}

\item The default \SubmitCmd{universe} for jobs is now 
\SubmitCmd{vanilla}, instead of \SubmitCmd{standard}.
The default can be changed using the configuration variable
\Macro{DEFAULT\_UNIVERSE}.

\item VMware \SubmitCmd{vm} universe jobs now have any BIOS settings saved in
an \File{nvram} file in the \SubmitCmd{vmware\_dir} given in the
job's submit file transferred to the execute machine, so that they
apply to the job's execution.

\item Daemons that become unresponsive are now killed using the
SIGABRT signal, which causes a core file to be dropped.
Setting the configuration variable \Macro{NOT\_RESPONDING\_WANT\_CORE}
to \Expr{False} will revert to the previous behavior that used
the SIGKILL signal.

\item The \Condor{job\_router} and the
\Condor{q} command with the \Opt{-better-analyze} option now
support more ClassAd functions than they previously did.  They now
support all ClassAd functions, except for those with names beginning
with the string \Code{stringList}.

\item \Condor{status} given the options \Opt{-submitters} \Opt{-xml}
no longer emits a single blank line when there are no submitters,
instead it prints valid XML output with an empty body.

\end{itemize}

\noindent Configuration Variable Additions and Changes:

\begin{itemize}

\item The HAD configuration variable \MacroNI{NEGOTIATOR\_STATE\_FILE}
has changed its name to \MacroNI{STATE\_FILE}.

\end{itemize}

\noindent Bugs Fixed:

\begin{itemize}

\item \Security A flaw was found and fixed that could allow an unauthenticated
user to cause Condor daemons to shut down,
and could allow running jobs to be removed from the queue.

% PR 952
\item Fixed a bug that caused \Condor{dagman} to stay in the Condor
queue, if \Condor{dagman} was accidentally submitted with an empty DAG
input file.

% PR 959
\item \Condor{submit\_dag} now generates a \File{.condor.sub} file with
the submit description file command \SubmitCmd{copy\_to\_spool}
set to \Expr{True}, to ease version upgrades while
large DAGs are running.

\item Fixed a problem in the \Condor{startd} when using
\MacroNI{STARTD\_SLOT\_EXPRS} for attributes that are sometimes
present and sometimes absent from the machine ClassAd.  This is most
typical of attributes that enter the machine ClassAd from the job, via
\MacroNI{STARTD\_JOB\_EXPRS}.  When the attribute went away from slot X
(for example, because the job on slot X finished), the corresponding
\MacroNI{SlotX\_<AttributeName>} attribute was not reliably removed from
all of the other slots.

\item Removed some redundant information from the \Condor{startd} 
advertisements to the \Condor{collector}, 
from within the private ClassAd that is not user-visible.
This fix reduces UDP traffic and memory usage generated by
the \Condor{startd} by about 20\Percent\
in the \Condor{collector} and \Condor{negotiator} daemons.

\item Fixed the \Condor{master} daemon to correctly preserve all command-line
arguments when restarting itself.  In some cases, not preserving \Code{argv[0]}
confused external utilities that monitor the \Condor{master} process by looking
at the output of \Prog{ps} or similar programs.  Also, not preserving
\Opt{-pid} and \Opt{-runfor} could cause unexpected behavior.

\item Fixed a bug that exhibited itself when
the configuration variable \MacroNI{NEGOTIATOR\_CONSIDER\_PREEMPTION}
was set to \Expr{False}, in which jobs
would not be matched to slots in the backfill state.  Corrected, slots doing
backfill are included in the matchmaking process.

\item The \Condor{job\_router} did not work while managing jobs from
multiple users when read access to the \Condor{schedd} required
authentication.  The \Condor{job\_router} was also not able to use
authentication methods other than FS.  Now it can use any
authentication method, as long as the resulting identity is listed in
the configuration variable
\MacroNI{QUEUE\_SUPER\_USERS} or the \Condor{job\_router} and
\Condor{schedd} are running as a Personal Condor in non-root mode.

% Commented out by Karen, as it provides no relevant information
% in the given form.
% \item Fixed a number of memory leaks.

\item Fixed a bug in the \Condor{schedd} daemon that could cause it to write
  an incorrect Unique ID to the event log's header.

\item Fixed a bug in the user log reader API that could cause it to
  incorrectly return a ULOG\_NO\_EVENT in rare cases.

\item Fixed a bug in the user log reader API that could cause it to
  crash if the application attempted to re-initialize the ReadUserLog
  object.  The code now detects this condition, and returns an error
  when the application attempts to re-initialization an already
  initialized ReadUserLog object.

\item Fixed a bug that limited the size of \File{stdin}, \File{stdout},
and \File{stderr} files in the vanilla universe to 2GBytes.

\item Fixed a bug that could cause the \Condor{starter} to EXCEPT upon 
completion or eviction of a \SubmitCmd{vm} universe job.
The error message that appeared in the \File{StarterLog} file was
\begin{verbatim}
  Write_Pipe: invalid pipe end
\end{verbatim}

\item When a held job is removed, the values of the attributes
\Attr{HoldReason}, \Attr{HoldReasonCode} and \Attr{HoldReasonSubCode}
are moved to \Attr{LastHoldReason}, \Attr{LastHoldReasonCode} and
\Attr{LastHoldReasonSubCode}. Before, a hold reason could be lost if a
removed job was subsequently held.

\item The executable attribute for amazon grid universe jobs no longer
needs to be a valid file path.

\item Improved error reporting when a Xen or VMware command fails in the
\SubmitCmd{vm} universe.

\item For \SubmitCmd{vm} universe jobs,
virtual floppy disks are no longer disabled.

\item Fixed a bug introduced in Condor 7.1.4 that caused Condor to
ignore the virtual machine status reported by Xen in the \SubmitCmd{vm} universe.

\item Fixed a 20-second delay in the start up of the \Condor{c-gahp} and
the \Condor{vm-gahp}.

\item Fixed a bug which caused the net mask to be published
  into the machine ClassAd incorrectly.

\item Fixed a bug introduced in Condor 7.1.4 which could cause any
  Condor daemon to crash if the level of debugging output \MacroNI{D\_ALL}
  is enabled when a \Condor{reconfig} command is issued.

\item Fixed a bug introduced in Condor 7.1.4 which caused standard universe
jobs to fail to start up, if security authentication, but not encryption was
enabled between the submit side and the execute side.

% Commented out by Karen, as it gives no relevant information to any
% reader of this version history.  
%\item Many bugs fixed in the \Condor{job\_router} hooks.

\item Fixed a bug with streaming \File{stdin}, \File{stdout}, and
\File{stderr} when using \Prog{glexec}.

% Commented out by Karen, as it gives no relevant information to any
% reader of this version history, and has nothing to do with bugs fixed.
% \item Many improvements in error propagation and debugging output.

\end{itemize}

\noindent Known Bugs:

\begin{itemize}

\item None.

\end{itemize}

\noindent Additions and Changes to the Manual:

\begin{itemize}

\item Initial documentation for dynamic provisioning is available
in section~ \ref{sec:SMP-dynamicprovisioning}.

\item Documentation for Kerberos authentication
(see section~ \ref{sec:Kerberos-Authentication})
and associated configuration variables has been updated.

\end{itemize}


%%%%      PLEASE RUN A SPELL CHECKER BEFORE COMMITTING YOUR CHANGES!
%%%      PLEASE RUN A SPELL CHECKER BEFORE COMMITTING YOUR CHANGES!
%%%      PLEASE RUN A SPELL CHECKER BEFORE COMMITTING YOUR CHANGES!
%%%      PLEASE RUN A SPELL CHECKER BEFORE COMMITTING YOUR CHANGES!
%%%      PLEASE RUN A SPELL CHECKER BEFORE COMMITTING YOUR CHANGES!

%%%%%%%%%%%%%%%%%%%%%%%%%%%%%%%%%%%%%%%%%%%%%%%%%%%%%%%%%%%%%%%%%%%%%%
\section{\label{sec:History-7-1}Development Release Series 7.1}
%%%%%%%%%%%%%%%%%%%%%%%%%%%%%%%%%%%%%%%%%%%%%%%%%%%%%%%%%%%%%%%%%%%%%%

This is the development release series of Condor.
The details of each version are described below.


%%%%%%%%%%%%%%%%%%%%%%%%%%%%%%%%%%%%%%%%%%%%%%%%%%%%%%%%%%%%%%%%%%%%%%
\subsection*{\label{sec:New-7-1-4}Version 7.1.4}
%%%%%%%%%%%%%%%%%%%%%%%%%%%%%%%%%%%%%%%%%%%%%%%%%%%%%%%%%%%%%%%%%%%%%%

\noindent Release Notes:

\begin{itemize}

\item The owner of the log file for the \Condor{vm-gahp}
has changed to the \Login{condor} user.
In Condor 7.1.2 and previous versions, it was owned by the
user that the virtual machine is started under.
Therefore, the owner of and permissions on an existing log file
are likely to be incorrect.
%Condor issues an error if the \Condor{gridmanager} is unable
%to read and write the existing file.
To correct the problem, an administrator may modify file
permissions such that the \Login{condor} user may read and
write the log file.
Alternatively, an administrator may delete the file, and
Condor will create a new file with the expected owner and
permissions.
In addition, the definition for \Macro{VM\_GAHP\_LOG}
in the \File{condor\_config.generic} file has changed for
Condor 7.1.3.

\item The \SubmitCmd{vm} universe no longer supports the use of 
the \SubmitCmd{xm}
command for running Xen virtual machines. The \SubmitCmd{virsh} tool
should be used instead.

\item Condor no longer supports the standard universe feature in its
ports to Solaris. We may resurrect this feature in the future if demand
for it on this port grows again to sufficient levels.

\end{itemize}


\noindent New Features:

\begin{itemize}

\item Local entries in the configuration file may now be specified
by pre-pending a local name and a period to the normal name.
Local settings take precedence over the other settings.
The local name can be specified on the command line to all daemons via
the new \Opt{-local-name} command line option. 

See section~\ref{sec:Config-File-Macros} 
for more details on how the local name will be used in the configuration,
and section~\ref{sec:DaemonCore-Arguments} 
for more details on the command line parameters.

\item Dynamic Startd Provisioning: New configuration options allow for slots
	to be broken into job-sized pieces. While this feature is still under
	ongoing development, we felt that what we had so far, although not yet
	fulfilling our complete vision, is useful enough in its present form to
	bring value to some installations.

% PR 947
\item \Condor{submit\_dag} is now automatically run recursively on
nested DAGs (unless the new \Opt{-no\_recurse} option is specified).
See \pageref{sec:DAGsinDAGs} for details.

% PR 947
\item Added the new \MacroNI{SUBDAG EXTERNAL} keyword (for specifying nested
DAGs) to \Condor{dagman}.  See \pageref{sec:DAGsinDAGs} for details.

\item It is now possible to have multiple rotations of the ``event
  log'' file, such as ``EventLog'', ``EventLog.1'', ``EventLog.2'', ...

\item The VM universe can now run VMware virtual machines on machines using
privilege separation without requiring the \Condor{vm-gahp} binary to be
setuid root. Running the \Condor{vm-gahp} as setuid root is no longer
supported for VMware or Xen.

\item Condor now supports the ability for the \Condor{master} to run a
  program as it shuts down.  This can be particularly useful for doing
  a graceful shutdown, followed by, a reboot.  This is
  accomplished through the new 
  \MacroNI{MASTER\_SHUTDOWN\_$<$Name$>$} configuration variable.
  The configuration variable \MacroNI{MASTER\_SHUTDOWN\_$<$Name$>$}
  is defined on page \pageref{param:MasterShutdownProgram}),
  and the manual page for \Condor{set\_shutdown}
  is on page~\pageref{man-condor-set-shutdown}.

\item The \Condor{lease\_manager} is a new daemon.  It
  provides a mechanism for managing leases to resources described by
  Condor's ClassAd mechanism.  These resources and leases are managed
  to be persistent.

\item VM universe now works with privilege separation (PrivSep)
for VMware jobs. Xen is still not supported in PrivSep mode.

\item Added the \Arg{DIR} directive for the \MacroNI{SPLICE} keyword in
the DAGMan language.
Please read section~\ref{sec:DAGSplicing} on page \pageref{sec:DAGSplicing} for
more information.

\item For gt4 type grid jobs (i.e. WS GRAM), include a request to retry
failed attempts at file clean-up in the RSL job description.

\item Improved the scalability of some algorithms used by the
\Condor{schedd} and \Condor{negotiator} when dealing with large
numbers of startds.

\item Added the ability for the \Condor{master} (actually, any
  DaemonCore process with children) to kill child
  processes that have quit responding SIGABRT instead of SIGKILL.
  This is for debugging purposes on UNIX systems, and is controlled by
  the new \MacroNI{NOT\_RESPONDING\_WANT\_CORE} configuration
  parameter.  If the child process is configured with
  \MacroNI{CREATE\_CORE\_FILES} enabled, the child process will then
  generate a core dump.
  This feature is currently implemented only on UNIX systems.

  See
  \MacroNI{NOT\_RESPONDING\_WANT\_CORE}
  on page \pageref{param:NotRespondingWantCore},
  \MacroNI{NOT\_RESPONDING\_TIMEOUT}
  on page\pageref{param:NotRespondingTimeout},
  and
  \MacroNI{CREATE\_CORE\_FILES}
  on page \pageref{param:CreateCoreFiles}
  for more details.

\item Condor can now be configured to keep a backup of the job queue
  log on a local file system in case \Condor{schedd} operations
  involving writes, flushes, or syncs to the job queue log fail.  This
  is most likely to happen when the job queue log is stored on a
  network file system like NFS. Such a backup enables an administrator
  to see that a job failed to submit, but does not perform any
  automatic recovery.  See below for the these configuration parameters.

\item Added preliminary support for ``Green Computing''.  This is
  supported only on Linux and Windows.
  See section~\ref{sec:power-man} on page~\pageref{sec:power-man} on
  ``Power Management'' for more details.

\end{itemize}

\noindent Configuration Variable Additions and Changes:

\begin{itemize}

\item Local versions of configuration parameters can now be specified
  via the use of the ``-local-name'' command line parameters (see the
  above ``New Features'' entry).

\item A new configuration parameter
  \MacroNI{EVENT\_LOG\_MAX\_ROTATIONS} has been added to allow
  multiple rotations of the event log file.
  See \pageref{param:EventLogMaxRotations} for details.

\item A new configuration parameter
  \MacroNI{EVENT\_LOG\_ROTATION\_LOCK} has been added to allow
  allow configuration of an alternate file for Condor to use while
  rotating event log files.
  See \pageref{param:EventLogRotationLock} for details.

\item The configuration parameter \MacroNI{MAX\_EVENT\_LOG} has been
  renamed to \MacroNI{EVENT\_LOG\_MAX\_SIZE}.  For backward
  compatibility, if \MacroNI{EVENT\_LOG\_MAX\_SIZE} is not defined,
  Condor will also try \MacroNI{MAX\_EVENT\_LOG}.
  See \pageref{param:EventLogMaxSize} for details.

\item The \Condor{vm-gahp} no longer requires its own configuration
  file. It now uses the normal Condor configuration file. Parameters
  that used to reside in the \Condor{vm-gahp}'s file should now be placed
  in the Condor configuration file.

\item The following VM universe-related configuration parameters have
  been removed:
  \begin{itemize}
  \item \MacroNI{VM\_GAHP\_CONFIG}
  \item \MacroNI{VM\_MAX\_MEMORY}
  \item \MacroNI{XEN\_CONTROLLER}
  \item \MacroNI{XEN\_VIF\_PARAMETER}
  \item \MacroNI{XEN\_NAT\_VIF\_PARAMETER}
  \item \MacroNI{XEN\_BRIDGE\_VIF\_PARAMETER}
  \item \MacroNI{XEN\_IMAGE\_IO\_TYPE}
  \end{itemize}

  \MacroNI{VMWARE\_LOCAL\_SETTINGS\_FILE}
  and \MacroNI{XEN\_LOCAL\_SETTINGS\_FILE} have been added. They allow
  a machine administrator to add settings to the virtual machine
  configuration files written by Condor for VMware and Xen.
  See \pageref{param:VMwareLocalSettingsFile} and
  \pageref{param:XenLocalSettingsFile} for details.

\item The configuration parameter family
  \MacroNI{MASTER\_SHUTDOWN\_$<$Name$>$} can be used in conjunction
  with \Condor{set\_shutdown} to cause the \Condor{master} to execute
  a specified program as it shuts down.  See
  \pageref{param:MasterShutdownProgram} and \Condor{set\_shutdown}
  manual page for more details.

\item The configuration parameter
  \MacroNI{NOT\_RESPONDING\_WANT\_CORE} controls the type of signal
  sent to child processes that DaemonCore has determined are no longer
  responding.  See the above discussion of the addition of this
  feature and \MacroNI{NOT\_RESPONDING\_WANT\_CORE} on page
  \pageref{param:NotRespondingWantCore} for details.

\item The configuration parameter \Macro{LOCAL\_QUEUE\_BACKUP\_DIR}
  should be set to the pathname of a directory that is writable by
  the Condor user and is located on a non-network file system.
  This is part of the ``Job Queue Backup'' feature, above.

\item The configuration parameter
  \Macro{LOCAL\_XACT\_BACKUP\_FILTER} controls whether or not the
  \Condor{schedd} will attempt to keep backups of transactions that
  were not written the job queue log.  If it is set to to
  \Expr{FAILED}, the \Condor{schedd} will attempt to keep a backup
  of the transaction in the local queue backup directory,
  defined by \MacroNI{LOCAL\_QUEUE\_BACKUP\_DIR},
  only if operations fail on the job queue log.  If it is set
  to none \Expr{NONE}, no backups should be performed even in the
  event of failure.  If it is set to \Expr{ALL}, then at all
  transactions should be backed up.  The \Expr{ALL} value will
  create quite a large number of files and slow the \Condor{schedd}
  substantially; it is only likely to be useful for users who are
  developing or debugging Condor.
  This is part of the Job Queue Backup feature.

\end{itemize}

\noindent Bugs Fixed:

\begin{itemize}

\item In some rare cases, the \Condor{startd} failed to fully preempt jobs.
The job itself was killed, but the \Condor{starter} process watching over
it would not be killed.  The slot would then stay in the Preempting state
indefinitely.

\item \Condor{q} performed poorly when querying a remote pool, using
\Opt{-pool}.  It was using an older latency-bound protocol even when
the remote \Condor{schedd} was new enough to use the improved protocol
that first appeared in version 6.9.3.

\item When using \Macro{USE\_VISIBLE\_DESKTOP} the user's (slot or owner)
access-control entry removed from the Desktop's access-control list.  This
fixes the previous behavior were users were added and never removed, 
resulting in an overflow in access-control list, which can only contain 
a fixed number of access-control entries.

\item Fixed a bug where if log line caching was enabled in \Condor{dagman}
and \Condor{dagman} failed during the recovery process, the cache would
stay active. Now the cache is disabled in all cases at the end of recovery.

\item Fixed a couple of bugs relevant only to the \Macro{GLEXEC\_STARTER}
mode of operation. One bug would result in the \Macro{SPOOL} directory being
deleted if local universe jobs (which are not supported in
\MacroNI{GLEXEC\_STARTER} mode) were submitted. The other bug prevented
COD jobs from running. Neither of these are problems for the newer
recommended \Macro{GLEXEC\_JOB} mode.

\item Fixed a bug that could cause the \Condor{procd} to crash, depending
on the timing of its process snapshots.

\item Fixed a bug that caused job status notifications from WS GRAM 4.2
servers to be lost.

\item Fixed a file descriptor leak in the \Condor{vm-gahp}.

\item Jobs now go on hold with a clear hold reason if a path to a
directory is put in the transfer files list.  Previously, the attempt
to run the job would simply fail and return to the idle state.

\item If \Macro{MAX\_EVENT\_LOG} set to 0, then let event log grow without 
  bounds.  Previously this behavior was broken, and setting
  \Macro{MAX\_EVENT\_LOG} to 0 resulted in the log rotating with every
  event.  Now it works as documented.

\end{itemize}

\noindent Known Bugs:

\begin{itemize}

\item When fixing the \Macro{USE\_VISIBLE\_DESKTOP} bug, a new one was
inadvertently introduced.  The bug manifests irrespective of the definition
of \Macro{USE\_VISIBLE\_DESKTOP}: the new code attempts to remove the current 
user's access-control entry from the Desktop's access-control list even when
it was not added by Condor.  This has the effect of inhibiting the creation
of new process for the logged on user.

\end{itemize}

\noindent Additions and Changes to the Manual:

\begin{itemize}

\item The extra space character injected into the names of Condor
daemons and programs has been removed.

\item Previously undocumented Condor Perl module subroutines have
been documented.

\end{itemize}



%%%%%%%%%%%%%%%%%%%%%%%%%%%%%%%%%%%%%%%%%%%%%%%%%%%%%%%%%%%%%%%%%%%%%%
\subsection*{\label{sec:New-7-1-3}Version 7.1.3}
%%%%%%%%%%%%%%%%%%%%%%%%%%%%%%%%%%%%%%%%%%%%%%%%%%%%%%%%%%%%%%%%%%%%%%

\noindent Release Notes:

\begin{itemize}

\item This developer release includes the majority of the bug fixes released
	in stable version 7.0.5, including the security patches documented in that
	release.  See section~\ref{sec:New-7-0-5} below.

\item Updated the version of Globus Toolkit: The Condor binaries are now
	linked against Globus v4.2.0.

\item Updated the version of OpenSSL: The Condor binaries are now linked
	against OpenSSL 0.9.8h.

\item Updated the version of GCB: The Condor binaries are now linked
	against GCB 1.5.6.

\item Changes to the \MacroNI{ALLOW\_*} and \MacroNI{DENY\_*} configuration
	variables no longer require the use of the \Opt{-full} option to
	\Condor{reconfig} upon reconfiguration.

\end{itemize}

\noindent New Features:

\begin{itemize}

\item Added a new mechanism termed \Term{Concurrency Limits}.  This
mechanism allows the Condor pool administrator to define an arbitrary
number of consumable resources in the configuration file of the
matchmaker.  The availability of these consumable resources will be taken
into account during the matchmaking process.  Individual jobs can specify
how many of each type of consumable resource is required.  
Typical applications of Concurrency Limits could include management of
software licenses, database connections, or any other consumable resource
that is external to Condor.  NOTE: Documentation still being written on
this feature.
See section~\ref{sec:Concurrency-Limits}) for documentation.

\item Added support for Condor to manage serial high throughput computing
workloads on the IBM Blue Gene supercomputer.  The IBM Blue Gene/P is now
a supported platform.

\item Extended Job Hooks (see section~\ref{sec:job-hooks}) to allow for
alternate transformation and/or monitoring engines for the Job Router (see
section~\ref{sec:JobRouter}.  Routing is still controlled by the Job
Router, but if Job Router Hooks are configured, then external programs or
scripts can be used to transform and monitor the job instead of Condor's
internal engine.

\item Added support for the new protocol for WS GRAM introduced in Globus
4.2. For each WS GRAM resource, Condor automatically determines whether it is
speaking the 4.0 or 4.2 version of the protocol and responds appropriately.
When setting \SubmitCmd{grid\_resource} in the submit file, use
\SubmitCmd{gt4} for both WS GRAM 4.0 and 4.2.

\item Added the ability for Windows slot users to load and run their jobs
within the context of their profile. 
This includes the \File{My Documents} directory 
hierarchy, its monikers, and the user's registry hive.
To use the profile, add a \SubmitCmd{load\_profile} command to the 
submit description file.  A current restriction prevents the use of
\SubmitCmd{load\_profile} 
in conjunction with \SubmitCmd{run\_as\_owner}. Please refer to 
section~\ref{sec:windows-load-profile} for further details.

\item The \File{StarterLog} file for local universe jobs now displays the job id
in each line in the file, so that interleaved messages relevant to
different jobs running concurrently can be identified.

\item Added the \Opt{-AllowVersionMismatch} command line option to
\Condor{submit\_dag} and \Condor{dagman} to (if absolutely necessary)
allow a version mismatch between \Condor{dagman} and the
\File{.condor.sub} file used to submit it.
This permits a Condor version mismatch between
\Condor{submit\_dag} and \Condor{dagman}).

\item Streamlined the protocol between submit and execute machines; in some
instances, fewer messages will be exchanged over the network.

\item When network requests are denied because of the authorization
policy, Condor now logs an explanation in the daemon log that denied
the request.  This helps the administrator understand why the policy
denied the request, in case it is not obvious.  A similar explanation
may be logged for requests that are accepted.  This is only generated
if \Macro{D\_SECURITY} is added to the daemon's debug options.

\end{itemize}

\noindent Configuration Variable Additions and Changes:

\begin{itemize}

\item Added the new configuration variable
\Macro{MAX\_PENDING\_STARTD\_CONTACTS}.  This limits the
number of simultaneous connection attempts by the \Condor{schedd} when
it is requesting claims from the \Condor{startd}s.  The intention is
to protect the \Condor{schedd} from being overloaded by authentication
operations.  The default is 0, which indicates no limit.

\item Added the new configuration variable
\Macro{SEC\_INVALIDATE\_SESSIONS\_VIA\_TCP},  which
defaults to \Expr{True}.  Previously, attempts to use an invalid security
session resulted in a UDP rather than a TCP response.  In networks with
different firewall rules for UDP and TCP, the filtering of the session
invalidation messages was easily overlooked, since it would not
typically happen during the initial vetting of the pool.  If these
packets were filtered out, then at the subsequent \Condor{collector}
restart, no daemons would be able to advertise themselves to the
pool until their existing security sessions expired.  The old behavior
can be achieved by setting this configuration parameter to \Expr{False}.

\item Added the new configuration variable
\Macro{SEC\_ENABLE\_MATCH\_PASSWORD\_AUTHENTICATION}.
This is a special authentication mechanism designed to minimize
overhead in the \Condor{schedd} when communicating with the execute
machine.  Essentially, matchmaking results in a secret being shared
between the \Condor{schedd} and \Condor{startd}, and this is used to
establish a strong security session between the execute and submit
daemons without going through the usual security negotiation protocol.
This is especially important when operating at large scale over high
latency networks, as in a glidein pool with one submit machine and thousands of
execute machines on a network with 0.1 second round trip times.  See
\pageref{param:SecEnableMatchPasswordAuthentication} for
details.

\item Added configuration entry \Macro{GLEXEC\_JOB} which replaces the
functionality previously encapsulated in \Macro{GLEXEC\_STARTER}.  Using
\MacroNI{GLEXEC\_JOB} enables privilege separation in Condor via glexec in a
manner much more consistent with how Condor's own privilege separation
mechanism works.  Specifically, the user identity switching will now occur
between the \Condor{starter} and the actual user job.

\item Added configuration parameter \Macro{AMAZON\_GAHP\_WORKER\_MAX\_NUM}
to specify a ceiling on the number of threads spawned on the submit
machine to support jobs running on Amazon EC2.  Defaults to 5.

\end{itemize}

\noindent Bugs Fixed:

\begin{itemize}

\item Includes bug fixes from Condor v7.0.5, including the security fixes.
	See section~\ref{sec:New-7-0-5}.

\item Fixed a bug in the \Condor{schedd} that would cause it to
except if a crontab entry was incorrectly formatted.

\item Fixed a bug in the CondorView server (collector) that caused it
to except (crash) when it received a machine ClassAd without a valid state.
It now logs this under level \MacroNI{D\_ALWAYS} and ignores the ClassAd.

\item Fixed a bug from Condor version 7.1.2 that would cause 
Condor daemons to start
consuming a lot of cpu time after rare types of communication failures
during security negotiation.

\item Fixed a bug from Condor version 7.1.2 that in rare cases could cause
Condor to fail to recognize when a call to exec() fails on Unix
platforms.

\item Fixed problems with configuration parameter
\Macro{JOB\_INHERITS\_STARTER\_ENVIRONMENT} when using PrivSep.

\item Improved the deletion of Amazon EC2 jobs when the server is
unreachable.

\item Fixed problems with Condor parallel universe jobs when recovering from
	a reboot of the submit machine.

\end{itemize}

\noindent Known Bugs:

\begin{itemize}

\item None.

\end{itemize}

\noindent Additions and Changes to the Manual:

\begin{itemize}

\item None.

\end{itemize}

%%%%%%%%%%%%%%%%%%%%%%%%%%%%%%%%%%%%%%%%%%%%%%%%%%%%%%%%%%%%%%%%%%%%%%
\subsection*{\label{sec:New-7-1-2}Version 7.1.2}
%%%%%%%%%%%%%%%%%%%%%%%%%%%%%%%%%%%%%%%%%%%%%%%%%%%%%%%%%%%%%%%%%%%%%%

\noindent Release Notes:

\begin{itemize}

\item None.

\end{itemize}


\noindent New Features:

\begin{itemize}

\item Added \Procedure{formatTime}, a built-in ClassAd function to create a
  formatted representation of the time.  A detailed description of this
  function is available in section~\ref{sec:classadFunctions}, which
  documents all of the available built-in ClassAd functions.

\item Improved Condor's authentication handshake, so that daemons such
as the \Condor{schedd}, which initiate connections to other daemons,
spend less time waiting for responses.
Authentication over high latency
networks is still rather expensive in Condor, so it still may be
necessary to scale up by running more \Condor{schedd} and \Condor{collector}
daemons than one would need for equivalent workloads on a low latency network.
Additional improvements in this area are planned.

\end{itemize}

\noindent Configuration Variable Additions and Changes:

\begin{itemize}

\item None.

\end{itemize}

\noindent Bugs Fixed:

\begin{itemize}

\item Fixed a memory leak, introduced in Condor version 7.1.1, which caused the
  \Condor{startd} daemon to grow without bound.

% PR 945
\item Fixed a bug in \Condor{dagman} that caused the user log file of
the first node job in a DAG to get created with 0600 permissions,
regardless of the user's umask.  Note that this fix involved removing
the \Opt{-condorlog} and \Opt{-storklog} command-line arguments from
\Condor{submit\_dag} and \Condor{dagman}.

\item Fixed a problem from Condor version 7.1.1 that in some cases caused the
\Condor{starter} to stop sending updates about the job status or
to send updates too frequently.

\end{itemize}

\noindent Known Bugs:

\begin{itemize}

\item None.

\end{itemize}

\noindent Additions and Changes to the Manual:

\begin{itemize}

\item None.

\end{itemize}


%%%%%%%%%%%%%%%%%%%%%%%%%%%%%%%%%%%%%%%%%%%%%%%%%%%%%%%%%%%%%%%%%%%%%%
\subsection*{\label{sec:New-7-1-1}Version 7.1.1}
%%%%%%%%%%%%%%%%%%%%%%%%%%%%%%%%%%%%%%%%%%%%%%%%%%%%%%%%%%%%%%%%%%%%%%

\noindent Release Notes:

\begin{itemize}

\item None.

\end{itemize}


\noindent New Features:

\begin{itemize}

\item Added a new feature to \Condor{dagman} which caches the log lines
emitted to the dagman.out file when in recovery mode and emits the
cache as one call to the logging subsystem when the cache size limit is
reached. Under NFS conditions, this prevents an open and close per line
of the log and greatly improves performance. This feature is off by
default and is controlled by \Attr{DAGMAN\_DEBUG\_CACHE\_ENABLE}, which
takes a boolean, and  \Attr{DAGMAN\_DEBUG\_CACHE\_SIZE}, which is an
integer in bytes of how big the cache should be before flushing.

\item Included some Windows example jobs (submit files and binaries).

\item Added a new feature to the DAGMan language called splicing. Please
read section~\ref{sec:DAGSplicing} on page \pageref{sec:DAGSplicing}.

\item The Prepare Job Hook can now modify the job ClassAd before execution.
For a complete description of the new hook system, read
section~\ref{sec:job-hooks} on page~\pageref{sec:job-hooks}.

\item Condor now coerces the result of \$\$([]) expressions within
submit description files to strings.
This means that submit files can do simple arithmetic.
For example, you can describe a command-line argument as:

arguments = \$\$([\$(PROCESS)+100])

and \Condor{submit} will expand the argument to be the expected value.

\item Condor daemons now periodically update the \Code{ctime} of their
  log files, instead of the \Code{mtime}, as they previously did.
  At start up, the daemons use this \Code{ctime} 
  to determine how long they may have been down.

\item Added the capability to the \Condor{startd} to allow it to power 
  down machines based a user specified policy.  See 
  section~\ref{sec:power-man} on \pageref{sec:power-man} on
  Power Management for more details.

\item \Condor{off} now supports the \Opt{-peaceful} option for the
  \Condor{schedd}, in addition to the existing support that already existed for
  the \Condor{startd}.  When peacefully shut down,
  the \Condor{schedd} stops starting new
  jobs and waits for all running jobs to finish before exiting.  The
  default shut down behavior is still \Opt{-graceful}, which checkpoints
  and stops all running standard universe jobs and gracefully
  disconnects from other types of jobs in the hopes of later restarting
  and reconnecting to them without any disturbance to the running job.

\item The \Condor{job\_router} now supports deletion of attributes
  when transforming job ClassAds from vanilla to grid universe.  It also
  behaves more deterministically when choosing from multiple possible
  routes.  Rather than picking one at random, it uses a round-robin
  selection.

% PR 941
\item \Condor{dagman} now checks that its submit file was generated by
a \Condor{submit\_dag} with the same version as \Condor{dagman} itself.
It is a fatal error for the versions to differ.

\end{itemize}

\noindent Configuration Variable Additions and Changes:

\begin{itemize}

\item Added \Attr{DAGMAN\_DEBUG\_CACHE\_ENABLE} and 
  \Attr{DAGMAN\_DEBUG\_CACHE\_SIZE} which allow DAGMan to maintain a
  cache of log lines and write out the cache as one open/write/close
  sequence.  \Attr{DAGMAN\_DEBUG\_CACHE\_ENABLE} is a boolean
  which turns on the ability for caching and defaults to \Expr{False}.
  \Attr{DAGMAN\_DEBUG\_CACHE\_SIZE} is a positive integer and represents
  the size of the cache in bytes and defaults to 5 Megabytes.

\item The existing \Macro{BIND\_ALL\_INTERFACES} configuration variable
  now defaults to \Expr{True}.

\item Added the \Macro{HIBERNATE} expression, which, when evaluated in
  the context of each slot, determines if a machine should enter
  a low power state. See page~\pageref{param:Hibernate} for more 
  information.

\item Added the \Macro{HIBERNATE\_CHECK\_INTERVAL} configuration variable,
  which, if set to a non-zero value, enables the \Condor{startd} to place the 
  machine in a low power state based on the evaluation of the
  \MacroNI{HIBERNATE} expression.  See 
  page~\pageref{param:HibernateCheckInterval} for more information.

\item The existing \Macro{VALID\_SPOOL\_FILES} configuration variable
  now automatically includes \File{SCHEDD.lock},
  the lock file used for high availability \Condor{schedd} fail over.
  Other high availability lock files are not currently included.

\item Added the \Macro{SEC\_DEFAULT\_AUTHENTICATION\_TIMEOUT} configuration
  variable, where the definition \Expr{DEFAULT} may be replaced
  by the usual list of contexts for security settings
  (for example, \Expr{CLIENT}, \Expr{READ}, and \Expr{WRITE}).
  This specifies the number of seconds that Condor should
  allow for the authentication of network connections to complete.
  Previously, GSI authentication was hard-coded to allow 5 minutes
  for authentication.
  Now it uses the same default as all other methods: 20 seconds.

\item Added the \Macro{STARTER\_UPDATE\_INTERVAL\_TIMESLICE} configuration
  variable, which
  specifies the highest fraction of time that the \Condor{starter} should spend
  collecting monitoring information about the job, such as disk usage.
  It defaults to 0.1.  If checking the disk usage of the job takes a
  long time, the \Condor{starter} will monitor less frequently than 
  specified by \MacroNI{STARTER\_UPDATE\_INTERVAL}.

\end{itemize}

\noindent Bugs Fixed:

\begin{itemize}

\item Fixed a bug introduced in 7.1.0 affecting configurations in
which authentication of all communication between the \Condor{shadow}
and \Condor{schedd} is required.  This caused failure in the final update
after the job had finished running.  The result was that the job would return
to the idle state to run again.

\item Fixed a bug in Java universe where each slot would be told to
  potentially use all the memory on the machine.  Now, each JVM 
  receives the physical memory divided by the number of slots.

\item On Windows, slot users would sometimes show up in the Windows Welcome
  Screen.  This has now been resolved.
  The slot users need to be manually
  removed for this to take effect and the machine may need to be rebooted for
  the setting to be honored.

\item Fixed a bug in the ClassAd \Procedure{string} function.
  The function now properly converts integers and floats
  to their string representation.

\item The Windows Installer is now completely internationalized: it will no 
  longer fail to install because of a missing "Users" group; instead, it
  will use the regionally appropriate group.

\item Interoperability with Samba (as a PDC) has been improved.  Condor 
  uses a fast form of login during credential validation.  Unfortunately, 
  this login procedure fails under Samba, even if the credentials are 
  valid.  The new behavior is to attempt the fast login, and on failure, 
  fall back to the slower form.

\item Windows slot users no longer have the Batch Privilege added, nor 
  does Condor first attempt a Batch login for slot users.  This was 
  causing permission problems on hardened versions of Windows, such 
  as Windows Sever 2003, in that not interactive users lacked the 
  permission to run batch files (via the \Prog{cmd.exe} tool). This affected 
  any user submitting jobs that used batch files as the executable.

% issue [#1516]
\item If the \AdAttr{IWD} is not defined in a job classified
  ad that was either fetched by the \Condor{startd} via job hooks, or
  pushed to the \Condor{startd} via COD, the \Condor{starter} no
  longer treats this as a fatal error, and instead uses the temporary
  job execution sandbox as the initial working directory.

% Fixes requested by LIGO
\item Made some fixes to the new-style rescue DAG feature:
\begin{itemize}
\item \Condor{submit\_dag} no longer needs the \Opt{-force} flag if a rescue
DAG will be run, even if the files generated by \Condor{submit\_dag}
already exist.
\item \Condor{submit\_dag} with the \Opt{-force} flag now renames any
existing new-style rescue DAG files, and therefore runs the original DAG.
\end{itemize}

% PR 942
\item Fixed a problem that caused new-style rescue DAGs to fail when
\Condor{submit\_dag} is invoked with the \Opt{-usedagdir} flag.

\end{itemize}

\noindent Known Bugs:

\begin{itemize}

\item None.

\end{itemize}

\noindent Additions and Changes to the Manual:

\begin{itemize}

\item The manual now contains Windows installation instructions for
  controlling the configuration for the \SubmitCmd{vm} universe.

\end{itemize}



%%%%%%%%%%%%%%%%%%%%%%%%%%%%%%%%%%%%%%%%%%%%%%%%%%%%%%%%%%%%%%%%%%%%%%
\subsection*{\label{sec:New-7-1-0}Version 7.1.0}
%%%%%%%%%%%%%%%%%%%%%%%%%%%%%%%%%%%%%%%%%%%%%%%%%%%%%%%%%%%%%%%%%%%%%%

\noindent Release Notes:

\begin{itemize}

\item Upgrading to 7.1.0 from previous versions of Condor will make
existing Standard Universe jobs that have already run fail to match to
machines running Condor 7.1.0 unless the job previously ran on a
machine using the Red Hat 5.0 release of Condor.  This is because the
value of the \Attr{CheckpointPlatform} attribute of the machine
ClassAd has changed in order to better represent checkpoint
compatibility.  If this affects you, you can use \Condor{qedit} to
change the \Attr{LastCheckpointPlatform} attribute of existing
Standard Universe jobs to match the new \Attr{CheckpointPlatform}
advertised by the machine ClassAd where the job last ran.

\item Condor no longer supports root configuration files
(for example, \File{/etc/condor/condor\_config.root},
\File{~condor/condor\_config.root}, and
the file defined by the configuration variable
\MacroNI{LOCAL\_ROOT\_CONFIG\_FILE}).  This feature was intended to
give limited powers to a Unix administrator to configure some aspects
of Condor without gaining root powers.  However, given the flexibility
of the configuration system, we decided that this was not practical.
As long as Condor is started up as root, it should be clearly
understood that whoever has the ability to edit the Condor
configuration files can effectively run arbitrary programs as root.

\end{itemize}


\noindent New Features:

\begin{itemize}

\item In the past, Condor has always sent work to the execute machines
  by pushing jobs to the \Condor{startd}, either from the
  \Condor{schedd} or via \Condor{cod}.
  As of version 7.1.0, The \Condor{startd} now has the ability to pull
  work by fetching jobs via a system of plug-ins or hooks.
  Additional hooks are invoked by the \Condor{starter} to help manage
  work (especially for fetched jobs, but the \Condor{starter} hooks
  can be defined and invoked for other kinds of jobs as well).
  For a complete description of the new hook system, read
  section~\ref{sec:job-hooks} on page~\pageref{sec:job-hooks}.

% PR 888/921
\item Added the capability to insert commands into the \File{.condor.sub}
  file produced by \Condor{submit\_dag} with the \Opt{-append} and
  \Opt{-insert\_sub\_file} command-line arguments to \Condor{submit\_dag} and
  the \Macro{DAGMAN\_INSERT\_SUB\_FILE} configuration variable.
  See the \Condor{submit\_dag} manual page on
  page~\pageref{man-condor-submit-dag}
  and the configuration variable definition on
  page~\pageref{param:DAGManInsertSubFile} for more information.

\item For platforms running a Windows operating system, the \Attr{Arch}
  machine ClassAd attribute more correctly reflects the architectures
  supported.  Instead of values \AdStr{INTEL} and \AdStr{UNDEFINED},
  the values will now be: \AdStr{INTEL} for x86,
  \AdStr{IA64} for Intel Itanium,
  and \AdStr{X86\_64} for both AMD and Intel 64-bit processors.
  These values are listed in the unnumbered subsection labeled
  Machine ClassAd Attributes on page~\pageref{sec:Machine-ClassAd-Attributes}.

\item The Windows MSI installer now supports extended \SubmitCmd{vm} universe 
  options. These new options include: the ability to set the 
  networking type, how much memory the \SubmitCmd{vm} universe can use 
  on a host, and
  the ability to set the version of \Prog{VMware} installed on the host.

\item The \Condor{status} and \Condor{q} command line tools now have a
  version option which prints the version of those specific tools.  This
  can be useful when multiple versions of Condor are installed on the
  same machine.

\item The configuration variable \MacroNI{CONDOR\_VIEW\_HOST} may now
  contain a port number and may (if desired) refer to a
  \Condor{collector} daemon running on the same host as the
  \Condor{collector} that is forwarding ads.  It is also now possible to
  use the forwarded ads for matchmaking purposes.  For example, several
  collectors could forward ads to a single aggregating collector which
  a \Condor{negotiator} then uses as its source of information for
  matchmaking.

% PR 598/788
\item \Condor{dagman} deals with rescue DAGs in a more sophisticated
way; this is especially helpful for nested DAGs.
See the rescue DAG subsection~\pageref{sec:DAGMan-rescue} of the \Condor{dagman}
manual section for more information.

\item Additional logging details for unusual error cases to help 
identify problems.

\item A new (optional) daemon named \Condor{job\_router} has been
added, so far only on Unix.  It may be configured to transform vanilla
universe jobs into grid universe jobs, for example to send excess jobs
to other sites via Condor-C or Condor-G.  For details, see
page~\pageref{sec:JobRouter}.

\item Previously, \condor{q} \Opt{-better-analyze} was supported on most
but not all versions of Linux.  It is now supported on all Unix platforms
but not yet on Windows.

\end{itemize}

\noindent Configuration Variable Additions and Changes:

\begin{itemize}

\item Added new configuration variables
  \MacroNI{ALLOW\_CLIENT} and \MacroNI{DENY\_CLIENT} as
  client-side authorization controls.
  When using a mutual authentication method (such as GSI, SSL, or Kerberos),
  these variables allow the specification of
  which authenticated servers the Condor tools and daemons should
  trust when they form a connection to the server.
  Because of the addition of these variables,
  the GSI-specific, client-side authorization configuration variable
  \Macro{GSI\_DAEMON\_NAME} is retired, and no longer valid.

% PR 921
\item Added the \Macro{DAGMAN\_INSERT\_SUB\_FILE} variable, which allows a file
  of commands to be inserted into \File{.condor.sub} files generated
  by \Condor{submit\_dag}.  See page~\pageref{param:DAGManInsertSubFile}
  for more information.

\item The semantics of \MacroNI{CLAIM\_WORKLIFE} were previously not
clearly defined before the start of the first job.  A delay between
the \Condor{schedd} claiming a slot and the \Condor{shadow} starting a
job could be caused by the submit machine being very busy or by
\MacroNI{JOB\_START\_DELAY}.  Previously, such a delay would
unpredictably result in the first job being rejected if
\MacroNI{CLAIM\_WORKLIFE} expired during that time.  Now,
\MacroNI{CLAIM\_WORKLIFE} is defined to apply only after the first job
has started.  Therefore, setting it to zero has the effect of allowing
exactly one job per claim to run.  The default is still the special
value -1, which places no limit on how long the slot may continue
accepting new jobs from the \Condor{schedd} that claimed it.

% PR 598/788
\item Added the \Macro{DAGMAN\_OLD\_RESCUE} variable, which controls whether
\Condor{dagman} writes rescue DAGs in the old way.  See
page~\pageref{param:DAGManOldRescue} for more information.

% PR 598/788
\item Added the \Macro{DAGMAN\_AUTO\_RESCUE} variable, which controls
whether \Condor{dagman} automatically runs an existing rescue DAG.
See page~\pageref{param:DAGManAutoRescue} for more information.

% PR 598/788
\item Added the \Macro{DAGMAN\_MAX\_RESCUE\_NUM} variable, which
controls the maximum "new-style" rescue DAG number written or
automatically run by \Condor{dagman}.
See page~\pageref{param:DAGManMaxRescueNum} for more information.

\end{itemize}

\noindent Bugs Fixed:

\begin{itemize}

\item The Condor Build ID is now printed by \Condor{version} and placed 
  in the logs for machines running a Windows operating system.

\item \Condor{quill} and the \Condor{dbmsd} correctly register 
  themselves with the Windows firewall.

% PR 926
\item \Condor{submit\_dag} now avoids possibly running off the end
of the argument list if an argument requiring a value does not have one.

\item The \Condor{submit\_dag} \Opt{-debug} argument now must be
specified with at least \Opt{-de} to avoid conflict with the
\Opt{-dagman} argument.

\item Added missing information about the \Opt{-config} argument to
\Condor{submit\_dag}'s usage message.

% PR 927
\item \Condor{dagman} no longer considers duplicate edges in a DAG a
fatal error (it is now a warning).

\end{itemize}

\noindent Known Bugs:

\begin{itemize}

\item No hook is invoked if a fetched job does not contain enough data
  to be spawned by a \Condor{starter} or if other errors prevent the
  job from being run after the \Condor{startd} agrees to accept the
  work.
  This limitation will be addressed in a future version of Condor,
  most likely via the addition of a new hook invoked whenever the
  \Condor{starter} fails to spawn a job.
  For more information about the new hook system included in Condor
  version 7.1.0, read section~\ref{sec:job-hooks} on
  page~\pageref{sec:job-hooks}.

\end{itemize}

\noindent Additions and Changes to the Manual:

\begin{itemize}

\item Added \AdStr{WINNT60} for the Vista operating system to
  the documented list of possible values for the machine ClassAd
  attribute \AdAttr{OpSys}.

\end{itemize}

%%%%      PLEASE RUN A SPELL CHECKER BEFORE COMMITTING YOUR CHANGES!
%%%      PLEASE RUN A SPELL CHECKER BEFORE COMMITTING YOUR CHANGES!
%%%      PLEASE RUN A SPELL CHECKER BEFORE COMMITTING YOUR CHANGES!
%%%      PLEASE RUN A SPELL CHECKER BEFORE COMMITTING YOUR CHANGES!
%%%      PLEASE RUN A SPELL CHECKER BEFORE COMMITTING YOUR CHANGES!

%%%%%%%%%%%%%%%%%%%%%%%%%%%%%%%%%%%%%%%%%%%%%%%%%%%%%%%%%%%%%%%%%%%%%%
\section{\label{sec:History-7-0}Stable Release Series 7.0}
%%%%%%%%%%%%%%%%%%%%%%%%%%%%%%%%%%%%%%%%%%%%%%%%%%%%%%%%%%%%%%%%%%%%%%

This is a stable release series of Condor.
It is based on the 6.9 development series.
All new features added or bugs fixed in the 6.9 series are available
in the 7.0 series.
As usual, only bug fixes (and potentially, ports to new platforms)
will be provided in future 7.0.x releases.
New features will be added in the 7.1.x development series.

The details of each version are described below.

%%%%%%%%%%%%%%%%%%%%%%%%%%%%%%%%%%%%%%%%%%%%%%%%%%%%%%%%%%%%%%%%%%%%%%
\subsection*{\label{sec:New-7-0-0}Version 7.0.0}
%%%%%%%%%%%%%%%%%%%%%%%%%%%%%%%%%%%%%%%%%%%%%%%%%%%%%%%%%%%%%%%%%%%%%%

\noindent Release Notes:

\begin{itemize}

\item PVM support has been dropped.

\item The time zone for the \Prog{PostgreSQL} 8.2 database
  used with Quill on Windows machines must be explicitly set
  to use an abbreviation.
  This Windows environment variable is \verb@TZ@.
  Proper abbreviations for the value of this variable may be found
  within the \Prog{PostgreSQL} installation in a file,
  \File{share/timezonesets/<continent>.txt}, where
  \verb@<continent>@ is replaced by the continent of the 
  desired time zone.

\end{itemize}


\noindent New Features:

\begin{itemize}

\item The Windows MSI installer now supports VM Universe.

\item Eliminated the ``tarball in a tarball'' in our distribution.
  The contents of \File{release.tar} from the distribution tarball
  (for example, \File{condor-6.9.6-linux-x86-centos45-dynamic.tar.gz}) is now
  included in the distribution tarball.

\item Updated \Condor{configure} to match the above change.  The
  \Opt{--install} option now takes a directory path as its parameter,
  for example \Opt{--install=/path/to/release}.
  It previously took the path to
  the \File{release.tar} tarball.

\item Added \Condor{install}, which is a symlink to \Condor{configure}.
  Invoking 
\begin{verbatim}
    condor_install
\end{verbatim}
  is identical to running
\begin{verbatim}
    condor_configure --install=.
\end{verbatim}

\item Added the option \Opt{--prefix=dir} to \Condor{configure} and
  \Condor{install}.  This is an alias for \Opt{--install-dir=dir}.

\item Added the option \Opt{--backup} option to \Condor{configure} and
  \Condor{install}.  This option renames the target \File{sbin} directory,
  if the \Condor{master} daemon exits while in the target \File{sbin} directory.
  Previous versions of \Condor{configure} did this by default.

\item Changed the default behavior of \Condor{install} to exit with a
  warning if the target \File{sbin} directory exists,
  the \Condor{master} daemon is in the \File{sbin} directory,
  and neither the \Opt{--backup} nor \Opt{--overwrite} options are specified.
  This prevents \Condor{install} from improperly moving an \File{sbin}
  directory out of the way.
  For example,
\begin{verbatim}
    condor_install --prefix=/usr
\end{verbatim}
  will not move \File{/usr/sbin} out of the way unless
  the \Opt{--backup} option is also specified.

\item Updated the usage summary of \Condor{configure} and
  \Condor{install} to be much more readable.

\end{itemize}

\noindent Configuration Variable Additions and Changes:

\begin{itemize}

\item The new configuration variable
  \Macro{DEAD\_COLLECTOR\_MAX\_AVOIDANCE\_TIME} defines the maximum
  time in seconds that a daemon will fail over from a primary
  \Condor{collector} to a secondary \Condor{collector}.
  See section~\ref{param:DeadCollectorMaxAvoidanceTime} on
  page~\pageref{param:DeadCollectorMaxAvoidanceTime} for a
  complete definition.

\end{itemize}

\noindent Bugs Fixed:

\begin{itemize}

\item Fixed a memory leak in the \Condor{procd} daemon on Windows.

\item Fixed a problem that could cause Condor daemons to crash if a
  failure occurred when communicating with the \Condor{procd}.

\item Fixed a couple of problems that were preventing the
  \Condor{startd} from properly removing per-job directories
  when running with PrivSep.

\item The \Condor{startd} will no longer fail to initialize, 
  claiming the \MacroNI{EXECUTE} directory has improper permissions,
  when PrivSep is enabled.

\item Lookups of ClassAd attribute \Attr{CurrentTime} are now
  case-insensitive, just like all other attributes.

\item Fixed problems causing the following error message in the log file:

\footnotesize
\begin{verbatim}
ERROR: receiving new UDP message but found a short message still waiting to be closed (consumed=1). Closing it now.
\end{verbatim}
\normalsize

\item The existence of the executable given in the submit file is now 
  enforced (when transferring the executable and not using VM 
  universe).

\item The copy of \Condor{dagman} that ships with Condor is now automatically 
  added to the list of trusted programs in the Windows Firewall.

\item Removed \SubmitCmd{remove\_kill\_sig} from the submission file
  generated by \Condor{submit\_dag} on Windows.

\item Fixed the algorithm in the \Condor{negotiator} daemon, which
  with large numbers of machine ClassAds (for example, 10,000) 
  was causing long delays at the 
  beginning of each negotiation cycle.

\item Use of \MacroNI{MAX\_CONCURRENT\_UPLOADS} was resulting in a
  connection attempt from the \Condor{shadow} to the \Condor{schedd} with a
  fixed 10 second timeout, which is sometimes too small.  This timeout
  has been increased to be the same as other connection timeouts between
  the \Condor{shadow} and the \Condor{schedd}, and it now respects
  \MacroNI{SHADOW\_TIMEOUT\_MULTIPLIER}, so it can be adjusted if necessary.

\item Fixed a problem with \Macro{MAX\_CONCURRENT\_UPLOADS} and
  \Macro{MAX\_CONCURRENT\_DOWNLOADS}, which was sometimes allowing
  more than the configured number of concurrent transfers to happen.

\item Fixed a bug in the \Condor{schedd} that could cause it to crash due
  to file descriptor exhaustion when trying to send messages to hundreds of
  \Condor{startd}s simultaneously.

\item Fixed a 6.9.4 bug in the \Condor{startd} that would cause it to crash
  when a BOINC backfill job exited.

\item Since 6.9.4, when using glExec, configuring \MacroNI{SLOTx\_EXECUTE}
  would cause \Condor{starter} to fail when starting the job.

\item Fixed a bug from 6.9.5 which caused authentication failure for
  the pool password authentication method.

\item Fixed a bug that caused Condor daemons to crash when encountering
  some types of invalid ClassAd expressions.

\item Fixed a bug under Linux that could cause multi-process daemons
  lacking a log lock file to crash while rotating logs that have reached
  their maximum configured size.

\item Fixed a bug under Windows that sometimes caused connection attempts
  between Condor daemons to fail with Windows error number 10056.

\item Fixed a problem in which there are multiple 
  \Condor{collector} daemons in a pool
  for fault tolerance.  If the primary \Condor{collector} failed, the
  \Condor{negotiator} would fail over to the secondary \Condor{collector}
  indefinitely (or until the secondary \Condor{collector} also failed or the
  administrator ran \Condor{reconfig}).  This was a problem for users
  flocking jobs to the pool, because flocking currently only works with
  the primary \Condor{collector}.  Now, the \Condor{negotiator} will fail over
  for a restricted amount of time, up to
  \Macro{DEAD\_COLLECTOR\_MAX\_AVOIDANCE\_TIME} seconds.  The default
  is one hour, but if querying the dead primary \Condor{collector}
  takes very little
  time to fail, the \Condor{negotiator} may retry more frequently
  in order to remain
  responsive to flocked users.

\item Fixed a problem preventing the use of \Condor{q} \Opt{-analyze}
  with the \Opt{-pool} option.

\item Fixed a problem in the \Condor{negotiator} in which machines go
  unassigned when user priorities result in the machines getting split
  into shares that are rounded down to 0.  For example if there are 10
  machines and 100 equal priority submitters, then each submitter was
  getting 0.1 machines, which got rounded down to 0, so no machines were
  assigned to anybody.  The message in the \Condor{negotiator} log in this case
  was this:

\footnotesize
\begin{verbatim}
Over submitter resource limit (0) ... only consider startd ranks
\end{verbatim}
\normalsize

\item Fixed a problem introduced in 6.9.3 that would cause daemons to
  run out of file descriptors if they create sub-processes and are
  configured to use a lock file for the debug log.

\item Standard universe jobs now work properly when using PrivSep.

\item Fixed problem with PrivSep mode where a job that dumps core would
  not get the core file transferred back to the the submit host if the
  \SubmitCmd{transfer\_output\_files} submit option were used.

\item Fixed a bug that caused the \Condor{starter} to crash if a job
called \Condor{chirp} with the \Expr{get\_job\_attr} option.

\end{itemize}

\noindent Known Bugs:

\begin{itemize}

\item None.

\end{itemize}

\noindent Additions and Changes to the Manual:

\begin{itemize}

\item None.

\end{itemize}


% Oct 2009, as we release 7.4, Karen commented out inclusion of the
% 6.9 and 6.8 histories
%%%%      PLEASE RUN A SPELL CHECKER BEFORE COMMITTING YOUR CHANGES!
%%%      PLEASE RUN A SPELL CHECKER BEFORE COMMITTING YOUR CHANGES!
%%%      PLEASE RUN A SPELL CHECKER BEFORE COMMITTING YOUR CHANGES!
%%%      PLEASE RUN A SPELL CHECKER BEFORE COMMITTING YOUR CHANGES!
%%%      PLEASE RUN A SPELL CHECKER BEFORE COMMITTING YOUR CHANGES!

%%%%%%%%%%%%%%%%%%%%%%%%%%%%%%%%%%%%%%%%%%%%%%%%%%%%%%%%%%%%%%%%%%%%%%
\section{\label{sec:History-6-9}Development Release Series 6.9}
%%%%%%%%%%%%%%%%%%%%%%%%%%%%%%%%%%%%%%%%%%%%%%%%%%%%%%%%%%%%%%%%%%%%%%

This is the development release series of Condor.
The details of each version are described below.

%%%%%%%%%%%%%%%%%%%%%%%%%%%%%%%%%%%%%%%%%%%%%%%%%%%%%%%%%%%%%%%%%%%%%%
\subsection*{\label{sec:New-6-9-3}Version 6.9.3}
%%%%%%%%%%%%%%%%%%%%%%%%%%%%%%%%%%%%%%%%%%%%%%%%%%%%%%%%%%%%%%%%%%%%%%

\noindent Release Notes:

\begin{itemize}

\item The \Macro{SECONDARY\_COLLECTOR\_LIST} configuration variable has
  been removed.
  Sites relying on this variable should instead use the configuration
  variable \Macro{COLLECTOR\_HOST}. It may be used to
  define a list of \Condor{collector} daemon hosts.

\item Cleaned up and improved help information for \Condor{history}.

\end{itemize}


\noindent New Features:

\begin{itemize}

% condor-admin 15254
\item Added new configuration knob \Macro{STARTER\_UPLOAD\_TIMEOUT}
which sets the timeout for the starter to upload output files to the
shadow on job exit.  The default value is 30 seconds, which should
be sufficient for serial jobs.  For parallel jobs, this may need to
be increased if many large output files are sent back to the shadow
on job exit.

% Gnats PR 806
\item \Condor{dagman} now aborts the DAG on ``scary'' submit events.
These are submit events in which
the Condor ID of the event does not match the
expected value.
Previously, \Condor{dagman} printed a warning, but continued.
To restore Condor to the previous behavior,
set the new \MacroNI{DAGMAN\_ABORT\_ON\_SCARY\_SUBMIT} configuration variable
to \Expr{False}.

\item When the \Condor{master} detects that its GCB broker is unavailable
and there is a list of alternative brokers,
it will restart immediately if \Macro{MASTER\_WAITS\_FOR\_GCB\_BROKER} is
set to \Expr{False} instead of waiting for another broker to became available.
\Condor{glidein} now sets \MacroNI{MASTER\_WAITS\_FOR\_GCB\_BROKER}
to  \Expr{False} in its configuration file.

\item When using GCB and a list of brokers is available, the
\Condor{master} will now pick a random broker rather than the least-loaded
one.

\item All Condor daemons now evaluate some ClassAd expressions
  whenever they are about to publish an update to the
  \Condor{collector}.
  Currently, the two supported expressions are:
  \begin{description}
  \item[\Macro{DAEMON\_SHUTDOWN}]
    If \Expr{True}, the daemon will gracefully shut itself down and will not
    be restarted by the \Condor{master} (as if it sent itself a
    \Condor{off} command).
  \item[\Macro{DAEMON\_SHUTDOWN\_FAST}]
    If \Expr{True}, the daemon will quickly shut itself down and will not be
    restarted by the \Condor{master} (as if it sent itself a
    \Condor{off} command using the \Opt{-fast} option).
  \end{description}
  For more information about these expressions, see
  section~\ref{param:DaemonShutdown} on
  page~\pageref{param:DaemonShutdown}.

\item When the \Condor{master} sends email announcing that another daemon has
died, exited, or been killed, it now notes the name of the machine, the
daemon's name, and a summary of the situation in the Subject line.

\item Anyplace in a Condor configuration or submit description file where
wildcards may be used, you can now place wildcards at both the beginning
and end of the string pattern (i.e. match strings that contain the text
between the wildcards anywhere in the string). Previously, only one
wildcard could appear in the string pattern.

\item Added optional configuration setting
\MacroNI{NEGOTIATOR\_MATCH\_EXPRS}.  This allows the negotiator to
insert expressions into the matched ClassAd.  See
page~\pageref{param:NegotiatorMatchExprs} for more information.

\item Increased speed of ClassAd parsing.

\item Added \MacroNI{DEDICATED\_EXECUTE\_ACCOUNT\_REGEXP} and
deprecated the boolean setting
\MacroNI{EXECUTE\_LOGIN\_IS\_DEDICATED}, because the latter could not
handle a policy where some jobs run as the job owner and some run as
dedicated execution accounts.  Also added support for
\MacroNI{STARTER\_ALLOW\_RUNAS\_OWNER} under unix.  See
Section~\ref{param:DedicatedExecuteAccountRegexp} and
Section~\ref{sec:RunAsNobody} for more information.

\end{itemize}

\noindent Bugs Fixed:

\begin{itemize}

\item On unix systems, Condor can now handle file descriptors larger than
FD\_SETSIZE when using the select system call. Previously, file descriptors
largers than FD\_SETSIZE would cause memory corruption and crashes.

\item When an update to the \Condor{collector} from the
\Condor{startd} is lost, it is possible for multiple claims to the
same resource to be handed out by the \Condor{negotiator}.  This is
still true.  What is fixed is that these multiple claims will not
result in mutual annihilation of the various attempts to use the
resource.  Instead, the first claim to be successfully requested will
proceed and the others will be rejected.

\item \Condor{glidein} was setting \MacroNI{PREEN\_INTERVAL}=0 in the default
configuration, but this is no longer a legal value, as of 6.9.2.

\item \Condor{glidein} was not setting necessary configuration parameters
for \Condor{procd} in the default glidein configuration.

\item On Windows systems, the truncate function no longer leaks file handles.

\item \Condor{condor_quill} project was change from an executable project
to a static library project, allowing the Windows build to work.

\end{itemize}

\noindent Known Bugs:

\begin{itemize}

\item None.

\end{itemize}

%%%%%%%%%%%%%%%%%%%%%%%%%%%%%%%%%%%%%%%%%%%%%%%%%%%%%%%%%%%%%%%%%%%%%%
\subsection*{\label{sec:New-6-9-2}Version 6.9.2}
%%%%%%%%%%%%%%%%%%%%%%%%%%%%%%%%%%%%%%%%%%%%%%%%%%%%%%%%%%%%%%%%%%%%%%

\noindent Release Notes:

\begin{itemize}

%% This is important (and thus, I believe, worth of being a top
%% level release note) because it will surprise anyone upgrading
%% an existing pool or repackaging Condor binaries (say, for
%% custom glideins, or as .deb packages for a local pool.)
% This is part of the privilege seperation work, but the procd
% is required even if you're not turning privsep on.
% Questions should go to the privsep team: psilord, zmiller, etc.
\item As part of ongoing security enhancements, Condor now has a
new, required daemon: \Condor{procd}.  This daemon is
automatically started by the \Condor{master}, you do not need to
add it to \Macro{DAEMON\_LIST}.  
However, you must be certain to update the \Condor{master}
if you update any of the other Condor daemons.
%Commented out the below since the defaults are in the code.
%New installations should not
%need to do anything; the default configuration file is correctly
%set.  Installations upgrading to 6.9.2 from previous versions
%will need to ensure several things are done.  
%1. Be sure to install \Condor{procd} into your Condor \Macro{SBIN} directory. 
%2. Add ``\Code{PROCD = \$(SBIN)/condor\_procd}'' to your Condor configuration. 
%3. Add ``\Code{PROCD\_ADDRESS = \$(LOCK)/procd\_pipe}'' to your Condor configuration. 
%On Windows there are two additional steps:
%4. Be sure to install \Condor{softkill} into your Condor \Macro{SBIN} directory. 
%5. Add ``\Code{WINDOWS\_SOFTKILL = \$(SBIN)/condor\_softkill}'' to your Condor configuration. 

% This isn't quite so important, but it's not really a feature or
% a bug, just a change.  It is a change that may surprise some
% users.  The full list of settings impacted is
% pretty long.  So far the below is just a small fraction,
% primarily being added because an external user was suprised by
% this when testing a 6.9.2-prerelease. For anyone curious or
% inspired to flesh out the list, here's the checkin that caused
% this:
% http://bonsai.cs.wisc.edu/bonsai/cvsquery.cgi?who=danb&whotype=match&sortby=Date&date=explicit&mindate=02%2F23%2F2007+19%3A15&maxdate=02%2F23%2F2007+19%3A30
% (To do the search yourself, search for checkins by danb between
% 02/23/2007 19:15 and 02/23/2007 19:30 )
% To determine if a variable is impacted, look at the removed
% code and confirm that it used the default if the setting was 0.
% Then if the new code sets a minimum of 1 (the third argument to
% param_integer), it's impacted.
\item Some configuration settings that previously accepted 0 no
  longer do so.  Instead the daemon using the setting will exit
  with an error message listing the acceptable range to its log.
  For these settings 0 was equivalent to requesting the default.
  As this was undocumented and confusing behavior it is no longer
  present.  To request a setting use its default, either comment it
  out, or set it to nothing (``\Code{EXAMPLE\_SETTING=}'').
  Setting impacted include but are not limited to: 
  % From condor_master.V6/master.C 1.82 to 1.83:
  \Macro{MASTER\_BACKOFF\_CONSTANT},
  \Macro{MASTER\_BACKOFF\_CEILING},
  \Macro{MASTER\_RECOVER\_FACTOR},
  \Macro{MASTER\_UPDATE\_INTERVAL},
  \Macro{MASTER\_NEW\_BINARY\_DELAY},
  \Macro{PREEN\_INTERVAL},
  \Macro{SHUTDOWN\_FAST\_TIMEOUT},
  \Macro{SHUTDOWN\_GRACEFUL\_TIMEOUT},
  % From condor_master.V6/daemon.C 1.68 to 1.69:},
  \Macro{MASTER\_<name>\_BACKOFF\_CONSTANT},
  \Macro{MASTER\_<name>\_BACKOFF\_CEILING},

\item Version 1.4.1 of the Generic Connection Broker (GCB) is
  now used for building Condor.  This version of GCB fixes a timing bug
  where a client may incorrectly think a network connection has been established,
  and also guards against an unresponsive client from causing a denial of
  service by the broker.
  For more information about GCB, see section~\ref{sec:GCB} on
  page~\pageref{sec:GCB}. 

% I'm checking this in commented since I'm not surewhat disclosure
% policy we want to use. Only CDF (Igor) uses the GLEXEC_STARTER
% functionality, so I think it'd be wise to run it by him before
% documenting this publicly.
%
%\item Fixed a security vulnerability in the \Macro{GLEXEC\_STARTER}
%feature. In previous versions when the \Condor{startd} received the
%user proxy, it placed it in a temporary file that for a short window
%of time could be opened for reading by any user on the system.

\end{itemize}


\noindent New Features:

\begin{itemize}
\item On UNIX, an execute-side Condor installation can run without
root privileges and still execute jobs as different users, properly
clean up when a job exits, and correctly enforce policies specified by
the Condor administrator and resource owners. Privileged functionality
has been separated into a well-defined set of functions provided by a
setuid helper program. This feature currently does not work for the
standard or PVM universes.

%%\item added bogus ImageSize to bogus dedicated scheduler
%%jobAd used only for claiming.  This fixes some problems with
%%startd WANT_SUSPEND going to undefined, but we don't document
%%this bogus ad anywhere, so I'm not going to add it here.

\item Added support for EmailAttributes in the parallel universe.  
Previously, it was only valid in the vanilla and standard universes.

\item Added configuration parameter \Macro{DEDICATED\_SCHEDULER\_USE\_FIFO}
which defaults to true.  When false, the dedicated scheduler will
use a best-fit algorithm to schedule parallel jobs.  This setting is
not recommended, as it can cause starvation.  When true, the dedicated
scheduler will schedule jobs in a first-in, first-out manner.

\item Added \Opt{-dump} to \Condor{config\_val} which will print out
all of the macros defined in any of the configuration files found by
the program.
\Condor{config\_val} \Opt{-dump} \Opt{-v} will augment the output
with exactly what line and in what file each configuration variable
was found.
\Note: The output format of the \Opt{-dump} option will most likely
change in a future revision of Condor.

% Gnats PR 671
\item Node names in \Condor{dagman} DAG files can now be DAG
keywords, except for PARENT and CHILD.

\item Improved the log message when \Attr{OnExitRemove} or
\Attr{OnExitHold} evaluates to UNDEFINED.

% Gnats PR 796
\item Added the \Macro{DAGMAN\_ON\_EXIT\_REMOVE} configuration macro,
which allows customization of the \Attr{OnExitRemove} expression
generated by \Condor{submit\_dag}.

\item When using GCB, Condor can now be told to choose from a list of
brokers. \Macro{NET\_REMAP\_INAGENT} is now a space and comma separated
list of brokers. On start up, the \Condor{master} will query all of the
brokers and pick the least-used one for it and its children to use. If
none of the brokers are operational, then the \Condor{master} will wait
until one is working. This waiting can be disabled by setting 
\Macro{MASTER\_WAITS\_FOR\_GCB\_BROKER} to FALSE in the configuration file.
If the chosen broker fails and recovery is not possible or another broker
is available, the \Condor{master} will restart all of the daemons.

\item When using GCB, communications between parent and child
Condor daemons on the same host no longer use the GCB broker.
This improves scalability and also allows a single host to
continue functioning if the GCB broker is unavailable.

\item The \Condor{schedd} now uses non-blocking methods to send the
``alive'' message to the \Condor{startd} when renewing the job lease.
This prevents the \Condor{schedd} from blocking for 20 seconds while
trying to connect to a machine that has become disconnected from the
network.

\item \Condor{advertise} can read the classad to be advertised from
standard input.

\item Unix Condor daemons now reinitialize their DNS
configuration (e.g. IP addresses of the nameservers) on reconfig.

% Gnats PR 777
\item A configuration file for \Condor{dagman} can now be specified
in a DAG file or on the \Condor{submit\_dag} command line.

\item Added \Condor{cod} option \Opt{-lease} for creation of COD claims
with a limited duration lease.  This provides automatic cleanup of COD
claims that are not renewed by the user.  The default lease is infinitely
long, so existing behavior is unchanged unless \Opt{-lease} is explicitly
specified.

\item Added \Condor{cod} command \Opt{delegate\_proxy} which will
delegate an x509 proxy to the requested COD claim.
This is primarily useful for sites wishing to use glexec to spawn the
\Condor{starter} used for COD jobs.
The new command optionally takes an \Opt{-x509proxy} argument to
specify the proxy file.
If this argument is not present, \Condor{cod} will search for the
proxy using the same logic as \Condor{submit} does.

% This is barely a feature, but it's definitely not a bug fix. It's
% more of a change in behavior.
\item \Macro{STARTD\_DEBUG} can now be empty, indicating a default, minimal
log level. It now defaults to empty.
Previously it had to be non-empty and defaulted to include D\_COMMAND.

\item The addition of the \Condor{procd} daemon means that all process
family monitoring and control logic is no longer replicated in each
Condor daemon that needs it. This improves Condor's scalability,
particularly on machines with many processes.

\end{itemize}

\noindent Bugs Fixed:

\begin{itemize}

\item Under various circumstances, condor 6.9.1 daemons would abort
with the message, ``ERROR: Unexpected pending status for fake message
delivery.''  A specific example is when \Attr{OnExitRemove} or
\Attr{OnExitHold} evaluated to UNDEFINED.  This caused the
\Condor{schedd} to abort.

\item In Condor 6.9.1, the \Condor{schedd} would die during startup
when trying to reconnect to running jobs for which the \Condor{schedd}
could not find a startd ClassAd.  This would happen shortly after
logging the following message: ``Could not find machine ClassAds for
one or more jobs.  May be flocking, or machine may be down.
Attempting to reconnect anyway.''

\item Improved Condor's validity checking of configuration values.
For example, in some cases where Condor was expecting an integer but
was given an expression such as 12*60, it would silently interpret
this as 12.  Such cases now result in the condor daemon exiting
after issuing an error message into the log file.

\item When sending a \Code{WM\_CLOSE} message to a process on Windows,
Condor daemons now invoke the helper program \Condor{softkill} to do
so. This prevents the daemon from needing to temporarily switch away
from its dedicated service Window Station and Desktop. It also fixes a
bug where daemons would leak Window Station and Desktop handles. This
was mainly a problem in the \Condor{schedd} when running many scheduler
universe jobs.

\end{itemize}

\noindent Known Bugs:

\begin{itemize}

\item \Condor{glidein} generates a default config file that sets
\MacroNI{PREEN\_INTERVAL} to an invalid value (0).  To fix this,
remove the setting of \MacroNI{PREEN\_INTERVAL}.

\item There are a couple of known issues with Condor's
\Macro{GLEXEC\_STARTER} feature when used in conjunction with
COD. First, the \Condor{cod} tool invoked with the
Opt{delegate\_proxy} option will sometimes incorrectly report that the
operation has failed. In addition, the \MacroNI{GLEXEC\_STARTER}
feature will not work properly with COD unless the UID that the each
COD job runs as is different than the UID of the opportunistic job or
any other COD jobs that are running on the execute machine when the
COD claim is activated.

\end{itemize}



%%%%%%%%%%%%%%%%%%%%%%%%%%%%%%%%%%%%%%%%%%%%%%%%%%%%%%%%%%%%%%%%%%%%%%
\subsection*{\label{sec:New-6-9-1}Version 6.9.1}
%%%%%%%%%%%%%%%%%%%%%%%%%%%%%%%%%%%%%%%%%%%%%%%%%%%%%%%%%%%%%%%%%%%%%%

\noindent Release Notes:

\begin{itemize}

\item The 6.9.1 release contains all of the bug fixes and enhancements
  from the 6.8.x series up to and including version 6.8.3.

\item Version 1.4.0 of the Generic Connection Broker (GCB) library is
  now used for building Condor, and it is the 1.4.0 versions of the
  \Prog{gcb\_broker} and \Prog{gcb\_relay\_server} programs that are
  included in this release.
  This version of GCB includes enhancements used by Condor
  along with a new GCB-related command-line tool:
  \Prog{gcb\_broker\_query}.
  Condor 6.9.1 will not work properly with older versions of the
  \Prog{gcb\_broker} or \Prog{gcb\_relay\_server}.
  For more information about GCB, see section~\ref{sec:GCB} on
  page~\pageref{sec:GCB}. 

\end{itemize}

\noindent New Features:

\begin{itemize}

\item Improved the performance of the ClassAd matching algorithm,
which speeds up the \Condor{schedd} and other daemons.

\item Improved the scalability of the algorithm used by 
the \Condor{schedd} daemon to find runnable jobs.
This makes a noticeable difference in \Condor{schedd} daemon performance,
when there are on the order of thousands of jobs in the queue.

\item the \Dflag{COMMAND} debugging level has been enhanced to
log many more messages. 

\item Updated the version of DRMAA, which contains several great
improvements regarding scalability and race conditions.

% Gnats PR 774
\item Added the \Macro{DAGMAN\_SUBMIT\_DEPTH\_FIRST} configuration macro,
which causes \Condor{dagman} to submit ready nodes in more-or-less depth-first
order, if set to \Expr{True}.  The default behavior is to submit
the ready nodes in breadth-first order.

\item Added configuration parameter \Macro{USE\_PROCESS\_GROUPS}.
If it is set to \Expr{False},
then Condor daemons on Unix machines will not create new 
sessions or process groups. This is intended for use with Glidein, as
we have had reports that some batch systems cannot properly track jobs that
create new process groups. The default value is \Expr{True}.

\item The default value for the submit file command
\SubmitCmd{copy\_to\_spool} has been changed to \Expr{False}, because copying
the executable to the spool directory for each job (or job cluster) is almost
never desired.  Previously, the default was \Expr{True} in all
cases, except for grid universe jobs and remote submissions.

\item More types of file transfer errors now result in the job going
on hold, with a specific error message about what went wrong.  The new
cases involve failures to write output files to disk on the submit
side (for example, when the disk is full).
As always, the specific error number is
recorded in \Attr{HoldReasonSubCode}, so you can enforce an automated
error handling policy using \SubmitCmd{periodic\_release} or
\SubmitCmd{periodic\_remove}.

\item Added the \Macro{<SUBSYS>\_DAEMON\_AD\_FILE}
configuration variable, which is similar to the 
\MacroNI{<SUBSYS>\_ADDRESS\_FILE}.
This new variable will be used in future versions of Condor, but is
not necessary for 6.9.1.


\end{itemize}

\noindent Bugs Fixed:

\begin{itemize}

\item Fixed a bug in the \Condor{master} so that it will now send obituary
e-mails when it kills child processes that it considers hung.

\item \Condor{configure} used to always make a personal Condor with
\Opt{--install} even when \Opt{--type} called for only execute or
submit types.  Now, \Condor{configure} honors the \Opt{--type}
argument, even when using \Opt{--install}.
If \Opt{--type} is not specified, the default is to still install a
full personal Condor with the following daemons: 
\Condor{master}, \Condor{collector},
\Condor{negotiator}, \Condor{schedd}, \Condor{startd}. 

\item While removing, putting on hold, or vacating a large number of
jobs, it was possible for the \Condor{schedd} and the \Condor{shadow} to
temporarily deadlock with each other.  This has been fixed under Unix,
but not yet under Windows.

\item Communication from a \Condor{schedd} to a \Condor{startd}
now occurs in a nonblocking manner.
This fixes the problem of the \Condor{schedd} blocking 
when the claimed machine running the \Condor{startd}
cannot be reached, for example because the machine is turned off.

\end{itemize}

\noindent Known Bugs:

\begin{itemize}

\item Under various circumstances, condor 6.9.1 daemons abort
with the message, ``ERROR: Unexpected pending status for fake message
delivery.''  A specific example is when \Attr{OnExitRemove} or
\Attr{OnExitHold} evaluated to UNDEFINED, which causes the
\Condor{schedd} to abort.

\item In Condor 6.9.1, the \Condor{schedd} will die during startup
when trying to reconnect to running jobs for which the \Condor{schedd}
can not find a startd ClassAd.  This happens shortly after
logging the following message: ``Could not find machine ClassAds for
one or more jobs.  May be flocking, or machine may be down.
Attempting to reconnect anyway.''

\end{itemize}

%%%%%%%%%%%%%%%%%%%%%%%%%%%%%%%%%%%%%%%%%%%%%%%%%%%%%%%%%%%%%%%%%%%%%%
\subsection*{\label{sec:New-6-9-0}Version 6.9.0}
%%%%%%%%%%%%%%%%%%%%%%%%%%%%%%%%%%%%%%%%%%%%%%%%%%%%%%%%%%%%%%%%%%%%%%

\noindent Release Notes:

\begin{itemize}

\item The 6.9.0 release contains all of the bug fixes and enhancements
  from the 6.8.x series up to and including version 6.8.2.

% and a few \condor{gridmanager} bug fixes from 6.8.3.  *sigh* we need
% a real solution to this problem (like pointing to issue node ids) ;)

\end{itemize}


\noindent New Features:

\begin{itemize}


\item Preliminary support for using \Prog{glexec} on execute machines
has been added.  This feature causes the \Condor{startd} to spawn the
\Condor{starter} as the user that \Prog{glexec} determines based on
the user's GSI credential.

\item A ``per-job history files'' feature has been added to the
\Condor{schedd}. When enabled, this will cause the \Condor{schedd} to
write out a copy of each job's ClassAd when it leaves the job
queue. The directory to place these files in is determined by the
parameter \Macro{PER\_JOB\_HISTORY\_DIR}. It is the responsibility of
whatever external entity (for example, an accounting or monitoring system) is
using these files to remove them as it completes its processing.

\item \Condor{chirp} command now supports writing messages to the user log.

\item \Condor{chirp} getattr and putattr now send all classad getattr
and putattr commands to the proc 0 classad, which allows multiple proc
parallel jobs to use proc 0 as a scratch pad.

\item Parallel jobs now support an \Attr{AllRemoteHosts} attribute,
which lists all the hosts across all procs in a cluster.

\item The \Macro{DAGMAN\_ABORT\_DUPLICATES} configuration macro (which causes
\Condor{dagman} to abort itself if it detects another \Condor{dagman}
running on the same DAG) now defaults to \Expr{True} instead of
\Expr{False}.

\end{itemize}

\noindent Bugs Fixed:

\begin{itemize}

\item None.

\end{itemize}

\noindent Known Bugs:

\begin{itemize}

\item None.

\end{itemize}


%%%%%%%%%%%%%%%%%%%%%%%%%%%%%%%%%%%%%%%%%%%%%%%%%%%%%%%%%%%%%%%%%%%%%%%
\section{\label{sec:History-6-8}Stable Release Series 6.8}
%%%%%%%%%%%%%%%%%%%%%%%%%%%%%%%%%%%%%%%%%%%%%%%%%%%%%%%%%%%%%%%%%%%%%%

This is a stable release series of Condor.
It is based on the 6.7 development series.
All new features added or bugs fixed in the 6.7 series are available
in the 6.8 series.
As usual, only bug fixes (and potentially, ports to new platforms)
will be provided in future 6.8.x releases.
New features will be added in the forthcoming 6.9.x development series.

%%%%%%
% we need a summary of major new features since 6.6.x here.  trying to
% sort through the 21 different 6.7.x releases and all the new
% features is a huge amount of noise.  people just want to see a
% summary of the major new functionality.
% \Todo
%%%%%%

The 6.8.x series supports a different set of platforms than 6.6.x.
Please see the updated table of available platforms in
section~\ref{sec:Availability} on page~\pageref{sec:Availability}.

The details of each version are described below.


%%%%%%%%%%%%%%%%%%%%%%%%%%%%%%%%%%%%%%%%%%%%%%%%%%%%%%%%%%%%%%%%%%%%%%
\subsection*{\label{sec:New-6-8-1}Version 6.8.1}
%%%%%%%%%%%%%%%%%%%%%%%%%%%%%%%%%%%%%%%%%%%%%%%%%%%%%%%%%%%%%%%%%%%%%%

\noindent Release Notes:

\begin{itemize}

\item The PCRE (Perl Compatible Regular Expressions) library used by
Condor is now dynamically linked and shipped as a DLL with Condor for
Windows, rather than being statically linked.

\end{itemize}


\noindent New Features:

\begin{itemize}

% Gnats PR 610
\item Added an optional argument to the \Condor{dagman} ABORT-DAG-ON
command that allows the DAGMan exit code to be specified separately
from the node value that causes the abort; also, a DAG can now be
aborted on a zero exit code from a node.

% I implemented in condor_rm.  Todd is handling condor_schedd implementation.
\item Added \Macro{ALLOW\_FORCE\_RM} setting.  If this expressions evaluates to
TRUE then an \Condor{rm} -f attempt is allowed.  If it evaluated to FALSE, the
attempt is disallowed.  The expression is evaluated in the context of the job
ad.  If not specified the setting defaults to TRUE, matching the behavior of
previous Condor releases.

% Gnats PR 664
\item \Condor{dagman} will now reject DAGs for which any of the node
job user log files are on NFS (because of the unreliability of NFS
file locking, this can cause DAGs to fail).  This feature can be
turned off by setting the DAGMAN\_LOG\_ON\_NFS\_IS\_ERROR configuration
macro to \Expr{False} (the default is \Expr{True}).

\item \Condor{submit} can now be configured to reject jobs for which
the log file is on NFS (to do this, set the LOG\_ON\_NFS\_IS\_ERROR
configuration macro to \Expr{True}).  The default is that \condor{submit}
will simply issue a warning if a log file is on NFS.

\item Added the DAGMAN\_ABORT\_DUPLICATES config macro, which causes
\Condor{dagman} to attempt to detect at startup whether another
\Condor{dagman} is already running on the same DAG; if so, the second
\Condor{dagman} will abort itself.

\item There is now a configuration paramater
\MacroNI{NETWORK\_MAX\_PENDING\_CONNECTS} that may be used to limit the
maximum number of simultaneous network connection attempts.  This is
primarily relevant to \Condor{schedd}, which may try to connect to
large numbers of startds when claiming them.  The negotiator may also
connect to large numbers of startds when initiating security sessions
used for sending MATCH messages.  On Unix, the default is to allow up to
eighty percent of the process file descriptor limit.  On windows, the
default is 1600.

\end{itemize}

\noindent Bugs Fixed:

\begin{itemize}

\item Fixed a Quill bug that prevented it from running on Windows.  The
symptom was errors in the QuillLog like
\begin{verbatim}
POLLING RESULT: ERROR
\end{verbatim}

\item Fixed a bug in Quill where it would cause errors like
\begin{verbatim}
duplicate key violates unique constraint "history_vertical_pkey"
\end{verbatim}
in the QuillLog and the postgres log file.  These errors would then trigger
a significant slowdown in the performance of Quill and the database.  This
would only happpen when a job attribute would change type from a string
type to a numeric type, or vice versa.

\item When a large number of jobs (roughly 200 or more) are running from a
single schedd and those jobs are using job leases (the default in 6.8), it is
possible the schedd to enter a state where it crashes on startup until all of
the job leases expire.  This bug is fixed.
% The bug is more generic; it potentially hit any user of GenericQuery or
% CondorQuery, but we know of no users encountering the bug in any other cases.

%% This is the change to datathread.C
\item In those unusual cases where Condor is unable to create a new process,
it will shut down cleanly, eliminating a small possibility of data corruption.

\item Fixed a bug with the gt4 and nordugrid grid universe job types that
caused the stdout and stderr of a job to not be transferred correctly if
the filenames given had absolute paths.

% Gnats PR 711
\item \Condor{dagman} now echos warnings from \condor{submit} and
stork\_submit to the \File{dagman.out} file.

\item Fixed a bug introduced in 6.7.20 causing \Condor{ckpt\_server}
to exit immediately after starting up, unless Condor's security
negotation was disabled.

% This was a bug because the default configuration file and manual
% both claimed it defaulted to 1MB.
\item \Macro{MAX\_<SUBSYS>\_LOG} defaults to one megabyte, even if the
setting is missing from the configuration.  Previously it was 64 kilobytes.

% this is the change to condor_secman.C by zmiller
\item Fixed a bug related to non-blocking connect that could occasionally
cause Condor daemons to crash.

% collector.C change: eliminated useless fprintf(stderr).
\item Fixed a rare bug where an exceptionally large query to the
\Condor{collector} could cause it to crash.  The most common cause is a single
schedd restarting and trying to recover a large number of job leases at once.
More than approximately 250 running jobs on a single schedd would be necessary
to trigger this bug.

\item When using the \Macro{JOB\_PROXY\_OVERRIDE\_FILE} configuration
parameter, the X509 proxy will now be properly forwarded for Condor-C jobs.

\item Greatly reduced the chance that a Condor-C job in the REMOVED state
will become HELD due to an expired proxy or failure to talk to the remote
\Condor{schedd}.

\item Fixed a number of error and debug messages added in 6.7.20 that
were incorrectly reporting IP and port numbers.  These messages were
intended to report the peer's address, but they were reporting the
local address of the network socket instead.

\item Fixed a bug introduced in 6.7.20 which could cause Condor daemons to
die with the message ``PANIC -- OUT OF FILE DESCRIPTORS''.  The conditions
causing this were related to failed attempts to send updated status
to the collector, with both non-blocking updates and security negotiation
enabled (the defaults).

\item Under some conditions, when making TCP connections, Condor was
still trying to connect for the full duration of the operation timeout
(often 10 or 20 seconds), even if the connection attempt was refused
(e.g. because the port being accessed is not accepting connections).
Now, the connect operation finishes immediately after the first such
failure, allowing the Condor process to continue with other tasks.

\item Fixed the problems relating to credential cache problems in the Kerberos
authentication mechanism.  The current version of Kerberos is 1.4.3.

% changes to condor_auth_ssl.C by zmiller
\item Fixed bugs in the SSL authentication mechanism that caused the
\Condor{schedd} to crash when submitting a job (on UNIX) and caused
all tools and daemons to crash on Windows when using SSL.

\item Some of the binaries required to use Condor-C on Windows were
mistakenly not included in previous releases of Condor. This has been
fixed.

\item Fixed problem on Windows where the \Condor{startd} could fail to
include some attributes in its ClassAd. This would result in some jobs
incorrectly not being matched to that machine.  This only happened if
\Macro{CREDD\_HOST} was defined and Condor daemons on the execute
machine were unable to authenticate with the \Condor{credd}.

\item Fixed a \Condor{dagman} bug which had prevented the
\MacroU{DAGManJobId} attribute from being expanded in job submit files
(e.g., when used as the value of the \Macro{Priority} command).

\item Fixed a bug in \Condor{submit} that caused parallel universe jobs
submitted via Condor-C to become mpi universe jobs.

\item Fixed a bug which could cause Condor daemons to hang if they try
to write to the standard error stream (stderr) on some platforms.  In
general, this should never happen, but can due to third party
libraries (beyond our control) trying to write error or other messages.

\item Fixed \Condor{status} to report error messages.

\item Fixed a problem with \MacroNI{NEGOTIATOR\_CONSIDER\_PREEMPTION}=False
in which the fraction of the pool already being claimed by a user was
calculated using the wrong total number of startds.  This could cause
some startds to remain unclaimed, even when there were jobs available
to run on them.

\end{itemize}

\noindent Known Bugs:

\begin{itemize}

\item None.

\end{itemize}




%%%%%%%%%%%%%%%%%%%%%%%%%%%%%%%%%%%%%%%%%%%%%%%%%%%%%%%%%%%%%%%%%%%%%%
\subsection*{\label{sec:New-6-8-0}Version 6.8.0}
%%%%%%%%%%%%%%%%%%%%%%%%%%%%%%%%%%%%%%%%%%%%%%%%%%%%%%%%%%%%%%%%%%%%%%

\noindent Release Notes:

\begin{itemize}

\item The default configuration for Condor now requires that
\Macro{HOSTALLOW\_WRITE} be explicitly set.  Condor will refuse
to start if the default configuration is used unmodified.
Existing installations should not need to change anything.  For
those who desire the earlier default, you can set it to "*", but
note that this is potentially a security hole allowing anyone to
submit jobs or machines to your pool.

\item Most Linux distributions are now supported using dynamically
  linked binaries built on a RedHat Enterprise Linux 3 machine.
  Recent security patches to a number of Linux distributions have
  rendered the binaries built on RedHat 9 machines ineffective.
  The download pages have been changed to reflect this, but Linux users
  should be aware of this change.
  The recommended download for most x86 Linux users is now:
  \File{condor-6.8.0-linux-x86-rhel3-dynamic.tar.gz}.

\item Some log messages have been clarified or moved to different
  debugging levels.
  For example, certain messages that looked like errors were printed
  to \MacroNI{D\_ALWAYS}, even though nothing was wrong and the system was
  behaving as expected.

\item The new features and bugs fixed in the rest of this section only
  refer to changes made since the 6.7.20 release, not the last stable
  release (6.6.11).
  For a complete list of changes since 6.6.11, read the 6.7 version
  history in section~\ref{sec:History-6-7} on
  page~\pageref{sec:History-6-7}. 

\end{itemize}


\noindent New Features:

\begin{itemize}

\item Version 1.4 of the Condor DRMAA libraries are now included 
  with the Condor release.
  For more information about DRMAA, see section~\ref{API-DRMAA} on
  page~\pageref{API-DRMAA}.

\item Version 1.0.15 of the Condor GAHP is now used for Condor-G and
  Condor-C. 

% Gnats PR 710
\item Added the \Opt{-outfile\_dir} command-line argument to
\Condor{submit\_dag}.  This allows you to change the directory in which
\Condor{dagman} writes the \File{dagman.out} file.

\item Added a new \Opt{--summary} (also \Opt{-s}) option to the
\Condor{update\_stats} tool.  If enabled, this prevents it from
displaying the entire history for each machine and only displays the
summary info.

\end{itemize}

\noindent Bugs Fixed:

\begin{itemize}

\item Fixed a number of potential static buffer overflows in various
  Condor daemons and libraries.

\item Fixed some small memory leaks in the \Condor{startd},
  \Condor{schedd}, and a potential leak that effected all Condor
  daemons.

\item Fixed a bug in Quill which caused it to crash when certain
long attributes appeared in a job ad.

\item The startd would crash after a reconfig if the address of a
collector had not been resolved since the previous reconfig
(e.g. because DNS was down during that time).

\item Once a Condor daemon failed to lookup the IP address of the
collector (e.g. because DNS was down), it would fail to contact the
collector from that time until the next reconfig.  Now, each time Condor
tries to contact the collector, it generates a fresh DNS query if the
previous attempt failed.

% Gnats PR 707
\item When using Condor-C or the -s or -r command-line options to
\condor{submit}, the job's standard output and error would be placed
in the job's initial working directory, even if the job ad said to
place them in a different directory.

% Gnats PRs 501 and 663
\item Greatly sped up the parsing of large DAGs (by a factor of 50
or so) by using a hash table instead of linear search to find DAG nodes.

% Gnats PR 697
\item Fixed a bug in \Condor{dagman} that caused an EXECUTABLE\_ERROR
event from a node job to abort the DAG instead of just marking the
relevant node as failed.

\item Fixed a bug in \Condor{collector} that caused it to discard
machine ads that don't have an IP address field (either StartdIpAddr
or STARTD\_IP\_ADDR).  The \Condor{startd} will always produce a
StartdIpAddr field, but machine ads published through
\Condor{advertise} may not.

\item When using \MacroNI{BIND\_ALL\_INTERFACES} on a dual-homed
machine, a bug introduced in 6.7.18 was causing Condor daemons to
sometimes incorrectly report their IP addresses, which could cause
jobs to fail to start running.

\item Made the event checking in \Condor{dagman} less strict: 
added the new "allow duplicate events" value to the
\MacroNI{DAGMAN\_ALLOW\_EVENTS} macro (this value is part of the
default); 16 value now also allows terminate event before submit;
changed "allow all events" to "allow almost all events"
(all except "run after terminal event"), so it is more useful.

% Gnats PR 712
\item \Condor{dagman} and \Condor{submit\_dag} now report
\Opt{-NoEventChecks} as ignored rather than deprecated.

\item Fixed a bug in the \Condor{dagman} \Opt{-maxidle} feature:
a shadow exception event now puts the corresponding job into the
idle state in \Condor{dagman}'s internal count.

\item Fixed a problem on Windows where daemons would sometimes crash
when dealing with UNC path names.

\item Fixed a problem where the \Condor{schedd} on Windows would
incorrectly reject a job if the client provided an \Attr{Owner}
attribute that was correct but differed in case from the authenticated
name.

\item Fixed a \Condor{startd} crash introduced in version 6.7.20. This
crash would appear if an execute machine was matched for preemption
but then not claimed in time by the appropriate \Condor{schedd}.

\item Resolved an issue where the \Condor{startd} was unable to clean
up jobs' execute directories on Windows when the \Condor{master} was
started from the command line rather than as a service.

\item Added more patches to Condor's DRMAA interface to make it more
compatible with Sun Grid Engine's DRMAA interface.

\item Removed the unused \MacroNI{D\_UPDOWN} debug level and added the
  \MacroNI{D\_CONFIG} debug level.

\item Fixed a bug that caused \Condor{q} with the \Opt{-l} or \Opt{-xml}
arguments to print out duplicate attributes when using Quill.

\item Fixed a bug that prevented Condor-C jobs (universe grid jobs of type condor)
from submitting correctly if \MacroNI{QUEUE\_ALL\_USERS\_TRUSTED} is set to
True.

\item Fixed a bug that could cause the \Condor{negotiator} to crash if the
pool contains several different versions of the \Condor{schedd} and in the
config file \MacroNI{NEGOTIATOR\_MATCHLIST\_CACHING} is set to True.

\item Changed the default value for config file entry
\MacroNI{NEGOTIATOR\_MATCHLIST\_CACHING} from False to True.  When set to
True, this will instruct the negotiator to safely cache data in order to
improve matchmaking performance.

\item The Condor{master} now recognizes \Condor{quill} as a valid
  Condor daemon without any manual configuration on the part of site
  administrators.
  This simplifies the configuration changes required to enable Quill. 

\item Fixed a rare bug in the \Condor{starter} where if there was a
  failure transferring job output files back to the submitting host,
  it could hang indefinitely, and the job appeared as if it was
  continuing to run.

\end{itemize}


\noindent Known Bugs:

\begin{itemize}

\item None.

\end{itemize}


% Dec 2007, as we release 7.x, Karen commented out the older histories
%%%%%%%%%%%%%%%%%%%%%%%%%%%%%%%%%%%%%%%%%%%%%%%%%%%%%%%%%%%%%%%%%%%%%%%
\section{\label{sec:History-6-7}Development Release Series 6.7}
%%%%%%%%%%%%%%%%%%%%%%%%%%%%%%%%%%%%%%%%%%%%%%%%%%%%%%%%%%%%%%%%%%%%%%

This is the development release series of Condor.
The details of each version are described below.

%%%%%%%%%%%%%%%%%%%%%%%%%%%%%%%%%%%%%%%%%%%%%%%%%%%%%%%%%%%%%%%%%%%%%%
\subsection{\label{sec:New-6-7-9}Version 6.7.9}
%%%%%%%%%%%%%%%%%%%%%%%%%%%%%%%%%%%%%%%%%%%%%%%%%%%%%%%%%%%%%%%%%%%%%%

\noindent Release Notes:

\begin{itemize}

\item This release contains all of the bug fixes and improvements from
  the 6.6 stable series up to and including version 6.6.10.

\end{itemize}

\noindent New Features:

\begin{itemize}

\item The Parallel Universe has been added.

\item 
\index{environment variables!X509\_USER\_PROXY}
\index{X509\_USER\_PROXY}
The environment variable \Env{X509\_USER\_PROXY} is set to the
full path of the proxy if a proxy is associated with the job.
This is usually done using \SubmitCmd{x509userproxy} in the submit file.
This currently works in the local, java, and vanilla universes.

\item \Condor{submit} generates more precise error messages in 
some failure cases.

\item \Condor{hold}, \Condor{release} and \Condor{rm} now allow the user
to change the HoldReason, ReleaseReason or RemoveReason with the -reason
flag.

\item \Condor{dagman} no longer does a one-second sleep before each
submit if all node jobs have the same log file.  (The sleep is still
needed if there are multiple log files, for unambiguous ordering of
events during bootstrapping.)  Note that if \MacroNI{DAGMAN\_SUBMIT\_DELAY}
is specified, the specified delay takes effect whether or not all
jobs have the same log file.

\end{itemize}

\noindent Bugs Fixed:

\begin{itemize}

\item Many crashes related to running the Dedicated Scheduler have
been fixed.

\item Setting COLLECTOR\_HOST or NEGOTIATOR\_HOST with a port but without
a hostname no longer causes the \Condor{master} to crash.

\item The Condor-G Grid Monitor now works with Globus 4.0 pre-Web Services
GRAM.

\item Several deadlocks in the Condor-C GAHP server have been fixed.

\end{itemize}

%%%%%%%%%%%%%%%%%%%%%%%%%%%%%%%%%%%%%%%%%%%%%%%%%%%%%%%%%%%%%%%%%%%%%%
\subsection{\label{sec:New-6-7-8}Version 6.7.8}
%%%%%%%%%%%%%%%%%%%%%%%%%%%%%%%%%%%%%%%%%%%%%%%%%%%%%%%%%%%%%%%%%%%%%%

\noindent Release Notes:

\begin{itemize}

\item This release contains all of the bug fixes and improvements from
  the 6.6 stable series up to and including version 6.6.9.

\end{itemize}

\noindent New Features:

\begin{itemize}

\item Controlling whether or not a standard universe job asks the
\Condor{shadow} about how/where to open every single file can be better
controlled with the \Attr{want\_remote\_io} attribute in the submit
description file.
This attribute can be set to true or false and it is true be default.
If set to false, then this attribute forces a standard universe job in 
Condor to always look to the local file system when opening files and not
to contact the shadow. 
This increases performance of user jobs where the jobs open a very large
amount of files in a small space of time.
However, the user jobs must be matched to machines that have the same
UID\_DOMAIN and FILESYSTEM\_DOMAIN, as per vanilla universe jobs with a 
homogeneous file system.

\item \Condor{dagman} now has the capability to run more than one
independent DAG in a single \Condor{dagman} process.

\item User policy expressions (on\_exit\_remove and on\_exit\_hold)
now work for scheduler universe jobs.

\item TotalCpus and TotalMemory are now set in machine ads.

\item \Condor{dagman} now tolerates the "two terminated events for
a single job" bug by default.  There is a new bit in
\MacroNI{DAGMAN\_ALLOW\_EVENTS} to control whether this bug is considered
a fatal error in a \Condor{dagman} run.

\item Added a new debug formatting flag, \Dflag{PID}, that prints out
  the process id (PID) of the process writing a given entry to a log
  file.
  This is useful in Condor daemons (such as the \Condor{schedd}) where 
  the daemon can fork() multiple processes to perform various tasks
  and it is helpful to see what log messages are coming from forked
  process versus the main thread of execution.
  The default \Macro{SCHEDD\_DEBUG} in the sample configuration files
  shipped with Condor now includes this flag.

\item When \Condor{dagman} writes rescue files, each node is now
specified with the same number of retries as was specified in the
original DAG, rather than with only the ``remaining'' number of
retries based on the failed run.  The latter behavior can be restored
by setting \Macro{DAGMAN\_RESET\_RETRIES\_UPON\_RESCUE} to false.

\item Added ``Hawkeye'' capabilities to \Condor{schedd}.  It's
configured identically to that of \Condor{startd}, but using 
``SCHEDD'' in place of ``STARTD'', in particular for the
``SCHEDD\_CRON\_NAME'' macro.

\end{itemize}

\noindent Bugs Fixed:

\begin{itemize}

% See Gnats PRs 297 and 430.
\item Fixed a bug in \Condor{dagman} that prevented POST scripts
from being used with jobs that write XML-format logs.

\item The event-checking code used by \Condor{dagman} now defaults
to allowing an execute event before the submit event for the same
job; if this happens, there will be a warning, but the DAG will
continue.  See section~\ref{param:DAGManAllowEvents} for more info.

\item \Condor{userprio} option \Opt{-pool} was failing with ``Can't
find address for negotiator'' since version 6.7.5.

\item Fixed a bug the prevented SOAP clients from being able to access
a job's spooled data files if the \Condor{schedd} restarted.

\item Fixed a bug that caused the \Condor{gridmanager} to panic when
trying to retire a job from the queue that was already gone. This
could cause multiple terminate events to be logged for some jobs.

\item Fixed a bug that caused match-making to not work for Condor-C
jobs.

\item Added workaround for a Globus bug that can cause re-execution of
a completed GT2 job in the correct failure case (Globus bugzilla ticket
3411).

\item Properly extend the lifetime of GT4 jobs and credentials on the
remote server.

\end{itemize}

%%%%%%%%%%%%%%%%%%%%%%%%%%%%%%%%%%%%%%%%%%%%%%%%%%%%%%%%%%%%%%%%%%%%%%
\subsection{\label{sec:New-6-7-7}Version 6.7.7}
%%%%%%%%%%%%%%%%%%%%%%%%%%%%%%%%%%%%%%%%%%%%%%%%%%%%%%%%%%%%%%%%%%%%%%

\noindent Release Notes:

\begin{itemize}

\item This release contains all of the bug fixes and improvements from
  the 6.6 stable series up to and including version 6.6.9.

\end{itemize}

\noindent New Features:

\begin{itemize}

\item The \Expr{STARTD\_EXPRS} list can now be on a per-VM basis, and
entries on the list can also be specific to a VM. 
See ~\ref{sec:SMP-exps} for more details.

\item The \Macro{LOCAL\_CONFIG\_FILE} can now be overridden. 
This now allows files to include other local config files. 
See ~\ref{param:LocalConfigFile} for more info.

\item Resources that are claimed but suspended can now optionally 
not be charged for at the accountant. 
When the resource is unsuspended, the accountant will resume charging
for usage. 
This is controlled by the \Expr{NEGOTIATOR\_DISCOUNT\_SUSPENDED\_RESOURCES}
config file entry, and it defaults to false.

\item The \Attr{DAGManJobID} attribute which \condor{dagman} inserts
into the classad of every job it submits now contains only its cluster
ID (instead of a cluster.proc ID pair), so that it may be referenced
as an integer in DAG job submit files.  This allows, for example, a
user to automatically set the relative local queue priority of jobs
based on the \condor{dagman} job that submitted them, so that jobs
submitted by ``older'' DAGs will start before jobs submitted by
``newer'' DAGs (assuming they are otherwise identical).

\item GSI authentication can now be used when Condor-C jobs are submitted
from one \condor{schedd} to another.

\item File permissions are now preserved when a job's data files are
transferred between unix machines. File transfers that involve a windows
machine or older version of Condor remain as before.

\item Condor-C now supports the scheduler remote universe.

\item \condor{advertise} now publishes a ``MyAddress'' if none is provided
in the source ClassAd.  This will prevent the collector from throwing out
ads with no address (see Bugs Fixed).

\item Added a new \Condor{dagman} parameter \MacroNI{DAGMAN\_ALLOW\_EVENTS}
controlling which ``bad'' events are not considered fatal errors;
the \Opt{-NoEventChecks} command-line argument is deprecated and has no effect.

\item \Condor{fetchlog} now takes an optional log file extension in order to
select logs such as ``StarterLog.vm2''.

\end{itemize}


\noindent Bugs Fixed:

\begin{itemize}

\item Fixed a throughput performance bottle neck when standard universe
	jobs vacate when the user has specified \Attr{WantCheckpoint} equal to
	False in the submit file.

\item Added initial support for the \Syscall{getdents},
	\Syscall{getdents64}, \Syscall{glob}, and the family of functions
	\Syscall{opendir}, \Syscall{readdir}, \Syscall{closedir} for the
	standard universe.  

	It is recommended that you do not directly invoke \Syscall{getdents} 
	or \Syscall{getdents64}, but instead use the other POSIX functions
	specified above.

	There are two caveats: these calls will not work in heterogeneous
	contexts, and you may not call \Syscall{getdents} directly when 
	\Condor{compile}ing a 32-bit program while specifying the 64-bit
	interfaces for the Unix API.

\item In versions 6.7.4 through 6.7.6, Computing On Demand (COD)
  support was broken due to a bug in how Condor daemons parsed their
  command line arguments.
  The bug was introduced with the changes to provide a web services
  (SOAP) interface to Condor.
  This bug has been fixed and COD support is now working again.

\item In version 6.7.6, the \MacroNI{DAGParentNodeNames} attribute
which \Condor{dagman} adds to all DAG job classads could grow too long
and cause job submission to fail.  Now, if the
\MacroNI{DAGParentNodeNames} value would be too long to add to the job
classad, the attribute is instead left undefined and a warning is
emitted in the DAGMan debugging log.  This behavior means that such a
node can be reliably distinguished from a node with no parents, as the
latter will have a \MacroNI{DAGParentNodeNames} attribute defined but
empty.

\item In version 6.7.3, the value of the X509UserProxySubject job attribute
was changed in such a way that Condor-G jobs submitted by a newer
\condor{submit} to an older \condor{schedd} could fail to run. Now,
\condor{submit} reverts to the old behavior when talking to an old
\condor{schedd}.

\item Bug-fixes and improvements to grid\_type gt4:

  \begin{itemize}

  \item Condor will now delegate a single proxy to the GT4 server for
  multiple. If the local proxy is refreshed, Condor will forward the
  refreshed copy to the server.

  \item Exit codes are now recorded properly.

  \item \Macro{JAVA\_EXTRA\_ARGUMENTS} now used when invoking the GT4 GAHP
  server (which is written in java).

  \item If \Macro{LOWPORT} and \Macro{HIGHPORT} are set in the config file,
  the GT4 GAHP server will now obey the port restriction.

  \item Fixed a bug that caused Condor not to notice when some GT4 jobs
  completed.

  \item Fixed a bug in handling the job's environment for GT4 jobs. Condor
  incorrectly used ``<name>=<value>'' for each variable's name.

  \item Improved hold reason in certain cases when a GT4 job goes on hold.

  \item \condor{q} -globus now works properly for GT4 jobs. Also, the resource
  name in the user log execute event is printed properly for GT4 jobs.

  \item Fixed a bug that could cause Condor to not detect when a GT4 job
  completes. This was triggered by Condor not properly recognizing the
  StageOut Globus job state.

  \end{itemize}

\item Fixed a bug that can cause the \condor{gridmanager} to abort if
\Attr{PeriodicRelease} evaluates to true while it's putting a job on hold.

% See Gnats PR 470
\item Fixed a bug in \Condor{dagman} that
caused the DAG to be aborted if a job generated an executable error
event.

\item Fixed a bug in \Condor{dagman} on Windows that would cause it to
hang or crash on exit.

\item MPI universe jobs now honor the \Attr{JOB\_START\_DELAY}
configuration setting.

\item The \Condor{collector} now throws out startd, schedd, and License
ClassAds that don't have a valid IP address (used in it's hashing).  The
collector now correctly will fall back to ``MyAddress'' if it's provided.

% See Gnats PR 479.
\item Fixed a bug in \Condor{dagman} that could cause \Condor{dagman}
to fail an assertion if PRE or POST scripts are throttled with the
\Opt{-maxpre} or \Opt{-maxpost} \Condor{submit\_dag} command line flags.

\end{itemize}


%%%%%%%%%%%%%%%%%%%%%%%%%%%%%%%%%%%%%%%%%%%%%%%%%%%%%%%%%%%%%%%%%%%%%%
\subsection{\label{sec:New-6-7-6}Version 6.7.6}
%%%%%%%%%%%%%%%%%%%%%%%%%%%%%%%%%%%%%%%%%%%%%%%%%%%%%%%%%%%%%%%%%%%%%%

\noindent Release Notes:

\begin{itemize}

\item Version 6.7.6 contains all the bug fixes and improvements from
  the 6.6 stable series up to and including version 6.6.9.

\end{itemize}

\noindent New Features:

\begin{itemize}

\item Added support for libc's \Syscall(system) function for standard
	universe executables. This call is not checkpoint-safe in that
	the standard universe job could call it twice or more times
	in the event of a resumption from an earlier checkpoint. The
	invocation of this call by the shadow on behalf of the user
	job is controlled by a configuration file parameter called
	\Attr{SHADOW\_ALLOW\_UNSAFE\_REMOTE\_EXEC} and is off by default.
	The full environment of the user job is preserved during the
	invocation of \Syscall(system) and this might cause problems in 
	heterogeneous submission contexts of the user is not careful.

\item Added support for a web services (SOAP) interface to Condor.
  For more information, see and section~\ref{API-WebService} on
  page~\pageref{API-WebService}.

  \Note Due to a bug in gSOAP, the SOAP support in Condor 6.7.6 does
  not work with all SOAP toolkits.
  Some of the responses that gSOAP generates contain unqualified tags.
  Therefore, SOAP toolkits that are strict (such as gSOAP or .Net)
  will not accept these poorly formed responses.
  SOAP toolkits that are more lax in the responses they accept (such
  as Axis, SOAP::Lite, or ZSI) will work with version 6.7.6.
  This problem has already been fixed and the solution will be
  released in Condor version 6.7.7.

\item Added support for the GT4 grid\_type in Condor's grid universe.
  This new grid type supports jobs submitted to grid resources
  controlled by Globus Toolkit version 4 (GT4).

  New configuration settings are required to support jobs
  submitted for the GT4 grid type.
  These settings have been added to the default configuration files
  shipped with Condor, but sites that are upgrading an existing
  installation and choosing to keep their old configuration files must
  add these settings to allow GT4 jobs to work:
\begin{verbatim}
## The location of the wrapper for invoking GT4 GAHP server
GT4_GAHP = $(SBIN)/gt4_gahp
 
## The location of GT4 files. This should normally be lib/gt4
GT4_LOCATION = $(LIB)/gt4

## gt4-gahp requires gridftp server. This should be the address of gridftp
## server to use
GRIDFTP_URL_BASE = gsiftp://$(FULL_HOSTNAME)
\end{verbatim}

\item Condor version 6.7.6 includes the Stork data movement system, 
  the Condor Credential Daemon (\Condor{credd}), and support for using
  MyProxy for credential management.
  However, currently these are only supported in our release for Linux
  using the 2.4 kernel with glibc version 2.3 (RedHat 9, etc).
  All of these features require changes to the Condor configuration
  files to function properly.
  The default configuration files shipped with Condor already include
  all the new settings, but sites upgrading an existing installation
  must add these new settings to their Condor configuration.
  For a list of settings and more information, see
  section~\ref{sec:Stork-Config-File-Entries} on 
  page~\pageref{sec:Stork-Config-File-Entries} for Stork,
  section~\ref{sec:Credd-Config-File-Entries} on
  page~\pageref{sec:Credd-Config-File-Entries} for \Condor{credd},
  and section~\ref{sec:MyProxy-Config-File-Entries} on
  page~\pageref{sec:MyProxy-Config-File-Entries} for MyProxy.
  For more information about MyProxy, you can also see  
  \URL{http://grid.ncsa.uiuc.edu/myproxy}

\item Added preliminary support for the High Availability Daemon (HAD).

\item Added a new \MacroNI{SCHED\_UNIV\_RENICE\_INCREMENT}
configuration variable used by the \Condor{schedd} for scheduler
universe jobs, analogous to the existing
\MacroNI{JOB\_RENICE\_INCREMENT} variable used by the \Condor{startd}
for other job universes.  The \MacroNI{SCHED\_UNIV\_RENICE\_INCREMENT}
variable is undefined by default, and when undefined, defaults to 0
internally.

\item The relative priority of a user's own jobs in the local
\condor{schedd} queue is no longer limited to the range -20 to +20,
but can be any integer value.

\item DAGMan Improvements:

\begin{itemize}

  \item \Condor{dagman} now inserts a \MacroNI{DAGParentNodeNames}
  attribute into classad of all Condor jobs it submits, containing the
  names of the job's parents in the DAG.  The list is in the form of a
  comma-delimited string.

  \item Added the \Condor{dagman} arguments \Opt{-noeventchecks} and
  \Opt{-allowlogerror} to \Condor{submit\_dag}.

\end{itemize}

\item \Condor{glidein} Improvements:

\begin{itemize}

  \item Added \Condor{glidein} options for setting up GSI authentication.

  \item Added \Condor{glidein} option {-run\_here} for direct
  execution of Glidein, instead of submitting it for remote execution.
  You may also save a script for doing this and then run the script
  through whatever mechanism you want (like some batch system
  interface not supported by Condor-G).

\end{itemize}

\item Added support for the \Macro{NEGOTIATOR\_CYCLE\_DELAY}
  configuration setting, which is only intended for expert
  administrators.
  For more information, see section~\ref{param:NegotiatorCycleDelay}
  on page~\pageref{param:NegotiatorCycleDelay}.


\end{itemize}

\noindent Bugs Fixed:

\begin{itemize}

\item Removed case-sensitivity of command-line argument names in
\Condor{submit\_dag}.

\item Fixed the {-r} (remote schedd) option in \Condor{submit\_dag}.

\item Condor versions 6.7.1 through 6.7.5 exhibit a bug in
  which the commands \Condor{off}, \Condor{restart}, and
  \Condor{vacate} did not handle the \Opt{-pool} command-line option
  correctly.
  The bug caused these commands to correctly query the central manager
  of the remote pool,
  and to incorrectly send the command to the central manager machine.
  This bug has now been fixed, and these tools no longer send
  the command to the central manager machine.

\end{itemize}

\noindent Known Bugs:

\begin{itemize}

\item None.

\end{itemize}


%%%%%%%%%%%%%%%%%%%%%%%%%%%%%%%%%%%%%%%%%%%%%%%%%%%%%%%%%%%%%%%%%%%%%%
\subsection{\label{sec:New-6-7-5}Version 6.7.5}
%%%%%%%%%%%%%%%%%%%%%%%%%%%%%%%%%%%%%%%%%%%%%%%%%%%%%%%%%%%%%%%%%%%%%%
%  This was 6.7.4, but we took 6.7.4 away and replaced it quickly
%  with 6.7.5
\noindent Release Notes:

\begin{itemize}

\item None.

\end{itemize}


\noindent New Features:

\begin{itemize}

\item Added DAG aborting feature -- a DAG can be configured to
abort immediately if a node exits with a given exit value.

\item The dedicated scheduler can now preempt running MPI jobs from 
appropriately configured machines. See 
~\ref{sec:Configure-Dedicated-Preemption} for details.

\item The MPI universe now supports submit files with multiple procs (queue 
commands), each with distinct requirements.  This is useful for placing
the head node of an MPI job on a specific machine, and the rest of the 
nodes elsewhere. See ~\ref{sec:MPI} for details.

\item The \Condor{negotiator} now publishes its own ClassAd to the
  \Condor{collector} which includes the IP address and port where it
  is listening.
  This negotiator ClassAd can be viewed using the new
  \Opt{-negotiator} option with \Condor{status}.
  In addition to removing an unnecessary fixed port for the
  \Condor{negotiator}, this change corrects some problems with
  commands that attempted to communicate directly with the
  \Condor{negotiator}.
  These bugs were first listed in the Known Bugs section of the 6.6.0
  version history.

  To enable this feature and have the \Condor{negotiator} listen on a
  dynamic port, you must comment out the \Macro{NEGOTIATOR\_HOST}
  setting in your configuration file.
  The new example configuration files shipped with version 6.7.4 and
  later will already have this setting undefined.
  However, if you upgrade your binaries and retain an older copy of
  your configuration files, you should consider commenting out 
  \MacroNI{NEGOTIATOR\_HOST}.

  To disable this feature and have the \Condor{negotiator} still
  listen on a well-known port, you can uncomment the
  \MacroNI{NEGOTIATOR\_HOST} setting in the default configuration. 
  For example:
\begin{verbatim}
NEGOTIATOR_HOST = $(CONDOR_HOST)
\end{verbatim}

  Pools that are comprised of older versions of Condor and a 6.7.4 or
  later central manager machine should either continue to use their
  old \File{condor\_config} file (which will still have
  \MacroNI{NEGOTIATOR\_HOST} defined) or they should re-define the
  \MacroNI{NEGOTIATOR\_HOST} setting in the new example configuration
  files which are used during the installation process.

\item Added optional \Expr{DAGMAN\_RETRY\_SUBMIT\_FIRST} configuration
parameter that tells \Condor{dagman} whether to immediately retry
the submit if a node submit fails, or to put that job at the end of
the ready jobs queue.  The default is TRUE, which retries the failed
submit before trying to submit any other jobs.

\item The schedd now uses non-blocking connection attempts when contacting
startds.  This prevents the long (typically 40 second) hang of all schedd
operations when the connection attempt does not complete, due to
network problems.

\end{itemize}

\noindent Bugs Fixed:

\begin{itemize}

\item Fixed a performance problem with the standard universe when
\Syscall{gettimeofday} is called in a very tight loop by the application.

\item Fixed the default value of \Macro{OPSYS} in the MacOSX version
  of Condor.
  Once again, Condor reports \verb@OSX@ for all versions of MacOSX.
  This bug was introduced in version 6.7.3 of Condor.

\item Fixed a bug in \Condor{dagman} that caused it to be killed if
the \Expr{DAGMAN\_MAX\_SUBMIT\_ATTEMPTS} parameter was set to too
high a value.

\item Fixed a bug in \Condor{gridmanager} that caused it to crash if
the grid\_monitor was activated.

\item Fixed support for the getdents64() system call inside the
  standard universe on Linux and Solaris.

% Gnats PR 467
\item Fixed a bug in \Condor{dagman} that dealt
incorrectly with the problem of Condor sometimes writing both a
terminated and an aborted event for the same job. The spurious
aborted event is now ignored.

\end{itemize}

\noindent Known Bugs:

\begin{itemize}

\item None.

\end{itemize}


%%%%%%%%%%%%%%%%%%%%%%%%%%%%%%%%%%%%%%%%%%%%%%%%%%%%%%%%%%%%%%%%%%%%%%
\subsection{\label{sec:New-6-7-3}Version 6.7.3}
%%%%%%%%%%%%%%%%%%%%%%%%%%%%%%%%%%%%%%%%%%%%%%%%%%%%%%%%%%%%%%%%%%%%%%

\noindent Release Notes:

\begin{itemize}

\item This release contains all the bug fixes from the 6.6 stable
  series up to and including version 6.6.7, and some of the fixes that
  will be included in version 6.6.8.
  The bug fixes in version 6.6.8 that were not included in version
  6.7.3 are listed in a separate section of the 6.6.8 version
  history. 

\end{itemize}


\noindent New Features:

\begin{itemize}

\item Added Full Ports of Condor to Redhat Fedora Core 1, 2 and 3 on
the 32-bit x86 architecture. 
Please read the Linux platform specific
section~\ref{sec:platform-linux-fed} in this manual for more information
on caveats with this port.

\item Added a feature to \Condor{dagman} that will allow VARS names to include
numerics and underscores.

\item Added optional \Expr{COLLECTOR\_HOST\_FOR\_NEGOTIATOR} configuration parameter to indicate which \Condor{collector} the  \Condor{negotiator} on this (local) host should query first. This is designed to improve negotiation performance.

\item Added a new \Condor{dagman} capability to allow the DAG to continue
if it encounters a double run of the same node job (set the
\Expr{DAGMAN\_IGNORE\_DUPLICATE\_JOB\_EXECUTION} parameter to true to do this).

\item Added Condor-C: the "condor" grid\_type.  Condor-C allows jobs to be handed from one \Condor{schedd} to another \Condor{schedd}.

\item Added \Opt{setup\_here} option to \Condor{glidein} for cases where
direct installation is desired instead of submitting a setup job to the
remote gatekeeper.  (For example, this is useful when doing an installation
onto AFS.)

\item If \Attr{RemoteOwner} is exported via \Expr{STARTER\_VM\_EXPRS} into the
ad of other virtual machines, the \Condor{negotiator} automatically inserts
\Attr{RemoteUserPrio} into the ad as well, so policy expressions can now take
into account the priority of jobs running on other virtual machines on the
same host.

\item Linux 2.6 kernels do not update the access time for console devices,
so Condor was unable to detect if there has been activity at the keyboard
or mouse. As a work-around, Condor now polls /proc/interrupts to detect
if the keyboard has requested attention. This does not work for USB keyboards
or pseudo TTYs, so \Attr{ConsoleIdle} on 2.6 kernels will be wrong for some
devices. Future versions of Condor or Linux may correct this.

\item \Condor{dagman} no longer removes the X509\_USER\_PROXY environment 
variable.
This should allow users to set the environment variable before invoking 
\Condor{submit\_dag} and have the jobs submitted by \Condor{dagman} correctly
find the proxy file.

\end{itemize}

\noindent Bugs Fixed:

\begin{itemize}

\item Fixed a \Condor{dagman} bug that could cause it to leave jobs running
when aborting a DAG.

\item Fixed a \Condor{dagman} bug which, if its debug level was set to
zero (silent), could cause it to to improperly recognize persistent
\Condor{submit} failures.

\item Fixed a bug in Condor's file transfer mechanism that showed up
  when users tried to use streaming output for either STDOUT or
  STDERR.
  There were situations where Condor would attempt to transfer back
  the STDOUT or STDERR file from the execution host, even though these
  files didn't exist and all the data was already streamed back to the
  submit host.
  Now, if either \Attr{stream\_output} or \Attr{stream\_error} are set
  to true in the job submit description file, Condor will transfer any
  other output but will not attempt to transfer back STDOUT or STDERR.

\item The Condor user log library (libcondorapi) now correctly handles
  execute events that lack a hostname.

\end{itemize}

\noindent Known Bugs:

\begin{itemize}

\item Unfortunately, the default \Macro{OPSYS} value for the MacOSX
  version of Condor was incorrectly changed in version 6.7.3.
  Condor used to always report \verb@OSX@, but in version 6.7.3 it
  will report either \verb@OSX10_2@, \verb@OSX10_3@, or
  \verb@OSX_UNK@.
  This is wrong, since Condor jobs submitted to any version of OSX
  should be able to run on any other version of OSX, and the above
  change needlessly partitions resources and complicates things for
  end-users.
  Therefore, anyone running version 6.7.3 on MacOSX is encouraged to
  add the following line to their global \File{condor\_config} file:
\begin{verbatim}
OPSYS = OSX
\end{verbatim}

  If your pool is already running the new release, you can cause the
  above change to take effect by running the following command on your
  pool's central manager machine (or any machine listed in the
  \MacroNI{HOSTALLOW\_ADMINISTRATOR} list) after you have changed the
  \MacroNI{OPSYS} value in your configuration:
\begin{verbatim}
condor_reconfig -all
\end{verbatim}

  However, if you have already submitted jobs to your pool with the
  old \MacroNI{OPSYS} value, the \Attr{Requirements} expression in
  those jobs will still refer to the incorrect value.
  In this case, you should either a) wait for the jobs to complete
  before making the above change, b) remove the jobs and resubmit
  them after you've made the change, or c) manually run \Condor{qedit}
  on the jobs to change their \Attr{Requirements} expressions.

\item When running in recovery mode on a DAG that has PRE scripts,
\Condor{dagman} may attempt more than the specified number of retries
of a node (counting retries attempted during the first run of the
DAG).  This is because if a node fails because of the PRE script
failing, that fact is not recorded in the log, so that retry is missed
in recovery mode.

\end{itemize}



%%%%%%%%%%%%%%%%%%%%%%%%%%%%%%%%%%%%%%%%%%%%%%%%%%%%%%%%%%%%%%%%%%%%%%
\subsection{\label{sec:New-6-7-2}Version 6.7.2}
%%%%%%%%%%%%%%%%%%%%%%%%%%%%%%%%%%%%%%%%%%%%%%%%%%%%%%%%%%%%%%%%%%%%%%

\noindent Release Notes:

\begin{itemize}

\item Condor Version 6.7.2 includes some bug fixes from Version 6.6.7,
but none from Version 6.6.8.

\item MPI users who are upgrading from previous versions of Condor
to version 6.7.2 will need to modify the 
\Macro{MPI\_CONDOR\_RSH\_PATH} configuration macro of their dedicated
resource to be \MacroU{LIBEXEC} instead of \MacroUNI{SBIN}.
Users who are installing Condor version 6.7.2
for the first time will not need to make any changes.

\end{itemize}


\noindent New Features:

\begin{itemize}

\item Added an \Macro{INCLUDE} configuration file variable
   to define the location of header files shipped with Condor
   that are currently needed to be included when compiling
   Condor APIs.
   When \MacroNI{INCLUDE} is defined,
   \Condor{config\_val} can be used to list header files.


\item A Condor pool can now support multiple Collectors. This should
  improve stability due to automatic failover. All daemons will now
  send updates to ALL of the specified collectors. All daemons/tools
  will query the Collectors in sequence, until an appropriate 
  response is received. Thus if one (or more) of the Collectors are 
  down, the pool will continue to function normally, as long as 
  there is at least one functioning Collector. 
  You can specify multiple (comma-separated) collector host (and port) 
  addresses in the \Expr{COLLECTOR\_HOST} entry in the configuration
  file. A given \Condor{master} can only run one Collector.

\item When the \Condor{master} is started with the \Opt{-r} option to
  indicate that it should quite after a period of time, the
  \Condor{startd} will now indicate how much time is remaining before it
  exits. It does this by advertising TimeToLive in the machine
  ClassAd.

\item Added new macro \Macro{JOB\_START\_COUNT} that works in
conjunction with existing macro \Macro{JOB\_START\_DELAY} to 
throttle job starts.
Together, this macro pair provides greater flexibility
tuning job start rate given available \Condor{schedd} performance.

\item Added a \MacroNI{LIBEXEC} directory to the install process.
Support commands that
the Condor system needs will be added to this directory in future releases.
This directory should not be added to a user or system-wide path.  

\item Added the ability to decide for each file that condor transfers whether
it should be encrypted or not, using encrypt\_input\_files, 
dont\_encrypt\_input\_files, encrypt output files, and
        dont\_encrypt\_output\_files in the job's submit file.

\item Added DISABLE\_AUTHENTICATION\_IP\_CHECK which will work around problems
on dual-homed machines where the IP address is reported incorrectly to condor.
This is particularly a problem when using Kerberos on multi-homed machines.

\end{itemize}

\noindent Bugs Fixed:

\begin{itemize}

\item Fixed a bug on Linux systems caused by both 
      Condor and the Linux distribution having a library file 
      called \File{libc.a}.
      The problem caused the link step to fail on Condor API
      programs.
      The evaluation order to determine the location of library
      files caused use of the wrong file, given the duplicate naming.
      The bug is fixed by renaming the Condor library files.

\item When the \Condor{startd} is evaluating the state of each virtual
  machine (VM), it now refreshes any ClassAd attributes which are
  shared from other virtual machines (using \Expr{STARTD\_VM\_EXPRS})
  before it tries to evaluate.
  This way, if a given VM changes its state, all other VMs will
  immediately see this state change.

\item Fixed a bug where you couldn't transfer input files larger than 2 gigabytes.

\item Condor can now detect the size of memory on a Linux machine with the 2.6
kernel.

\item JAR files specified in the submit file were not being transfered
along with the job unless they were also explicitly placed in the list
of input files to transfer. Now, the JAR files are implicitly added to the
list of input files to transfer.

\end{itemize}

\noindent Known Bugs:

\begin{itemize}

\item None.

\end{itemize}




%%%%%%%%%%%%%%%%%%%%%%%%%%%%%%%%%%%%%%%%%%%%%%%%%%%%%%%%%%%%%%%%%%%%%%
\subsection{\label{sec:New-6-7-1}Version 6.7.1}
%%%%%%%%%%%%%%%%%%%%%%%%%%%%%%%%%%%%%%%%%%%%%%%%%%%%%%%%%%%%%%%%%%%%%%

\noindent Release Notes:

\begin{itemize}

\item Version 6.7.1 contains all of the features, ports, and bug fixes
  from the previous stable series, up to and including version 6.6.6.
  There are a few additional bugs that have been fixed in the 6.6.x
  stable series which have not yet been released, but which will
  appear in version 6.6.7.
  These bug fixes have been included in version 6.7.1, and appear in
  the ``Bugs fixes included from version 6.6.7'' list below.
  In addition, a number of new features and some bug fixes have been
  made, which are described below in more detail.

\item None.

\end{itemize}


\noindent New Features:

\begin{itemize}

\item Added an option to DAGMan's retry ability. If a DAG specifies
  something like ``RETRY job 10 unless-exit 9'', then the retries will
  only happen if the node doesn't exit with a value of 9. 

\item Condor-G can now submit jobs to Globus 3.2 (WS) (for jobs with 
  \Expr{universe = grid}, \Expr{grid\_type = gt3}). Submitting to Globus 
  3.0 (as in Condor 6.7.0) is no longer supported. Submitting to pre-WS 
  Globus (2.x) is still supported (\Expr{grid\_type = gt2}).

\item Added new startd policy expression MaxJobRetirementTime.  This
specifies the maximum amount of time (in seconds) that the startd
is willing to wait for a job to finish on its own when the startd
needs to preempt the job (for owner preemption, negotiator preemption,
or graceful startd shutdown).

\item Added -peaceful shutdown/restart mode.  This will shut down the
startd without killing any jobs, effectively treating both
\Expr{MaxJobRetirementTime} and \Expr{GRACEFUL\_SHUTDOWN\_TIMEOUT} as
infinite.  The default shutdown/restart mode is still -graceful, which
behaves according to whatever \Expr{MaxJobRetirementTime} and
\Expr{GRACEFUL\_SHUTDOWN\_TIMEOUT} are.  The behavior of -fast mode
is unchanged; it kills jobs immediately, regardless of the other
timeout settings.

\item Jobs can now be submitted as ``noop'' jobs. Jobs submitted with
  \Expr{noop\_job = true} will not be executed by Condor, and instead will
   immediately have a terminate event written to the job log file and 
   removed from the queue. This is useful for DAGs where the pre-script
   determines the job should not run.

\item Added preliminary support for the Tool Daemon Protocol (TDP)
  into Condor.
  This protocol is still under development, but the goal is to provide
  a generic way for scheduling systems (daemons) to interact with
  monitoring tools.
  Assuming this protocol is adopted by other scheduling systems and by
  various monitoring tools, it would allow arbitrary combinations of
  tools and schedulers to co-exist, function properly, and provide
  monitoring services for jobs running under the schedulers.
  This initial support allows users to specify a ``tool'' that should
  be spawned along-side their regular Condor job.
  On Linux, the ability to have the batch Condor job suspend
  immediately upon start-up is also implemented, which allows a
  monitoring tool to attach with ptrace() before the job's main()
  function is called.

\end{itemize}

\noindent Bugs Fixed:

\begin{itemize}

\item Fixed a significant memory leak in the \Condor{schedd} that was
  introduced in version 6.7.0.
  In 6.7.0, the \Condor{schedd} would leak a copy of ClassAd for every
  job it tried to spawn (on average, around 2000 bytes per job).

\item Fixed the bugs in Condor's MPI support that were introduced in
  version 6.7.0.
  Condor now supports MPI jobs linked with MPICH 1.2.4 and older.
  Improved Condor's log messages and email notifications when MPI jobs
  run on multiple virtual machines (the messages now include the
  appropriate ``vmX'' identifier, not just the hostname).
  Unfortunately, due to changes in MPICH between version 1.2.4 and
  1.2.5, Condor's MPI support is not compatible with MPICH 1.2.5.
  We will be addressing this problem in a future release.

\end{itemize}

\noindent Bugs fixes included from version 6.6.7:

\begin{itemize}

\item Fixed an important bug in the low-level code that Condor uses to
  transfer files across a network.
  There were certain temporary failure cases that were being treated
  as permanent, fatal errors.
  This resulted in file transfers that aborted prematurely, causing
  jobs to needlessly re-run.
  The code now gracefully recovers from these temporary errors.
  This should significantly help throughput for some sites,
  particularly ones that transfer very large files as output from
  their jobs.

\item Fixed a number of bugs in the \Opt{-format} option to \Condor{q}
  and \Condor{status}.
  Now, these tools will properly handle printing boolean expressions
  in all cases.
  Previously, depending on how the boolean evaluated, either the
  expression was printed, or the tool could crash.
  Furthermore, the tools do a better job of handling the different 
  types of format conversion strings and printing out the appropriate
  value.
  For example, if a user tries to print out a boolean attribute with
  \verb@condor_status -format "%d\n" HasFileTransfer@, the
  \Condor{status} tool will evaluate \Attr{HasFiletransfer} and print
  either a 0 or a 1 (FALSE or TRUE).
  If, on the other hand, a user tries to print out a boolean attribute
  with \verb@condor_status -format "%s\n" HasFileTransfer@, the
  \Condor{status} tool will print out the string ``FALSE'' or ``TRUE''
  as appropriate.

\item The ClassAd attribute scope resolution prefixes, \texttt{MY.}
  and \texttt{TARGET.}, are no longer case sensitive.

\item \Condor{dagman} now does better checking for inconsistent events
(such as getting multiple terminate events for a single job).  This
checking can be disabled with the \Opt{-NoEventChecks} command-line
option.

\end{itemize}

\noindent Known Bugs:

\begin{itemize}

\item None.

\end{itemize}




%%%%%%%%%%%%%%%%%%%%%%%%%%%%%%%%%%%%%%%%%%%%%%%%%%%%%%%%%%%%%%%%%%%%%%
\subsection{\label{sec:New-6-7-0}Version 6.7.0}
%%%%%%%%%%%%%%%%%%%%%%%%%%%%%%%%%%%%%%%%%%%%%%%%%%%%%%%%%%%%%%%%%%%%%%

\noindent Release Notes:

\begin{itemize}

\item Version 6.7.0 contains all of the features, ports, and bug fixes
  from the previous stable series, up to and including version 6.6.4.
  In addition, a number of new features and some bug fixes have been
  made, which are described below in more detail.

\end{itemize}


\noindent New Features:

\begin{itemize}

\item Added support for vanilla and Java jobs to reconnect when the
  connection between the submitting and execution nodes is lost for
  any reason.
  Possible reasons for this disconnect include: network outages,
  rebooting the submit machine, restarting the Condor daemons on the
  submit machine, etc.
  If the execution machine is rebooted or the Condor daemons are
  restarted, reconnection is not possible.
  To take advantage of this reconnect feature, jobs must be submitted
  with a \Attr{JobLeaseDuration}.
  There are new events in the UserLog related to disconnect and
  reconnect.

\item Added a new Condor tool, \Condor{vacate\_job}.
  This command is similar to \Condor{vacate}, except the kinds of
  arguments it takes define jobs in a job queue, not machines to
  vacate.
  For example, a user can vacate a specific job id, all the jobs in a
  given cluster, all the jobs matching a job queue constraint, or even
  all jobs owned by that user.
  The owner of a job can always vacate their own jobs, regardless of
  the pool security policy controlling \Condor{vacate} (which is an
  administrative command which acts directly on machines).
  See the new command reference, section~\ref{man-condor-vacate-job}
  on page~\pageref{man-condor-vacate-job} for details.
  
\item Added a new ``High Availability'' service to the \Condor{master}.
   You can now specify a daemon which can have ``fail over'' capabilities
   (i.e. the master on another machine can start a matching daemon if the
   first one fails).  Currently, this is only available over a shared
   file system (i.e. NFS), and has only been tested for the \Condor{schedd}.

\item Scheduler universe jobs on UNIX can now specify a
  \Attr{HoldKillSig}, the signal that should be sent when the job is
  put on hold.
  If not specified, the default is to use the \Attr{KillSig}, and if
  that is not defined, the job will be sent a SIGTERM.
  The submit file keyword to use for defining this signal is
  \AdAttr{hold\_kill\_sig}, for example,
  \verb@hold_kill_sig = SIGUSR1@.

\item The \Condor{startd} can now support policies on SMP machines
  where each virtual machine (VM) has knowledge of the other VMs on
  the same host.
  For example, if a job starts running on one of the VMs, a job
  running on another VM could immediately be suspended.
  This is accomplished by using the new configuration variable
  \Macro{STARTD\_VM\_EXPRS}, which is a list of ClassAd attribute
  names that should be shared across all VMs on the machine.
  For each VM on the machine, every attribute in this list is looked
  up in the VM-specific machine ClassAd, the attribute name is given a
  prefix indicating what VM it came from, and then inserted into the
  machine ClassAds of all the other VMs.

\item The \Condor{startd} publishes four new attributes into the
  machine ClassAds it generates when it is in the Claimed state:
  \Attr{TotalJobRunTime}, \Attr{TotalJobSuspendTime},
  \Attr{TotalClaimRunTime}, \Attr{TotalClaimSuspendTime}.
  These attributes keep track of the total time the resource was
  either running a job (in the Busy activity) or had a job suspended,
  regardless of how many suspend/resume cycles the job went through.
  The first two attributes (with ``Job'' in the name) keep track for a
  single job (i.e. since the last time the resource was
  Claimed/Idle). 
  The last two attributes (with ``Claim'' in the name) keep track of
  these totals across all jobs that ran under the same claim
  (i.e. since the last state change into the Claimed state).

\item Added a \Opt{-num} option to the \Condor{wait} tool to wait for
   a specified number of jobs to finish.

\item Added a configuration option \Macro{STARTER\_JOB\_ENVIRONMENT}
   so the admin can configure the default environment inherited by
   user jobs.

\item Added a (configurable, defaults to off) feature to the \Condor{schedd}
   to allow backup the spool file before doing anything else.

\item The "Continuous" option of the \Condor{startd} ``cron'' jobs is
being deprecated.   It's being replaced by two new options which
control separate aspects of it's behavior:
\begin{itemize}
\item "WaitForExit" specifies the "exit timing" mode
\item "ReConfig" specifies that the job can handle SIGHUPs, and it should 
be sent a SIGHUP when the \Condor{startd} is reconfigured.
\end{itemize}

\item A lot of the items logged by the \Condor{startd} ``cron'' logic,
changed to D\_FULLDEBUG (from D\_ALWAYS), etc.

\item Added \Macro{NEGOTIATOR\_PRE\_JOB\_RANK} and
\Macro{NEGOTIATOR\_POST\_JOB\_RANK}.  These expressions are applied
respectively before and after the user-supplied job rank when deciding
which of the possible matches to choose.  (The existing expression
\Macro{PREEMPTION\_RANK} is applied after
\Macro{NEGOTIATOR\_POST\_JOB\_RANK}.)  The pool administrator may use
these expressions to steer jobs in ways that improve the overall
performance of the pool.  For example, using the pre job rank,
preemption may be avoided as long as there are idle machines, even
when the user-supplied rank expression prefers a machine that happens
to be busy.  Using the post job rank, one could steer jobs towards
machines that are known to be dedicated to batch jobs, or one could
enforce breadth-first instead of depth-first filling of a cluster of
multi-processor machines.

\item Added the ability for Condor to transfer files larger than 2G on
platforms that support large files.  This works automatically for
transferred executables, input files and output files.

\item Added the ability for jobs to stream back standard input, output, and
error files while running.  This is activated by the \Opt{stream\_input},
\Opt{stream\_output}, and \Opt{stream\_error} options to \Condor{submit}.
Note that this feature is incompatible with the new feature described
above where the shadow and starter can reconnect in certain
circumstances. 

\item Added support for vanilla jobs to be mirrored on a second
  \Condor{schedd}. The jobs are submitted to the second \Condor{schedd}
  on hold and will be released if the second \Condor{schedd} hasn't
  heard from the first \Condor{schedd} (actually, a \Condor{gridmanager}
  running under the first \Condor{schedd}) for a configurable amount of
  time. Once the second \Condor{schedd} releases the jobs, the first
  \Condor{schedd} acts as a mirror, reflecting the state of the jobs on
  the second \Condor{schedd}.
  To use this mirroring feature, jobs must be submitted
  with a \Attr{mirror\_schedd} parameter in the submit file and require
  no file transfer.

\end{itemize}


\noindent Bugs Fixed:

\begin{itemize}

\item Fixed a bug in the \Condor{startd} ``cron'' logic which caused the
\Condor{startd} to except when trying to delete a job that could never
be run (i.e. invalid executable, etc).

\item Fixed a bug in \Condor{startd} ``cron'' logic which caused it to
not detect when the starting of a ``job'' failed.

\item Fixed several bugs in the reconfiguration handling of the
\Condor{startd} ``cron'' logic.  In particular, even if the job has
the "reconfig" option set (or "continuous"), the job(s) won't be sent
a SIGHUP when the startd first starts, or when the job itself is first
run (until it outputs its first output block, defined by the "-"
separator).

\end{itemize}


\noindent Known Bugs:

\begin{itemize}

\item Condor's MPI support (for MPICH 1.2.4) was broken by other
  changes in version 6.7.0.
  Support for MPI jobs will return in Condor version 6.7.1.

\end{itemize}
\begin{center}
\begin{table}[hbt]
\begin{tabular}{|ll|} \hline
\emph{Architecture} & \emph{Operating System} \\ \hline \hline
Hewlett Packard PA-RISC (both PA7000 and PA8000 series) & HPUX 10.20 \\ \hline
Sun SPARC Sun4m,Sun4c, Sun UltraSPARC & Solaris 2.6, 2.7, 8, 9 \\ \hline
Silicon Graphics MIPS (R5000, R8000, R10000) & IRIX 6.5 (clipped) \\ \hline
Intel x86 & Red Hat Linux 7.1, 7.2, 7.3, 8.0 \\
 & Red Hat Linux 9 \\
 & Windows 2000 Professional and Server, 2003 Server (clipped) \\
 & Windows XP Professional (clipped) \\ \hline
ALPHA & Digital Unix 4.0 \\
 & Red Hat Linux 7.1, 7.2, 7.3 (clipped) \\
 & Tru64 5.1 (clipped) \\ \hline
PowerPC & Macintosh OS X (clipped) \\
 & AIX 5.2L (clipped) \\ \hline
Itanium & Red Hat Linux 7.1, 7.2, 7.3 (clipped) \\
 & SuSE Linux Enterprise 8.1 (clipped) \\ \hline
\end{tabular}
\caption{\label{supported-platforms}Condor 6.7.0 supported platforms}
\end{table}
\end{center}

%%%%%%%%%%%%%%%%%%%%%%%%%%%%%%%%%%%%%%%%%%%%%%%%%%%%%%%%%%%%%%%%%%%%%%%
\section{\label{sec:History-6-6}Stable Release Series 6.6}
%%%%%%%%%%%%%%%%%%%%%%%%%%%%%%%%%%%%%%%%%%%%%%%%%%%%%%%%%%%%%%%%%%%%%%

This is a stable release series of Condor.
It is based on the 6.5 development series.
All new features added or bugs fixed in the 6.5 series are available
in the 6.6 series.
The details of each version are described below.

%%%%%%%%%%%%%%%%%%%%%%%%%%%%%%%%%%%%%%%%%%%%%%%%%%%%%%%%%%%%%%%%%%%%%%
\subsection{\label{sec:New-6-6-11}Version 6.6.11}
%%%%%%%%%%%%%%%%%%%%%%%%%%%%%%%%%%%%%%%%%%%%%%%%%%%%%%%%%%%%%%%%%%%%%%

\noindent Release Notes:

\begin{itemize}

\item A security team at UW-Madison is conducting an onging security
audit of the Condor system and has identified a few important
vulnerabilities.
Condor versions 6.6.11 and 6.7.18 fix these security problems and
other bugs.
There have been no reported exploits, but all sites are urged to
upgrade immediately.

The Condor Team will publish detailed reports of these vulnerabilities
on 2006-04-24, 4 weeks from the date when the fixes were first
released (2006-03-27).
This will allow all sites time to upgrade before enough information to
exploit these bugs is widely available.

\end{itemize}

%\noindent New Features:
%
%\begin{itemize}
%
%\item None.
%
%\end{itemize}

\noindent Security Bugs Fixed:

\begin{itemize}

\item Bugs in previous versions of Condor could allow any user who can
submit jobs on a machine to gain access to the ``condor'' account
(or whatever non-privileged user the Condor daemons are running as).
This bug can not be exploited remotely, only by users already logged
onto a submit machine in the Condor pool.

\item The security of the ``\condor{config\_val} -set'' feature was
found to be insufficient, so this feature is now disabled by default.
There are new configuration settings to enable this feature in a
secure manner.
Please read the descriptions of \Macro{ENABLE\_RUNTIME\_CONFIG},
\Macro{ENABLE\_PERSISTENT\_CONFIG} and \Macro{PERSISTENT\_CONFIG\_DIR}
in the example configuration file shipped with the latest Condor
releases, or in section~\ref{param:EnableRuntimeConfig} on
page~\pageref{param:EnableRuntimeConfig}. 

\end{itemize}


\noindent Other bugs fixed that are included in version 6.7.18:

\begin{itemize}

% gnats PR #646
\item Fixed a bug which could cause the \Condor{collector} to crash
  when it receives certain types of malformed ads.

\item Fixed a bug which caused the \Condor{collector} incorrectly
  handle ads in which the \AdAttr{UpdateInterval} attribute is set.
  In particular, the previous versions of the \Condor{collector} will
  use the \AdAttr{UpdateInterval} value as the maximum \Term{lifetime}
  of the ad when aging the ads, which could cause it to remove the ad
  prematurely.
  The \Condor{collector} now looks at the \AdAttr{ClassAdLifetime}
  attribute, and uses its value (if set).
  \Note No current Condor daemons are publishing either of these
  attributes, but may do so in the future.

\end{itemize}

\noindent Bugs fixed that are included in version 6.7.14:

\begin{itemize}

\item Fixed a rare problem in the \Condor{negotiator} where a poorly
  formed classad from a single \Condor{schedd} could halt negotiation
  for the entire pool.
  This poorly formed ad could only happen in extrememly rare
  circumstances, but it was possible.
  Now, the \Condor{negotiator} will simply ignore poorly formed
  classads and continue to negotiate with any other \Condor{schedd} in
  the system that has idle jobs.

\item Fixed a bug which caused log messages which should contain
  ``PRIV\_USER\_FINAL'' to be ``PRIV\_USER\_FINALPRIV\_FILE\_OWNER''.
  It's also possible that this same bug could cause crashes if any
  daemon attempts to log a message which would refer to
  ``PRIV\_FILE\_OWNER''.

\item Fixed a bug which caused the \Condor{starter} to exit with an
  error when the sum total of the file transfer size exceeded 2G.
  This, in turn, caused a ``shadow exeception'', and the job would
  fail.

\end{itemize}


\noindent Bugs fixed that are included in version 6.7.11:

\begin{itemize}

\item In very rare cases, the \Condor{startd} could get into an
infinite loop if a job it was managing was suspended and then there
were fatal errors trying to send commands to evict the corresponding
\Condor{starter}.
This bug has been fixed, and the \Condor{startd} will now correctly
recover (and cleanup all processes) if it fails to send commands to a
starter managing a suspended job.

\item Condor on Solaris has been patched to work around a Solaris stdio
limitation of 255 maximum file descriptors.  Before this patch, heavily
loaded Condor daemons running on Solaris, particularly the \Condor{schedd},
could exit complaining about lack of file descriptors for dprintf.

\item Fixed a bug where the \Condor{starter} would follow symbolic links to
directories, when calculating job disk usage.  This could cause an incorrect
job disk usage calculation, or hang the starter upon encountering an infinite
directory loop.  This bug only affected Unix platforms.

\item For Globus jobs, the Rematch expression is now evaluated when a
submit fails (in addition to when a submit commit times out).

\item Fixed a bug that caused the \Condor{gridmanager} to go into an
infinite loop if an entry in the job's environment string was missing
an equals sign.

\end{itemize}

\noindent Bugs fixed that are included in version 6.7.9:

\begin{itemize}

\item Fixed a bug where the \Condor{startd} would erroneously compute the 
console idle time utilizing a file called /proc/interrupts on unix machines
that were not linux. 

\item Fixed a bug where the \Condor{negotiator} might dump core if it was
reconfiged in the middle of a negotiation cycle.

\item Fixed a bug where the \Condor{negotiator} might dump core if a startd
had a name longer than 63 bytes. 

\item Fixed a bug that could cause \Condor{userprio} to crash if the
data it gets back from the \Condor{negotiator} is invalid.

\item Fixed a bug where
\Macro{DEFAULT\_PRIO\_FACTOR} was ignored if 
\Macro{ACCOUNTANT\_LOCAL\_DOMAIN} was not defined.

\end{itemize}

\noindent Bugs fixes irrelevant to the 6.7 series:

\begin{itemize}

\item Added the \Opt{-NoEventChecks} and the \Opt{-AllowLogError}
command-line flags to \Condor{submit\_dag} and the \Condor{submit\_dag}
man page (they were already in \Condor{dagman}).
Added \Opt{-r} and \Opt{-debug} to the \Condor{submit\_dag}
man page (they were already in \Condor{submit\_dag}, just not
documented).

\item Made command-line arguments case insensitive in the Windows
version of \Condor{submit\_dag}; also fixed log file checks in
that version.

\end{itemize}

\noindent Known Bugs:

\begin{itemize}

\item None.

\end{itemize}


%%%%%%%%%%%%%%%%%%%%%%%%%%%%%%%%%%%%%%%%%%%%%%%%%%%%%%%%%%%%%%%%%%%%%%
\subsection{\label{sec:New-6-6-10}Version 6.6.10}
%%%%%%%%%%%%%%%%%%%%%%%%%%%%%%%%%%%%%%%%%%%%%%%%%%%%%%%%%%%%%%%%%%%%%%

\noindent Release Notes:

\begin{itemize}

\item Most of the fixes included in this release were also included in
  version 6.7.7 (see below).

\item The \MacroNI{QUEUE\_CLEAN\_INTERVAL} timer is reset during a 
\Condor{schedd} reconfig only if this timer value has been changed.
Previously, the timer was reset during all \Condor{schedd} reconfigs, which
could prevent the \File{job\_queue.log} file from being cleaned.  Note that
this timer is always reset upon a \Condor{schedd} startup.  See the
related change for truncating the \File{job\_queue.log} below, for this same
release.

\item Previously, the \Condor{schedd} would over-react and exit if it
tried to send a user email and \MacroNI{SMTP\_SERVER} was undefined;
now it simply prints an error in the SchedLog and moves on.

\end{itemize}

%\noindent New Features:
%
%\begin{itemize}
%
%\item None.
%
%\end{itemize}

\noindent Bugs fixed that are included in version 6.7.7:

\begin{itemize}

\item Fixed a bug that could cause the file \File{job\_queue.log} in
	the Condor SPOOL directory to grow unnecessarily large, thereby
	slowing down the startup and/or shutdown times for the \Condor{schedd}
	daemon.

\item Fixed a critical bug where the console idle time for PS/2 keyboards
	and mice was not being updated correctly.

\item Fixed a bug in the \Condor{collector} that could cause it to
crash when parsing certain types of invalid ClassAds.  In particular, if
a Machine, Schedd or License ClassAd sent to the \Condor{collector} has
an IP address field which is empty (which should never happen), the
\Condor{collector} will crash.

\item Fixed some bugs in how the \Condor{schedd} handles a graceful
  shutdown (either because of a \Condor{off}) or a \verb@SIGTERM@ on
  UNIX): 
\begin{itemize}
  \item There was a minor bug if \Macro{JOB\_START\_DELAY} was set to
     0 that would prevent the \Condor{schedd} from correctly cleaning
     up during graceful shutdown.
     Now, the \Condor{schedd} will properly shutdown, even if
     \MacroNI{JOB\_START\_DELAY} is set to 0.

  \item Fixed a bug when there are scheduler universe jobs that were
    recently submitted to the queue.
    Previously, the shutdown code would not evict scheduler universe
    jobs that had been submitted since the last
    \Macro{SCHEDD\_INTERVAL} (which defaults to 5 minutes).
    So, if a user submitted a scheduler universe job and then someone
    shutdown Condor on that machine, the \Condor{schedd} would wait
    until the next \MacroNI{SCHEDD\_INTERVAL} had elapsed before
    evicting the job.
    Now, the schedd will always attempt to evict scheduler universe
    jobs during a shutdown, without waiting for this interval to pass.
\end{itemize}

\item A number of Windows-specific bugs were fixed:
\begin{itemize}
  \item It was possible under certain circumstances for execute
  directories to not be cleaned up properly. This has been fixed.

  \item Certain Asian locales would cause the \Condor{starter} to crash
  due to character translation problems. This has been fixed.

  \item Condor will now properly report memory sizes that exceed 2 GB.

  \item The \Condor{starter} would be unable to run jobs if the \verb@LOG@
  path had a period (.) in it. This has been fixed.

  \item The \Condor{startd} would leak memory, especially on SMP
  machines. This has been fixed.

  \item The \Condor{master} would crash immediately on Windows 2003
  Server if the firewall was enabled. This has been fixed.

\end{itemize}

% See Gnats PR 479. 
\item Fixed a bug in \Condor{dagman} that could cause \Condor{dagman}
to fail an assertion if PRE or POST scripts are throttled with the
\Opt{-maxpre} or \Opt{-maxpost} \Condor{submit\_dag} command line flags.

\end{itemize}

\noindent Bugs fixed that are NOT included in version 6.7.7:

\begin{itemize}

\item Fixed a bug where enabling the grid\_monitor for any globus
job handled by something other than a hard-coded list of jobmanager names
would cause the job to stay idle forever.  The hard-coded list of
jobmanager names was: condor, fork, lsf, pbs, and remote.  A jobmanager
by any other name (e.g. condor\_rh9, or lcgpbs) would cause the problem.
This bug was originally fixed in internal releases of 6.7.0, but it was
reintroduced by mistake in all public releases.

\item Fix the way \Condor{version} handles command line arguments
  (there were a number of problems and inconsistencies) and added a
  \Opt{-help} option and usage message.

\item Fixed some memory leaks in the \Condor{startd} that would be
induced by calling \Condor{reconfig} or \Condor{status} \Opt{-d}.

\item By design, Condor daemons will exit if their parent process
exits. On Windows, a bug introduced in v6.5.x series broke this
behavior. This is now fixed.

\item On Windows, users would often observe the \Condor{master} failing to
add exceptions for the Condor daemons to the Windows Firewall on Windows
XP SP2 or Windows 2003 Server SP1. The \Condor{master} will
now retry for a longer period of time to add these exceptions,
and the number of retries has now been made configurable. See
section~\ref{param:WindowsFirewallFailureRetry} on
page~\pageref{param:WindowsFirewallFailureRetry} for more information.

\end{itemize}

\noindent Known Bugs:

\begin{itemize}

\item None.

\end{itemize}

%%%%%%%%%%%%%%%%%%%%%%%%%%%%%%%%%%%%%%%%%%%%%%%%%%%%%%%%%%%%%%%%%%%%%%
\subsection{\label{sec:New-6-6-9}Version 6.6.9}
%%%%%%%%%%%%%%%%%%%%%%%%%%%%%%%%%%%%%%%%%%%%%%%%%%%%%%%%%%%%%%%%%%%%%%

\noindent Release Notes:

\begin{itemize}

\item Most of the fixes included in this release were also included in
  version 6.7.5.
  However, at the end of this section, a few fixes that were added to
  6.6.9 after 6.7.5 was released are mentioned separately.

\end{itemize}

%\noindent New Features:
%
%\begin{itemize}
%
%\item None.
%
%\end{itemize}

\noindent Bugs fixed that are included in version 6.7.5:

\begin{itemize}

\item Fixed a security bug in the \Condor{schedd} that could enable a
maliciously modified \Condor{submit} tool to overwrite files in the Condor
\Macro{SPOOL} subdirectory, including the job queue.

\item Fixed a bug where under very pathological file permission failure
conditions with a standard universe job, there would be a cycle of an
execute event followed by a termination event in the user log when the
job had not actually ran.


\end{itemize}

\noindent Bugs fixed that are NOT included in version 6.7.5:

\begin{itemize}

\item Fixed a memory management bug introduced in version 6.6.8 that
  could result in deallocated memory being referenced after a child
  process forked from a Condor daemon exits.

\item Fixed bugs in some Condor tools that failed to locate
  \Condor{startd} daemons that contained multiple \verb&@& signs in
  their \Attr{Name} attribute.
  For example, a virtual machine from a multiple-CPU \Condor{startd}
  spawned using glidein would have the name:
  \verb&vm1@[pid]@[hostname]&.
  All Condor tools that need to communicate with a \Condor{startd}
  like this will now succeed.

\item Removed a fixed-length buffer in the code that handled the
  \Macro{SUBSYS\_EXPRS} config file setting.
  Previously, if any attributes referred to were larger than
  approximate 1000 bytes, Condor daemons would crash.
  Now, there is no limit to the size of the attributes listed in 
  \MacroNI{SUBSYS\_EXPRS}.
  For more information about this setting, see
  section~\ref{param:SubsysExprs} on page~\pageref{param:SubsysExprs}.

\item Fixed a bug which would cause Condor to fail to cache user GID
information and potentially overwhelm NIS servers.

\item Fixed another bug which could cause UDP machine updates to be
dropped by the \Condor{collector}.

\end{itemize}

\noindent Known Bugs:

\begin{itemize}

\item If a DAG node has both retries and a POST script, and the
actual Condor job for the node fails, the POST script is not
run except after the last retry of the job (or if the job
succeeds).  (The POST script should be run each time the node
job is run, whether the job succeeds or not.)

\item Occasionally, Condor generates both a terminated event and
an aborted event for a job that is aborted.  If this happens for a
DAG node job, \Condor{dagman} considers this an error
and aborts the DAG.  If you run into this problem, you can avoid
the abort by adding the \Opt{-NoEventChecks} flag to argument list
in the \Condor{dagman} submit file generated by \Condor{submit\_dag}
(you have to do \Condor{submit\_dag} \Arg{-no\_submit} and hand-edit
the resulting submit file).  However, if you get the
double events on a node that has retries, \Condor{dagman} will assert.
The only fix for this is to upgrade to a 6.7.5 or newer \Condor{dagman}.
You can do this by simply installing a newer \Condor{dagman} executable,
without any other changes to your Condor installation.  It is fine to
run a 6.7 \Condor{dagman} on a 6.6 Condor installation.

\item \item In a DAG, if a node job generates an executable error event,
the DAG is aborted.  This can be worked around by adding the
\Opt{-NoEventChecks} flag to argument list in the \Condor{dagman}
submit file generated by \Condor{submit\_dag} (you have to do
\Condor{submit\_dag} \Arg{-no\_submit} and hand-edit the resulting
submit file).

\end{itemize}


%%%%%%%%%%%%%%%%%%%%%%%%%%%%%%%%%%%%%%%%%%%%%%%%%%%%%%%%%%%%%%%%%%%%%%
\subsection{\label{sec:New-6-6-8}Version 6.6.8}
%%%%%%%%%%%%%%%%%%%%%%%%%%%%%%%%%%%%%%%%%%%%%%%%%%%%%%%%%%%%%%%%%%%%%%

\noindent Release Notes:

\begin{itemize}

\item Most of the fixes included in this release were also included in
  version 6.7.3.
  However, at the end of this section, a few fixes that were added to
  6.6.8 after 6.7.3 was released are mentioned separately.

\end{itemize}


\noindent New Features:

\begin{itemize}

\item None.

\end{itemize}

\noindent Bugs Fixed:

\begin{itemize}

\item In version 6.6.7, we fixed bugs related to the
  \Opt{-format} option to various Condor tools.
  However, some sites were using \Opt{-format} in ways we did not
  expect, by not specifying any '%' conversion string in the format
  string at all.
  This used to work, given the old buggy code that handled
  \Opt{-format}, but the changes in version 6.6.7 broke this, and
  format strings without a '%' conversion string were ignored.
  Now, if the format string does not contain a '%' conversion string,
  the attribute name which follows it is once again ignored, and the
  format string is printed directly without any modification.
  For example, to print out the machine's \Attr{Name} (always defined)
  and the \Attr{RemoteUser} (only defined if the machine is claimed),
  and always print a newline (to keep the formatting legible), this
  command will now work:
\begin{verbatim}
% condor_status -f "%s " Name -f "%s " RemoteUser -f "\n" bogus
bird.cs.wisc.edu biguser@raven.cs.wisc.edu
condor.cs.wisc.edu
dodo.cs.wisc.edu biguser@raven.cs.wisc.edu
lark.cs.wisc.edu biguser@raven.cs.wisc.edu
raven.cs.wisc.edu
...
\end{verbatim}

\item Windows bug fixes:
\begin{itemize}

\item Fixed a bug in that would cause Condor to fail to gracefully
shutdown user jobs that are console applications (including batch
scripts).

\item Fixed an issue that would cause \Condor{store\_cred} to fail
if the user did not have \verb@NETWORK@ logon rights.

\item \Condor{store\_cred} \Opt{query} command would appear to succeed,
even if the stored credential was invalid (e.g. the password was changed
but the password stash was not updated). This has been fixed.


\item Fixed a bug that would cause the \Condor{startd} to crash under
certain conditions during job eviction. This bug was introduced in Condor
version 6.6.6.

\item Fixed a bug that would cause \Condor{dagman} to crash if it was
submitted as a non-Administrator user.

\item Fixed a bug that would cause Condor to occasionally kill processes
that didn't belong to it during job eviction or daemon restarts.

\item On startup, the \Condor{master} would occasionally fail to add the
daemons to the Windows XP firewall exception list because of a race with
the Windows SharedAccess service. This bug has been fixed.

\item If a user submitted a job with an invalid executable, the starter
would often wedge until the job was preempted. Now, the starter attempts
to detect invalid executables and prevent wedging.

\item Fixed issues that would cause \Condor{startd} to ``disappear''
from the pool because of dropped machine ad updates. This fix applies
to all platforms, but the symptoms were exhibited predominantly on
Windows machines.

\item Fixed a bug that could cause \Macro{HIGHPORT} and
\Macro{LOWPORT} parameters to be ignored if a Windows machine ran for
several weeks without being rebooted.

\end{itemize}

\item Starting with RedHat 9, newer versions of Linux began to produce
  core files named \File{core.<pid>}.
  This broke functionality in Condor that managed and transferred back
  any core file created by the job, since the \Condor{starter} was
  unable to locate the proper file.
  Now, Condor will correctly transfer back core files, even if they
  are created as \File{core.<pid>}.
  This functionality works in all universes, and is independent of
  Condor's file transfer mechanism.


\item Fixed a bug that was causing \Condor{startd} to consume large
amounts of memory over long periods of time.

\item Fixed a bug that was causing \Condor{startd} to fail to start up
with the message, "caInsert: Can't insert CpuBusy into target ClassAd."

\item Fixed a long-standing bug in Condor regarding the configuration
  settings \Macro{LOWPORT} and \Macro{HIGHPORT}.
  When these were enabled (to restrict Condor's port usage to a
  specified range), Condor would fail to set the
  \texttt{SO\_KEEPALIVE} option on sockets it created.
  This meant that in the case of a hard machine failure (such as a
  sudden power outage, etc) on one machine, Condor daemons
  communicating with that machine would never notice it had died.
  Now, the \texttt{SO\_KEEPALIVE} option is properly set on all
  sockets, even with \MacroNI{LOWPORT} and \MacroNI{HIGHPORT}
  defined. 

\item Fixed a bug that caused \Condor{rm} \Opt{-forcex} to not remove
  jobs that make use of \AdAttr{leave\_in\_queue}.
  If invoked using a cluster id, username, or constraint expression,
  \Condor{rm} would report success but the jobs would remain in the queue.
  Now, the jobs will leave the queue.

\item When a held job is released, job ad attributes HoldReasonCode and
  HoldReasonSubCode are now properly moved to LastHoldReasonCode and
  LastHoldReasonSubCode.

\item Fixed a bug that would cause the \Attr{RemoveReason} attribute
  for a job 
  to be set incorrectly in some circumstances.
  Specifically, this was when a job
  was not running and a \AdAttr{periodic\_remove} expression
  caused the job to be cancelled.

\item Fixed \Condor{submit} such that submit description file
  commands written with syntax both of
  \verb@ThisStyle@ and \verb@this_style@ will work.

\item Fixed a very rare but serious bug in Condor that was originally
  introduced in version 6.3.0.
  Under exceptional circumstances (a very heavily loaded machine where
  a huge number of processes are being spawned all the time, and where
  the \Condor{schedd} is managing many thousands of jobs in the
  queue), it was possible for the \Condor{schedd} to run a job twice.
  We have fixed the underlying problem that lead to the
  \Condor{schedd} making this mistake, rendering this error
  impossible.

\item Fixed a bug that occurred when submitted Condor-G jobs while
  using the grid monitor. If the grid job monitor returned a FAILED
  status for a job while the jobmanager is asleep, the \Condor{gridmanager}
  could sometimes end up in a loop, continuously restarting the remote
  Globus jobmanager then putting it back to sleep.

\end{itemize}

\noindent Known Bugs:

\begin{itemize}

\item None

\end{itemize}

\noindent Bugs fixed that are not included in version 6.7.3:

\begin{itemize}

\item Fixed a discrepancy in the \Macro{SUBSYS\_ADDRESS\_FILE}
  setting.
  Previously, this setting did not work for \MacroNI{SUBSYS} values of
  \ShortExpr{COLLECTOR} or \ShortExpr{NEGOTIATOR} (for example, defining
  \MacroNI{COLLECTOR\_ADDRESS\_FILE} had no effect).
  Now, if either of these is defined in the configuration file,
  the corresponding Condor daemon will write out the address
  and port it is using to the specified file.
  Normally, the \Condor{collector} and \Condor{negotiator} listen on a
  well-known, fixed port.
  However, on single-machine, Personal Condor installations,
  these address files allow all of the Condor daemons and tools to locate
  the \Condor{collector} and \Condor{negotiator}, even if they are
  using a dynamically assigned port.
  For more information about the \MacroNI{SUBSYS\_ADDRESS\_FILE}
  setting, please see the description in
  section~\ref{param:SubsysAddressFile} on
  page~\pageref{param:SubsysAddressFile}.
  For more information about using non-standard ports for the
  \Condor{collector} and \Condor{negotiator}, please see the
  description of ``Non Standard Ports for Central Managers'' in
  section~\ref{sec:Ports-NonStandard} on
  page~\pageref{sec:Ports-NonStandard}.

\end{itemize}



%%%%%%%%%%%%%%%%%%%%%%%%%%%%%%%%%%%%%%%%%%%%%%%%%%%%%%%%%%%%%%%%%%%%%%
\subsection{\label{sec:New-6-6-7}Version 6.6.7}
%%%%%%%%%%%%%%%%%%%%%%%%%%%%%%%%%%%%%%%%%%%%%%%%%%%%%%%%%%%%%%%%%%%%%%

\noindent Release Notes:

\begin{itemize}

\item None.

\end{itemize}

\noindent New Features:

\begin{itemize}

\item Added a feature to the \Condor{master} which automatically adds
the Condor daemons to the Windows Firewall exception list. This only
applies to machines running Windows XP SP2.

\end{itemize}

\noindent Bugs Fixed:

\begin{itemize}

\item Fixed a bug specific to Windows that could cause, in rare occurrences
due to a race condition, Condor to fail to properly signal the job to
suspend, continue, or preempt.

\item When Condor transfers the job executable using the file transfer
  mechanism, it used to leave the binary sitting as a world-writable
  file inside the execute directory on UNIX.
  Now, executable files transferred by Condor have the proper
  permissions (mode 0755).

\item Fixed an important bug in the low-level code that Condor uses to
  transfer files across a network.
  There were certain temporary failure cases that were being treated
  as permanent, fatal errors.
  This resulted in file transfers that aborted prematurely, causing
  jobs to needlessly re-run.
  The code now gracefully recovers from these temporary errors.
  This should significantly help throughput for some sites,
  particularly ones that transfer very large files as output from
  their jobs.
 
\item Fixed a bug in the file transfer mechanism which caused
  segmentation faults when very long input/output/intermediate file
  lists were used.

\item Fixed a number of bugs in the \Opt{-format} option to \Condor{q}
  and \Condor{status}.
  Now, these tools will properly handle printing boolean expressions
  in all cases.
  Previously, depending on how the boolean evaluated, either the
  expression was printed, or the tool could crash.
  Furthermore, the tools do a better job of handling the different 
  types of format conversion strings and printing out the appropriate
  value.
  For example, if a user tries to print out a boolean attribute with
  \verb@condor_status -format "%d\n" HasFileTransfer@, the
  \Condor{status} tool will evaluate \Attr{HasFiletransfer} and print
  either a 0 or a 1 (FALSE or TRUE).
  If, on the other hand, a user tries to print out a boolean attribute
  with \verb@condor_status -format "%s\n" HasFileTransfer@, the
  \Condor{status} tool will print out the string ``FALSE'' or ``TRUE''
  as appropriate.

\item The ClassAd attribute scope resolution prefixes, \texttt{MY.} and
\texttt{TARGET.}, are no longer case sensitive.

\item \Condor{dagman} now generates a fatal error if any node submit
files are missing the log file attribute.  This behavior can be
overridden with the \Opt{-AllowLogError} command-line option.

\item \Condor{dagman} now does better checking for inconsistent events
(such as getting multiple terminate events for a single job).  This
checking can be disabled with the \Opt{-NoEventChecks} command-line
option.

\item Under Tru64, Condor would sometimes fail to start a job while
	setting the resource limits on behalf of the job.
	This error appears to be the result of a kernel issue.
	A workaround has been implemented which will leave the limits
	of the job unmodified and run the job when this specific error
	situation arises.

\item On Windows, occasionally Condor would exhibit erratic behavior
when a machine resumes from sleeping. This has been fixed.

\item On Windows, occasionally Condor would fail to bind to any available
interfaces due to a mishandling of a function return value. This has
been fixed.

\end{itemize}

\noindent Known Bugs:

\begin{itemize}

\item None.

\end{itemize}


%%%%%%%%%%%%%%%%%%%%%%%%%%%%%%%%%%%%%%%%%%%%%%%%%%%%%%%%%%%%%%%%%%%%%%
\subsection{\label{sec:New-6-6-6}Version 6.6.6}
%%%%%%%%%%%%%%%%%%%%%%%%%%%%%%%%%%%%%%%%%%%%%%%%%%%%%%%%%%%%%%%%%%%%%%

\noindent Release Notes:

\begin{itemize}

\item A \Condor{dagman} job will fail and report a cycle in the DAG
when XML logs are used in a single or multiple log format. The Post
Script  completion event does not get converted to XML and Dagman
never sees them complete or fail because of the format of the event.

\end{itemize}


\noindent New Features:

\begin{itemize}

\item The checkpoint server has moved from contrib module status to being
	a normal part of Condor.

\item When the first start running, all Condor daemons will now try to
  print to their log file the full path to the binary they are
  executing. 
  Unfortunately, we can only reliably get this information on Linux,
  Solaris, MacOSX, and Windows platforms.
  On other platforms, this information will only be printed to the log
  file in certain cases that depend on how the daemon was invoked.
  This new feature was added to aid in debugging problems where sites
  were not running the version of the Condor daemons they thought they
  were due to problems in custom-built startup scripts.
 
\item \Condor{wait} is now available in the Windows port.

\item Added a fix to the accountant that allows users to specify user 
	priorities with \Condor{userprio} before any jobs have been submitted. 

\item Added support for running batch files under Windows when using the
	\Macro{STARTD\_CRON} or \Macro{USER\_JOB\_WRAPPER} attributes.

\item Moved from Globus 2.2.2 to Globus 2.2.4 for Condor-G, except for 
	the DUX 4.0f platform.

\end{itemize}

\noindent Bugs Fixed:

\begin{itemize}

\item Windows bug fixes:
\begin{itemize}
  \item Fixed a bug which could cause Condor to kill processes that
    aren't related to Condor or the job it was running at the time.

  \item Fixed a problem that could cause daemons or tools to crash
    when they looked up information about processes running on the
    system.

  \item Fixed a problem with the collector dropping TCP updates with
    pools larger than roughly 20 machines. This issue only occurs with
    \Macro{UPDATE\_COLLECTOR\_WITH\_TCP} enabled.

  \item Fixed an issue with \Condor{store\_cred} reporting success when
   in fact under certain circumstances the store command actually failed.

  \item Removed \Condor{kbdd\_dll}. It is no longer used.

  \item Fixed an issue with \Condor{birdwatcher} that caused it to
    leak resource handles.

  \item Fixed an issue with the Windows port of \Condor{dagman} that
    would cause it to crash when POST scripts were used.

\end{itemize}

\item Fixed a bug where the environment of jobs in any universe could
  be corrupted.

\item The \Condor{startd} now properly cleans up execute directories on
  root-squashed NFS mounts.

\item Fixed a problem where the \Condor{starter} could crash if the
  job it was running used Condor's file transfer mechanism and the
  full path names to the job's files became longer than a few hundred
  characters.

\item The \Attr{image\_size} attribute of a job on Mac OS X is much
  closer to the values that \Prog{ps} returns.
  Previously it would be highly inflated.

\item Fixed a memory leak in the \Condor{gridmanager}.

\item Added the \Opt{-Storklog} argument to \Condor{submit\_dag} to make it
	compatible with the older perl script of the same name.

\item Removed support for the \Opt{-libc} option for \Condor{version}.

\item Added a fix to \Condor{compile} where if our internal \Prog{ld} managed
	to not be invoked during linking of a standard universe executable, 
	a warning is emitted.

\item Fixed a minor bug in the file transfer mechanism.  Specifically,
  if a VANILLA job had \Expr{when\_to\_transfer\_output} set to
  ON\_EXIT\_OR\_EVICT, wrote more than one output file, and was
  actually evicted, the condor \Condor{shadow} would have a fatal
  run-time error (shadow exception) and your job would be rerun.

\item DAGMan bug fixes:
\begin{itemize}
  \item If submit files for individual nodes referred to the same log
    file with different paths, \Condor{dagman} would read log events
    incorrectly and the DAG would fail.
    \Condor{dagman} is now able to recognize that the different paths
    actually refer to the same log file.

  \item Fixed a bug where DAGMan failed to monitor Stork job logs.

  \item If a node submit file doesn't specify a log file, the warning
	message now gets printed out in the the DAGMan log file.

  \item Fixed a bug that caused \Condor{dagman} to fail if first node
        submit file has continuation in log file line.


\end{itemize}

\item Bugs related to configuration
\begin{itemize}
  \item Fixed a bug where Condor daemons could crash if
    \Macro{COLLECTOR\_HOST} or \Macro{NEGOTIATOR\_HOST} was defined to
    be something bogus.

  \item Fixed potential crash in the \Condor{collector} when
    \Macro{COLLECTOR\_NAME} was too long.

  \item The default setting for \Macro{POOL\_HISTORY\_DIR} is no
    longer \Macro{SPOOL}.
    Using the spool directory would result in history files being
    obliterated by \Condor{preen}.

\end{itemize}

\item Fixed a bug which could result in a daemon crashing while it was
	writing to its logfile.

\item Fixed a signal handling bug in the checkpoint server which could
  cause the daemon to hang sometimes.

\item The Kerberos map file now tolerates spaces on either side of the
	equals sign instead of generating a parse error.

\item The \Opt{-analyze} option to \Condor{q} is only meaningful for certain
  universes.  \Condor{q} now warns if the output might not be meaningful. 

\item Java universe: when jar files are transferred to the execute
  machine (with \Expr{should\_transfer\_files} or
  \Expr{transfer\_input\_files}) the \Condor{starter} will use the
  local path (in the execute directory) for the jarfiles, instead of
  the original path specified in the submit file.

\item Previously, if a scheduler universe job died with a signal, the
  \Condor{schedd} would write multiple (conflicting) events into the
  UserLog file: a terminate event and an abort event.
  Now, only the terminate event is written, not the abort event.

\item Fixed a minor bug where if the \Condor{schedd} crashed or was
  killed at just the wrong moment while a job was being removed
  because the \Attr{periodic\_remove} expression had evaluated to
  TRUE, the job might have been successfully removed but the
  \Attr{RemoveReason} attribute could have been lost.
  Now, both actions are taken together atomically.
  If a job is successfully removed, it will always have a
  \Attr{RemoveReason} attribute.

\item Fixed a memory leak in the \Condor{collector}.

\end{itemize}

\noindent Known Bugs:

\begin{itemize}

\item None.

\end{itemize}




%%%%%%%%%%%%%%%%%%%%%%%%%%%%%%%%%%%%%%%%%%%%%%%%%%%%%%%%%%%%%%%%%%%%%%
\subsection{\label{sec:New-6-6-5}Version 6.6.5}
%%%%%%%%%%%%%%%%%%%%%%%%%%%%%%%%%%%%%%%%%%%%%%%%%%%%%%%%%%%%%%%%%%%%%%

\noindent Release Notes:

\begin{itemize}

\item None.

\end{itemize}

\noindent New Features:

\begin{itemize}

\item None.

\end{itemize}

\noindent Bugs Fixed:

\begin{itemize}

\item Fixed a bug introduced in Condor version 6.6.2 that could cause
      \Condor{dagman} to segfault while parsing some DAG files, or
      fail to recognize already-completed nodes in a rescue DAG.

\item Fixed a bug in \Condor{dagman}, whereby it could fail to
      automatically discover a Condor job's userlog file if the job's
      submit file did not have whitespace surrounding the equal sign
      on the log file line.

\item Fixed a bug in \Condor{submit} that appears to only have
  effected OSX machines.
  Previously, submit files that only defined a single job and used
  \verb@queue@ without any numerical modifiers would result in an
  error like this: 
\footnotesize
\begin{verbatim}
     ERROR: "test.sub" doesn't contain any "queue" commands -- no jobs queued
\end{verbatim}
\normalsize
  Now, \Condor{submit} will properly process and submit the job from
  job description files that contain a single \verb@queue@ statement
  with no modifiers.

\item Fixed a bug in the AIX \Condor{starter} that was causing the
starter to sometimes kill itself when the job completed.  Because this
happened before the \Condor{starter} reported the job completion back
to the \Condor{shadow}, such a job would be restarted.

\item Fixed a few memory and registry handle leaks in the \Condor{schedd}
and \Condor{startd}. These leaks particularly affected Windows systems.

\item On Windows, Condor was known to have trouble accessing config files
with UNC paths (with appropriate permissions set). This has been fixed.

\item On Windows, \Condor{store\_cred} would fail if the account did not
have \verb@Log on Locally@ privileges, even if the account was allowed
to log in interactively. This has been fixed.

\item Fixed a bug on Windows that would cause the \Condor{schedd} to
crash if \Dflag{FULLDEBUG} was turned on, and the submitting user
account did not have Administrator access rights.

\end{itemize}

\noindent Known Bugs:

\begin{itemize}

\item \Condor{dagman} can fail to detect a job's progress if another
      job in the DAG specifies the same underlying userlog file using
      a different path or filename (e.g., log=foo and log=./foo) in
      its submit file.

\end{itemize}



%%%%%%%%%%%%%%%%%%%%%%%%%%%%%%%%%%%%%%%%%%%%%%%%%%%%%%%%%%%%%%%%%%%%%%
\subsection{\label{sec:New-6-6-4}Version 6.6.4}
%%%%%%%%%%%%%%%%%%%%%%%%%%%%%%%%%%%%%%%%%%%%%%%%%%%%%%%%%%%%%%%%%%%%%%

\noindent Release Notes:

\begin{itemize}

\item This version only contains platform-specific bug fixes.
  Therefore, it was only released for the two effected platforms. 

\end{itemize}

\noindent Bugs Fixed:

\begin{itemize}

\item Fixed the bug in Condor's file transfer mechanism for Mac OSX
  that was introduced in version 6.6.3.

\end{itemize}

\noindent Known Bugs:

\begin{itemize}

\item None.

\end{itemize}



%%%%%%%%%%%%%%%%%%%%%%%%%%%%%%%%%%%%%%%%%%%%%%%%%%%%%%%%%%%%%%%%%%%%%%
\subsection{\label{sec:New-6-6-3}Version 6.6.3}
%%%%%%%%%%%%%%%%%%%%%%%%%%%%%%%%%%%%%%%%%%%%%%%%%%%%%%%%%%%%%%%%%%%%%%

\noindent Release Notes:

\begin{itemize}

\item The Globus universe support for versions of Globus prior to 2.2 (specifically, those using GRAM 1.5 or earlier) has been removed.

\end{itemize}


\noindent New Features:

\begin{itemize}

\item The Globus universe now supports submitting jobs to Globus Toolkit 3.2 installations.

\end{itemize}

\noindent Bugs Fixed:

\begin{itemize}

\item The negotiator no longer crashes when a grid site ClassAd sets WantAdRevaluate but does not contain an UpdateSequenceNumber.

\item Globus universe jobs were failing to go on hold when a \$\$() expression
could not be expanded.

\item On Windows, the system-wide TEMP variable is included in the
execute environment if it is not specified in the submit file.

\item Fixed a rarely-occurring bug when  the child process forked by the schedd gets stuck in an infinite loop when the user does ``condor\_submit -s''. This should also fix problems when the child process forked by the collector would sometimes get stuck in an infinite loop when \Expr{ COLLECTOR\_QUERY\_WORKERS > 0 } in the config file.

\end{itemize}

\noindent Known Bugs:

\begin{itemize}

\item The Condor file transfer mechanism is broken on Mac OSX in
  Condor version 6.6.3.
  OSX users should either upgrade to version 6.6.4, or install a
  patched \Condor{starter} binary available from
  \URL{http://www.cs.wisc.edu/condor/binaries/condor-6.6.3-patch1-MacOSX-PPC.tar.Z}. 

\end{itemize}






%%%%%%%%%%%%%%%%%%%%%%%%%%%%%%%%%%%%%%%%%%%%%%%%%%%%%%%%%%%%%%%%%%%%%%
\subsection{\label{sec:New-6-6-2}Version 6.6.2}
%%%%%%%%%%%%%%%%%%%%%%%%%%%%%%%%%%%%%%%%%%%%%%%%%%%%%%%%%%%%%%%%%%%%%%

\noindent Release Notes:

\begin{itemize}

\item There will be another release, 6.6.3, within a few weeks.  We decided to
	release this version now because it adds the AIX platform and has some bug
	fixes which we thought important enough for a release.  However, if you are
	not affected by the bugs fixed (see below) you may wish to wait for 6.6.3.
     

\end{itemize}


\noindent New Features:

\begin{itemize}

\item Clipped support for AIX 5.2.
      This means VANILLA universe only - no checkpointing or STANDARD universe.

\item The setting \Macro{GRIDMANAGER\_GLOBUS\_COMMIT\_TIMEOUT} allows
   configuring the two phase commit timeout in Globus.  This maps to the
   two\_phase setting in the Globus RSL.

\item Added a new configuration variable,
      \Macro{DAGMAN\_MAX\_SUBMIT\_ATTEMPTS}, that controls how many
      times in a row \Condor{dagman} will attempt to execute
      \Condor{submit} for a given job before giving up.  It cannot be
      set to less than 1 attempt, or more than 10; if left undefined,
      it defaults to 6.

\item Added a new tool \Condor{updates\_stats} to dump out the update
statistics information from ClassAds in a human readable format.
Condor 6.6.1, by default, publishes ``update statistics'' into the
ClassAds as published by the \Condor{collector}.  This program parses
this output and displays it to the user in a readable format.

\item Changed the default \Condor{dagman} behavior so that it doesn't
      check for cycles at startup, only at runtime, since the former
      could be expensive for large DAGs.  Added a boolean
      \Attr{DAGMAN\_STARTUP\_CYCLE\_DETECT} config attribute to
      re-enable cycle-detection at startup.

\item \Condor{dagman} now offers a configuration variable,
      \Macro{DAGMAN\_MAX\_SUBMITS\_PER\_INTERVAL}, which controls how
      many individual jobs \Condor{dagman} will submit in a row before
      servicing other requests (such as a \Condor{rm}).

\item The grid\_monitor now automatically detects jobmanager scripts on the
      remote gatekeeper.  Previously it was limited to supporting the condor,
	  fork, lsf, pbs, and remote jobmanager scripts.

\item A new parameter, \Macro{SEC\_DEBUG\_PRINT\_KEYS}, controls whether or not
      the keys used for encryption get printed into the log.
	  The default is false.


\end{itemize}

\noindent Bugs Fixed:

\begin{itemize}

\item Jobs that make use of Condor's file transfer mechanism were not
automatically authorized to read/write input/output files when
flocking to machines that did not happen to be in the
\Macro{HOST\_ALLOW\_WRITE} list.  This bug has existed since 6.3.

\item Eliminated a small chance that a grid\_monitor log file or state file
    might be reused.  The unique identifying numbers are now unique across
	the entire gridmanager, not each Globus resource.

\item Eliminated a race condition which might cause the grid monitor to
	erroneously decide that the status file was broken when in fact it
	was being uploaded and was empty.

\item The grid monitor now attempts to restart transfers in the event of
    globus-url-copy hanging.

\item Removed some settings from the default configuration files
  shipped with Condor that are no longer used in the code.

\item Fixed bugs in \Condor{dagman} parsing of submit files (to determine
  node log files).  Previously, a submit file line beginning with
  "log" (e.g., "LogLock = True") would be interpreted as a log file
  line.  Also, if "log" was defined twice in the submit file,
  \Condor{dagman} would incorrectly use the first definition, rather than
  the last.

\item Re-added PVM support for IRIX 6.5.

\item Fixed an indirect bug whereby \Condor{dagman} could fail with an
assertion error if it encounters both a terminate and a abort event in
the userlog for the same job; this can happen due to a bug in the
\Condor{schedd}, which is not yet fixed.

\item \Condor{dagman} now works right with nodes that have an initialdir
  specified in the node submit file.  (Previously, specifying
  an initialdir only worked if the log file path was absolute.)

\item \Condor{dagman} now responds more quickly to a request to be
      removed from the queue (via \Condor{rm}), even if it is in the
      midst of submitting jobs.  Previously, \Condor{dagman} would
      finish submitting all ready jobs before responding to a removal
      request, which could take a long time, and forced it to
      immediately remove all the jobs it had just submitted
      unnecessarily.

\item Fixed keyboard idle reporting on Mac OS X. Previously, the code
      would often return -1 on newer hardware. 

\end{itemize}

\noindent Known Bugs:

\begin{itemize}

\item If a scheduler universe job terminates via a signal, the 
      \Condor{schedd} logs both a terminate event and an abort event
      to the userlog. 

\item Keyboard activity is not reported for pseudo-ttys on Mac OS X, only
      the physically connected keyboard

\end{itemize}


%%%%%%%%%%%%%%%%%%%%%%%%%%%%%%%%%%%%%%%%%%%%%%%%%%%%%%%%%%%%%%%%%%%%%%
\subsection{\label{sec:New-6-6-1}Version 6.6.1}
%%%%%%%%%%%%%%%%%%%%%%%%%%%%%%%%%%%%%%%%%%%%%%%%%%%%%%%%%%%%%%%%%%%%%%

\noindent Release Notes:

\begin{itemize}

\item \Condor{analyze} is not included in the downloads of Version 6.6.1.
  The existing binary from Version 6.6.0 is likely to work on all platforms
  for which it was released.

\end{itemize}


\noindent New Features:

\begin{itemize}

\item Added full support (including standard universe jobs with
  checkpointing and remote system calls) for Linux i386 RedHat 9
  (using gcc/g++ version 3.2.2 and glibc version 2.3.2). 

\item Added full support (including standard universe jobs with
  checkpointing and remote system calls) for Linux i386 RedHat 8
  (using gcc/g++ version 3.2 and glibc version 2.2.93). 

\item The time it takes \Condor{dagman} to submit jobs has been
      reduced slightly to improve up the startup time of large DAGs.

\item In order to help reduce load on the \Condor{schedd} when
      \Condor{dagman} is submitting jobs, there is a new config
      variable, \Macro{DAGMAN\_SUBMIT\_DELAY}, to specify the number
      of seconds \Condor{dagman} will sleep before submitting each
      job.

\item Enabled the ``update statistics'' in the \Condor{collector} by
      default in both the executable and in the default configuration.

\item Command-line arguments to \Condor{dagman} are now handled
      case-insensitively.

% commented out since i'm not sure we want to make a big deal of this
% new config file knob. -derek 1/12/04
% \item Added a new configuration macro
%     \Macro{MAX\_CLAIM\_ALIVES\_MISSED}, described on
%     page~\pageref{param:MaxClaimAlivesMissed}.
%     This setting controls how many keep alive messages a startd is
%     willing to miss before it releases the claim from a given schedd.

\item Added support for Condor-G and strong authentication to Condor
  for IRIX 6.5, but removed support for checkpointing and remote
  system calls.
  We plan to add support in Condor for IRIX's kernel-level
  checkpointing in a future release.

\item Added a \Opt{-p} option to \Condor{store\_cred} so that users
can now specify the the password on the command line instead of getting
prompted for it.

\item The gahp\_server helper process for Condor-G includes patches from
the LHC Computing Grid Project to increase data transfer performance of
the Condor-G client. Previous versions of Condor-G could bog down in
accepting new transfer requests, producing a variety of errors. 

\item Added a new configuration setting,
  \Macro{SUBMIT\_SEND\_RESCHEDULE} which controls whether or not
  \Condor{submit} should automatically send a \Condor{reschedule}
  command when it is done.
  Previously, \Condor{submit} would always send this reschedule so
  that the \Condor{schedd} knew to start trying to find matches for
  the new jobs.
  However, for submit machines that are managing a huge number of jobs
  (thousands or tens of thousands), this step would hurt performance
  in such a way that it became an obstacle to scalability.
  In this case, an administrator can set
  \MacroNI{SUBMIT\_SEND\_RESCHEDULE} to \verb@FALSE@, this extra
  step is not performed, and the \Condor{schedd} will try to find
  matches whenever the periodic timer in the \Condor{negotiator}
  (\MacroNI{NEGOTIATOR\_INTERVAL}) goes off.

\item Pool administrators can now specify the length of time before
  the \Condor{starter} sends its initial update to the
  \Condor{shadow} by defining
  \Macro{STARTER\_INITIAL\_UPDATE\_INTERVAL}. 
  The default is 8 seconds.
  This setting would not normally need changing except to fine-tune a
  heavily loaded system.

\item Administrators can now specify the default session duration for
  each Condor subsystem.
  This allows for fine tuning the image size of running Condor daemons
  if the memory footprint is a concern.
  The default for tools is 1 minute, the default for \Condor{submit}
  is one hour, and the default for daemons is 100 days.
  This does not mean that tools cannot run more than one minute or
  submit cannot run for more than an hour; it only affects memory
  usage.

\item Added new configuration setting
  \Macro{GRID\_MONITOR\_HEARTBEAT\_TIMEOUT}.
  If this many
  seconds pass without hearing from the grid\_monitor, it is
  assumed to be dead.  Defaults to 300 (5 minutes).  Increasing
  this number will improve the ability of the grid\_monitor to
  survive in the face of transient problems but will also
  increase the time before Condor notices a problem.  Prior to
  this change the gridmanager always waited 5 minutes, the user
  could not change the setting.

\item Added new configuration setting
  \Macro{GRID\_MONITOR\_RETRY\_DURATION}.  
  If something goes wrong
  with the grid\_monitor at a particular site (like
  \MacroNI{GRID\_MONITOR\_HEARTBEAT\_TIMEOUT} expiring), it will be retried
  for this many seconds.  Defaults to 900 (15 minutes).  If we
  can't successfully get it going again the grid monitor will be
  disabled for that site until 60 minutes have passed.  Prior to
  this change the condor\_gridmanager wait 60 minutes after any
  failure.

\end{itemize}



\noindent Bugs Fixed:

\begin{itemize}

\item Fixed bugs related to network communication and timeouts that
  impact scalability in Condor:
  \begin{itemize}
    \item Fixed a bug inside Condor's network communication layer that 
      could result in Condor daemons blocking trying to read more data
      after a socket had already been closed.
    \item Fixed a \Condor{negotiator} bug that could, in certain rare
      circumstances, cause a \Condor{schedd} to hang for five minutes
      while trying to communicate with it.
    \item Fixed a bug in which TCP connections would re-authenticate
      needlessly when Condor's strong authentication was enabled.
      This was not harmful but incurred a bit of overhead, especially
      when using Kerberos authentication.
  \end{itemize}

\item Fixed bugs related to network security sessions which were
  getting cleared out.
  If the timing was unfortunate, this could cause some jobs to fail
  immediately after completion.
  So, Condor no longer clears out security sessions periodically (it
  used to happen every 8 hours) nor does it do so when a daemon
  receives a \Condor{reconfig} command.

\item Fixed a bug in the standard universe where C++ code that threw an 
exception would result in abortion of the executable instead of the
delivery of the exception. This bug affects Condor version 6.6.0 for
Redhat 7.x.

\item Fixed a \Condor{shadow} bug that could result in a fatal error
  if the following 3 conditions were met: (1) the job enables Condor's
  file transfer mechanism, (2) the job wants Condor to automatically
  figure out what files to transfer back (the default), and (3) the
  job does not specify a userlog.

\item Fixed bug whereby \Condor{dagman}, if removed from the queue via
      \Condor{rm}, could fail to remove all of its submitted jobs if
      any of their submit events had not yet appeared in the userlog.

\item Fixed a few bugs in \Condor{preen}:
  \begin{itemize}
  \item It will no longer potentially remove files related to a valid
    Computing on Demand (COD) claim on an otherwise idle machine.
  \item \Condor{preen} will no longer keep reporting that it had
    successfully removed a directory which was in fact failing to be
    removed.
  \end{itemize}

\item Fixed the faulty argument parsing in \Condor{rm},
  \Condor{release}, and \Condor{hold}.
  Before you could accidentally type \verb@condor_rm -analyze@, and it
  would remove all of your jobs.
  Now it gives an error.

\item On Windows, when you type a command like
  \verb@condor_reconfig.exe@ instead of \verb@condor_reconfig@, you no
  longer get an error.

\item Fixed a bug on Windows that would cause ``GetCursorPos() failed''
  to appear repeatedly in the StartLog. The startd now uses a different
  function to track mouse activity that does not have a tendency to fail.

\item Fixed a bug on Windows that would prevent some \Condor{shadow}
  daemons from obtaining a lock to their log file under heavy load, and
  thus causing them to EXCEPT().

\item Fixed a bug on Windows where file transfers would incorrectly fail
because of bad permissions when using domain accounts with nested groups,
or when UNC paths were used.

\item Fixed the bug where the \Condor{starter} would fail to transfer
  back core files created by Vanilla, Java and MPI universe jobs.
  This bug was introduced in Condor version 6.5.2.
  Now, Condor correctly transfers back any core files created by
  faulty user jobs in any job universe.

\item In some circumstances, \Condor{history} would fail to read
  information about some jobs, and would report errors. In particular,
  when jobs had large environments, it would fail. This has been
  corrected.

\item Fixed a rare bug affecting \Condor{dagman} when job-throttling
      was enabled: if \Condor{dagman} was removed from the queue
      together with some of its own jobs (e.g., via \verb@condor_rm -a@),
      it would quickly submit new jobs to replace them before
      recognizing that it needs to exit.  It now shuts down
      immediately without submitting and then removing these
      unnecessary jobs.

\item Fixed a potential security problem that was introduced in Condor
  version 6.5.5 when the \Macro{REQUIRE\_LOCAL\_CONFIG\_FILE}
  configuration setting was added.
  This setting used to default to FALSE if it was not defined in the
  configuration files.
  It now defaults to TRUE.
  If administrators define local configuration files for the machines
  in their pool, it should be a fatal error if those files don't exist
  unless the administrators actively disable this check by defining 
  \MacroNI{REQUIRE\_LOCAL\_CONFIG\_FILE} to be FALSE.

\item Fixed a bug on Windows that would cause the \Condor{startd} to
EXCEPT() if the \Condor{starter} exited and left orphaned processes to
be cleaned up. This bug first appeared in 6.5.0.

\item Fixed a bug on Windows that would cause graceful shutdowns on
Windows (such as when \verb@condor_vacate@  is called) to fail to
complete.

\item The gahp\_server helper program, which provides Globus services
to Condor-G, was always dynamically linked, even in statically-linked
releases.
The statically linked distributions of Condor now include a static
gahp\_server.

\item Fixed minor bug in parsing XML user log files that contain empty
  strings. 

\item Fixed the messages written to the Condor daemon log files in
  various error conditions to be more informative and clear:
  \begin{itemize}
  \item The error message in the SchedLog that indicates that swap
    space has been depleted has been rephrased so it appears to be
    significant.
  \item Certain serious error messages are now being written to the 
    \Dflag{ALWAYS} debug level that used to only appear if other debug
    levels were enabled.
  \item Clarified log messages related to errors looking up user
    information in the passwd database on UNIX and for creating
    dynamic users on Windows.
  \item Log messages related to keep-alives sent between the
    \Condor{schedd} and \Condor{startd} (written to \Dflag{PROTOCOL})
    now include the \Attr{ClaimId} on both sides, so that it is easier
    to find potential problems and figure out which keep-alive
    messages correspond to what resources.
  \item Added more useful information to certain errors relating to 
    security sessions and strong authentication.
  \item Fixed the formatting of some messages to correctly include a
    newline at the end of the message.
  \end{itemize}

\item Fixed a bug in the \Condor{configure} installation tool.
  Previously, it would set \MacroNI{MAIL\_PATH}, which doesn't exist
  in Condor and had no effect.
  Now, \Condor{configure} correctly sets \Macro{MAIL}, instead.

\item Fixed bug in userlog code in the CondorAPI library to prevent
  segmentation faults.

\item Clarified log messages for Condor-G's GridmanagerLog,
  especially those relating to the grid monitor.

\item Fixed potential race condition when using the grid monitor.  
  Condor-G now identifies partial grid monitor status updates and 
  waits for the update to complete.


\item The grid\_monitor is slightly more robust in the face of
  unexpected behavior by the Globus jobmanager.  This is only a
  partial fix, for complete success you really need the Globus
  patch at
  \URL{http://bugzilla.globus.org/bugzilla/show\_bug.cgi?id=1425}

\item Internal timeouts in the grid\_monitor have been increased,
  increasing robustness during transient errors.

\end{itemize}

\noindent Known Bugs:

\begin{itemize}

\item Submission of MPI jobs from a Unix machine to run on Windows
machines (or vice versa) fails for machine\_count > 1.  This is
not a new bug.  Cross-platform submission of MPI jobs between
Unix and Windows has always had this problem.

\item A multiple install of Condor's standard universe support libraries
onto an NFS server for the purposes of having a heterogeneous mix of Linux
distribution revisions all being able to utilize the same \Condor{compile}
does not function correctly if Redhat 9 is one of the distributions.

\end{itemize}


%%%%%%%%%%%%%%%%%%%%%%%%%%%%%%%%%%%%%%%%%%%%%%%%%%%%%%%%%%%%%%%%%%%%%%
\subsection{\label{sec:New-6-6-0}Version 6.6.0}
%%%%%%%%%%%%%%%%%%%%%%%%%%%%%%%%%%%%%%%%%%%%%%%%%%%%%%%%%%%%%%%%%%%%%%

\noindent New Features:

\begin{itemize}

\item The \Condor{dagman} debugging log now reports the total number
      of ``Un-Ready'' Nodes (i.e. those waiting for unfinished
      dependencies) in its periodic summaries.  In the past, the
      omission of this state led to confusion because the total of all
      reported job states didn't always match the total number of jobs
      in the DAG.

\item Most Condor commands (\Condor{on}, \Condor{off},
  \Condor{restart}, \Condor{reconfig}, \Condor{vacate},
  \Condor{checkpoint}, \Condor{reschedule}) now support a \Opt{-all}
  command-line option to specify which daemons to act on.
  This is more efficient and much easier to use than previous methods
  for accomplishing the same effect.
  Using \Opt{-all} with \Condor{off} correctly leaves the existing
  \Condor{master} processes running on each host, so that a subsequent
  \Condor{on} would work.
  See section~\ref{sec:Pool-Shutdown-and-Restart} on
  page~\pageref{sec:Pool-Shutdown-and-Restart} for more details on
  proper use of \Opt{-all} with \Condor{off} and \Condor{on}

\end{itemize}

\noindent Bugs Fixed:

\begin{itemize}

\item Fixed a bug under Solaris 8 with Update 6+, and Solaris 9 where
Condor would incorrectly report the console and mouse idle times as zero.

\item The standard-universe fetch\_files feature was not cleaning up
temporary files on the execution machine.

\item In rare circumstances, a Linux kernel bug results in conflicting
information about system boot time (\File{/proc/stat} and
\File{/proc/uptime}). 
Specifically, the "btime" field in \File{/proc/stat} suddenly jumps to
the present moment and then stays at that value.  This
was resulting in incorrect estimation of process ages, which caused
Condor's estimation of CondorLoadAvg to be completely wrong.  A more
robust heuristic is now being used.

\item A long configuration line with with continuation lines can cause the
config file parser to not properly skip the leading whitespace from
the continued lines.  This has been corrected.

\item The Grid Monitor now will automatically probe for and work with
``unknown'' batch systems.

\item Fixed a bug where under certain circumstances \Condor{dagman}
      would fail to detect an unsuccessful invocation of
      \Condor{submit}, and would instead report the job as
      successfully submitted with job id 0.0.

\item Fixed a bug which was causing problems when a periodic\_remove
expression for a scheduler universe job evaluates to true.  Under
these conditions, the schedd did not log the job termination to the
job log.  Additionally, the schedd would exit with an error status.

\item Fixed a recently-introduced \Condor{dagman} bug where the number
      of node retries (specified with the RETRY keyword) wasn't being
      updated after some failures; instead, the node would be allowed
      to retry indefinitely if it kept failing.

\item Fixed a recently-introduced bug where shutting down the
      \Condor{schedd} caused \Condor{dagman} to remove all its jobs
      from the queue and write a rescue file, rather than simply
      exiting so that it could recover automatically upon restart.

\item Changed the default ``Periodic Expression Interval'' parameter
(PERIODIC\_EXPR\_INTERVAL) from 60 seconds to 300 seconds.

\item Whenever \Condor{reconfig} was used to re-configure multiple
  daemons which included the \Condor{collector} for a pool, the
  command would start to fail after the \Condor{collector} was
  reconfigured due to problems with security sessions in Condor's
  strong authentication code.
  This situation no longer causes problems for the \Condor{reconfig}
  tool, and it can properly re-configure multiple daemons at once,
  even if one of them is the \Condor{collector} for a pool.

\item Most Condor commands (\Condor{on}, \Condor{off},
  \Condor{restart}, \Condor{reconfig}, \Condor{vacate},
  \Condor{checkpoint}, \Condor{reschedule}) now check to make sure
  they are not sending a duplicate command if the user specifies the
  same target machine or daemon twice.  For example:
\begin{verbatim}
     condor_reconfig hostname1 hostname2 hostname1
\end{verbatim}
  will only send a single reconfig command to \verb@hostname1@.

\item Fixed a bug in the HPUX version of Condor which was causing the
startd to occasionally abort operation.  This has been in Condor since
version 6.1.1.

\item The Condor daemons will no longer overwhelm NIS servers
when large numbers of daemons are running. Condor now caches
uid and group information internally, and refreshes the
cache entries on a specified interval (which defaults to 5
minutes). See section~\ref{param:PasswdCacheRefresh} on
page~\pageref{param:PasswdCacheRefresh} for more details.

\end{itemize}

\noindent Known Bugs:

\begin{itemize}

\item The \Condor{preen} program does not know about Computing on
  Demand (COD) claims.
  If there are no regular Condor jobs on a given machine, but there
  are COD claims, and \Condor{preen} is spawned, it will remove files
  related to the COD claims.
  In version 6.6.0, sites using COD are encouraged to disable
  \Condor{preen} by commenting out the \MacroNI{PREEN} setting in the
  config files.
  This bug has been fixed in Condor version 6.6.1.

\item Normally, if a user's job crashes and creates a core file on a
  remote execution machine, the \Condor{starter} will automatically
  transfer the core file back to the submit machine.
  However, beginning in Condor version 6.5.2, if a vanilla, Java, or
  MPI universe job creates a core file, the \Condor{starter} will fail
  to transfer it back.
  This bug will be fixed in version 6.6.1.
  
\item There are a few bugs related to Condor tools failing to
  correctly locate the \Condor{negotiator} daemon.
  These bugs usually show up if a site is using non-standard ports for
  the central manager daemon.
  However, some of the bugs show up regardless of if the negotiator is
  listening on the standard port or not. 

  \begin{itemize}
    \item \verb@condor_config_val -negotiator@ queries the
          \Condor{collector}, instead of querying the
          \Condor{negotiator} like it should.  

    \item Using the \Opt{-pool} option to \verb@condor_q -analyze@
          will not work.
          The tool will fail to find and query the \Condor{negotiator}
          for user priorities which it needs to determine why jobs may
          not be running.

    \item The Condor tools that support either the \Opt{-negotiator}
          or \Opt{-collector} options do not work when a user also
          specifies the \Opt{-pool} to define a remote pool to
          communicate with.
          The tools print a somewhat confusing message in this case.

    \item Most Condor tools that support \verb@-pool hostname@ will
          also recognize \verb@-pool hostname:port@ if the remote
          \Condor{collector} is listening on a non-standard port.
          However, the \Condor{findhost} tool does not work if given a
          \Opt{-pool} option that includes a port.

  \end{itemize}

\end{itemize}

\begin{center}
\begin{table}[hbt]
\begin{tabular}{|ll|} \hline
\emph{Architecture} & \emph{Operating System} \\ \hline \hline
Hewlett Packard PA-RISC (both PA7000 and PA8000 series) & HPUX 10.20 \\ \hline
Sun SPARC Sun4m,Sun4c, Sun UltraSPARC & Solaris 2.6, 2.7, 8, 9 \\ \hline
Silicon Graphics MIPS (R5000, R8000, R10000) & IRIX 6.5 \\ \hline
Intel x86 & Red Hat Linux 7.1, 7.2, 7.3 \\
 & Red Hat Linux 8 (clipped) \\ \hline
 & Red Hat Linux 9 (clipped) \\ \hline
 & Windows NT 4.0 Workstation and Server (clipped) \\ \hline
 & Windows 2000 Professional and Server, 2003 Server (clipped) \\ \hline
 & Windows XP Professional (clipped) \\ \hline
ALPHA & Digital Unix 4.0 \\
 & Red Hat Linux 7.1, 7.2, 7.3 (clipped) \\ \hline
 & Tru64 5.1 (clipped) \\ \hline
PowerPC & Macintosh OS X (clipped) \\
Itanium & Red Hat Linux 7.1, 7.2, 7.3 (clipped) \\
\end{tabular}
\caption{\label{6.6.0-supported-platforms}Condor version 6.6.0 supported platforms}
\end{table}
\end{center}


% Feb 2007 -- still in the manual source, just not incorporating
% these old histories into the finished product, thereby
% reducing the size of the manual by 200 pages
%%%%%%%%%%%%%%%%%%%%%%%%%%%%%%%%%%%%%%%%%%%%%%%%%%%%%%%%%%%%%%%%%%%%%%%
\section{\label{sec:History-6-5}Development Release Series 6.5}
%%%%%%%%%%%%%%%%%%%%%%%%%%%%%%%%%%%%%%%%%%%%%%%%%%%%%%%%%%%%%%%%%%%%%%

This is the development release series of Condor,
The details of each version are described below.

%%%%%%%%%%%%%%%%%%%%%%%%%%%%%%%%%%%%%%%%%%%%%%%%%%%%%%%%%%%%%%%%%%%%%%
\subsection*{\label{sec:New-6-5-5}Version 6.5.5}
%%%%%%%%%%%%%%%%%%%%%%%%%%%%%%%%%%%%%%%%%%%%%%%%%%%%%%%%%%%%%%%%%%%%%%

\noindent New Features:

\begin{itemize}

\item Condor-G jobs can now use matchmaking. See
  Section~\ref{sec:Condor-G-Matchmaking}. 

\item \Condor{advertise} has a new option, -tcp.

\item Added initial support for the \Condor{gridshell}.
  Documentation is not yet available, but will be coming soon.

\item Added the \Macro{TRUST\_UID\_DOMAIN} config file setting.
  For more information about this feature, see
  section~\ref{param:TrustUidDomain} on
  page~\pageref{param:TrustUidDomain}. 

\item Added the \Macro{REQUIRE\_LOCAL\_CONFIG\_FILE} config file setting.
  If this setting is False, the absence of the file specified in the 
  \Macro{LOCAL\_CONFIG\_FILE} config file setting is not treated as an error.

\item Added a new command, \Condor{wait}, which will watch a log file
  until a job or set of jobs complete.

\end{itemize}

\noindent Bugs Fixed:

\begin{itemize}

\item The \Condor{starter} was reporting ImageSize to be much bigger
than reality for multi-threaded jobs in Linux.  If the jobs were ever evicted,
this could cause them to never match to another machine or to be
unnecessarily restricted.

\item The \Condor{starter} no longer seg-faults when attempting to run
  a job as nobody on HPUX.
  This bug was introduced in version 6.3.2 and effected all job
  universes.

\item Fixed a bug related to defining \Macro{CONDOR\_IDS} in the
  configuration file.
  This new feature was added in version 6.5.3, but it did not work
  properly with the Standard and PVM universes.
  Now, all parts of Condor should work correctly without permission
  problems if \MacroNI{CONDOR\_IDS} is defined in the config file.

\item Under certain rare circumstances (e.g., running \Condor{rm} on a
      \Condor{dagman} job which was running a POST script for a node
      which had no PRE script defined), DAGMan could dereference a
      nonexistent pointer and segfault.  This has now been fixed.

\item Fixed a minor bug in the \Opt{-jobad} option for the
  \Condor{cod} command-line tool when activiating Computing On Demand
  (COD) jobs.
  Now, the parser for the user-specified ClassAd will ignore comments
  and whitespace.

\item In Standard Universe, a transient failure to read the user's
password file entry (e.g. with an overloaded NIS server) could
result in the job running to completion and exiting the queue
without updating the user log.

\item Fixed a bug which prevented the user from specifying Hawkeye
jobs with colons and / or spaces in the executable name.  The
\MacroNI{HAWKEYE\_JOBS} macro now allows for the individual fields to be
quoted, solving this problem.

\end{itemize}

\noindent Known Bugs:

\begin{itemize}

\item There is a suspected-but-not-yet-confirmed bug in DAGMan's
      \Opt{-MaxJobs} feature that can lead to a segfault which
      re-occurs during recovery.  Users are advised to avoid using
      this feature until the bug is found and fixed.  \Condor{dagman}
      jobs submitted without the \Opt{-MaxJobs} feature are not
      affected.  Likewise, the \Opt{-MaxPre} and \Opt{-MaxPost}
      features are not affected and can be safely used.

\item For large numbers of jobs, \Condor{analyze} may run out of memory
      and fail.
\end{itemize}



%%%%%%%%%%%%%%%%%%%%%%%%%%%%%%%%%%%%%%%%%%%%%%%%%%%%%%%%%%%%%%%%%%%%%%
\subsection*{\label{sec:New-6-5-4}Version 6.5.4}
%%%%%%%%%%%%%%%%%%%%%%%%%%%%%%%%%%%%%%%%%%%%%%%%%%%%%%%%%%%%%%%%%%%%%%

\noindent New Features:

\begin{itemize}

\item Added the \Opt{-jobad} option to the \Condor{cod} command-line
  tool when activiating Computing On Demand (COD) jobs.
  This allows the user to specify dynamic attributes at the point when
  they request the job to start, instead of having to define
  everything ahead of time in the Condor configuration files at the
  execution site.

\end{itemize}

\noindent Bugs Fixed:

\begin{itemize}

\item The \Condor{startd} will now immediately exit with a fatal error
  if there is a syntax error in any of the policy expressions
  (\MacroNI{START}, \MacroNI{SUSPEND}, \MacroNI{PREEMPT}, etc) as
  defined in the configuration file.
  Before, the \Condor{startd} might run for a long time before
  reporting the error, or would silently ignore the expression.
  For example, if the \MacroNI{START} expression contained a syntax
  error, the \Condor{startd} would never match against any resources
  without warning the administrator that the \MacroNI{START}
  expression could not be parsed.

\item The \Condor{starter} now checks to make sure that the value of
      the \MacroNI{JOB\_RENICE\_INCREMENT} expression is within the
      valid range of 0 to 19, and adjusts out-of-range values to the
      closest valid value, printing a warning to the StarterLog.
      Previously, no warnings were printed.  Also, under Windows,
      values above 19 used to be incorrectly adjusted to 0 (i.e.,
      normal priority) instead of 19 (i.e., idle priority class).

\item When Condor is testing to see if a given submit machine is in
  the same \MacroNI{UID\_DOMAIN} as the execution machine, it now uses 
  a case-insensitive test.
  Previous versions of Condor that used a case-sensitive comparison
  caused problems for sites that have mixed-case hostnames in their
  DNS records.

\item When the \Condor{shadow} is trying to claim a \Condor{startd},
  if the IP address of the \Condor{shadow} cannot be resolved, the
  \Condor{startd} used to refuse the request.
  Now, the \Condor{startd} accepts the request and uses the IP address
  of the \Condor{shadow} in all log messages, tool output, and the
  \Attr{ClientMachine} ClassAd attribute that are normally set to the
  hostname.

\item Process handles would intermittantly be lost on Windows, causing
  the \Condor{startd} and other daemons to EXCEPT(). This has been fixed.

\end{itemize}

\noindent Known Bugs:

\begin{itemize}

\item None.

\end{itemize}


%%%%%%%%%%%%%%%%%%%%%%%%%%%%%%%%%%%%%%%%%%%%%%%%%%%%%%%%%%%%%%%%%%%%%%
\subsection*{\label{sec:New-6-5-3}Version 6.5.3}
%%%%%%%%%%%%%%%%%%%%%%%%%%%%%%%%%%%%%%%%%%%%%%%%%%%%%%%%%%%%%%%%%%%%%%

% Karen's table
\begin{center}
\begin{table}[hbt]
\begin{tabular}{|ll|} \hline
\emph{Architecture} & \emph{Operating System} \\ \hline \hline
Hewlett Packard PA-RISC (both PA7000 and PA8000 series) & HPUX 10.20 \\ \hline
Sun SPARC Sun4m,Sun4c, Sun UltraSPARC & Solaris 2.6, 2.7, 8, 9 \\ \hline
Silicon Graphics MIPS (R5000, R8000, R10000) & IRIX 6.5 \\ \hline
Intel x86 & Red Hat Linux 6.2, 7.2 \\
 & Red Hat Linux 8 (clipped) \\ \hline
 & Red Hat Linux 9 (clipped) \\ \hline
 & Windows NT 4.0 (clipped) \\ \hline
 & Windows NT 2000 (clipped) \\ \hline
 & Windows NT XP (clipped) \\ \hline
ALPHA & Digital Unix 4.0 \\
 & Red Hat Linux 7.2 (clipped) \\ \hline
 & Tru64 5.1 (clipped) \\ \hline
PowerPC & Macintosh OS X (clipped) \\
Itanium & Red Hat Linux 7.2 (clipped) \\
\end{tabular}
\caption{\label{vers-hist-sup-plat}Condor 6.5.3 supported platforms}
\end{table}
\end{center}

\noindent New Features:

\begin{itemize}

\item Added support for \$RANDOM\_CHOICE(xxx,yyy,...) in
   the configuration file parsing and in the submit description file,
   so we can now do things like:
\begin{verbatim}
   CKPT_SERVER_HOST = $RANDOM_CHOICE(check1.example.com,check2.example.com)
\end{verbatim}

\item Added a new installation and configuration tool to Condor.
  Instead of using the complicated \Condor{install} script to install
  and configure Condor, users can now try the simplified
  \Condor{configure} script.
  The new method does not ask any interactive questions.
  Instead, you set the few installation options you need to specify
  for your site as command-line arguments.
  For most sites, this method is much better than the question-driven
  \Condor{install} script, and we recommend that new users of Condor
  try \Condor{configure}.
  All the options it supports are documented in the new
  \Condor{configure} man page, section~\ref{man-condor-configure} on
  page~\pageref{man-condor-configure}. 
  However, since this is the first public release of the new script,
  there may be some problems with how it works under certain
  conditions, so we are still including \Condor{install} in the 6.5.3 
  release if you run into problems with \Condor{configure} and would
  prefer to use the old method. 

\item You can now define \MacroNI{CONDOR\_IDS} in the Condor
  configuration files, not just as an environment variable.
  This setting is used to specify what effective user id the Condor
  daemons should run as when they are spawned with root privileges.
  This is the effective user id that Condor daemons write to their log
  files as, manipulate the job queue, and so on.
  Therefore, the \File{log} and \File{spool} directories should be
  owned by this user.
  If the Condor daemons are spawned as root and \verb@CONDOR_IDS@ is
  not set, Condor searches for a ``condor'' user in the local user
  information database (the \File{/etc/passwd} file, NIS, etc).
  For more information about the \verb@CONDOR_IDS@ setting, see
  section~\ref{sec:uids} on page~\pageref{sec:uids} and/or
  section~\ref{param:CondorIds} on page~\pageref{param:CondorIds}.

\item Added the \Dflag{HOSTNAME} debugging level to print out verbose
  messages regarding how Condor resolves hostnames, domain names,
  IP addresses, and so on.
  If set, the Condor daemons also print out messages showing how they
  initialize their own local hostname and IP address.
  This is useful for sites that are having trouble getting Condor to
  work because of problems with their DNS or NIS installation.

\item Improvements for Computing On Demand (COD) support: 
  \begin{itemize}
    \item Simplified the interface to the \Condor{cod} command-line tool
      for managing COD claims.
      All commands that require a ClaimID argument no longer require
      \Opt{-name} or \Opt{-addr}, since the ClaimID contains the
      contact information for the \Condor{startd} that created it. 
    \item If you request a claim from a specific virtual machine with
      the \Opt{-name} option to \Condor{cod\_request} (for example
      \verb$condor_cod_request -name vm3@hostname$)
      the tool will now automatically append a requirement to your
      request to ensure that your claim comes from the virtual machine
      you requested.
    \item \verb@condor_status -cod -long@ now provides the ClaimID
      string for any COD claims in your pool. 
      This is useful in case you misplace or do not save the ClaimID
      returned to you from \Condor{cod\_request}.
  \end{itemize}

\item Improvements for PVM support:
  \begin{itemize}
  \item Added support for \Procedure{pvm\_export}.
  \item Added support for any tasks the Master spawns to also spawn other
	tasks. Before this fix, only the master process could spawn tasks.
  \item Added the ability for multiple tasks to be run on a resource
	simultaneously. 
  \item Added support for Condor-PVM to run multiple tasks under a single
	Condor-PVM starter. Currently, this feature must be turned on
	by a specific option in the submit description file.
  \end{itemize}

\item Improvements for Windows:
  \begin{itemize}
  \item Added support for encrypting the execute directory
    on Windows 2000/XP. This feature is enabled by setting
    \MacroNI{ENCRYPT\_EXECUTE\_DIRECTORY} = True.
  \item Added the \MacroNI{VMx\_USER} config file parameter for specifying
    an account to run jobs as (instead of the condor nobody accounts). A
    different account can be specified for each VM.
  \end{itemize}

\item Added new option \Opt{-dagman} to \Condor{submit\_dag} which allows 
specification of an alternate \Condor{dagman} executable. 

\item Added new entry in a DAG input file that adds a macro (using the
"+" syntax) for use in the submit description file.  The new entry
is called VARS.

\item Improved \Opt{-config} option to \Condor{config\_val} to print
      descriptive lines to stderr and filenames to stdout, so that its
      output can be cleanly piped into tools like xargs.

\item On Unix, added the \MacroNI{VMx\_USER} config file parameter for
specifying an account to run jobs as instead of ``nobody''. (i.e. when
\Macro{UID\_DOMAIN}s do not match and \Macro{SOFT\_UID\_DOMAIN} =
false). A different account can be specified for each VM.

\item Added \Macro{EXECUTE\_LOGIN\_IS\_DEDICATED}. When set to True,
this tells the \Condor{starter} to track job processes by login, instead
of by process tree, which prevents so-called lurker processes. This is
turned off by default.

\end{itemize}

\noindent Bugs Fixed:

\begin{itemize}

\item Fixed the serious bug with periodic checkpointing that was
  introduced in version 6.5.2.

\item Fixed a security hole in the \Condor{schedd} when it was running
  not as root on UNIX.
  Starting with Condor version 6.5.1, if you did not run the
  \Condor{schedd} as root, it would incorrectly allow other users to
  modify the job queue.
  In particular, a user could remove another user's jobs, to submit
  jobs to another user's personal \Condor{schedd}, and so on.
  This bug has now been fixed, and the \Condor{schedd} will no longer
  allow users to remove or modify other user's jobs.

\item Fixed some bugs in the \Condor{kbdd} that prevented it from
  communicating with the \Condor{startd} to send updates about console
  activity (key presses and mouse movements).
  The \Condor{kbdd} is only needed on Digital Unix and IRIX platforms,
  so this bug did not effect most Condor users.

\item Fixed a minor bug in the \Condor{startd} that caused it to exit
  with a fatal error when it fails to execute the \Condor{starter}. 
  This problem is now handled gracefully and is not considered a fatal
  error. 

\item Fixed a minor problem introduced in version 6.5.2 in
  \Condor{submit}.
  New attributes for specifying Condor's File Transfer mechanism were
  added in version 6.5.2.
  \Condor{submit} was supposed to be backwards compatible with old job
  description files, but in a few cases, it would incorrectly give a
  fatal error when it saw certain combinations of the old syntax.
  Now, all old job description files are properly recognized by the
  new \Condor{submit}.
  For more information on using the Condor File Transfer mechanism,
  see section~\ref{sec:file-transfer} on
  page~\pageref{sec:file-transfer}, or the \Condor{submit} man
  page on page~\pageref{man-condor-submit}.

\item Fixed a minor problem (introduced in version 6.5.2 while fixing
  another bug) that made it difficult to use a \Condor{master} to
  spawn the Hawkeye daemons.
  Now, if you install Hawkeye on a machine, add \verb@HAWKEYE@ to your
  \MacroNI{DAEMON\_LIST}, and define \MacroNI{HAWKEYE} to be the path to
  the \Prog{hawkeye\_master} program, the \Condor{master} will spawn
  everything correctly.

\item PVM-related bugs:
  \begin{itemize}
    \item Fixed the \Attr{Environment} attribute in the submit description 
		to be honored by both the master process and the slave tasks.
    \item Fixed a few PVM deadlock scenarios during PVM startup.
  \end{itemize}

\item Fixed a bug which caused the ``KFlops'' published in the
startd's ClassAd to have a large variance on fast CPUs.

\item Fixed a bug which could cause inter-daemon communication
problems.  A side effect of this fix, however, causes all daemons to
use 3 more file descriptors.

\item Fixed a misleading error message generated by \Condor{q} when it
  can not find the address of the \Condor{schedd} to query.
  Previously, it suggested a fairly obscure problem as the likely
  source of the error.
  The same error occurs when the \Condor{schedd} is not running, which
  is a far more common situation.

\item Fixed a couple of related of bugs which could cause the
 \Condor{collector} to seg-fault under unusual circumstances.

\item Fixed a minor bug where Condor could get confused if a machine
  had only a fully qualified domain name in DNS, but no simple
  hostname without the domain name.
  In this case, Condor daemons and tools used to exit with a fatal
  error.
  Now, they function properly, since there's no reason for them to
  consider this a problem.

 \item Fixed a minor bug on Windows where the \Condor{starter} would
   incorrectly determine the VM number if StarterLog path contained
   periods.

\end{itemize}

\noindent Known Bugs:

\begin{itemize}

\item The VARS entry within a DAG input file is not propagated
to the rescue DAG.

\end{itemize}


%%%%%%%%%%%%%%%%%%%%%%%%%%%%%%%%%%%%%%%%%%%%%%%%%%%%%%%%%%%%%%%%%%%%%%
\subsection*{\label{sec:New-6-5-2}Version 6.5.2}
%%%%%%%%%%%%%%%%%%%%%%%%%%%%%%%%%%%%%%%%%%%%%%%%%%%%%%%%%%%%%%%%%%%%%%

\noindent Known Bugs:
\begin{itemize}

\item There is a serious bug with periodic checkpointing that was
  introduced in version 6.5.2.
  This bug only effects jobs submitted to the Standard universe, since
  those are the only jobs that can perform periodic checkpoints in
  Condor. 
  As soon as the \Macro{PERIODIC\_CHECKPOINT} expression evaluates to
  TRUE on a machine, the \Condor{startd} will get in a loop where it
  continues to request a periodic checkpoint every 5 seconds until the
  job is evicted from the machine.
  The work-around for this bug in version 6.5.2 is to set
  \verb@PERIODIC_CHECKPOINT = False@ in your \File{condor\_config}
  file.

\end{itemize}

\noindent New Features:
\begin{itemize}

\item The collector and negotiator can now run on configurable
ports, instead of relying on hard-coded values.
To use this feature, many places in Condor where you could previously
only provide a hostname now understand ``hostname:port'' notation.
For example, in your config file, you can now use:

\begin{verbatim}
  COLLECTOR_HOST = $(CONDOR_HOST):9650
  NEGOTIATOR_HOST = $(CONDOR_HOST):9651
  FLOCK_TO = your-other.collector.domain.org:7002
\end{verbatim}

If you define \Macro{COLLECTOR\_HOST} in this way, all Condor tools
will automatically use the specified port if you are using them in the
local pool (so you do not need to use any special options to the tools
to get them to find your \Condor{collector} listening on the new port).

In addition, the \Opt{-pool} option to all Condor tools now
understands the ``hostname:port'' notation for remote pools.
To query a remote pool with a collector listening on a non-standard
port, you can use this:

\begin{verbatim}
condor_status -pool your-other.collector.domain.org:7002
\end{verbatim}

\item When the Condor daemons start up, they now log the names of the
Configuration files they are using, right after the startup banner.

\item The \Condor{config\_val} program now has a \Opt{-verbose} option
  which will tell you in which configuration file and line number a
  condor configuration parameter is defined, and a \Opt{-config}
  option which will simply list all of the configuration files in use
  by Condor.

\item There are new attributes to control Condor's file transfer
  mechanism.
  Not only will they hopefully be more clear and easy to use than the
  old ones, they also provide new functionality.
  There is now an option to only transfer files ``IF\_NEEDED''.
  In this case, if the job is matched with a machine in the same
  \Attr{FileSystemDomain}, the files are not transfered, but if the
  job runs at a machine in a different \Attr{FileSystemDomain}, the
  files are transfered automatically.
  For more information on using the Condor File Transfer mechanism,
  see section~\ref{sec:file-transfer} on
  page~\pageref{sec:file-transfer}, or the \Condor{submit} man
  page on page~\pageref{man-condor-submit}.

\item Added support for Condor's new ``Computing on Demand''
  functionality.
  Documentation is not yet available, but will be coming soon.

\item DAGMan no longer requires that all jobs in a DAG have the same
  log file.

\item The value of the \Macro{JOB\_RENICE\_INCREMENT} configuration
      parameter can now be an arbitrary ClassAd expression rather than
      just a static integer.  The expression will be evaluated by the
      \Condor{starter} for each job just before it runs, and can refer
      to any attribute in the job ClassAd.

\item The \Condor{collector} now publishes statistics about the running
  jobs.  In particular, it now publishes the number of jobs running in
  each Universe, both as a snapshot, and the maximum of the snapshots
  for each month.

\item The UNIX \Condor{collector} has the ability to fork off child
  processes to handle queries.  The \Macro{COLLECTOR\_QUERY\_WORKERS}
  parameter is used to specify the maximum number of these worker
  processes and defaults to zero.

\item All daemons now publish ``sequence number'' and ``start time''
information in their ClassAds.

\item The Collector now maintains and publishes update statistics
using the above ClassAd ``sequence number'' and ``start time''
information.  History information is stored for the past
COLLECTOR\_CLASS\_HISTORY\_SIZE updates and is also published in the
Collector's ClassAd as a hex string.

\item The \Condor{schedd} can now use TCP connections to send updates
  to pools that it is configured to flock to.
  You can now define \Macro{TCP\_UPDATE\_COLLECTORS} list and any
  collectors listed there, including ones the \Condor{schedd} is
  flocking with, will be updated with TCP.
  Also, the \Condor{master} uses the same list to decide if it should
  use TCP to update any collectors listed in the
  \Macro{SECONDARY\_COLLECTORS\_LIST}.
  For more infomation on TCP collector updates in Condor and if your
  site would want to enable them, read
  section~\ref{sec:tcp-collector-update} on ``Using TCP to Send 
  Collector Updates'' on page~\pageref{sec:tcp-collector-update}.

\item You no longer need to define \Macro{FLOCK\_VIEW\_SERVERS} in
  your config file if you have configured a \Condor{schedd} to flock
  to other pools.
  This is now handled automatically, so you only have to define
  \Macro{FLOCK\_TO}.

\item If no \Macro{STARTD\_EXPRS} is specified, the \Condor{startd}
  now defaults to ``JobUniverse''.

\item Globus universe jobs under Condor-G now send email on job
completion based on the notification setting in the submit file.

\item When submitting a Globus universe job to Condor-G the
input, output, and error files can now optionally be an http,
https, ftp, or gsiftp URL.  (The actual file transfer is handled
by globus-url-copy on the remote site.)

\end{itemize}

\noindent Bugs Fixed:
\begin{itemize}

\item DAGMan now correctly reports an error and rejects DAGs which
      contain two nodes with the same name, regardless of their case.
      (DAGMan has rejected duplicate node names since Condor 6.4.6,
      but until now it would fail to do so if there was any difference
      in their case.)

\item When \Condor{submit\_dag} checks job submit files for proper
      ``log'' statements, it now correctly recognizes lines with
      leading whitespace.

\item Fixed a minor bug whereby DAGMan was not removing its lock file
      after successful completion.

\item Fixed a bug introduced in version 6.5.0 whereby the
      \Macro{UID\_DOMAIN} attributes of jobs and resources were being
      compared in a case-sensitive manner, resulting in erroneous
      failures.

\item The \Condor{master} used to always pass a \Opt{-f} on the
  command line to all daemons defined in the \Macro{DAEMON\_LIST}
  config file setting.
  However, if you include entries which are not Condor daemons, the
  \Condor{master} will no longer add a \Opt{-f}.

\item The files specified in \Attr{transfer\_input\_files} and/or 
  \Attr{transfer\_output\_files} can now contain spaces in the
  filenames.
  If there are multiple names, they \emph{must} be seperated by a
  comma.
  Any spaces are considered part of a filename, not separators between
  filenames.
  This allows filenames containing spaces (which are common on Windows
  platforms) to be easily described.

\end{itemize}


%%%%%%%%%%%%%%%%%%%%%%%%%%%%%%%%%%%%%%%%%%%%%%%%%%%%%%%%%%%%%%%%%%%%%%
\subsection*{\label{sec:New-6-5-1}Version 6.5.1}
%%%%%%%%%%%%%%%%%%%%%%%%%%%%%%%%%%%%%%%%%%%%%%%%%%%%%%%%%%%%%%%%%%%%%%

\noindent New Features:
\begin{itemize}

\item DAGMan now supports both Condor computational jobs and Stork data
placement (DaP) jobs.  (See \URL{http://www.cs.wisc.edu/condor/stork/}
for more info on Stork.)

\item Starter exceptions, such as failure to open standard input, are
now recorded in the user log.

\item Added new \Opt{-forcex} argument to \Condor{rm} to force the
immediate local removal of (typically Globus universe) jobs in the 'X'
state, regardless of their remote state.

\end{itemize}

\noindent Bugs Fixed:
\begin{itemize}

\item When transfer\_files is being used, the path to the stdout/stderr
files was not being respected.  After hese files have been transferred,
they are now copied to the location specified in the submit file.

\item A DAGMan bug introduced in Condor 6.5.0 has been fixed, where
DAGMan could crash (with a failed assertion) when recovering from a
rescue DAG.

\item Fixed a bug in the example condor\_config.generic and
hawkeye\_config files, where COLLECTOR\_HOST was being included in the
default STARTD\_EXPRS in non-string form, resulting in an invalid value
for that attribute in machine classads.

\end{itemize}

\noindent Known Bugs:
\begin{itemize}

\item DAGMan doesn't detect when users mistakenly specify two
DAG nodes with the same node name; instead it waits for the
same node to complete twice, which never happens, and so DAGMan
goes off into never-never land.

\end{itemize}

%%%%%%%%%%%%%%%%%%%%%%%%%%%%%%%%%%%%%%%%%%%%%%%%%%%%%%%%%%%%%%%%%%%%%%
\subsection*{\label{sec:New-6-5-0}Version 6.5.0}
%%%%%%%%%%%%%%%%%%%%%%%%%%%%%%%%%%%%%%%%%%%%%%%%%%%%%%%%%%%%%%%%%%%%%%
\noindent New Features:
\begin{itemize}

\item A fresh value of RemoteWallClock is now used when evaluating
user policy expressions, such as periodic\_remove.

\item The IOProxy handler now handles escaped characters (whitespace)
in filenames.

\item condor.boot is now configured to work automatically with Red Hat
chkconfig.

\item A new log\_xml option has been added to condor\_submit. It is
documented in the condor\_submit portion of the manual.

\item A new DAGMan option to produce dot files was added. Dot is a
program that creates visualizations of DAGs. This feature is
documented in Section~\ref{sec:DAGMan}.

\item The email report from condor\_preen is now less cryptic, and
more self-explanatory.

\item Specifying full device paths (e.g., ``/dev/mouse'') instead of bare
device names (e.g., ``mouse'') in CONSOLE\_DEVICES in the config file is no
longer an error.

\item The condor\_submit tool now prints a more helpful, specific error if
the specified job executable is not found, or can't be accessed.

\item The startd ``cron'' (Hawkeye) now permits zero length ``prefix''
strings.

\item A number of new Hawkeye modules have been added, and most have
various bug fixes and improvements.

\item Added support for a new config parameter, Q\_QUERY\_TIMEOUT, which
defines the timeout that \Condor{q} uses when communicating with the
\Condor{schedd}.

\item Added the ability to use TCP to send ClassAd updates to the
Condor{collector}, though the feature is disabled by default.
Read section~\ref{sec:tcp-collector-update} on ``Using TCP to Send
Collector Updates'' on page~\pageref{sec:tcp-collector-update} for
more details and a discussion of when a site would need to enable this 
functionality.

\item Scheduler Universe has been ported to Windows. This enables
\Condor{dagman} to run on Windows as well.

\end{itemize}

\noindent Bugs Fixed:
\begin{itemize}

\item Hawkeye will no longer busy-loop if a ``continuous mode''
module with a period of 0 fails to execute for some reason.  (Now, for
continuous mode modules, a period of 0 is automatically reset to be 1, and
a warning appears in the log.)

\item Fixed a very rare potential bug when initializing a user log
file.
Improved the error messages generated when there are problems
initializing the user log to include string descriptions of the
errors, not just the error number (errno).

\item The default value in the config files for \Macro{LOCK} is now
defined in terms of \Macro{LOG}, instead of using \Macro{LOCAL\_DIR}
and appending ``log''.
This is a very minor correction, but in the rare cases where the log
directory is being redefined for some reason, we usually want that to
apply to the lock files, as well.
Of course, if the log directory is on NFS, \Macro{LOCK} should still
be customized to point to a directoy on a local file system.

\item Fixed a bug in the \Condor{schedd} where Scheduler Universe jobs
  with \Macro{copy\_to\_spool} = false would fail.

\end{itemize}

\noindent Known Bugs:
\begin{itemize}

\item DAGMan doesn't detect when users mistakenly specify two
DAG nodes with the same node name; instead it waits for the
same node to complete twice, which never happens, and so DAGMan
goes off into never-never land.

\end{itemize}

%%%%%%%%%%%%%%%%%%%%%%%%%%%%%%%%%%%%%%%%%%%%%%%%%%%%%%%%%%%%%%%%%%%%%%%
\section{\label{sec:History-6-4}Stable Release Series 6.4}
%%%%%%%%%%%%%%%%%%%%%%%%%%%%%%%%%%%%%%%%%%%%%%%%%%%%%%%%%%%%%%%%%%%%%%

This is the stable release series of Condor.
New features will be added and tested in the 6.5 development series. 
The details of each version are described below.

%%%%%%%%%%%%%%%%%%%%%%%%%%%%%%%%%%%%%%%%%%%%%%%%%%%%%%%%%%%%%%%%%%%%%%
\subsection{\label{sec:New-6-2-0}Version 6.4.8}
%%%%%%%%%%%%%%%%%%%%%%%%%%%%%%%%%%%%%%%%%%%%%%%%%%%%%%%%%%%%%%%%%%%%%%

\noindent New Features:

\begin{itemize}

\item None.

\end{itemize}

\noindent Bugs Fixed:

\begin{itemize}

\item The starter would crash if the execute directory contained
symlinks from child to parent directories.

\end{itemize}

\noindent Known Bugs:

\begin{itemize}

\item None.

\end{itemize}

%%%%%%%%%%%%%%%%%%%%%%%%%%%%%%%%%%%%%%%%%%%%%%%%%%%%%%%%%%%%%%%%%%%%%%
\subsection{\label{sec:New-6-4-7}Version 6.4.7}
%%%%%%%%%%%%%%%%%%%%%%%%%%%%%%%%%%%%%%%%%%%%%%%%%%%%%%%%%%%%%%%%%%%%%%
\noindent New Features:
\begin{itemize}

\item. None

\end{itemize}

\noindent Bugs Fixed:
\begin{itemize}

\item Fixed major problem with inflated job counts in \Condor{status} -submitter 
\item Fixed a problem with the UserLog parser if the Aborted, Held, or Removed
event did not have the optional reason string that caused those events to be
ignored.

\item Fixed a problem with Condor-G on some older Linux distributions that
would cause it to crash on startup because of an invalid file descriptor for
stderr

\end{itemize}
\noindent Known Bugs:
\begin{itemize}

\item None.

\end{itemize}

%%%%%%%%%%%%%%%%%%%%%%%%%%%%%%%%%%%%%%%%%%%%%%%%%%%%%%%%%%%%%%%%%%%%%%
\subsection{\label{sec:New-6-4-6}Version 6.4.6}
%%%%%%%%%%%%%%%%%%%%%%%%%%%%%%%%%%%%%%%%%%%%%%%%%%%%%%%%%%%%%%%%%%%%%%

\noindent New Features:
\begin{itemize}

\item Clarified the output of \Condor{q} -analyze.

\end{itemize}

\noindent Bugs Fixed:
\begin{itemize}

\item Fixed a major bug that caused standard universe Condor jobs to
crash whenever they ran in a pool with the \Macro{LOWPORT} and
\Macro{HIGHPORT} config file settings enabled.
These settings restrict what ports Condor will use in your pool.
To allow standard universe jobs to run in such a pool, you must relink
your executable with \Condor{compile} after the 6.4.6 condor libraries
are installed in your Condor \File{lib} directory.

\item When more than 512 distinct users submit to Condor or Condor-G,
the \Condor{schedd} no longer crashes. 

\item DAGMan now correctly reports an error and rejects DAGs
which contain two nodes with the same name; in the past DAGMan
did not detect this, and would not exit even when the DAG was
complete.

\item DAGMan now correctly reports an error when the null string
is passed to any of its arguments which require a filename.

\item The \Opt{-format} option to \Condor{q} and \Condor{status} can
now be used to print out boolean expressions in ClassAds, not just
strings, integers and floating point numbers.
Any boolean expression will now be treated like a string, so be sure
to use \verb@%s@ as the conversion character in the formatting string
you pass to \Opt{-format}.

\item If no \Attr{Rank} is specified in the job description file, or
in the \Macro{DEFAULT\_RANK} config file variable, the default value
is now ``0.0'' not just ``0''.
Since \Attr{Rank} is supposed to be a floating point value, the
``0.0'' ensures that the \Opt{-format} option to tools like \Condor{q}
and \Condor{status} will always treat this variable as a float.

\item Fixed the help message for the \Condor{rm}, \Condor{hold} and
\Condor{release} tools to be more helpful, and to mention the
\Opt{-constraint} option.

\item When you ran \Condor{q} in a way that it needed to query the
\Condor{collector} and that query failed, the error message used to be
very short and cryptic.
Now, there is a verbose message that explains what went wrong and
possible ways you can fix the problem.
Also fixed a misleading error message in \Condor{q} that was
displayed under very rare circumstances.

\end{itemize}

\noindent Packaging Changes:

\begin{itemize}

\item Begining with Condor version 6.4.6, all of the Condor-G related
binaries (\Condor{gridmanager}, \Condor{glidein}, and
\Prog{gahp\_server}) are also included in the full release of Condor.
So, if you are using the full Condor system and want to use Condor's
grid-enabled functionality, you no longer need to download and install
a separate ``contrib module''.  
However, if all you want is Condor-G, the contrib module is still
available.
For more information, please the chapter on ``Grid Computing'' on
page~\pageref{grid-computing}.

\item Fixed a process family managment bug in the \Condor{startd} on 
Windows. This bug was preventing the \Condor{startd} from detecting
the \Condor{starter} as a member of its process family.

\end{itemize}

%%%%%%%%%%%%%%%%%%%%%%%%%%%%%%%%%%%%%%%%%%%%%%%%%%%%%%%%%%%%%%%%%%%%%%
\subsection{\label{sec:New-6-4-3}Version 6.4.3}
%%%%%%%%%%%%%%%%%%%%%%%%%%%%%%%%%%%%%%%%%%%%%%%%%%%%%%%%%%%%%%%%%%%%%%

\noindent New Features:
\begin{itemize}

\item Added a \Opt{-hold} and \Opt{-held} option to \Condor{q} which 
displays the reason that the job had been held.

\end{itemize}

\noindent Bugs Fixed:
\begin{itemize}

\item Fixed a bug where more than one space between arguments to a job
in the java universe would result in it being invoked with and incorrect
list arguments.

\item Removed renaming of the executable to \Prog{condor\_exec} in the java
universe. This fixes a bug where the JVM was looking at its path to determine
its installation directory.

\item Fixed a bug and resulting null pointer exception in the java universe
because under certain conditions, Condor would invoke the JVM incorrectly.

\item Fixed serveral error reporting messages to be more precise.

\item When the NIS environment was being used, the \Condor{starter} daemon
would produce heavy amounts of NIS traffic. This has been fixed.

\item Binary characters in the \File{StarterLog} file and a possible
segmentation fault have been fixed.

\item Fixed \Cmd{select}{2} in the standard universe on our Linux ports.

\item Fixed a small bug in \Condor{q} that was displaying the wrong
username for ``niceuser'' jobs.

\item Fixed a bug where, in the standard universe, you could not open a file
whose name had spaces in it.

\item Fixed a bug in DAGMan where pre and post scripts would fail to
run if the DAG description file had extra whitespace.
Also, reworded the error messages DAGMan produces when it fails to
parse the DAG description file to be more clear and helpful for
solving the problem.

\item Fixed some misleading error messages in the Condor log files
when there were permission problems trying to execute a program. 

\item Condor for Windows will now run on Windows XP.

\item Condor for Windows now supports the Java Universe.

\item Users logged into Windows Domain accounts rather than local accounts
can submit jobs.

\item Potential Windows registry bloating bug fixed. Condor for Windows no
longer creates and deletes an account on the execute machine each time a
job is run. Instead, a single account for each VM on the execute machine is
created once and enabled or disabled as needed.

\item Cross-submits from Windows to Unix and from Unix to Windows are now
supported, provided that both platforms are running Condor 6.4 series daemons.

\item Free disk space is now reported accurately on Windows.

\item A rare but serious bug that could allow non-Condor processes to be added
to the Condor process family on Windows has been fixed.

\item Condor for Windows will now also run 16-bit applications.

\item Fixed a minor bug where certain integer attributes in the
\File{condor\_config} file might not have been properly parsed if they
were defined in terms of other config file attributes, using the
\MacroUNI{attribute} notation.  

\end{itemize}

\noindent Known Bugs:
\begin{itemize}

\item You may not open a file in the standard universe whose name contains a
colon ``:''.

\end{itemize}

%%%%%%%%%%%%%%%%%%%%%%%%%%%%%%%%%%%%%%%%%%%%%%%%%%%%%%%%%%%%%%%%%%%%%%
\subsection{\label{sec:New-6-4-2}Version 6.4.2}
%%%%%%%%%%%%%%%%%%%%%%%%%%%%%%%%%%%%%%%%%%%%%%%%%%%%%%%%%%%%%%%%%%%%%%
\noindent New Features:
\begin{itemize}

\item. This release mirrored the Condor-G release, and has no new features.

\end{itemize}

\noindent Bugs Fixed:
\begin{itemize}
\item None.

\end{itemize}
\noindent Known Bugs:
\begin{itemize}

\item None.

\end{itemize}

%%%%%%%%%%%%%%%%%%%%%%%%%%%%%%%%%%%%%%%%%%%%%%%%%%%%%%%%%%%%%%%%%%%%%%
\subsection{\label{sec:New-6-4-1}Version 6.4.1}
%%%%%%%%%%%%%%%%%%%%%%%%%%%%%%%%%%%%%%%%%%%%%%%%%%%%%%%%%%%%%%%%%%%%%%
\noindent New Features:
\begin{itemize}

\item None.

\end{itemize}

\noindent Bugs Fixed:
\begin{itemize}

\item Users are now allowed to answer ``none'' when prompted by the
installer to provide a Java JVM path. This avoids an endless loop and
leaves the Java abilities of Condor unconfigured.

\end{itemize}

\noindent Known Bugs:
\begin{itemize}

\item None.

\end{itemize}

%%%%%%%%%%%%%%%%%%%%%%%%%%%%%%%%%%%%%%%%%%%%%%%%%%%%%%%%%%%%%%%%%%%%%%
\subsection{\label{sec:New-6-4-0}Version 6.4.0}
%%%%%%%%%%%%%%%%%%%%%%%%%%%%%%%%%%%%%%%%%%%%%%%%%%%%%%%%%%%%%%%%%%%%%%

\noindent New Features:

\begin{itemize}

\item If a job universe is not specified in a submit description file, 
\Condor{submit}  will check the config file for \Macro{DEFAULT\_UNIVERSE}
instead of always choosing the standard universe. 

\item The \Macro{D\_SECONDS} debug flag is deprecated. Seconds are now always
included in logfiles. 

\item For each daemon listed in \Macro{DAEMON\_LIST}, you can now control the
environment variables of the daemon with a config file setting of the form
\Macro{DAEMONNAME\_ENVIRONMENT}, where \MacroNI{DAEMONNAME} is the name of a
daemon listed in \Macro{DAEMON\_LIST}. For more information, see
section~\ref{sec:Master-Config-File-Entries}.

\end{itemize}

\noindent Bugs Fixed:

\begin{itemize}

\item Fixed a bug in the new starter where if the submit file set no
arguments, the job would receive one argument of zero length.

\end{itemize}

\noindent Known Bugs:

\begin{itemize}

\item None.

\end{itemize}

%%%%%%%%%%%%%%%%%%%%%%%%%%%%%%%%%%%%%%%%%%%%%%%%%%%%%%%%%%%%%%%%%%%%%%%
\section{\label{sec:History-6-3}Development Release Series 6.3}
%%%%%%%%%%%%%%%%%%%%%%%%%%%%%%%%%%%%%%%%%%%%%%%%%%%%%%%%%%%%%%%%%%%%%%

This is the second development release series of Condor.

It contains numerous enhancements over the 6.2 stable series.
For example:

\begin{itemize}

\item Support for Kerberos and X.509 authentication.

\item Support for transferring files needed by jobs (for all universes
except standard and PVM)

\item Support for MPICH jobs.

\item Support for JAVA jobs.

\item 
Condor DAGMan is dramatically more reliable and efficient, and offers
a number of new features.

\end{itemize}

The 6.3 series has many other improvements over the 6.2 series, and
may be available on newer platforms.  The new features, bugs fixed,
and known bugs of each version are described below in detail.


%%%%%%%%%%%%%%%%%%%%%%%%%%%%%%%%%%%%%%%%%%%%%%%%%%%%%%%%%%%%%%%%%%%%%%
\subsection{\label{sec:New-6-3-3}Version 6.3.3}
%%%%%%%%%%%%%%%%%%%%%%%%%%%%%%%%%%%%%%%%%%%%%%%%%%%%%%%%%%%%%%%%%%%%%%

\noindent New Features:

\begin{itemize}

\item Added support for Kerberos and X.509 authentication in Condor.  

\item Added the ability for vanilla jobs on Unix to use Condor's file
transfer mechanism so that you don't have to rely on a shared file
system.  

\item Added support for MPICH jobs on Windows NT and 2000.

\item Added support for the JAVA universe.

\item When you use \Condor{hold} and \Condor{release}, you now see an
entry about the event in the UserLog file for the job.

\item Whenever a job is removed, put on hold, or released (either by a
Condor user or by the Condor system itself), there is a ``reason''
attribute placed in the job ad and written to the UserLog file.  
If a job is held, \Attr{HoldReason} will be set.
If a job is released, \Attr{ReleaseReason} will be set.
If a job is removed, \Attr{RemoveReason} will be set.
In addition, whenever a job's status changes,
\Attr{EnteredCurrentStatus} will contain the epoch time when the
change took place.

\item The error messages you get from \Condor{rm}, \Condor{hold} and
\Condor{release} have all been updated to be more specific and
accurate. 

\item Condor users can now specify a policy for when their jobs should
leave the queue or be put on hold.
They can specify expressions that are evaluated periodically, and
whenever the job exits.
This policy can be used to ensure that the job remains in the queue
and is re-run until it exits with a certain exit code, that the job
should be put on hold if a certain condition is true, and so on. 
If any of these policy expressions result in the job being removed
from the queue or put on hold, the UserLog entry for the event
includes a string describing why the action was taken.

\item Changed the way Condor finds the various \Condor{shadow} and
\Condor{starter} binaries you have installed on your machine.
Now, you can specify a \Macro{SHADOW\_LIST} and a
\Macro{STARTER\_LIST}.
These are treated much like the \MacroNI{DAEMON\_LIST} setting, they
specify a list of attribute names, each of which point to the actual
binary you want to use.
On startup, Condor will check these lists, make sure all the binaries
specified exist, and find out what abilities each program provides.
This information is used during matchmaking to ensure that a job which
requires a certain ability (like having a new enough version of Condor
to support transferring files on Unix) can find a resource that
provides that ability.

\item Added new security feature to offer fine-grained control over
what configuration values can be modified by \Condor{config\_val}
using \Arg{-set} and related options.
Pool administrators can now define lists of attributes that can be set
by hosts that authenticate to the various permission levels of
Condor's host based security (for example, \DCPerm{WRITE},
\DCPerm{ADMINISTRATOR}, etc).
These lists are defined by attributes with names like
\Macro{SETTABLE\_ATTRS\_CONFIG} and
\Macro{STARTD\_SETTABLE\_ATTRS\_OWNER}. 
For more information about host-based security in Condor, see
section~\ref{sec:Host-Security} on page~\pageref{sec:Host-Security}.
For more information about how to configure the new settings, see the
same section of the manual.
In particular, see section~\ref{sec:Host-Security} on
page~\pageref{sec:Host-Security}. 

\item Greatly improved the handling of the ``soft kill signal'' you
can specify for your job.
This signal is now stored as a signal name, not an integer, so that it
works across different platforms.
Also, fixed some bugs where the signal numbers were getting translated
incorrectly in some circumstances.

\item Added the \Arg{-full} option to \Condor{reconfig}.
The \Arg{-full} option causes the Condor daemon to clear its cache of
DNS information and some other expensive operations.
So, the regular \Condor{reconfig} is now more light-weight, and can
be used more frequently without undue overhead on the Condor daemons. 
The default \Condor{reconfig} has also been changed so that it will
work from any host with \DCPerm{WRITE} permission in your pool,
instead of requiring \DCPerm{ADMINISTRATOR} access.

\item Added the \Macro{EMAIL\_DOMAIN} config file setting.
This allows Condor administrators to define a default domain where
Condor should send email if whatever \Macro{UID\_DOMAIN} is set to
would yield invalid email addresses.
For more information, see section~\ref{param:EmailDomain} on
page~\pageref{param:EmailDomain}.

\item
Added support for Red Hat 7.2.

\item When printing out the UserLog, we now only log a new event for
``Image size of job updated'' when the new value is different than the
existing value.

\end{itemize}

\noindent Bugs Fixed:

\begin{itemize}

\item
Fixed a bug in Condor-PVM where it was possible that a machine would be 
placed into the virtual machine, but then ignored by Condor for the purposes
of scheduling tasks there.

\item
Under Solaris, the checkpointing libraries could segfault while determining
the page size of the machine. 
This has been fixed.

\item
In a heavily loaded submit machine, the \Condor{schedd} would time out
authentication checks with its shadows. 
This would cause the shadows to
exit believing the \Condor{schedd} had died placing jobs into the idle
state and the \Condor{schedd} to exhibit poor performance.
This timeout problem has been corrected.

\item
Removed use of the bfd libary in the Condor Linux distribution. 
This will make the dynamic versions of the Condor executables have a
higher chance of being usable when Red Hat upgrades.

\item
When you specify ``STARTD\_HAS\_BAD\_UTMP = True'' in the config files
on a linux machine with a 2.4+ kernel, the \Condor{startd} would report
an error stating some of the tty entries in /dev. This would result
in incorrect tty activity sampling causing jobs to not be migrated or
incorrectly started on a resource. This has now been corrected.

\item 
When you specify ``GenEnv = True'' in a \Condor{submit} file,
your environment is no longer restricted to 10KB.

\item
The three-digit event numbers which begin each job event in the
userlog were incorrect for some events in Condor 6.3.0 and 6.3.1.
Specifically, ULOG\_JOB\_SUSPENDED, ULOG\_JOB\_UNSUSPENDED,
ULOG\_JOB\_HELD, ULOG\_JOB\_RELEASED, ULOG\_GENERIC, and
ULOG\_JOB\_ABORTED had incorrect event numbers.  This has now been
corrected.

\Note This means userlog-parsing code written for Condor 6.3.0 or
6.3.1 development releases may not work reliably with userlogs
generated by other versions of Condor, and visa-versa.  Userlog events
will remain compatible between all stable releases of Condor, however,
and with post-6.3.1 releases in this development series.

\item
The \Condor{run} script now correctly exits when it sees a job aborted
event, instead of hanging, waiting for a termination event.

\item
Until now, when a DAG node's Condor job failed, the node failed,
regardless of whether its POST script succeeded or failed.  This was a
bug, because it prevented users from using POST scripts to evaluate
jobs with non-zero exit codes and deem them successful anyway.  This
has now been fixed -- a node's success is equal to its POST script's
success -- but the change may affect existing DAGs which rely on the
old, broken behavior.  Users utilizing POST scripts must now be sure
to pass the POST script the job's return value, and return it again,
if they do not wish to alter it; otherwise failed jobs will be masked
by ignorant POST scripts which always succeed.

\end{itemize}

\noindent Known Bugs:

\begin{itemize}
\item The HP-UX Vendor C++ CFront compiler does not work with \Condor{compile}
if exception handling is enabled with +eh.

\item The HP-UX Vendor aCC compiler does not work at all with Condor.
\end{itemize}

%%%%%%%%%%%%%%%%%%%%%%%%%%%%%%%%%%%%%%%%%%%%%%%%%%%%%%%%%%%%%%%%%%%%%%
\subsection{\label{sec:New-6-3-2}Version 6.3.2}
%%%%%%%%%%%%%%%%%%%%%%%%%%%%%%%%%%%%%%%%%%%%%%%%%%%%%%%%%%%%%%%%%%%%%%

Version 6.3.2 of Condor was only released as a version of
``Condor-G''.
This version of Condor-G is not widely deployed.
However, to avoid confusion, the Condor developers did not want to
release a full Condor distribution with the same version number.


%%%%%%%%%%%%%%%%%%%%%%%%%%%%%%%%%%%%%%%%%%%%%%%%%%%%%%%%%%%%%%%%%%%%%%
\subsection{\label{sec:New-6-3-1}Version 6.3.1}
%%%%%%%%%%%%%%%%%%%%%%%%%%%%%%%%%%%%%%%%%%%%%%%%%%%%%%%%%%%%%%%%%%%%%%

\noindent New Features:
\begin{itemize}

\item
Added support for an \AdAttr{x509proxy} option in
\Condor{submit}. There is now a seperate \Condor{GridManager} for each
user and proxy pair. This will be detailed in a future release of
Condor.
 
\item
More Condor DAGMan improvements and bug fixes:

\begin{itemize}

\item 
Added a \oArgnm{-dag} flag to \Condor{q} to more succinctly display dags
and their ownership.

\item
Added a new event to the Condor userlog at the completion of a POST
script.  This allows DAGMan, during recovery, to know which POST
scripts have finished succesfully, so it no longer has to re-run them
all to make sure.

\item
Implemented separate \Arg{-MaxPre} and \Arg{-MaxPost} options to limit
the number of simultaneously running PRE and POST scripts.  The
\Arg{-MaxScripts} option is still available, and is equivalent to
setting both \Arg{-MaxPre} and \Arg{-MaxPost} to the same value.

\item
Added support for a new ``Retry'' parameter in the DAG file, which
instructs DAGMan to automatically retry a node a configurable number
of times if its PRE Script, Job, or POST Script fail for any reason.

\item
Added timestamps to all DAGMan log messages.

\item
Fixed a bug whereby DAGMan would clean up its lock file without
creating a rescue file when killed with SIGTERM.

\item
DAGMan no longer aborts the DAG if it encounters executable error or
job aborted events in the userlog, but rather marks the corresponding
DAG nodes as ``failed'' so the rest of the DAG can continue.

\item
Fixed a bug whereby DAGMan could crash if it saw userlog events for
jobs it didn't submit.

\end{itemize}

\item Added port restriction capabilities to Condor so you can specify a range
of ports to use for the communication between Condor Daemons.

\item To improve performance: if there's no \Macro{HISTORY} file
specified, don't connect back to the schedd to report your exit info on
successful compeletion, since the schedd is simply going to discard that
info anyway.

\item Added the macro \Macro{SECONDARY\_COLLECTOR\_LIST} to tell the
master to send classads to an additional list of collectors so you can
do administration commands when the primary collector is down.

\item When a job checkpoints it askes the shadow whether or not it
should and if so where. This fixes some flocking bugs and increases
performance of the pool.

\item Added match rejection diagnostics in \Condor{q} \oArgnm{-analyze} to
give more information on why a particular job hasn't started up yet.

\item Added \oArgnm{--vms} argument to \Condor{glidein} that enables the
control of how many virtual machines to start up on the target platform.

\item Added capability to the config file language to retrieve environment
variables while being processed.

\item Added capability to make default user user priority factor configurable
with the \Macro{DEFAULT\_PRIORITY\_FACTOR} macro in the config files.

\item Added full support for Red Hat 7.1 and the gcc 2.96 compiler. However,
the standard universe binaries must still be statically linked.

\item When jobs are suspended or unsuspended, an event is now written into
the user job log.

\item Added \oArgnm{-a} flag to \Condor{submit} to add/override attributes
specified in the submit file.

\item Under Unix, added the ability for a submittor of a job to describe when
and how a job is allowed/not allowed to leave the queue. For example, if
a job has only run for 5 minutes, but it was supposed to have run an hour 
minimum, then do not let the job leave the queue but restart it instead.

\item New environment variable available CONDOR\_SCRATCH\_DIR available
in a standard or vanilla job's environment that denotes temporary space
the job can use that will be cleaned up automatically when the job leaves
from the machine.

\item Not exactly a new feature, but some internal parts of Condor had been
fixed up to try and improve the memory footprint of a few of our daemons.

\end{itemize}

\noindent Bugs Fixed:
\begin{itemize}

\item Fixed a bug where \Condor{q} would produce wildly inaccurate run time
reports of jobs in the queue.

\item Fixed it so that if the condor scheduler fails to notify the
administrator through email, just print a warning and do not except.

\item Fixed a bug where \Condor{submit} would incorrectly create the user
log file.

\item Fixed a bug where a job queue sorted by date with \Condor{q} would
be displayed in descending instead of ascending order.

\item Fixed and improved error handling when \Condor{submit} fails.

\item Numerous fixes in the Condor User Log System.

\item Fixed a bug where when Condor inspects its on disk job queue log,
it would do it with case sensitivity. Now there is no case sensitivity.

\item Fixed a bug in \Condor{glidein} where it have trouble figuring out
the architecture of a minimally installed HP-UX machine.

\item Fixed it so that email to the user has the word ``condor'' capitalized
in the subject.

\item Fixed a situation where when a user has multiple schedulers submitting
to the same pool, the Negotiator would starve some of the schedulers.

\item Added a feature whereby if a transfer of an executable
from a submission machine to an execute machine fails, Condor
will retry a configurable numbers of times denotated by the
\Macro{EXEC\_TRANSFER\_ATTEMPTS} macro. This macro defaults to three if
left undefined. This macro exists only for the Unix port of Condor.

\item Fixed a bug where if a schedd had too many rejected clusters during a
match phase, it would ``except'' and have to be restarted by the master.

\end{itemize}

\noindent Known Bugs:
\begin{itemize}
\item The HP-UX Vendor C++ CFront compiler does not work with \Condor{compile}
if exception handling is enabled with +eh.

\item The HP-UX Vendor aCC compiler does not work at all with Condor.
\end{itemize}

%%%%%%%%%%%%%%%%%%%%%%%%%%%%%%%%%%%%%%%%%%%%%%%%%%%%%%%%%%%%%%%%%%%%%%
\subsection{\label{sec:New-6-3-0}Version 6.3.0}
%%%%%%%%%%%%%%%%%%%%%%%%%%%%%%%%%%%%%%%%%%%%%%%%%%%%%%%%%%%%%%%%%%%%%%

\noindent New Features:
\begin{itemize}

\item Added support for running MPICH jobs under Condor.

\end{itemize}

\noindent
Many Condor DAGMan improvements and bug fixes:

\begin{itemize}

\item
PRE and POST scripts now run asynchronously, rather than synchronously
as in the past.  As a result, DAGMan now supports a \Arg{-MaxScripts}
option to limit the number of simultaneously running PRE and POST
scripts.

\item
Whether or not POST scripts are always executed after failed jobs is
now configurable with the \Arg{-NoPostFail} argument.

\item
Added a \Arg{-r} flag to \Condor{submit\_dag} to submit DAGMan to a
remote \Condor{schedd}.

\item
Made the arguments to \Condor{submit\_dag} case-insensitive.

\item
Fixed a variety of bugs in DAGMan's event handling, so DAGMan should
no longer hang indefinitely after failed jobs, or mistake one job's
userlog events for those of another.

\item
DAGMan's error handling and logging output have been substantially
clarified and improved.  For example, DAGMan now prints a list of
failed jobs when it exits, rather than just saying ``some jobs
failed''.

\item
Jobs submitted by a \Condor{dagman} job now have \AdAttr{DAGManJobId}
and \AdAttr{DAGNodeName} in the job classad.

\item
Fixed a \Condor{submit\_dag} bug preventing the submission of DAGMan
Rescue files.

\item
Improved the handling of userlog errors (less crashing, more coping).

\item
Fixed a bug when recovering from the userlog after a crash or reboot.

\item
Fixed bugs in the handling of \Arg{-MaxJobs}.

\item
Added a \Arg{-a line} argument to \Condor{submit} to add a line to the
submit file before processing (overriding the submit file).

\item
Added a \Arg{-dag} flag to \Condor{q} to format and sort DAG jobs
sensibly under their DAGMan master job.

\end{itemize}

\noindent Known Bugs:

\begin{itemize}

\item \Condor{kbdd} doesn't work properly under Compaq Tru64 5.1, and
as a result, resources may not leave the ``Unclaimed'' state
regardless of keyboard or pty activity.  Compaq Tru64 5.0a and earlier
do work properly.

\end{itemize}

%%%%%%%%%%%%%%%%%%%%%%%%%%%%%%%%%%%%%%%%%%%%%%%%%%%%%%%%%%%%%%%%%%%%%%%
\section{\label{sec:History-6-2}Stable Release Series 6.2}
%%%%%%%%%%%%%%%%%%%%%%%%%%%%%%%%%%%%%%%%%%%%%%%%%%%%%%%%%%%%%%%%%%%%%%

This is the second stable release series of Condor.
All of the new features developed in the 6.1 series are now considered
stable, supported features of Condor.
New releases of 6.2.0 should happen infrequently and will only include
bug fixes and support for new platforms.
New features will be added and tested in the 6.3 development series. 
The details of each version are described below.

%%%%%%%%%%%%%%%%%%%%%%%%%%%%%%%%%%%%%%%%%%%%%%%%%%%%%%%%%%%%%%%%%%%%%%
\subsubsection{\label{sec:New-6-2-0}Version 6.2.0}
%%%%%%%%%%%%%%%%%%%%%%%%%%%%%%%%%%%%%%%%%%%%%%%%%%%%%%%%%%%%%%%%%%%%%%

\noindent New Features Over the 6.0 Release Series

\begin{itemize}

\item Support for running multiple jobs on SMP (Symmetric
Mutli-Processor) machines.

\Todo

\end{itemize}

\noindent Known Bugs:

\begin{itemize}

\item None.

\Todo

\end{itemize}


%%%%%%%%%%%%%%%%%%%%%%%%%%%%%%%%%%%%%%%%%%%%%%%%%%%%%%%%%%%%%%%%%%%%%%%
\section{\label{sec:History-6-1}Development Release Series 6.1}
%%%%%%%%%%%%%%%%%%%%%%%%%%%%%%%%%%%%%%%%%%%%%%%%%%%%%%%%%%%%%%%%%%%%%%

This was the first development release series.
It contains numerous enhancements over the 6.0 stable series.
For example:

\begin{itemize}
\item Support for running multiple jobs on SMP machines
\item Enhanced functionality for pool administrators
\item Support for PVM, MPI and Globus jobs
\item Support for \Term{Flocking} jobs across different Condor pools
\end{itemize}

The 6.1 series has many other improvements over the 6.0 series, and  
is available on more platforms.  
The new features, bugs fixed, and known bugs of each version are
described below in detail.

%%%%%%%%%%%%%%%%%%%%%%%%%%%%%%%%%%%%%%%%%%%%%%%%%%%%%%%%%%%%%%%%%%%%%%
\subsection*{\label{sec:New-6-1-17}Version 6.1.17}
%%%%%%%%%%%%%%%%%%%%%%%%%%%%%%%%%%%%%%%%%%%%%%%%%%%%%%%%%%%%%%%%%%%%%%

This version is the 6.2.0 ``release candidate''.  
It was publically released in Feburary of 2001, and it will be released
as 6.2.0 once it is considered ``stable'' by heavy testing at the 
UW-Madison Computer Science Department Condor pool.

\noindent New Features:

\begin{itemize}

\item Hostnames in the HOSTALLOW and HOSTDENY entries are now case-insensitive.

\item It is now possible to submit NT jobs from a UNIX machine.

\item The NT release of Condor now supports a USE\_VISIBLE\_DESKTOP parameter. 
If true, Condor will allow the job to create windows on the desktop of the
execute machine and interact with the job. This is particularly useful for 
debugging why an application will not run under Condor.

\item The \Condor{startd} contains support for the new MPI dedicated 
scheduler that will appear in the 6.3 development series. This will allow
you to use your 6.2 Condor pool with the new scheduler.

\item Added a \Opt{mixedcase} option to \Condor{config\_val} to allow 
for overriding the default of lowercasing all the config names

\item Added a pid\_snapshot\_interval option to the config file to
control how often the \Condor{startd} should examine the running 
process family. It defaults to 50 seconds.

\end{itemize}

\noindent Bugs Fixed:

\begin{itemize}

\item Fixed a bug with the \Condor{schedd} reaching the MAX\_JOBS\_RUNNING
mark and properly calculating Scheduler Universe jobs for preemption.

\item Fixed a bug in the \Condor{schedd} loosing track of \Condor{startd}s 
in the initial claiming phase. This bug affected all platforms, but was most
likely to manifest on Solaris 2.6

\item CPU Time can be greater than wall clock time in Multi-threaded
apps, so do not consider it an error in the UserLog.

\item \Condor{restart} \Opt{-master} now works correctly.
 
\item Fixed a rare condition in the \Condor{startd} that could corrupt
memory and result in a signal 11 (SIGSEGV, or segmentation violation).

\item Fixed a bug that would cause the ``execute event'' to not be
logged to the UserLog if the binary for the job resided on AFS.

\item Fixed a race-condition in Condor's PVM support on SMP machines
(introduced in version 6.1.16) that caused PVM tasks to be associated
with the wrong daemon.

\item Better handling of checkpointing on large-memory Linux machines.

\item Fixed random occasions of job completion email not being sent.

\item It is no longer possible to use \Condor{user\_prio} to set a priority of less
than 1.

\item Fixed a bug in the job completion email statistics.
Run Time was being underreported when the job completed after doing a
periodic checkpoint.

\item Fixed a bug that caused CondorLoadAvg to get stuck at 0.0 on
Linux when the system clock was adjusted.

\item Fixed a \Condor{submit} bug that caused all machine\_count
commands after the first queue statement to be ignored for PVM jobs.

\item PVM tasks now run as the user when appropriate instead of always
running under the UNIX ``nobody'' account.

\item Fixed support for the PVM group server.

\item PVM uses an environment variable to communicate with it's children
instead of a file in /tmp. This file previously could become overwritten
by mulitple PVM jobs.

\item \Condor{stats} now lives in the ``bin'' directory instead of ``sbin''.

\end{itemize}

\noindent Known Bugs:

\begin{itemize}

\item The \Condor{negotiator} can crash if the Accountantnew.log file becomes
corrupted. This most often occurs if the Central Manager runs out of diskspace. 

\end{itemize}

%%%%%%%%%%%%%%%%%%%%%%%%%%%%%%%%%%%%%%%%%%%%%%%%%%%%%%%%%%%%%%%%%%%%%%
\subsection*{\label{sec:New-6-1-16}Version 6.1.16}
%%%%%%%%%%%%%%%%%%%%%%%%%%%%%%%%%%%%%%%%%%%%%%%%%%%%%%%%%%%%%%%%%%%%%%

\noindent New Features:

\begin{itemize}

\item Condor now supports multiple pvmds per user on a machine.  Users
can now submit more than one PVM job at a time, PVM tasks can now run
on the submission machine, and multiple PVM tasks can run on SMP
machines.  \Condor{submit} no longer inserts default job requirements
to restrict PVM jobs to one pvmd per user on a machine.  This new
functionality requires the \Condor{pvmd} included in this (and future)
Condor releases.  If you set ``PVM\_OLD\_PVMD = True'' in the Condor
configuration file, \Condor{submit} will insert the default PVM job
requirements as it did in previous releases.  You must set this if you
don't upgrade your \Condor{pvmd} binary or if your jobs flock with pools
that user an older \Condor{pvmd}.

\item The NT release of Condor no longer contains debugging
information.
This drastically reduces the size of the binaries you must install.  

\end{itemize}

\noindent Bugs Fixed:

\begin{itemize}

\item The configuration files shipped with version 6.1.15 contained a
number of errors relating to host-based security, the configuration of
the central manager, and a few other things.
These errors have all been corrected.

\item Fixed a memory management bug in the \Condor{schedd} that could
cause it to crash under certain circumstances when machines were taken
away from the schedd's control.

\item Fixed a potential memory leak in a library used by the
\Condor{startd} and \Condor{master} that could leak memory while
Condor jobs were executing.

\item Fixed a bug in the NT version of Condor that would result in
faulty reporting of the load average.

\item The \Condor{shadow.pvm} should now correctly return core files
when a task or \Condor{pvmd} crashes.

\item This release fixes a memory error introduced in version
6.1.15 that could crash the \Condor{shadow.pvm}.

\item Some \Condor{pvmd} binaries in previous releases included
debugging code we added that could cause the \Condor{pvmd} to crash.
This release includes new \Condor{pvmd} binaries for all platforms
with the problematic debugging code removed.

\item Fixed a bug in the \Opt{-unset} options to \Condor{config\_val}
that was introduced in version 6.1.15.
Both \Opt{-unset} and \Opt{-runset} work correctly, now.

\end{itemize}

\noindent Known Bugs:

\begin{itemize}

\item None.

\end{itemize}

%%%%%%%%%%%%%%%%%%%%%%%%%%%%%%%%%%%%%%%%%%%%%%%%%%%%%%%%%%%%%%%%%%%%%%
\subsection*{\label{sec:New-6-1-15}Version 6.1.15}
%%%%%%%%%%%%%%%%%%%%%%%%%%%%%%%%%%%%%%%%%%%%%%%%%%%%%%%%%%%%%%%%%%%%%%

\noindent New Features:

\begin{itemize}

\item In the job submit description file passed to \Condor{submit}, 
a new style of macro (with two dollar-signs) can reference attributes
from the machine ClassAd.  This new style macro can be used in the
job's \MacroNI{Executable}, \MacroNI{Arguments}, or \MacroNI{Environment}
settings in the submit description file.  For example, if you have both
Linux and Solaris machines in your pool, the following submit description
file will run either foo.INTEL.LINUX or foo.SUN4u.SOLARIS27 as appropiate,
and will pass in the amount of memory available on that machine on the
command line:
\begin{verbatim}
	executable = foo.$$(Arch).$$(Opsys)
	arguments = $$(Memory)
	queue
\end{verbatim}

\item The \DCPerm{CONFIG} security access level now controls the
modification of daemon configurations using \Condor{config\_val}.  For
more information about security access levels, see
section~\ref{sec:Host-Security} on
page~\pageref{sec:Host-Security}.

\item The \Macro{DC\_DAEMON\_LIST} macro now indicates to the
\Condor{master} which processes in the \Macro{DAEMON\_LIST} use
Condor's DaemonCore inter-process communication mechanisms.  This
allows the \Condor{master} to monitor both processes developed with or
without the Condor DaemonCore library.

\item The new \Macro{NEGOTIATE\_ALL\_JOBS\_IN\_CLUSTER} macro can be
use to configure the \Condor{schedd} to not assume (for efficiency)
that if one job in a cluster can't be scheduled, then no other jobs in
the cluster can be scheduled.
If \Macro{NEGOTIATE\_ALL\_JOBS\_IN\_CLUSTER} is set to True, the
\Condor{schedd} will now always try to schedule each individual job in
a cluster.

\item The \Condor{schedd} now automatically adds any machine it is
matched with to its HOSTALLOW\_WRITE list.
This simplifies setting up a machine for flocking, since the
submitting user doesn't have to know all the machines where the job
might execute, they only have to know what central manager they wish
to flock to.
Submitting users must trust a central manager they report to, so this
doesn't impact security in any way.

\item Some static limits relating to the number of jobs which can be 
simultaneously started by the \Condor{schedd} has been removed.

\item The default Condor config file(s) which are installed by
the installation program have been re-organized for greater 
clarity and simplicity.  

\end{itemize}

\noindent Bugs Fixed:

\begin{itemize}

\item In the STANDARD Universe, jobs submitted to Condor could segfault
if they opened multiple files with the same name.  Usually this bug
was exposed when users would submit jobs without specifying a file
for either stdout or stderr; in this case, both would default to 
\File{/dev/null}, and this could trigger the problem.

\item The Linux 2.2.14 kernel, which is used by default with Red Hat 6.2,
has a serious bug can cause the machine to lock up when 
the same socket is used for repeated connection attempts.   Thus, 
previous versions of Condor could cause the 2.2.14 kernel to hang
(lots of other applications could do this as well).  The Condor Team
recommends that you upgrade your kernel to 2.2.16 or later.  However,
in v6.1.15 of Condor, a patch was added to the Condor networking
layer so that Condor would not trigger this Linux kernel bug.

\item If no email address was specified when the job was submitted
with \Condor{submit}, completion email was being sent to 
user@submit-machine-hostname.  This is not the correct behavior.  Now 
email goes by default to user@uid-domain, where uid-domain is
defined by the \MacroNI{UID\_DOMAIN} setting in the config file.

\item The \Condor{master} can now correctly shutdown and restart the
\Condor{checkpoint\_server}.

\item Email sent when a SCHEDULER Universe job compeltes now has the
correct From: header.

\item In the STANDARD universe, jobs which call sigsuspend() will 
now receive the correct return value.

\item Abnormal error conditions, such as the hard disk on the submit
machine filling up, are much less likely to result in a job disappearing
from the queue.

\item The \Condor{checkpoint\_server} now correctly reconfigures when
a \Condor{reconfig} command is received by the \Condor{master}.

\item Fixed a bug with how the \Condor{schedd} associates jobs with
machines (claimed resources) which would, under certain circumstances,
cause some jobs to remain idle until other jobs in the queue complete
or are preempted.

\item A number of PVM universe bugs are fixed in this release.
Bugs in how the \Condor{shadow.pvm} exited, which caused jobs to hang
at exit or to run multiple times, have been fixed.
The \Condor{shadow.pvm} no longer exits if there is a problem starting
up PVM on one remote host.
The \Condor{starter.pvm} now ignores the periodic checkpoint command
from the startd.  Previously, it would vacate the job when it received
the periodic checkpoint command.
A number of bugs with how the \Condor{starter.pvm} handled
asynchronous events, which caused it to take a long time to clean up
an exited PVM task, have been fixed.
The \Condor{schedd} now sets the status correctly on multi-class PVM
jobs and removes them from the job queue correctly on exit.
\Condor{submit} no longer ignores the machine\_count command for PVM
jobs.
And, a problem which caused pvm\_exit() to hang was diagnosed:
PVM tasks which call pvm\_catchout() to catch the output of
child tasks should be sure to call it again with a NULL argument to
disable output collection before calling pvm\_exit().

\item The change introduced in 6.1.13 to the \Condor{shadow} regarding
when it logged the execute event to the user log produced situations
where the shadow could log other events (like the shadow exception
event) before the execute event was logged.
Now, the \Condor{shadow} will always log an execute event before it
logs any other events.
The timing is still improved over 6.1.12 and older versions, with the
execute event getting logged after the bulk of the job initialization
has finished, right before the job will actually start executing.
However, you will no longer see user logs that contain a ``shadow
exception'' or ``job evicted'' message without a ``job executing''
event, first.

\item \Syscall{stat} and varient calls now go through the file table to
get the correct logical size and access times of buffered files.
Before, \Syscall{stat} used to return zero size on a buffered file that had
not yet been synced to disk.

\end{itemize}

\noindent Known Bugs:

\begin{itemize}

\item On IRIX 6.2, C++ programs compiled with GNU C++ (g++) 2.7.2 and
linked with the Condor libraries (using \Condor{compile}) will not
execute the constructors for any global objects.
There is a work-around for this bug, so if this is a problem for you,
please send email to \Email{condor-admin@cs.wisc.edu}.

\item In HP-UX 10.20, \Condor{compile} will not work correctly with HP's
C++ compiler. 
The jobs might link, but they will produce incorrect output, or die with
a signal such as SIGSEGV during restart after a checkpoint/vacate cycle.
However, the GNU C/C++ and the HP C compilers work just fine.

\item The \Syscall{getrusage} call does not work always as expected in
STANDARD Universe jobs.  
If your program uses \Syscall{getrusage}, it 
could decrease incorrectly by a second
across a checkpoint and restart.  In addition, the time it takes
Condor to restart from a checkpoint is included in the usage times
reported by \Syscall{getrusage}, and it probably should not be.

\end{itemize}


%%%%%%%%%%%%%%%%%%%%%%%%%%%%%%%%%%%%%%%%%%%%%%%%%%%%%%%%%%%%%%%%%%%%%%
\subsection*{\label{sec:New-6-1-14}Version 6.1.14}
%%%%%%%%%%%%%%%%%%%%%%%%%%%%%%%%%%%%%%%%%%%%%%%%%%%%%%%%%%%%%%%%%%%%%%

\noindent New Features:

\begin{itemize}

\item Initial supported added for Red Hat Linux 6.2 (i.e. glibc 2.1.3).

\end{itemize}

\noindent Bugs Fixed:

\begin{itemize}

\item In version 6.1.13, periodic checkpoints would not occur (see the
Known Bugs section for v6.1.13 listed below).  This bug, which only
impacts v6.1.13, has been fixed.

\end{itemize}

\noindent Known Bugs:

\begin{itemize}

\item The \Syscall{getrusage} call does not work properly inside
``standard'' jobs.  
If your program uses \Syscall{getrusage}, it will not report correct values
across a checkpoint and restart.
If your program relies on proper reporting from \Syscall{getrusage}, you
should either use version 6.0.3 or 6.1.10.

\item While Condor now supports many networking calls such as
\Syscall{socket} and \Syscall{connect}, (see the description below of this
new feature added in 6.1.11), on Linux, we cannot at this time support
\Syscall{gethostbyname} and a number of other database lookup calls.
The reason is that on Linux, these calls are implemented by bringing in a
shared library that defines them, based on whether the machine is using
DNS, NIS, or some other database method.
Condor does not support the way in which the C library tries to explicitly
bring in these shared libraries and use them.
There are a number of possible solutions to this problem, but the Condor
developers are not yet agreed on the best one, so this limitation might not
be resolved by 6.1.14.

\item In HP-UX 10.20, \Condor{compile} will not work correctly with HP's
C++ compiler. 
The jobs might link, but they will produce incorrect output, or die with
a signal such as SIGSEGV during restart after a checkpoint/vacate cycle.
However, the GNU C/C++ and the HP C compilers work just fine.

\item When a program linked with the Condor libraries (using \Condor{compile})
is writing output to a file, \Syscall{stat}--and variant calls,
will return zero for the size of the file if the program has not yet
read from the file or flushed the file descriptors.
This is a side effect of the file buffering code in Condor and will be
corrected to the expected semantic.

\item On IRIX 6.2, C++ programs compiled with GNU C++ (g++) 2.7.2 and
linked with the Condor libraries (using \Condor{compile}) will not
execute the constructors for any global objects.
There is a work-around for this bug, so if this is a problem for you,
please send email to \Email{condor-admin@cs.wisc.edu}.

\end{itemize}
%%%%%%%%%%%%%%%%%%%%%%%%%%%%%%%%%%%%%%%%%%%%%%%%%%%%%%%%%%%%%%%%%%%%%%
\subsection*{\label{sec:New-6-1-13}Version 6.1.13}
%%%%%%%%%%%%%%%%%%%%%%%%%%%%%%%%%%%%%%%%%%%%%%%%%%%%%%%%%%%%%%%%%%%%%%

\noindent New Features:

\begin{itemize}

\item Added \Macro{DEFAULT\_IO\_BUFFER\_SIZE} and
\Macro{DEFAULT\_IO\_BUFFER\_BLOCK\_SIZE} to config parameters to allow
the administrator to set the default file buffer sizes for user jobs
in \Condor{submit}.

\item There is no longer any difference in the configuration file
syntax between ``macros'' (which were specified with an ``='' sign)
and ``expressions'' (which were specified with a ``:'' sign).  
Now, all config file entries are treated and referenced as macros. 
You can use either ``='' or ``:'' and they will work the same way. 
There is no longer any problem with forward-referencing macros
(referencing macros you haven't yet defined), so long as they are
eventually defined in your config files (even if the forward reference
is to a macro defined in another config file, like the local config
file, for example).

\item \Condor{vacate} now supports a \Opt{-fast} option that forces
Condor to hard-kill the job(s) immediately, instead of waiting for
them to checkpoint and gracefully shutdown.

\item \Condor{userlog} now displays times in days+hours:minutes format
instead of total hours or total minutes.

\item The \Condor{run} command provides a simple front-end to
\Condor{submit} for submitting a shell command-line as a vanilla
universe job.

\item Solaris 2.7 SPARC, 2.7 INTEL have been added to the
list of ports that now support remote system calls and checkpointing.

\item Any mail being sent from Condor now shows up as having been sent from
the designated Condor Account, instead of root or ``Super User''.

\item The \Condor{submit} ``hold'' command may be used to submit jobs
to the queue in the hold state.  Held jobs will not run until released
with \Condor{release}.

\item It is now possible to use checkpoint servers in remote pools
when flocking even if the local pool doesn't use a checkpoint server.
This is now the default behavior (see the next item).

\item \Macro{USE\_CKPT\_SERVER} now defaults to True if a checkpoint
server is available.  It is usually more efficient to use a checkpoint
server near the execution site instead of storing the checkpoint back
to the submission machine, especially when flocking.

\item All Condor tools that used to expect just a hostname or address 
(\Condor{checkpoint}, \Condor{off}, \Condor{on}, \Condor{restart},
\Condor{reconfig}, \Condor{reschedule}, \Condor{vacate}) to specify
what machine to effect, can now take an optional \Opt{-name} or
\Opt{-addr} in front of each target.
This provides consistancy with other Condor tools that require the
\Opt{-name} or \Opt{-addr} options.
For all of the above mentioned tools, you can still just provide
hostnames or addresses, the new flags are not required.

\item Added \Opt{-pool} and \Opt{-addr} options to \Condor{rm},
\Condor{hold} and \Condor{release}.

\item When you start up the \Condor{master} or \Condor{schedd} as any
user other than ``root'' or ``condor'' on Unix, or ``SYSTEM'' on NT,
the daemon will have a default \Attr{Name} attribute that includes
both the username of the user who the daemon is running as and the
full hostname of the machine where it is running.

\item Clarified our Linux platform support.  We now officially
support the Red Hat 5.2 and 6.x distributions, and although other Linux
distributions (especially those with similar libc versions) may work,
they are not tested or supported.

\item The schedd now periodically updates the run-time counters in the
job queue for running jobs, so if the schedd crashes, the counters
will remain relatively up-to-date.  This is controlled by the
\Macro{WALL\_CLOCK\_CKPT\_INTERVAL} parameter.

\item The \Condor{shadow} now logs the ``job executing'' event in the
user log after the binary has been successfully transfered, so that
the events appear closer to the actual time the job starts running.
This can create some somewhat unexpected log files.  
If something goes wrong with the job's initialization, you might see
an ``evicted'' event before you see an ``executing'' event.

\end{itemize}

\noindent Bugs Fixed:

\begin{itemize}

\item Fixed how we internally handle file names for user jobs. This
fixes a nasty bug due to changing directories between checkpoints.

\item Fixed a bug in our handling of the \Macro{Arguments} macro in
the command file for a job. If the arguments were extremely long, or
there were an extreme number of them, they would get corrupted when the
job was spawned.

\item Fixed DAGMan. It had not worked at all in the previous release.

\item Fixed a nasty bug under Linux where file seeks did not work
correctly when buffering was enabled.

\item Fixed a bug where \Condor{shadow} would crash while sending job
completion e-mail forcing a job to restart multiple times and the user
to get multiple completion messages.

\item Fixed a long standing bug where Fortran 90 would occasionally
truncate its output files to random sizes and fill them with zeros.

\item Fixed a bug where \Syscall{close} did not propogate its return
value back to the user job correctly.

\item If a SIGTERM was delivered to a \Condor{shadow}, it used to
remove the job it was running from the job queue, as if \Condor{rm}
had been used.
This could have caused jobs to leave the queue unexpectedly.
Now, the \Condor{shadow} ignores SIGTERM (since the \Condor{schedd}
knows how to gracefully shutdown all the shadows when it gets a
SIGTERM), so jobs should no longer leave the queue prematurely.
In addition, on a SIGQUIT, the shadow now does a fast shutdown, just
like the rest of the Condor daemons.

\item Fixed a number of bugs which caused checkpoint restarts
to fail on some releases of Irix 6.5 (for example, when migrating from
a mips4 to a mips3 CPU or when migrating between machines with
different pagesizes).

\item Fixed a bug in the implementation of the \Syscall{stat} family
of remote system calls on Irix 6.5 which caused file opens in Fortran
programs to sometimes fail.

\item Fixed a number of problems with the statistics reported in the
job completion email and by \Condor{q} \Opt{-goodput}, including the
number of checkpoints and total network usage.  Correct values will
now be computed for all new jobs.

\item Changes in \Macro{USE\_CKPT\_SERVER} and
\Macro{CKPT\_SERVER\_HOST} no longer cause problems for jobs in the
queue which have already checkpointed.

\item Many of the Condor administration tools had a bug where they
would suffer a segmentation violation if you specified a \Opt{-pool} 
option and did not specify a hostname.
This case now results in an error message instead.

\item Fixed a bug where the \Condor{schedd} could die with a
segmentation violation if there was an error mapping an IP address
into a hostname.

\item Fixed a bug where resetting the time in a large negative direction
caused the \Condor{negotiator} to have a floating point error on some
platforms.

\item Fixed \Condor{q}'s output so that certain arguments are not ignored.

\item Fixed a bug in \Condor{q} where issuing a \Opt{-global} with a
fairly restrictive \Opt{-constraint} argument would cause garbage to be
printed to the terminal sometimes.

\item Fixed a bug which caused jobs to exit without completing a
checkpoint when preempted in the middle of a periodic checkpoint.
Now, the jobs will complete their periodic checkpoint in this case
before exiting.
\end{itemize}

\noindent Known Bugs:

\begin{itemize}

\item Periodic checkpoints do not occur.  Normally, when the config
file attribute \Macro{PERIODIC\_CHECKPOINT} evaluates to True, 
Condor performs a periodic checkpoint of the running job.  This
bug has been fixed in v6.1.14.  \Note there is a work-around to permit
periodic checkpoints to occur in v6.1.13: include the attribute name
``PERIODIC\_CHECKPOINT'' to the attributes 
listed in the \Macro{STARTD\_EXPRS} entry in the config file.

\item The \Syscall{getrusage} call does not work properly inside
``standard'' jobs.  
If your program uses \Syscall{getrusage}, it will not report correct values
across a checkpoint and restart.
If your program relies on proper reporting from \Syscall{getrusage}, you
should either use version 6.0.3 or 6.1.10.

\item While Condor now supports many networking calls such as
\Syscall{socket} and \Syscall{connect}, (see the description below of this
new feature added in 6.1.11), on Linux, we cannot at this time support
\Syscall{gethostbyname} and a number of other database lookup calls.
The reason is that on Linux, these calls are implemented by bringing in a
shared library that defines them, based on whether the machine is using
DNS, NIS, or some other database method.
Condor does not support the way in which the C library tries to explicitly
bring in these shared libraries and use them.
There are a number of possible solutions to this problem, but the Condor
developers are not yet agreed on the best one, so this limitation might not
be resolved by 6.1.14.

\item In HP-UX 10.20, \Condor{compile} will not work correctly with HP's
C++ compiler. 
The jobs might link, but they will produce incorrect output, or die with
a signal such as SIGSEGV during restart after a checkpoint/vacate cycle.
However, the GNU C/C++ and the HP C compilers work just fine.

\item When writing output to a file, \Syscall{stat}--and variant calls,
will return zero for the size of the file if the program has not yet
read from the file or flushed the file descriptors,
This is a side effect of the file buffering code in Condor and will be
corrected to the expected semantic.

\item On IRIX 6.2, C++ programs compiled with GNU C++ (g++) 2.7.2 and
linked with the Condor libraries (using \Condor{compile}) will not
execute the constructors for any global objects.
There is a work-around for this bug, so if this is a problem for you,
please send email to \Email{condor-admin@cs.wisc.edu}.

\end{itemize}

%%%%%%%%%%%%%%%%%%%%%%%%%%%%%%%%%%%%%%%%%%%%%%%%%%%%%%%%%%%%%%%%%%%%%%
\subsection*{\label{sec:New-6-1-12}Version 6.1.12}
%%%%%%%%%%%%%%%%%%%%%%%%%%%%%%%%%%%%%%%%%%%%%%%%%%%%%%%%%%%%%%%%%%%%%%

Version 6.1.12 fixes a number of bugs from version 6.1.11.
If you linked your ``standard'' jobs with version 6.1.11, you should
upgrade to 6.1.12 and re-link your jobs (using \Condor{compile}) as soon as
possible.

\noindent New Features:

\begin{itemize}

\item None.

\end{itemize}

\noindent Bugs Fixed:

\begin{itemize}

\item A number of system calls that were not being trapped by the Condor
libraries in version 6.1.11 are now being caught and sent back to the
submit machine.
Not having these functions being executed as remote system calls prevented
a number of programs from working, in particular Fortran programs, and
many programs on IRIX and Solaris platforms.

\item Sometimes submitted jobs report back as having no owner and have
\Bold{-????-} in the status line for the job. This has been fixed.

\item \Condor{q} \Opt{-io} has been fixed in this release.

\end{itemize}

\noindent Known Bugs:

\begin{itemize}

\item The \Syscall{getrusage} call does not work properly inside
``standard'' jobs.  
If your program uses \Syscall{getrusage}, it will not report correct values
across a checkpoint and restart.
If your program relies on proper reporting from \Syscall{getrusage}, you
should either use version 6.0.3 or 6.1.10.

\item While Condor now supports many networking calls such as
\Syscall{socket} and \Syscall{connect}, (see the description below of this
new feature added in 6.1.11), on Linux, we cannot at this time support
\Syscall{gethostbyname} and a number of other database lookup calls.
The reason is that on Linux, these calls are implemented by bringing in a
shared library that defines them, based on whether the machine is using
DNS, NIS, or some other database method.
Condor does not support the way in which the C library tries to explicitly
bring in these shared libraries and use them.
There are a number of possible solutions to this problem, but the Condor
developers are not yet agreed on the best one, so this limitation might not
be resolved by 6.1.13.

\item In HP-UX 10.20, \Condor{compile} will not work correctly with HP's
C++ compiler. 
The jobs might link, but they will produce incorrect output, or die with
a signal such as SIGSEGV during restart after a checkpoint/vacate cycle.
However, the GNU C/C++ and the HP C compilers work just fine.

\item When writing output to a file, \Syscall{stat}--and variant calls,
will return zero for the size of the file if the program has not yet
read from the file or flushed the file descriptors,
This is a side effect of the file buffering code in Condor and will be
corrected to the expected semantic.

\item On IRIX 6.2, C++ programs compiled with GNU C++ (g++) 2.7.2 and
linked with the Condor libraries (using \Condor{compile}) will not
execute the constructors for any global objects.
There is a work-around for this bug, so if this is a problem for you,
please send email to \Email{condor-admin@cs.wisc.edu}.

\item The \Opt{-format} option in \Condor{q} has no effect when querying
remote machines with the \Opt{-n} option.

\item \Condor{dagman} does not work at all in this release. 
The behaviour of its failure is to exit immediately with a success and
to not perform any work. It will be fixed in the next release of Condor.

\end{itemize}


%%%%%%%%%%%%%%%%%%%%%%%%%%%%%%%%%%%%%%%%%%%%%%%%%%%%%%%%%%%%%%%%%%%%%%
\subsection*{\label{sec:New-6-1-11}Version 6.1.11}
%%%%%%%%%%%%%%%%%%%%%%%%%%%%%%%%%%%%%%%%%%%%%%%%%%%%%%%%%%%%%%%%%%%%%%

\noindent New Features:

\begin{itemize}

\item \Condor{status} outputs information for held jobs instead of
MaxRunningJobs when supplied with \Opt{-schedd} or \Opt{-submitter}.

\item \Condor{userprio} now prints 4 digit years (for Y2K compiance). 
If you give a two digit date, it also will assume that 1/1/00 is 1/1/2000
and not 1/1/1900.

\item IRIX 6.5 has been added to the list of ports that now support
remote system calls and checkpointing.

\item \Condor{q} has been fixed to be faster and much more memory
efficient.  This is much more obvious when getting the queue from
\Condor{schedd}'s that have more than 1000 jobs.

\item Added support for support for socket() and pipe() in standard
jobs.  Both sockets and pipes are created on the executing machine.
Checkpointing is deferred anytime a socket or pipe is open.

\item Added limited support for select() and poll() in standard jobs.
Both calls will work only on files opened locally.

\item Added limited support for fcntl() and ioctl() in standard jobs.
Both calls will be performed remotely if the control-number is understood
and the third argument is an integer.

\item Replaced buffer implementation in standard jobs.
The new buffer code reads and writes variable sized chunks.
It will never issue a read to satisfy a write.  Buffering is enabled
by default.

\item Added extensive feedback on I/O performance in the user's email.

\item Added \Opt{-io} option to \Condor{q} to show I/O statistics.

\item Removed libckpt.a and libzckpt.a.  To build for standalone
checkpointing, just do a regular \Condor{compile}.
No -standalone option is necessary.

\item The checkpointing library now only re-opens files when they are
actually used.  If files or other needed resources cannot be found
at restart time, the checkpointer will fail with a verbose error.

\item The \Attr{RemoteHost} and \Attr{LastRemoteHost} attributes in
the job classad now contain hostnames instead IP address and port
numbers.  The \Opt{-run} option of older versions of \Condor{q} is not
compatible with this change.

\item Condor will now automatically check for compatibility between
the version of the Condor libraries you have linked into a standard
job (using \Condor{compile}) and the version of the \Condor{shadow}
installed on your submit machine.
If they are incompatible, the \Condor{shadow} will now put your job on
hold.  
Unless you set ``Notification = Never'' in your submit file, Condor
will also send you email explaining what went wrong and what you can
do about it.

\item All Condor daemons and tools now have a \Attr{CondorPlatform}
string, which shows which platform a given set of Condor binaries was
built for.
In all places that you used to see \Attr{CondorVersion}, you will now
see both \Attr{CondorVersion} and \Attr{CondorPlatform}, such as in
each daemon's ClassAd, in the output to a \Opt{-version} option (if
supported), and when running \Prog{ident} on a given Condor binary. 
This string can help identify situations where you are running the 
wrong version of the Condor binaries for a given platform (for
example, running binaries built for Solaris 2.5.1 on a Solaris 2.6
machine).   

\item Added commented-out settings in the default
\File{condor\_config} file we ship for various SMP-specific settings
in the \Condor{startd}.
Be sure to read section~\ref{sec:Configuring-SMP} on ``Configuring the
Startd for SMP Machine'' on page~\pageref{sec:Configuring-SMP} for
details about using these settings. 

\item \Condor{rm}, \Condor{hold}, and \Condor{release} all support
\Opt{-help} and \Opt{-version} options now.

\end{itemize}

\noindent Bugs Fixed:

\begin{itemize}

\item A race condition which could cause the \Condor{shadow} to not
exit when its job was removed has been fixed.
This bug would cause jobs that had been removed with \Condor{rm} to
remain in the queue marked as status ``X'' for a long time.
In addition, Condor would not shutdown quickly on hosts that had hit
this race condition, since the \Condor{schedd} wouldn't exit until all
of its \Condor{shadow} children had exited.

\item A signal race condition during restart of a Condor job has
been fixed.

\item In a Condor linked job, \Syscall{getdomainname} is now
supported. 

\item IRIX 6.5 can give negative time reports for how long a process has been
running. We account for that now in our statistics about usage times.

\item The \Condor{status} memory error introduced in version 6.1.10
has been fixed.

\item The \Macro{DAEMON\_LIST} configuration setting is now case
insensitive.

\item Fixed a bug where the \Condor{schedd}, under rare circumstances,
cause another schedd's jobs not to be matched.

\item The free disk space is now properly computed on Digital Unix.
This fixed problems where the \Attr{Disk} attribute in the
\condor{startd} classad reported incorrect values.

\item The config file parser now detects incremental macro definitions
correctly (see section~\ref{sec:Config-File-Macros} on
page~\pageref{sec:Config-File-Macros}).  Previously, when a macro (or
expression) being defined was a substring of a macro (or expression)
being referenced in its definition, the reference would be erroneously
marked as an incremental definition and expanded immediately.  The
parser now verifies that the entire strings match.

\end{itemize}

\noindent Known Bugs:

\begin{itemize}

\item The output for \condor{q} \Opt{-io} is incorrect and will likely show
zeroes for all values.  A fixed version will appear in the next release.

\end{itemize}

%%%%%%%%%%%%%%%%%%%%%%%%%%%%%%%%%%%%%%%%%%%%%%%%%%%%%%%%%%%%%%%%%%%%%%
\subsection*{\label{sec:New-6-1-10}Version 6.1.10}
%%%%%%%%%%%%%%%%%%%%%%%%%%%%%%%%%%%%%%%%%%%%%%%%%%%%%%%%%%%%%%%%%%%%%%

\noindent New Features:

\begin{itemize}

\item \Condor{q} now accepts \texttt{-format} parameters like \Condor{status}

\item \Condor{rm}, \Condor{hold} and \Condor{release} accept
  \texttt{-constraint} parameters like \Condor{status}

\item \Condor{status} now sorts displayed totals by the first column.
(This feature introduced a bug in \Condor{status}.  See ``Known Bugs''
below.)

\item Condor version 6.1.10 introduces ``clipped'' support for Sparc
Solaris version 2.7.
This version does not support checkpointing or remote system calls.
Full support for Solaris 2.7 will be released soon.

\item Introduced code to enable Linux to use the standard C library's
I/O buffering again, instead of relying on the Condor I/O buffering
code (which is still in beta testing).  

\end{itemize}

\noindent Bugs Fixed:

\begin{itemize}

\item The bug in checkpointing introduced in version 6.1.9 has been
fixed.
Checkpointing will now work on all platforms, as it always used to.  
Any jobs linked with the 6.1.9 Condor libraries will need to be
relinked with \Condor{compile} once version 6.1.10 has been installed
at your site. 

\end{itemize}

\noindent Known Bugs:

\begin{itemize}

\item The \AdAttr{CondorLoadAvg} attribute in the \Condor{startd} has
some problems in the way it is computed.
The CondorLoadAvg is somewhat inaccurate for the first minute a job
starts running, and for the first minute after it completes.
Also, the computation of CondorLoadAvg is very wrong on NT.
All of this will be fixed in a future version.

\item A memory error may cause \Condor{status} to die with SIGSEGV
(segmentation violation) when displaying totals or cause incorrect
totals to be displayed.  This will be fixed in version 6.1.11.

\end{itemize}


%%%%%%%%%%%%%%%%%%%%%%%%%%%%%%%%%%%%%%%%%%%%%%%%%%%%%%%%%%%%%%%%%%%%%%
\subsection*{\label{sec:New-6-1-9}Version 6.1.9}
%%%%%%%%%%%%%%%%%%%%%%%%%%%%%%%%%%%%%%%%%%%%%%%%%%%%%%%%%%%%%%%%%%%%%%

\noindent New Features:

\begin{itemize}

\item Added full support for Linux 2.0.x and 2.2.x kernels using
libc5, glibc20 and glibc21.
This includes support for Red Hat 6.x, Debian 2.x and other popular
Linux distributions.
Whereas the Linux machines had once been fragmented across libc5 and
GNU libc, they have now been reunified.
This means there is no longer any need for the ``LINUX-GLIBC'' OpSys
setting in your pool: all machines will now show up as ``LINUX''.
Part of this reunification process was the removal of dynamically
linked user jobs on Linux.
\Condor{compile} now forces static linking of your Standard Universe
Condor jobs. 
Also, please use \Condor{compile} on the same machine on which you
compiled your object files.

\item Added \Condor{qedit} utility to allow users to modify job
attributes after submission.  See the new manual page on
page~\pageref{man-condor-qedit}.

\item Added \OptArg{{-runfor}{minutes}} option to daemonCore to have
the daemon gracefully shut down after the given number of minutes.

\item Added support for statfs(2) and fstatfs(2) in user jobs. We support 
only the fields
\textit{f\_bsize, f\_blocks, f\_bfree, f\_bavail, f\_files, f\_ffree} from
the structure statfs. This is still in the experimental stage.

\item Added the \Opt{-direct} option to \Condor{status}.
After you give \Opt{-direct}, you supply a hostname, and
\Condor{status} will query the \Condor{startd} on the specified host
and display information directly from there, instead of querying the
\Condor{collector}.
See the manual page on page~\pageref{man-condor-submit} for details. 

\item Users can now define \Macro{NUM\_CPUS} to override the automatic
computation of the number of CPUs in your machine.
Using this config setting can cause unexpected results, and is not
recommended. 
This feature is only provided for sites that specifically want this
behavior and know what they are doing.

\item The \Opt{-set} and \Opt{-rset} options to \Condor{config\_val}
have been changed to allow administrators to set both macros and
expressions.
Previously, \Condor{config\_val} assumed you wanted to set
expressions.
Now, these two options each take a single argument, the string
containing exactly what you would put into the config file, so you can
specify you want to create a macro by including an ``='' sign, or an
expression by including a ``:''.
See section~\ref{sec:Intro-to-Config-Files} on
page~\pageref{sec:Intro-to-Config-Files} for details on macros
vs. expressions.
See the \Condor{config\_val} man page on
page~\pageref{man-condor-config-val} for details on
\Condor{config\_val}.  

\item If the directory you specified for LOCK (which holds lock files
used by Condor) doesn't exist, Condor will now try to create that
directory for you instead of giving up right away.

\item If you change the \Attr{COLLECTOR\_HOST} setting and reconfig
the \Condor{startd}, the startd will ``invalidate'' its ClassAds at
the old collector before it starts reporting to the new one.

\end{itemize}

\noindent Bugs Fixed:

\begin{itemize}

\item Fixed a major bug dealing with the group access a Condor job is
started with.
Now, Condor jobs are started with all the groups the job's owner is
in, not just their default group.
This also fixes a security hole where user jobs could be started up in
access groups they didn't belong to.

\item Fixed a bug where there was a needless limitation on the number of open
file descriptors a user job could have.

\item Fixed a standalone checkpointing bug where we weren't blocking signals
in critical sections and causing file table corruption at checkpoint
time.

\item Fixed a linker bug on Digital Unix 4.0 concerning fortran where
the linker would fail on \_\_uname and \_\_sigsuspend.

\item Fixed a bug in \Condor{shadow} that would send incorrect job
completion email under Linux.

\item Fixed a bug in the remote system call of \Syscall{fchdir} that caused
a garbage file descriptor to be used in Standard Universe jobs.

\item Fixed a bug in the \Condor{shadow} which was causing \Condor{q}
\Opt{-goodput} to display incorrect values for some jobs.

\item Fixed some minor bugs and made some minor enhancements in the
\Condor{install} script.
The bugs included a typo in one of the questions asked, and incorrect
handling for the answers of a few different questions.
Also, if DNS is misconfigured on your system, \Condor{install} will
try a few ways to find your fully qualified hostname, and if it still
can't determine the correct hostname, it will prompt the user for it. 
In addition, we now avoid one installation step in cases were it is
not needed. 

\item Fixed a rare race condition that could delay the completion of
large clusters of short running jobs. 

\item Added more checking to the various arguments that might be
passed to \Condor{status}, so that in the case of bad input,
\Condor{status} will print an error message and exit, instead of
performing a segmentation fault.
Also, when you use the \Opt{-sort} option, \Condor{status} will only
display ClassAds where the attributes you use to sort are defined.

\item Fixed a bug in the handling of the config files created by
using the \Opt{-set} or \Opt{-rset} options to \Condor{config\_val}.
Previously, if you manually deleted the files that were created, you
could cause the affected Condor daemon to have a segmentation fault.
Now, the daemons simply exit with a fatal error but still have a
chance to clean up.

\item Fixed a bug in the \Opt{-negotiator} option for most Condor
tools that was causing it to get the wrong address.

\item Fixed a couple of bugs in the \Condor{master} that could cause
improper shutdowns. 
There were cases during shutdown where we would restart a daemon
(because we previously noticed a new executable, for example).
Now, once you begin a shutdown, the \Condor{master} will not restart
anything. 
Also, fixed a rare bug that could cause the \Condor{master} to stop
checking the timestamps on a daemon.

\item Fixed a minor bug in the \Opt{-owner} option to
\Condor{config\_val} that was causing \Condor{init} not to work.

\item Fixed a bug where the \Condor{startd}, while it was already
shutting down, was allowing certain actions to succeed that should
have failed.
For example, it allowed itself to be matched with a user looking for
available machines, or to begin a new PVM task.

\end{itemize}

\noindent Known Bugs:

\begin{itemize}

\item The \AdAttr{CondorLoadAvg} attribute in the \Condor{startd} has
some problems in the way it is computed.
The CondorLoadAvg is somewhat inaccurate for the first minute a job
starts running, and for the first minute after it completes.
Also, the computation of CondorLoadAvg is very wrong on NT.
All of this will be fixed in a future version.

\item There is a serious bug in checkpointing when using Condor's
I/O buffering for ``standard'' jobs.
By default, Linux uses Condor buffering in version 6.1.9 for all
standard jobs.
The bug prevents checkpointing from working more than once.
This renders the \Condor{vacate} and \Condor{checkpoint} commands
useless, and jobs will just be killed without a checkpoint when
machine owners come back to their machines.

\end{itemize}


%%%%%%%%%%%%%%%%%%%%%%%%%%%%%%%%%%%%%%%%%%%%%%%%%%%%%%%%%%%%%%%%%%%%%%
\subsection*{\label{sec:New-6-1-8}Version 6.1.8}
%%%%%%%%%%%%%%%%%%%%%%%%%%%%%%%%%%%%%%%%%%%%%%%%%%%%%%%%%%%%%%%%%%%%%%

\begin{itemize}

\item Added \Term{file\_remaps} as command in the job submit file given to
STANDARD universe jobs.
A Job can now specify that it would like to have files be remapped
from one file to another.
In addition you can specify that files should be read from the local machine
by specifing them.
See the \Condor{submit} manual page on page~\pageref{man-condor-submit} for
more details.

\item Added \Term{buffer\_size} and \Term{buffer\_block\_size} so that STANDARD
universe jobs can specify that they wish to have I/O buffering turned on.
Without buffering, all I/O requests in the STANDARD universe are sent back
over the network to be executed on the submit machine.  
With buffering, read ahead, write behind, and seek batch buffering is
performed to minimize network traffic and latency.
By default, jobs do not specify buffering, however, for many situations buffering
can drastically increase throughput.  See the \Condor{submit} manual page
on page~\pageref{man-condor-submit} for more details.

\item The \Condor{schedd} is much more memory efficient handling clusters
with hundreds/thousands of jobs.  
If you submit large clusters, your submit machine will only use a fraction
of the amount of RAM it used to require.  
\Note The memory savings will only be realized for new clusters submitted
after the upgrade to v6.1.8 -- clusters which previously existed in the
queue at upgrade time will still use the same amount of RAM in the
\Condor{schedd}.

\item Submitting jobs, especially submitting large clusters containing many
jobs, is much faster.

\item Added a \Opt{-goodput} option to \Condor{q}, which displays
statistics about the execution efficiency of STANDARD universe jobs.

\item Added FS\_REMOTE method of user authentication to possible values
of the configuration option \Macro{AUTHENTICATION\_METHODS} to fix problems
with using the \Opt{-r} remote scheduler option of \Condor{submit}.
Additionally, the user authentication protocol has changed, so previous
versions of Condor programs cannot co-exist with this new protocol.

\item Added a new utility and documentation for \Condor{glidein} which uses 
Globus resources to extend your local pool to use remote Globus machines as 
part of your Condor pool.

\item Fixed more bugs in the handling of the stat() system call
and its relatives on Linux with glibc.
This was causing problems mainly with Fortran I/O, though other I/O
related problems on glibc Linux will probably be solved now.

\item Fixed a bug in various Condor tools (\Condor{status},
\Condor{user\_prio}, \Condor{config\_val}, and \Condor{stats}) that
would cause them to seg fault on bad input to the \Opt{-pool} option. 

\item Fixed a bug with the \Opt{-rset} option to \Condor{config\_val} which
could crash the Condor daemon whose configuration was being changed.

\item Added \Term{allow\_startup\_script} command to the job submit
description file which is given to \Condor{submit}.  This allows the
submission of a startup script to the STANDARD universe.  See 

\item Fixed a bug in the \Condor{schedd} where it would get into an
infinite loop if the persistant log of the job queue got corrupted.  
The \Condor{schedd} now correctly handles corrupted log files.

\item The full release tar file now contains a \File{dagman}
subdirectory in the \File{examples} directory.
This subdirectory includes an example DAGMan job, including a README
(in both ASCII and HTML), a Makefile, and so on.

\item Condor will now insert an environment variable, \Env{CONDOR\_VM}, into
the environment of the user job.  
This variable specifies which SMP ``virtual machine'' the job was started on.
It will equal either vm1, vm2, vm3, \Dots , depending upon which virtual
machine was matched.
On a non-SMP machine, \Env{CONDOR\_VM} will always be set to vm1.

\item Fixed some timing bugs introduced in v6.1.6 which could occur when
Condor tries to simultaneously start a large number of jobs submitted from a
single machine.

\item Fixed bugs when Condor is told to gracefully shutdown; Condor no
longer starts up new jobs when shutting down.  Also, the \Condor{schedd}
progressively checkpoints running jobs during a graceful shutdown instead of
trying to vacate all the job simultaneously.  The rate at which the shutdown
occurs is controlled by the \Macro{JOB\_START\_DELAY} configuration
parameter (see page~\pageref{param:JobStartDelay}).

\item Fixed a bug which could cause the \Condor{master} process to exit if
the Condor daemons have been hung for a while by the operating system (if,
for instance, the LOG directory was placed on an NFS volume and the NFS
server is down for an extended period).

\item Previously, removing a large number of jobs with \Condor{rm} would
result in the \Condor{schedd} being unresponsive for a period of time
(perhaps leading to timeouts when running \Condor{q}).  The \Condor{schedd}
has been improved to multitask the removal of jobs while servicing new
requests.

\item Added new configuration parameter \Macro{COLLECTOR\_SOCKET\_BUFSIZE}
which controls the size of TCP/IP buffers used by the \Condor{collector}.
For more info, see section~ref{param:CollectorSocketBufsize} on
page~pageref{param:CollectorSocketBufsize}.

\item Fixed a bug with the \Opt{-analyze} option to \Condor{q}: in some
cases, the RANK expression would not be evaluated correctly.  This could
cause the output from \Opt{-analyze} to be in error.

\item When running on a multi-CPU (SMP) Hewlett-Packard machine, fixed bugs
computing the system load average.

\item Fixed bug in \Condor{q} which could cause the RUN\_TIME reported to
be temporarily incorrect when jobs first start running. 

\item The \Condor{startd} no longer rapidly sends multiple ClassAds one
right after another to the Central Manager when its state/activity is in
rapid transition.  Also, on SMP machines, the \Condor{startd} will only send
updates for 4 nodes per second (to avoid overflowing the central manager when
reporting the state of a very large SMP machine with dozens of CPUs).

\item Reading a parameter with \Condor{config\_val} is now allowed from any
machine with Host-IP READ permission.
Previsouly, you needed ADMINISTRATOR permission.  
Of course, setting a parameter still requires ADMINISTRATOR permission.

\item Worked around a bug in the StreamTokenizer Java class from Sun
that we use in the CondorView client Java applet.
The bug would cause errors if usernames or hostnames in your pool
contained ``-'' or ``\_'' characters.
The CondorView applet now gets around this and properly displays all
data, including entries with the ``bad'' characters.

\end{itemize}

%%%%%%%%%%%%%%%%%%%%%%%%%%%%%%%%%%%%%%%%%%%%%%%%%%%%%%%%%%%%%%%%%%%%%%
\subsection*{\label{sec:New-6-1-7}Version 6.1.7}
%%%%%%%%%%%%%%%%%%%%%%%%%%%%%%%%%%%%%%%%%%%%%%%%%%%%%%%%%%%%%%%%%%%%%%

\Note Version 6.1.7 only adds support for platforms not supported in
6.1.6.  
There are no bug fixes, so there are no binaries released for any
other platforms. 
You do not need 6.1.7 unless you are using one of the two platforms we
released binaries for.

\begin{itemize}

\item Added ``clipped'' support for Alpha Linux machines running the
2.0.X kernel and glibc 2.0.X (such as Red Hat 5.X).
We do not yet support checkpointing and remote system calls on this
platform, but we can start ``vanilla'' jobs.
See section~\ref{sec:Choosing-Universe} on
page~\pageref{sec:Choosing-Universe} for details on vanilla
vs. standard jobs.

\item Re-added support for Intel Linux machines running the 2.0.X
Linux kernel, glibc 2.0.X, using the GNU C compiler (gcc/g++ 2.7.X) or
the EGCS compilers (versions 1.0.X, 1.1.1 and 1.1.2).
This includes Red Hat 5.X, and Debian 2.0.
\Bold{Red Hat 6.0 and Debian 2.1 are not yet supported, since they use
glibc 2.1.X and the 2.2.X Linux kernel.}
Future versions of Condor will support all combinations of kernels,
compilers and versions of libc.

\end{itemize}


%%%%%%%%%%%%%%%%%%%%%%%%%%%%%%%%%%%%%%%%%%%%%%%%%%%%%%%%%%%%%%%%%%%%%%
\subsection*{\label{sec:New-6-1-6}Version 6.1.6}
%%%%%%%%%%%%%%%%%%%%%%%%%%%%%%%%%%%%%%%%%%%%%%%%%%%%%%%%%%%%%%%%%%%%%%

\begin{itemize}

\item Added \Term{file\_remaps} as command in the job submit file given to
\Condor{submit}.
This allows the user to explicitly specify where to find a given file (e.g.
either on the submit or execute machine), as well as remap file access to a
different filename altogether.

\item Changed the way that \Condor{master} spawns daemons and
\Condor{preen} which allows you to specify command line arguments for
any of them, though a \Macro{SUBSYS\_ARGS} setting.
Previously, when you specified \Macro{PREEN}, you added the command
line arguments directly to that setting, but that caused some
problems, and only worked for \Condor{preen}.
\Bold{Once you upgrade to version 6.1.6, if you continue to use your
old \File{condor\_config} files, you must change the \Macro{PREEN}
setting to remove any arguments you have defined and place those
arguments into a separate config setting, \Macro{PREEN\_ARGS}.}
See section~\ref{sec:Master-Config-File-Entries}, ``\condor{master}
Config File Entries'', on
page~\pageref{sec:Master-Config-File-Entries} for more details.

\item Fixed a very serious bug in the Condor library linked in with
\Condor{compile} to create standard jobs that was causing
checkpointing to fail in many cases.  
Any jobs that were linked with the 6.1.5 Condor libraries should
probably be removed, re-linked, and re-submitted. 

\item Fixed a bug in \Condor{userprio} that was introduced in version
6.1.5 that was preventing it from finding the address of the
\Condor{negotiator} for your pool.

\item Fixed a bug in \Condor{stats} that was introduced in version
6.1.5 that was preventing it from finding the address of the
\Condor{collector} for your pool.

\item Fixed a bug in the way the \Opt{-pool} option was handled by
many Condor tools that was introduced in version 6.1.5. 


\item \Condor{q} now displays job \emph{allocation time} by default, instead
of displaying CPU time.  
Job allocation time, or RUN\_TIME, is the amount of wall-clock time the job
has spent running.  
Unlike CPU time information which is only updated when a job is
checkpointed, the allocation time displayed by \Condor{q} is continuously
updated, even for vanilla universe jobs.  
By default, the allocation time displayed will be the total time across all
runs of the job.  
The new \Opt{-currentrun} option to \Condor{q} can be used to display the
allocation time for solely the current run of the job.
Additionally, the \Opt{-cputime} option can be used to view job CPU times as
in earlier versions of Condor.

\item \Condor{q} will display an error message if there is a timeout
fetching the job queue listing from a \condor{schedd}.  Previously,
\Condor{q} would simply list the queue as empty upon a communication error.

\item The \condor{schedd} daemon has been updated to verify all queue access
requests via Condor's IP/Host-Based Security mechanism (see
section~\ref{sec:Host-Security}).

\item Fixed a bug on platforms which require the \Condor{kbdd} (currently
Digital Unix and IRIX).  
This bug could have allowed Condor to start a job within the first five
minutes after the Condor daemons had been started, even if there is a user
typing on the keyboard.

\item \Condor{release} now gives an error message if the user tries to
release a job which either does not exist or is not in the hold state.

\item Added a new config file parameter, \Macro{USER\_JOB\_WRAPPER}, which
allows administrators to specify a file to act as a ``wrapper'' script
around all jobs started by Condor. 
See inside section~\ref{param:UserJobWrapper}, on 
page~\pageref{sec:Starter-Config-File-Entries}, for more details.

\item \Condor{dagman} now permits the backslash character (``\Bs'') to be used
as a line-continuation character for DAG Input Files, just like the
\condor{config} files.

\item The Condor version string is now included in all Condor
libraries.
You can now run \Prog{ident} on any program linked with
\Condor{compile} to view which version of the Condor libraries you
linked with.
In addition, the format of the version string changed in 6.1.6.
Now, the identifier used is ``CondorVersion'' instead of ``Version''
to prevent any potential ambiguity.
Also, the format of the date changed slightly.

\item The SMP startd can now handle dynamic reconfiguration of the
number of each type of virtual machine being reported.
This allows you, during the normal running of the startd, to increase
or decrease the number of CPUs that Condor is using.
If you reconfigure the startd to use less CPUs than it currently has
under its control, it will first remove CPUs that have no Condor jobs
running on them.
If more CPUs need to be evicted, the startd will checkpoint jobs and
evict them in reverse rank order (using the startd's \Macro{Rank}
expression).
So, the lower the value of the rank, the more likely a job will be
kicked off.

\item The SMP startd contrib module's \Condor{starter} no longer makes
a call that was causing warning messages about ``ERROR: Unknown System
Call (-58) - system call not supported by Condor'' when used with the
6.0.X \Condor{shadow}.
This was a harmless call, but removing the call prevents the error
message.

\item The SMP contrib module now includes the \Condor{checkpoint} and
\Condor{vacate} programs, which allow you to vacate or checkpoint jobs
on individual CPUs on the SMP, instead of checkpointing or vacating
everything.  
You can now use ``\condor{vacate} vm1@hostname'' to just vacate the
first virtual machine, or ``\condor{vacate} hostname'' to vacate all
virtual machines. 

\item Added support for SMP Digital Unix (Alpha) machines.

\item Fixed a bug that was causing an overflow in the computation of
free disk and swap space on Digital Unix (Alpha) machines.

\item The \Condor{startd} and \Condor{schedd} now can ``invalidate''
their classads from the collector.
So, when a daemon is shut down, or a machine is reconfigured to 
advertise fewer virtual machines, those changes will be instantly
visible with \Condor{status}, instead of having to wait 15 minutes for
the stale classads to time-out.

\item The \Condor{schedd} no longer forks a child process (a ``schedd
agent'') to claim available \Condor{startd}s.  
You should no longer see multiple \condor{schedd} processes running on
your machine after a negotiation cycle.
This is now accomplished in a non-blocking manner within the
\Condor{schedd} itself.

\item The startd now adds an \Attr{VirtualMachineID} attribute to
each virtual machine classad it advertises.
This is just an integer, starting at 1, and increasing for every
different virtual machine the startd is representing.
On regular hosts, this is the only ID you will ever see.
On SMP hosts, you will see the ID climb up to the number of different
virtual machines reported.
This ID can be used to help write more complex policy expressions on
SMP hosts, and to easily identify which hosts in your pool are in fact
SMP machines.

\item Modified the output for \Condor{q} -run for scheduler and PVM
universe jobs.  The host where the scheduler universe job is running
is now displayed correctly.  For PVM jobs, a count of the current
number of hosts where the job is running is displayed.

\item Fixed the \Condor{startd} so that it no longer prints lots of
ProcAPI errors to the log file when it is being run as non-root.

\item \Macro{FS\_PATHNAME} and \Macro{VOS\_PATHNAME} are no longer
used.  AFS support now works similar to NFS support, via the
\Macro{FILESYSTEM\_DOMAIN} macro.

\item Fixed a minor bug in the \File{Condor.pm} perl module that was
causing it to be case-sensitive when parsing the Condor submit file.
Now, the perl module is properly case-insensitive, as indicated in the
documentation.

\end{itemize}

%%%%%%%%%%%%%%%%%%%%%%%%%%%%%%%%%%%%%%%%%%%%%%%%%%%%%%%%%%%%%%%%%%%%%%
\subsection*{\label{sec:New-6-1-5}Version 6.1.5}
%%%%%%%%%%%%%%%%%%%%%%%%%%%%%%%%%%%%%%%%%%%%%%%%%%%%%%%%%%%%%%%%%%%%%%

\begin{itemize}

\item Fixed a nasty bug in \Condor{preen} that would cause it to
remove files it shouldn't remove if the \Condor{schedd} and/or
\Condor{startd} were down at the time \Condor{preen} ran.
This was causing jobs to mysteriously disappear from the job queue.

\item Added preliminary support to Condor for running on machines with
multiple network interfaces.
On such machines, users can specify the IP address Condor should use
in the \Macro{NETWORK\_INTERFACE} config file parameter on each host. 
In addition, if the pool's central manager is on such a machine, users
should set the \Macro{CM\_IP\_ADDR} parameter to the ip address you wish
to use on that machine.
See section~\ref{sec:Multiple-Interfaces} on
page~\pageref{sec:Multiple-Interfaces} for more details.

\item The support for multiple network interfaces introduced bugs in
\Condor{userprio}, \Condor{stats}, CondorPVM, and the \Opt{-pool}
option to many Condor tools.
All of these will be fixed in version 6.1.6.

\item Fixed a bug in the remote system call library that was
preventing certain Fortran operations from working correctly on
Linux.  

\item The Linux binaries for GLIBC we now distribute are compiled on a
Red Hat 5.2 machine.
If you're using this version of Red Hat, you might have better luck
with the dynamically linked version of Condor than previous releases
of Condor.
Sites using other GLIBC Linux distributions should continue to use the
statically linked version of Condor.

\item Fixed a bug in the \Condor{shadow} that could cause it to die
with signal 11 (segmentation violation) under certain rare
circumstances. 

\item Fixed a bug in the \Condor{schedd} that could cause it to die
with signal 11 (segmentation violation) under certain rare
circumstances. 

\item Fixed a bug in the \Condor{negotiator} that could cause it to
die with signal 8 (floating point exception) on Digital Unix
machines. 

\item The following shadow parameters have been added to control
checkpointing: \Macro{COMPRESS\_PERIODIC\_CKPT},
\Macro{COMPRESS\_VACATE\_CKPT}, \Macro{PERIODIC\_MEMORY\_SYNC},
\Macro{SLOW\_CKPT\_SPEED}.  See
section~\ref{sec:Shadow-Config-File-Entries} on
page~\pageref{sec:Shadow-Config-File-Entries} for more details.
In addition, the shadow now honors the \Attr{CkptWanted} flag in a job
classad, and if it is set to ``False'', the job will never
checkpoint.

\item Fixed a bug in the \Condor{startd} that could cause it to
report negative values for the CondorLoadAvg on rare occasions. 

\item Fixed a bug in the \Condor{startd} that could cause it to die
with a fatal exception in situations where the act of getting claimed
by a remote schedd failed for some reason.  
This resulted in the \Condor{startd} exiting on rare occasions with a
message in its log file to the effect of \texttt{ERROR ``Match timed
out but not in matched state''}.

\item Fixed a bug in the \Condor{schedd} that under rare circumstances
could cause a job to be left in the ``Running'' state even after the
\Condor{shadow} for that job had exited.

\item Fixed a bug in the \Condor{schedd} and various tools that
prevented remote read-only access to the job queue from working.
So, for example, \texttt{condor\_q -name foo}, if run on any machine
other than foo, wouldn't display any jobs from foo's queue. 
This fix re-enables the following options to \Condor{q} to work:
\Opt{submitter}, \Opt{name}, \Opt{global}, etc.

\item Changed the \Condor{schedd} so that when starting jobs, it
always sorts on the cluster number, in addition to the date the jobs
were enqueued and the process number within clusters, so that if many
clusters were submitted at the same time, the jobs are started in
order.

\item Fixed a bug in \Condor{compile} that was modifying the
\Env{PATH} environment variable by adding things to the front of it.
This would potentially cause jobs to be compiled and linked with a
different version of a compiler than they thought they were getting.  

\item Minor change in the way the \Condor{startd} handles the
\Dflag{LOAD} and \Dflag{KEYBOARD} debug flags.  
Now, each one, when set, will only display every
\Macro{UPDATE\_INTERVAL}, regardless of the startd state.
If you wish to see the values for keyboard activity or load average
every \Macro{POLLING\_INTERVAL}, you must enable \Dflag{FULLDEBUG}. 

\end{itemize}

%%%%%%%%%%%%%%%%%%%%%%%%%%%%%%%%%%%%%%%%%%%%%%%%%%%%%%%%%%%%%%%%%%%%%%
\subsection*{\label{sec:New-6-1-4}Version 6.1.4}
%%%%%%%%%%%%%%%%%%%%%%%%%%%%%%%%%%%%%%%%%%%%%%%%%%%%%%%%%%%%%%%%%%%%%%

\begin{itemize}

\item Fixed a bug in the socket communication library used by Condor
that was causing daemons and tools to die on some platforms (notably,
Digital Unix) with signal 8, SIGFPE (floating point exception).

\item Fixed a bug in the usage message of many Condor tools that
mentioned a \Opt{-all} option that isn't yet supported. 
This option will be supported in future versions of Condor.

\item Fixed a bug in the filesystem authentication code used to
authenticate operations on the job queue that left empty temporary
files in /tmp.  
These files are now properly removed after they are used.

\item Fixed a minor bug in the totals \Condor{status} displays when
you use the \Opt{ckptsrvr} option.

\item Fixed a minor syntax error in the \Condor{install} script that
would cause warnings.

\item the \File{Condor.pm} Perl module is now included in the
\File{lib} directory of the main release directory.

\end{itemize}

%%%%%%%%%%%%%%%%%%%%%%%%%%%%%%%%%%%%%%%%%%%%%%%%%%%%%%%%%%%%%%%%%%%%%%
\subsection*{\label{sec:New-6-1-3}Version 6.1.3}
%%%%%%%%%%%%%%%%%%%%%%%%%%%%%%%%%%%%%%%%%%%%%%%%%%%%%%%%%%%%%%%%%%%%%%

\Note There are a lot of new, unstable features in 6.1.3.  
PLEASE do not install all of 6.1.3 on a production pool.
Almost all of the bug fixes in 6.1.3 are in the \Condor{startd} or
\Condor{starter}, so, unless you really know what you're doing, we
recommend you just upgrade SMP-Startd contrib module, not the entire
6.1.3 release. 

\begin{itemize}

\item Owners can now specify how the SMP-Startd partitions the system
resources into the different types and numbers of virtual machines,
specifying the number of CPUs, megs of RAM, megs of swap space, etc.,
in each.
Previously, each virtual machine reported to Condor from an SMP
machine always had one CPU, and all shared system resources were
evenly divided among the virtual machines.

\item Fixed a bug in the reporting of virtual memory and disk space on
SMP machines where each virtual machine represented was advertising
the total in the system for itself, instead of its own share.
Now, both the totals, and the virtual machine-specific values are
advertised.  

\item Fixed a bug in the \Condor{starter} when it was trying to
suspend jobs.
While we always killed all of the processes when we were trying to
vacate, if a vanilla job forked, the starter would sometimes not
suspend some of the children processes.
In addition, we could sometimes miss a standard universe job for
suspending as well.
This is all fixed.

\item Fixed a bug in the SMP-Startd's load average computation that
could cause processes spawned by Condor to not be associated w/ the
Condor load average.
This would cause the startd to over-estimate the owner's load average,
and under-estimate the Condor load, which would cause a cycle of
suspending and resuming a Condor job, instead of just letting it run.

\item Fixed a bug in the SMP-Startd's load average computation that
could cause certain rare exceptions to be treated as fatal, when in
fact, the Startd could recover from them.

\item Fixed a bug in the computation of the total physical memory on
some platforms that was resulting in an overflow on machines with
lots of ram (over 1 gigabyte).

\item Fixed some bugs that could cause \Condor{starter} processes to
be left as zombies underneath the \Condor{startd} under very rare
conditions.  

\item For sites using AFS, if there are problems in the
\Condor{startd} computing the AFS cell of the machine it's running on,
the startd will exit with an error message at start-up time.

\item Fixed a minor bug in \Condor{install} that would lead to a
syntax error in your config file given a certain set of installation
options.  

\item Added the \Opt{-maxjobs} option to the \Condor{submit\_dag}
script that can be used to specify the maximum number of jobs Condor
will run from a DAG at any given time.
Also, \Condor{submit\_dag} automatically creates a ``rescue DAG''.
See section~\ref{sec:DAGMan} on page~\pageref{sec:DAGMan} for details
on DAGMan.

\item Fixed bug in ClassAd printing when you tried to display an
integer or float attribute that didn't exist in the given ClassAd. 
This could show up in \Condor{status}, \Condor{q}, \Condor{history},
etc. 

\item Various commands sent to the Condor daemons now have separate
debug levels associated with them.
For example, commands such as ``keep-alives'', and the command sent
from the \Condor{kbdd} to the \Condor{startd} are only seen in the
various log files if \Dflag{FULLDEBUG} is turned on, instead of
\Dflag{COMMAND}, which the default and now enabled for all daemons on
all platforms by default.
Administrators retaining their old configuration when upgrading to
this version are encouraged to enable \Dflag{COMMAND} in the
\Macro{SCHEDD\_DEBUG} setting.  
In addition, for IRIX and Digital Unix machines, it should be enabled
in the \Macro{STARTD\_DEBUG} setting as well.
See section~\ref{sec:Daemon-Logging-Config-File-Entries} on
page~\pageref{sec:Daemon-Logging-Config-File-Entries} for details on
debug levels in Condor.

\item New debug levels added to Condor: 
\begin{itemize}
\item \Dflag{NETWORK}, used by various daemons in Condor to report
various network statistics about the Condor daemons. 
\item \Dflag{PROCFAMILY}, used to report information about various
families of processes that are monitored by Condor.
For example, this is used in the \Condor{startd} when monitoring the
family of processes spawned by a given user job for the purposes of
computing the Condor load average.
\item \Dflag{KEYBOARD}, used by the \Condor{startd} to print out
statistics about remote tty and console idle times in the
\Condor{startd}.
This information used to be logged at \Dflag{FULLDEBUG}, along with
everything else, so now, you can see just the idle times, and/or have
the information stored to a separate file.
\end{itemize}

\item Added a \Opt{-run} option to \Condor{q}, which displays
information for running jobs, including the remote host where each job
is running.

\item Macros can now be incrementally defined.  See
section~\ref{sec:Config-File-Macros} on
page~\pageref{sec:Config-File-Macros} for more details.

\item \Condor{config\_val} can now be used to set configuration
variables.  See the man page on page~\pageref{man-condor-config-val}
for more details.

\item The job log file now contains a record of network activity.  The
evict, terminate, and shadow exception events indicate the number of
bytes sent and received by the job for the specific run.  
The terminate event additionally indicates totals for the life of the
job.

\item \Macro{STARTER\_CHOOSES\_CKPT\_SERVER} now defaults to true.
See section~\ref{param:StarterChoosesCkptServer} on
page~\pageref{param:StarterChoosesCkptServer} for more details.

\item The infrastructure for authentication within Condor has been
overhauled, allowing for much greater flexibility in supporting new
forms of authentication in the future.
This means that the 6.1.3 schedd and queue management tools (like
\Condor{q}, \Condor{submit}, \Condor{rm} and so on) are incompatible
with previous versions of Condor.

\item Many of the Condor administration tools have been improved to
allow you to specify the ``subsystem'' you want them to effect.  
For example, you can now use ``\condor{reconfig} -startd'' to just
have the startd reconfigure itself.
Similarly, \condor{off}, \condor{on} and \condor{restart} can now all 
work on a single daemon, instead of machine-wide.
See the man pages (section~\ref{sec:command-reference} on
page~\pageref{sec:command-reference}) or run any command with \Opt{-help}
for details. 
\Note The usage message in 6.1.3 incorrectly reports \Opt{-all} as a
valid option.

\item Fixed a bug in the Condor tools that could cause a segmentation
violation in certain rare error conditions.

\end{itemize}

%%%%%%%%%%%%%%%%%%%%%%%%%%%%%%%%%%%%%%%%%%%%%%%%%%%%%%%%%%%%%%%%%%%%%%
\subsection*{\label{sec:New-6-1-2}Version 6.1.2}
%%%%%%%%%%%%%%%%%%%%%%%%%%%%%%%%%%%%%%%%%%%%%%%%%%%%%%%%%%%%%%%%%%%%%%

\begin{itemize}

\item Fixed some bugs in the \Condor{install} script.
Also, enhanced \Condor{install} to customize the path to perl in
various perl scripts used by Condor.

\item Fixed a problem with our build environment that left some files
out of the \File{release.tar} files in the binary releases on some
platforms. 

\item \Condor{dagman}, ``DAGMan'' (see section~\ref{sec:DAGMan} on 
page~\pageref{sec:DAGMan} for details) is now included in the
development release by default.

\item Fixed a bug in the computation of the total physical memory in
HPUX machines that was resulting in an overflow on machines with
lots of ram (over 1 gigabyte).
Also, if you define ``MEMORY'' in your config file, that value will
override whatever value Condor computes for your machine.

\item Fixed a bug in \Condor{starter.pvm}, the PVM version of the
Condor starter (available as an optional ``Contrib module''), when you
disabled \Macro{STARTER\_LOCAL\_LOGGING}.
Now, having this set to ``False'' will properly place debug messages
from \Condor{starter.pvm} into the \File{ShadowLog} file of the
machine that submitted the job (as opposed to the \File{StarterLog}
file on the machine executing the job).  

\end{itemize}


%%%%%%%%%%%%%%%%%%%%%%%%%%%%%%%%%%%%%%%%%%%%%%%%%%%%%%%%%%%%%%%%%%%%%%
\subsection*{\label{sec:New-6-1-1}Version 6.1.1}
%%%%%%%%%%%%%%%%%%%%%%%%%%%%%%%%%%%%%%%%%%%%%%%%%%%%%%%%%%%%%%%%%%%%%%

\begin{itemize}

\item Fixed a bug in the \Condor{startd} where we compute the load
average caused by Condor that was causing us to get the wrong values.
This could cause a cycle of continuous job suspends and job resumes.

\item Beginning with this version, any jobs linked with the Condor
checkpoint libraries will use the zlib compression code (used by gzip
and others) to compress periodic checkpoints before they are written
to the network.  
These compressed checkpoints are uncompressed at startup time.  
This saves network bandwidth, disk space, as well as time (if the
network is the bottleneck to checkpointing, which it usually is). 
In future versions of Condor, all checkpoints will probably be
compressed, but at this time, it is only used for periodic
checkpoints.  
Note, you have to relink your jobs with the \Condor{compile} command
to have this feature enabled.
Old jobs (not relinked) will continue to run just fine, they just
won't be compressed.

\item \Condor{status} now has better support for displaying checkpoint
server ClassAds. 

\item More contrib modules from the development series are now
available, such as the checkpoint server, PVM support, and the
CondorView server.  

\item Fixed some minor bugs in the UserLog code that were causing
problems for DAGMan in exceptional error cases.

\item Fixed an obscure bug in the logging code when \Dflag{PRIV} was
enabled that could result in incorrect file permissions on log files. 

\end{itemize}

%%%%%%%%%%%%%%%%%%%%%%%%%%%%%%%%%%%%%%%%%%%%%%%%%%%%%%%%%%%%%%%%%%%%%%
\subsection*{\label{sec:New-6-1-0}Version 6.1.0}
%%%%%%%%%%%%%%%%%%%%%%%%%%%%%%%%%%%%%%%%%%%%%%%%%%%%%%%%%%%%%%%%%%%%%%

\begin{itemize}

\item Support has been added to the \condor{startd} to run multiple
jobs on SMP machines.
See section~\ref{sec:Configuring-SMP} on
page~\pageref{sec:Configuring-SMP} for details about setting up and
configuring SMP support.

\item The expressions that control the \condor{startd} policy for
vacating, jobs has been simplified.
See section~\ref{sec:Configuring-Policy} on
page~\pageref{sec:Configuring-Policy} for complete details on the new
policy expressions, and section~\ref{sec:V60-Policy-diffs} on
page~\pageref{sec:V60-Policy-diffs} for an explanation of what's
different from the version 6.0 expressions.

\item We now perform better tracking of processes spawned by Condor.
If children die and are inherited by init, we still know they belong
to Condor.
This allows us to better ensure we don't leave processes lying around
when we need to get off a machine, and enables us to have a much more
accurate computation of the load average generated by Condor (the
\Attr{CondorLoadAvg} as reported by the \Condor{startd}). 

\item The \condor{collector} now can store historical information
about your pool state.
This information can be queried with the \Condor{stats} program (see
the man page on page~\pageref{man-condor-stats}), which is used by the
\Condor{view} Java GUI, which is available as a separate contrib
module.

\item Condor jobs can now be put in a ``hold'' state with the
\Condor{hold} command.
Such jobs remain in the job queue (and can be viewed with \Condor{q}),
but there will not be any negotiation to find machines for them.
If a job is having a temporary problem (like the permissions are 
wrong on files it needs to access), the job can be put on hold until
the problem can be solved.
Jobs put on hold can be released with the \Condor{release} command.

\item \condor{userprio} now has the notion of \Term{user factors} as a
way to create different groups of users in different priority levels.
See section~\ref{sec:UserPrio} on page~\pageref{sec:UserPrio} for
details.
This includes the ability to specify a local priority domain, and all
users from other domains get a much worse priority.

\item Usage statistics by user is now available from
\condor{userprio}.
See the man page on page~\pageref{man-condor-userprio} for details. 

\item The \condor{schedd} has been enhanced to enable ``flocking'',
where it seeks matches with machines in multiple pools if its requests
cannot be serviced in the local pool.
See section~\ref{sec:Flocking} on page~\pageref{sec:Flocking} for more
details.

\item The \condor{schedd} has been enhanced to enable \condor{q} and
other interactive tools better response time.

\item The \condor{schedd} has also been enhanced to allow it to check
the permissions of the files you specify for input, output, error and
so on.  
If the schedd doesn't have the required access rights to the files,
the jobs will not be submitted, and \Condor{submit} will print an
error message.

\item When you perform a \Condor{rm} command, and the job you removed
was using a ``user log'', the remove event is now recorded into the
log. 

\item Two new attributes have been added to the job classad when it 
begins executing: \Attr{RemoteHost} and \Attr{LastRemoteHost}.
These attributes list the IP address and port of the startd that is
either currently running the job, or the last startd to run the job
(if it's run on more than one machine). 
This information helps users track their job's execution more closely,
and allows administrators to troubleshoot problems more effectively. 

\item The performance of checkpointing was increased by using larger
buffers for the network I/O used to get the checkpoint file on and off
the remote executing host (this helps for all pools, with or without
checkpoint servers). 

\end{itemize}


%%%%%%%%%%%%%%%%%%%%%%%%%%%%%%%%%%%%%%%%%%%%%%%%%%%%%%%%%%%%%%%%%%%%%%%
\section{\label{sec:History-6-0}Stable Release Series 6.0}
%%%%%%%%%%%%%%%%%%%%%%%%%%%%%%%%%%%%%%%%%%%%%%%%%%%%%%%%%%%%%%%%%%%%%%

6.0 is the first version of Condor with \Term{ClassAds}.
It contains many other fundamental enhancements over version 5.
It is also the first official stable release series, with a
development series (6.1) simultaneously available.


%%%%%%%%%%%%%%%%%%%%%%%%%%%%%%%%%%%%%%%%%%%%%%%%%%%%%%%%%%%%%%%%%%%%%%
\subsection{\label{sec:New-6-0-3}Version 6.0.3}
%%%%%%%%%%%%%%%%%%%%%%%%%%%%%%%%%%%%%%%%%%%%%%%%%%%%%%%%%%%%%%%%%%%%%%

\begin{itemize}

\item Fixed a bug that was causing the hostname of the submit machine
that claimed a given execute machine to be incorrectly reported by the
\Condor{startd} at sites using NIS.

\item Fixed a bug in the \Condor{startd}'s benchmarking code that
could cause a floating point exception (SIGFPE, signal 8) on very,
very fast machines, such as newer Alphas.

\item Fixed an obscure bug in \Condor{submit} that could happen when
you set a requirements expression that references the ``Memory''
attribute.
The bug only showed up with certain formations of the requirement
expression.

\end{itemize}


%%%%%%%%%%%%%%%%%%%%%%%%%%%%%%%%%%%%%%%%%%%%%%%%%%%%%%%%%%%%%%%%%%%%%%
\subsection{\label{sec:New-6-0-2}Version 6.0.2}
%%%%%%%%%%%%%%%%%%%%%%%%%%%%%%%%%%%%%%%%%%%%%%%%%%%%%%%%%%%%%%%%%%%%%%

\begin{itemize}

\item Fixed a bug in the \Syscall{fcntl} call for Solaris 2.6 that was
causing problems with file I/O inside Fortran jobs.

\item Fixed a bug in the way the \Macro{DEFAULT\_DOMAIN\_NAME}
parameter was handled so that this feature now works properly.  

\item Fixed a bug in how the \Macro{SOFT\_UID\_DOMAIN} config file
parameter was used in the \Condor{starter}.
This feature is also documented in the manual now (see
section~\ref{param:SoftUidDomain} on
page~\pageref{param:SoftUidDomain}).

\item You can now set the RunBenchmarks expression to ``False'' and
the \Condor{startd} will never run benchmarks, not even at startup
time. 

\item Fixed a bug in \Syscall{getwd} and \Syscall{getcwd} for sites
that use the NFS automounter.
his bug was only present if user programs tried to call
\Syscall{chdir} themselves.
Now, this is supported. 

\item Fixed a bug in the way we were computing the available virtual
memory on HPUX 10.20 machines.

\item Fixed a bug in \Condor{q} -analyze so it will correctly identify
more situations where a job won't run.

\item Fixed a bug in \Condor{status} -format so that if the requested 
attribute isn't available for a given machine, the format string
(including spaces, tabs, newlines, etc) is still printed, just the
value for the requested attribute will be an empty string. 

\item Fixed a bug in the \Condor{schedd} that was causing
\Condor{history} to not print out the first ClassAd attribute of all
jobs that have completed

\item Fixed a bug in \Condor{q} that would cause a segmentation fault
if the argument list was too long.

\end{itemize}

%%%%%%%%%%%%%%%%%%%%%%%%%%%%%%%%%%%%%%%%%%%%%%%%%%%%%%%%%%%%%%%%%%%%%%
\subsection{\label{sec:New-6-0-1}Version 6.0.1}
%%%%%%%%%%%%%%%%%%%%%%%%%%%%%%%%%%%%%%%%%%%%%%%%%%%%%%%%%%%%%%%%%%%%%%

\begin{itemize}

\item Fixed bugs in the \Syscall{getuid}), \Syscall{getgid},
\Syscall{geteuid}, and \Syscall{getegid} system calls. 

\item Multiple config files are now supported as a list specified via
the \Macro{LOCAL\_CONFIG\_FILE} variable. 

\item \Macro{ARCH} and \Macro{OPSYS} are now automatically determined
on all machines (including HPUX 10 and Solaris). 

\item Machines running IRIX now correctly suspend vanilla jobs.

\item \Condor{submit} doesn't allow root to submit jobs.

\item The \Condor{startd} now notices if you have changed
\Macro{COLLECTOR\_HOST} on reconfig.

\item Physical memory is now correctly reported on Digital Unix when
daemons are not running as root. 

\item New \MacroU{SUBSYSTEM} macro in configuration files that changes
based on which daemon is reading the file (i.e. STARTD, SCHEDD, etc.) 
See section~\ref{sec:Condor-Subsystem-Names}, ``Condor Subsystem
Names'' on page~\pageref{sec:Condor-Subsystem-Names} for a complete
list of the subsystem names used in Condor.

\item Port to HP-UX 10.20.  

\item \Syscall{getrusage} is now a supported system call.  
This system call will allow you to get resource usage about the entire
history of your condor job.

\item Condor is now fully supported on Solaris 2.6 machines (both
Sparc and Intel). 

\item Condor now works on Linux machines with the GNU C library.  
This includes machines running Red Hat 5.x and Debian 2.0. 
In addition, there seems to be a bug in Red Hat that was causing the
output from \Condor{config\_val} to not appear when used in scripts
(like \Condor{compile}).
We put in explicit calls to flush the I/O buffers before
\Condor{config\_val} exits, which seems to solve the problem.

\item Hooks have been added to the checkpointing library to help
support the checkpointing of PVM jobs.

\item Condor jobs can now send signals to themselves when running in
the standard universe.
You do this just as you normally would:
\begin{verbatim}
        kill( getpid(), signal_number )
\end{verbatim}
Trying to send a signal to any other process will result in
\Syscall{kill} returning -1.

\item Support for NIS has been improved on Digital Unix and IRIX.

\item Fixed a bug that would cause the negotiator on IRIX machines to
never match jobs with available machines.  

\end{itemize}

%%%%%%%%%%%%%%%%%%%%%%%%%%%%%%%%%%%%%%%%%%%%%%%%%%%%%%%%%%%%%%%%%%%%%%
\subsection{\label{sec:New-6-0-pl4}Version 6.0 pl4}
%%%%%%%%%%%%%%%%%%%%%%%%%%%%%%%%%%%%%%%%%%%%%%%%%%%%%%%%%%%%%%%%%%%%%%

\Note Back in the bad old days, we used this evil ``patch level''
version number scheme, with versions like ``6.0pl4''.
This has all gone away in the current versions of Condor. 

\begin{itemize}

\item Fixed a bug that could cause a segmentation violation in the 
\Condor{schedd} under rare conditions when a \Condor{shadow} exited.

\item Fixed a bug that was preventing any core files that user jobs
submitted to Condor might create from being transferred back to the
submit machine for inspection by the user who submitted them.

\item Fixed a bug that would cause some Condor daemons to go into an
infinite loop if the "ps" command output duplicate entries.
This only happens on certain platforms, and even then, only under rare
conditions.
However, the bug has been fixed and Condor now handles this case
properly.

\item Fixed a bug in the \Condor{shadow} that would cause a
segmentation violation if there was a problem writing to the user log
file specified by "log = filename" in the submit file used with
\Condor{submit}.

\item Added new command line arguments for the Condor daemons to support
saving the PID (process id) of the given daemon to a file, sending a
signal to the PID specified in a given file, and overriding what
directory is used for logging for a given daemon.
These are primarily for use with the \Condor{kbdd} when it needs to be
started by XDM for the user logged onto the console, instead of
running as root.
See section~\ref{sec:kbdd} on ``Installing the \Condor{kbdd}'' on
page~\pageref{sec:kbdd} for details.

\item Added support for the \Macro{CREATE\_CORE\_FILES} config file
parameter.  
If this setting is defined, Condor will override whatever limits you
have set and in the case of a fatal error, will either create core
files or not depending on the value you specify ("true" or "false").

\item Most Condor tools (\Condor{on}, \Condor{off},
\Condor{master\_off}, \Condor{restart}, \Condor{vacate},
\Condor{checkpoint}, \Condor{reconfig}, \Condor{reconfig\_schedd},
\Condor{reschedule}) can now take the IP address and port you want to
send the command to directly on the command line, instead of only
accepting hostnames. 
This IP/port must be passed in a special format used in Condor (which
you will see in the daemon's log files, etc).
It is of the form: \Sinful{ip.address:port}.  
For example: \Sinful{123.456.789.123:4567}.

\end{itemize}

%%%%%%%%%%%%%%%%%%%%%%%%%%%%%%%%%%%%%%%%%%%%%%%%%%%%%%%%%%%%%%%%%%%%%%
\subsection{\label{sec:New-6-0-pl3}Version 6.0 pl3}
%%%%%%%%%%%%%%%%%%%%%%%%%%%%%%%%%%%%%%%%%%%%%%%%%%%%%%%%%%%%%%%%%%%%%%

\begin{itemize}

\item Fixed a bug that would cause a segmentation violation if a
machine was not configured with a full hostname as either the official
hostname or as any of the hostname aliases.

\item If your host information does not include a fully qualified
hostname anywhere, you can specify a domain in the
\Macro{DEFAULT\_DOMAIN\_NAME} parameter in your global config file
which will be appended to your hostname whenever Condor needs to use a
fully qualified name.

\item All Condor daemons and most tools now support a "-version"
option that displays the version information and exits.

\item The \Condor{install} script now prompts for a short description
of your pool, which it stores in your central manager's local config
file as \Macro{COLLECTOR\_NAME}.
This description is used to display the name of your pool when sending
information to the Condor developers.

\item When the \Condor{shadow} process starts up, if it is configured
to use a checkpoint server and it cannot connect to the server, the
shadow will check the \Macro{MAX\_DISCARDED\_RUN\_TIME} parameter.  
If the job in question has accumulated more CPU minutes than this
parameter, the \Condor{shadow} will keep trying to connect to the
checkpoint server until it is successful.
Otherwise, the \Condor{shadow} will just start the job over from
scratch immediately.

\item If Condor is configured to use a checkpoint server, it will only
use the checkpoint server.
Previously, if there was a problem connecting to the checkpoint
server, Condor would fall back to using the submit machine to store
checkpoints.
However, this caused problems with local disks filling up on machines
without much disk space.

\item Fixed a rare race condition that could cause a segmentation
violation if a Condor daemon or tool opened a socket to a daemon and
then closed it right away.

\item All TCP sockets in Condor now have the "keep alive" socket option
enabled.
This allows Condor daemons to notice if their peer goes away in a hard
crash.

\item Fixed a bug that could cause the \Condor{schedd} to kill jobs
without a checkpoint during its graceful shutdown method under certain
conditions.

\item The \Condor{schedd} now supports the
\Macro{MAX\_SHADOW\_EXCEPTIONS} parameter.
If the \Condor{shadow} processes for a given match die due to a fatal
error (an exception) more than this number of times, the
\Condor{schedd} will now relinquish that match and stop trying to
spawn \Condor{shadow} processes for it.

\item The "-master" option to \Condor{status} now displays the \Attr{Name}
attribute of all \Condor{master} daemons in your pool, as opposed
to the \Attr{Machine} attribute.
This helps for pools that have submit-only machines joining them, for
example.

\end{itemize}

%%%%%%%%%%%%%%%%%%%%%%%%%%%%%%%%%%%%%%%%%%%%%%%%%%%%%%%%%%%%%%%%%%%%%%
\subsection{\label{sec:New-6-0-pl2}Version 6.0 pl2}
%%%%%%%%%%%%%%%%%%%%%%%%%%%%%%%%%%%%%%%%%%%%%%%%%%%%%%%%%%%%%%%%%%%%%%

\begin{itemize}

\item In patch level 1, code was added to more accurately find the
full hostname of the local machine.
Part of this code relied on the resolver, which on many platforms is a
dynamic library.
On Solaris, this library has needed many security patches and the
installation of Solaris on our development machines produced binaries
that are incompatible with sites that haven't applied all the security
patches.
So, the code in Condor that relies on this library was simply removed
for Solaris.

\item Version information is now built into Condor.
You can see the \Attr{CondorVersion} attribute in every daemon's
ClassAd. 
You can also run the UNIX command "ident" on any Condor binary to see
the version. 

\item Fixed a bug in the "remote submit" mode of \Condor{submit}.
The remote submit wasn't connecting to the specified schedd, but was
instead trying to connect to the local schedd.

\item Fixed a bug in the \Condor{schedd} that could cause it to exit
with an error due to its log file being locked improperly under
certain rare circumstances.

\end{itemize}

%%%%%%%%%%%%%%%%%%%%%%%%%%%%%%%%%%%%%%%%%%%%%%%%%%%%%%%%%%%%%%%%%%%%%%
\subsection{\label{sec:New-6-0-pl1}Version 6.0 pl1}
%%%%%%%%%%%%%%%%%%%%%%%%%%%%%%%%%%%%%%%%%%%%%%%%%%%%%%%%%%%%%%%%%%%%%%

\begin{itemize}

\item \Condor{kbdd} bug patched: On Silicon Graphics and DEC Alpha
ports, if your X11 server is using Xauthority user authentication, and
the \Condor{kbdd} was unable to read the user's \File{.Xauthority}
file for some reason, the \Condor{kbdd} would fall into an infinite 
loop.

\item When using a Condor Checkpoint Server, the protocol between the
Checkpoint Server and the \Condor{schedd} has been made more robust
for a faulty network connection. Specifically, this improves
reliability when submitting jobs across the Internet and using a
remote Checkpoint Server.

\item Fixed a bug concerning \Macro{MAX\_JOBS\_RUNNING}: The parameter
\MacroNI{MAX\_JOBS\_RUNNING} in the config file controls the maximum
number of simultaneous \Condor{shadow} processes allowed on your
submission machine.
The bug was the number of shadow processes could, under certain
conditions, exceed the number specified by
\MacroNI{MAX\_JOBS\_RUNNING}. 

\item Added new parameter \Macro{JOB\_RENICE\_INCREMENT} that can be
specified in the config file.
This parameter specifies the UNIX nice level that the \Condor{starter}
will start the user job.
It works just like the \Cmd{renice}{1} command in UNIX. 
Can be any integer between 1 and 19; a value of 19 is the lowest
possible priority.

\item Improved response time for \Condor{userprio}.

\item Fixed a bug that caused periodic checkpoints to happen more
often than specified.

\item Fixed some bugs in the installation procedure for certain
environments that weren't handled properly, and made the documentation
for the installation procedure more clear.

\item Fixed a bug on IRIX that could allow vanilla jobs to be started
as root under certain conditions.
This was caused by the non-standard uid that user "nobody" has on
IRIX.
Thanks to Chris Lindsey at NCSA for help discovering this bug.

\item On machines where the \File{/etc/hosts} file is misconfigured to
list just the hostname first, then the full hostname as an alias,
Condor now correctly finds the full hostname anyway.

\item The local config file and local root config file are now only
found by the files listed in the \Macro{LOCAL\_CONFIG\_FILE} and
\Macro{LOCAL\_ROOT\_CONFIG\_FILE} parameters in the global config
files.
Previously, \File{/etc/condor} and user condor's home directory
(\~condor) were searched as well.
This could cause problems with submit-only installations of Condor at
a site that already had Condor installed.

\end{itemize}

%%%%%%%%%%%%%%%%%%%%%%%%%%%%%%%%%%%%%%%%%%%%%%%%%%%%%%%%%%%%%%%%%%%%%%
\subsection{\label{sec:New-6-0-pl0}Version 6.0 pl0}
%%%%%%%%%%%%%%%%%%%%%%%%%%%%%%%%%%%%%%%%%%%%%%%%%%%%%%%%%%%%%%%%%%%%%%

\begin{itemize}

\item Initial Version 6.0 release.

\end{itemize}

