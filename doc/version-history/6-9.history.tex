%%%      PLEASE RUN A SPELL CHECKER BEFORE COMMITTING YOUR CHANGES!
%%%      PLEASE RUN A SPELL CHECKER BEFORE COMMITTING YOUR CHANGES!
%%%      PLEASE RUN A SPELL CHECKER BEFORE COMMITTING YOUR CHANGES!
%%%      PLEASE RUN A SPELL CHECKER BEFORE COMMITTING YOUR CHANGES!
%%%      PLEASE RUN A SPELL CHECKER BEFORE COMMITTING YOUR CHANGES!

%%%%%%%%%%%%%%%%%%%%%%%%%%%%%%%%%%%%%%%%%%%%%%%%%%%%%%%%%%%%%%%%%%%%%%
\section{\label{sec:History-6-9}Development Release Series 6.9}
%%%%%%%%%%%%%%%%%%%%%%%%%%%%%%%%%%%%%%%%%%%%%%%%%%%%%%%%%%%%%%%%%%%%%%

This is the development release series of Condor.
The details of each version are described below.

%%%%%%%%%%%%%%%%%%%%%%%%%%%%%%%%%%%%%%%%%%%%%%%%%%%%%%%%%%%%%%%%%%%%%%
\subsection*{\label{sec:New-6-9-3}Version 6.9.3}
%%%%%%%%%%%%%%%%%%%%%%%%%%%%%%%%%%%%%%%%%%%%%%%%%%%%%%%%%%%%%%%%%%%%%%

\noindent Release Notes:

\begin{itemize}

\item As of version 6.9.3, the entire Condor system has undergone a
  major terminology change.
  For almost 10 years, Condor has used the term \Term{virtual machine}
  or \Term{vm} to refer to each distinct resource that could run a
  Condor job (for example, each of the CPUs on an SMP machine).
  Back when we chose this terminology, it made sense, since each of
  these resource was like an independent machine in a pool, with
  its own state, ClassAd, claims, and so on.
  However, in recent years, the term \Term{virtual machine} is now
  almost universally associated with the kinds of virtual machines
  created using tools such as VMware and Xen.  Entire operating systems
  run inside a given process, usually emulating the underlying
  hardware on a host machine.
  So, to avoid confusion with these other kinds of virtual machines,
  the old  \Term{virtual machine} terminology has been replaced by
  the term \Term{slot}.

  Numerous configuration settings, command-line arguments to Condor
  tools, ClassAd attribute names, and so on, have all been
  modified to reflect the new \Term{slot} terminology.
  In general, the old settings and options will still work, but are
  now retired and may disappear in the future.

\item The \Condor{install} installation script has
  been removed.
  All sites should use \Condor{configure} when setting up a new Condor
  installation.

\item The \Macro{SECONDARY\_COLLECTOR\_LIST} configuration variable has
  been removed.
  Sites relying on this variable should instead use the configuration
  variable \Macro{COLLECTOR\_HOST}. It may be used to
  define a list of \Condor{collector} daemon hosts.

\item Cleaned up and improved help information for \Condor{history}.

\end{itemize}


\noindent New Features:

\begin{itemize}

\item Numerous scalability and performance improvements.  Given enough
memory, the schedd can now handle much larger job queues (e.g. 10s of
thousands) without the severe degradation in performance that used to
be the case.

\item Added the \Macro{START\_LOCAL\_UNIVERSE} and \Macro{START\_SCHEDULER\_UNIVERSE}
parameters for the \Condor{schedd}. This allows administrators to control whether
a Local/Scheduler universe job will be started. This expression is evaluated
against the job's ClassAd before the \AdAttr{Requirements} expression.

\item All Local and Scheduler universe jobs now have their \AdAttr{Requirements} 
expressions evaluated before execution. If the expression evaluates to false, the
job will not be allowed to begin running. In previous versions of Condor, Local 
and Scheduler universe jobs could begin execution without the \Condor{schedd} checking
the validity of the \AdAttr{Requirements}.

\item Added \MacroNI{SCHEDD\_INTERVAL\_TIMESLICE} and
\MacroNI{PERIODIC\_EXPR\_TIMESLICE}.  These indicate the maximum
fraction of time that the schedd will spend on the respective
activities.  Previously, these activities were done on a fixed
interval, so with very large job queue sizes, the fraction of time
spent was increasing to unreasonable levels.

\item Under Intel Linux, added \Macro{USE\_CLONE\_TO\_CREATE\_PROCESSES}.
This defaults to true and results in scalability improvements for processes
using large amounts of memory (e.g. a schedd with a lot of jobs in the queue).

\item Jobs in the parallel universe now can have \$\$ expanded in their
ads in the same way as other universes.

\item Local universe jobs now support policy expression evaluation, which includes
the \AdAttr{ON\_EXIT\_REMOVE}, \AdAttr{ON\_EXIT\_HOLD}, \AdAttr{PERIODIC\_REMOVE},
\AdAttr{PERIODIC\_HOLD}, and \AdAttr{PERIODIC\_RELEASE} attributes. The periodic
expressions are evaluated at intervals determined by the
\Macro{PERIODIC\_EXPR\_INTERVAL} configuration macro.

\item Jobs can be scheduled to executed periodically, similar to the crontab
functionality found in Unix systems. The \Condor{schedd} calculates the next
runtime for a job based on the new \AdAttr{CRON\_MINUTE}, \AdAttr{CRON\_HOUR},
\AdAttr{CRON\_DAY\_OF\_MONTH}, \AdAttr{CRON\_MONTH}, and
\AdAttr{CRON\_DAY\_OF\_WEEK} attributes. A preparation time defined by the
\AdAttr{CRON\_PREP\_TIME} attribute allows a job to be submitted to the
execution machine before the actual time the job is to begin execution.
Jobs that would like to be run repeatedly will need to define the
the \AdAttr{ON\_EXIT\_REMOVE} attribute properly so that they are
re-queued after executing each time.

\item Condor now looks for its configuration file in \File{/usr/local/etc}
if the \Macro{CONDOR\_CONFIG} environment variable is not set and there is
no condor\_config file located in \File{/etc/condor}. This allows a default
Condor installation to be more compatible with Free BSD.

\item If a user job requests streaming input or output in the submit
file, the job can now run with job leases and the job will continue
to run for the lease duration should the submit machine crash.  Previously,
jobs with streaming i/o would be evicted if the submit machine crashed.
While the submit machine is down, if the job tried to issue a streaming
read or write, the job will block until the submit machine returns or the
job lease expires.

\item Ever since version 6.7.19, \Condor{submit} has added a default
  job lease duration of 20 minutes to all jobs that support these
  leases.
  However, there was no way to disable this functionality if a user
  did not want job lease semantics.
  Now, a user can place \verb@job_lease_duration = 0@ in their submit
  file to manually disable the job lease.

% condor-admin 15254
\item Added new configuration knob \Macro{STARTER\_UPLOAD\_TIMEOUT}
which sets the timeout for the starter to upload output files to the
shadow on job exit.  The default value is 200 seconds, which should
be sufficient for serial jobs.  For parallel jobs, this may need to
be increased if many large output files are sent back to the shadow
on job exit.

% Gnats PR 806
\item \Condor{dagman} now aborts the DAG on ``scary'' submit events.
These are submit events in which
the Condor ID of the event does not match the
expected value.
Previously, \Condor{dagman} printed a warning, but continued.
To restore Condor to the previous behavior,
set the new \Macro{DAGMAN\_ABORT\_ON\_SCARY\_SUBMIT} configuration variable
to \Expr{False}.

\item When the \Condor{master} detects that its GCB broker is unavailable
and there is a list of alternative brokers,
it will restart immediately if \Macro{MASTER\_WAITS\_FOR\_GCB\_BROKER} is
set to \Expr{False} instead of waiting for another broker to became available.
\Condor{glidein} now sets \MacroNI{MASTER\_WAITS\_FOR\_GCB\_BROKER}
to  \Expr{False} in its configuration file.

\item When using GCB and a list of brokers is available, the
\Condor{master} will now pick a random broker rather than the least-loaded
one.

\item All Condor daemons now evaluate some ClassAd expressions
  whenever they are about to publish an update to the
  \Condor{collector}.
  Currently, the two supported expressions are:
  \begin{description}
  \item[\Macro{DAEMON\_SHUTDOWN}]
    If \Expr{True}, the daemon will gracefully shut itself down and will not
    be restarted by the \Condor{master} (as if it sent itself a
    \Condor{off} command).
  \item[\Macro{DAEMON\_SHUTDOWN\_FAST}]
    If \Expr{True}, the daemon will quickly shut itself down and will not be
    restarted by the \Condor{master} (as if it sent itself a
    \Condor{off} command using the \Opt{-fast} option).
  \end{description}
  For more information about these expressions, see
  section~\ref{param:DaemonShutdown} on
  page~\pageref{param:DaemonShutdown}.

\item When the \Condor{master} sends email announcing that another daemon has
died, exited, or been killed, it now notes the name of the machine, the
daemon's name, and a summary of the situation in the Subject line.

\item Anyplace in a Condor configuration or submit description file where
wild cards may be used, you can now place wild cards at both the beginning
and end of the string pattern (i.e. match strings that contain the text
between the wild cards anywhere in the string). Previously, only one
wild card could appear in the string pattern.

\item Added optional configuration setting
\Macro{NEGOTIATOR\_MATCH\_EXPRS}.  This allows the negotiator to
insert expressions into the matched ClassAd.  See
page~\pageref{param:NegotiatorMatchExprs} for more information.

\item Increased speed of ClassAd parsing.

\item Added \Macro{DEDICATED\_EXECUTE\_ACCOUNT\_REGEXP} and
deprecated the boolean setting
\Macro{EXECUTE\_LOGIN\_IS\_DEDICATED}, because the latter could not
handle a policy where some jobs run as the job owner and some run as
dedicated execution accounts.  Also added support for
\Macro{STARTER\_ALLOW\_RUNAS\_OWNER} under Unix.  See
Section~\ref{param:DedicatedExecuteAccountRegexp} and
Section~\ref{sec:RunAsNobody} for more information.

\item All Condor daemons now publish a \Attr{MyCurrentTime} attribute
  which is the current local time at the time the update was generated
  and sent to the \Condor{collector}.
  This is in addition to the \Attr{LastHeardFrom} attribute which is
  inserted by the \Condor{collector} (the current local time at the
  collector when the update is received).

\item \Condor{history} now accepts partial command line
arguments.  For example, -constraint can be abbreviated -const.
This brings \Condor{history} in line with other Condor command
line tools.

\item \Condor{history} can now emit ClassAds formatted as XML with
the new -xml option.
This brings \Condor{history} more in line \Condor{q}.

\item The \verb@$$@ substitution macro syntax now supports the insertion
of literal \verb@$$@ characters through the use of \verb@$$(DOLLARDOLLAR)@.
Also, \verb@$$@ expansion is no longer recursive, so if the value being
substituted in place of a \verb@$$@ macro itself contains \verb@$$@ characters,
these are no longer interpreted as substitution macros but are instead
inserted literally.

\item When started as root on a Linux 64-bit x86 machine, Condor daemons will
now leave core files in the log directory when they crash.  This matches
Condor's behavior on most other Unix-like operating systems, including
32-bit x86 versions of Linux.
% Code: Google's coredumper library is now used on Linux x86-64.

\item The \Env{\_CONDOR\_SLOT} variable is now placed into the
  environment for jobs of all universes.
  This variable indicates what slot a given job is running on, and
  will have the same value as the \AdAttr{SlotID} from the machine
  classified ad where the job is running.
  The \Env{\_CONDOR\_SLOT} variable replaces the deprecated
  \Env{CONDOR\_VM} environment variable, which was only defined for
  standard universe jobs.

\item Added a \Macro{USE\_PROCD} configuration parameter. If this
parameter is set to true for a given daemon, the daemon will use the
\Condor{procd} program to monitor process families. If set to false,
the daemon will execute process family monitoring logic on its
own. The \Condor{procd} is more scalable and is also an essential
piece in the ongoing privilege separation effort. The disadvantage of
using the ProcD is that it is newer, less-hardened code.

\end{itemize}

\noindent Configuration Variable Additions and Changes:

\begin{itemize}

\item The \Macro{SECONDARY\_COLLECTOR\_LIST} configuration variable has
  been removed.
  Sites relying on this variable should instead use the configuration
  variable \Macro{COLLECTOR\_HOST} to
  define a list of \Condor{collector} daemon hosts.

\item Added new configuration variables \Macro{START\_LOCAL\_UNIVERSE}
  and \Macro{START\_SCHEDULER\_UNIVERSE} for the \Condor{schedd} daemon.
  These boolean expressions default to \Expr{True}.
  \MacroNI{START\_LOCAL\_UNIVERSE} is relevant only to local universe jobs.
  \MacroNI{START\_SCHEDULER\_UNIVERSE} is relevant only to scheduler 
  universe jobs.
  These new variables allow an administrator to define
  a \MacroNI{START} expression specific to these jobs. 
  The expression is evaluated
  against the job's ClassAd before the \AdAttr{Requirements} expression.

\item Added new configuration variables \Macro{SCHEDD\_INTERVAL\_TIMESLICE}
  and \Macro{PERIODIC\_EXPR\_TIMESLICE}.  These configuration variables
  address a scalability issue for very large job queues.
  Previously, the \Condor{schedd} daemon handled an activity related
  to counting jobs, as well as the activity related to evaluating
  periodic expressions for jobs at the fixed time interval of 5 minutes.
  With large job queues, the fraction of the \Condor{schedd} daemon
  execution time devoted to these two activities became excessive,
  such that it could be doing little else.
  The fixed time interval is now gone, and Condor calculates the amount
  of time spent on the two activities, using these new configuration
  variables to calculate an appropriate time interval.
  
  Each is a floating point value within the range
  (noninclusive) 0.0 to 1.0.
  Each determines the maximum fraction of the time interval that the 
  \Condor{schedd} daemon  will spend on the respective
  activity.
  \MacroNI{SCHEDD\_INTERVAL\_TIMESLICE} defaults to the value 0.05,
  such that the calculated time interval will be 20 * the amount
  of time spent on the counting jobs activity.
  \MacroNI{PERIODIC\_EXPR\_TIMESLICE} defaults to the value 0.01,
  such that the calculated time interval will be 100 * the amount
  of time spent on the periodic expression evaluation activity.

\item Added new configuration variable 
  \Macro{USE\_CLONE\_TO\_CREATE\_PROCESSES}, relevant only to the
  Intel Linux platform.  
  This boolean value defaults to \Expr{True}, and it results in scalability
  improvements for Condor processes using large amounts of memory.
  These processes may clone themselves instead of forking themselves.
  An example of the improvement occurs for a \Condor{schedd}
  deamon with a lot of jobs in the queue.

\item Added new configuration variable \Macro{STARTER\_UPLOAD\_TIMEOUT},
  which allows a configurable time (in seconds) for a timeout used by the 
  \Condor{starter}.
  The default value of 200 seconds replaces the previously hardcoded
  value of 20 seconds.
  This timeout before job failure is to upload output files to the
  \Condor{shadow} upon job exit.
  The default value should be sufficient for serial jobs.
  For parallel jobs, it may need to
  be increased if there are many large output files.

\item Added new configuration variable \Macro{DAGMAN\_ABORT\_ON\_SCARY\_SUBMIT}.
  This boolean variable defaults to \Expr{True}, and causes
  \Condor{dagman} to abort the DAG on ``scary'' submit events.
  These are submit events in which
  the Condor ID of the event does not match the expected value.
  Previously, \Condor{dagman} printed a warning, but continued.
  To restore Condor to the previous behavior,
  set \MacroNI{DAGMAN\_ABORT\_ON\_SCARY\_SUBMIT} to \Expr{False}.

\item Added new configuration variable \Macro{NEGOTIATOR\_MATCH\_EXPRS}.
  It causes the \Condor{negotiator} to
  insert expressions into the matched ClassAd.  See
  page~\pageref{param:NegotiatorMatchExprs} for details.

\item Added new configuration variable
  \Macro{DEDICATED\_EXECUTE\_ACCOUNT\_REGEXP} to replace the retired 
  \Macro{EXECUTE\_LOGIN\_IS\_DEDICATED},
  because \MacroNI{EXECUTE\_LOGIN\_IS\_DEDICATED} could not
  handle a policy where some jobs run as the job owner and others run as
  dedicated execution accounts.  Also added support for
  the existing configuration variable
  \Macro{STARTER\_ALLOW\_RUNAS\_OWNER} under Unix.  See
  Section~\ref{param:DedicatedExecuteAccountRegexp} and
  Section~\ref{sec:RunAsNobody} for more information.

\item Added new configuration variable \Macro{USE\_PROCD}.
  This boolean variable defaults to \Expr{False} for the
  \Condor{master}, and \Expr{True} for all other daemons.
  When \Expr{True}, the daemon will use the
  \Condor{procd} program to monitor process families.
  When \Expr{False}, a daemon will execute process family
  monitoring logic on its own.
  The \Condor{procd} is more scalable and is also an essential
  piece in the ongoing privilege separation effort. The disadvantage of
  using the \Condor{procd} is that it is newer, less-hardened code.

\end{itemize}

\noindent Bugs Fixed:

\begin{itemize}

\item When the \Condor{schedd} is issued a graceful shutdown command, any jobs
that running with a job lease are allowed to keep running. When the \Condor{schedd}
starts back up at a later time, it will spawn \Condor{shadow} to reconnect
to the jobs if they are still executing. This mimics the same behavior as
a fast shutdown.

\item On Unix systems, Condor can now handle file descriptors larger than
FD\_SETSIZE when using the select system call. Previously, file descriptors
larger than FD\_SETSIZE would cause memory corruption and crashes.

\item When an update to the \Condor{collector} from the
\Condor{startd} is lost, it is possible for multiple claims to the
same resource to be handed out by the \Condor{negotiator}.  This is
still true.  What is fixed is that these multiple claims will not
result in mutual annihilation of the various attempts to use the
resource.  Instead, the first claim to be successfully requested will
proceed and the others will be rejected.

\item \Condor{glidein} was setting \Macro{PREEN\_INTERVAL}=0 in the default
configuration, but this is no longer a legal value, as of 6.9.2.

\item \Condor{glidein} was not setting necessary configuration parameters
for \Condor{procd} in the default glidein configuration.

\item In 6.9.2, Condor daemons crashed after failing to authenticate a
network connection.

\item \Condor{status} will now accurately report the \Attr{ActvtyTime}
  (activity time) value in Condor pools where not all machines are in
  the same timezone, or if there is clock-skew between the hosts.

\item Fixed the known issue in Condor 6.9.2 where using the
\Macro{EXECUTE\_LOGIN\_IS\_DEDICATED} setting on UNIX platforms would
cause the \Condor{procd} to crash.

\item Failure when activating a COD claim no longer will result in an
opportunistic job running on the same \Condor{startd} being left
suspended. This problem was most likely to be seen when using the
\Macro{GLEXEC\_STARTER} feature.

\item In Condor 6.9.2 for Tru64 UNIX, the \Condor{master} would
immediately fail if started as root. This problem has been fixed.

\item Condor 6.9.2 introduced a problem where the \Condor{master}
would fail if started as root with the UID part of the
\Macro{CONDOR\_IDS} parameter set to 0 (root). This issue has been
fixed.

\end{itemize}

\noindent Known Bugs:

\begin{itemize}

\item None.

\end{itemize}

%%%%%%%%%%%%%%%%%%%%%%%%%%%%%%%%%%%%%%%%%%%%%%%%%%%%%%%%%%%%%%%%%%%%%%
\subsection*{\label{sec:New-6-9-2}Version 6.9.2}
%%%%%%%%%%%%%%%%%%%%%%%%%%%%%%%%%%%%%%%%%%%%%%%%%%%%%%%%%%%%%%%%%%%%%%

\noindent Release Notes:

\begin{itemize}

%% This is important (and thus, I believe, worth of being a top
%% level release note) because it will surprise anyone upgrading
%% an existing pool or repackaging Condor binaries (say, for
%% custom glideins, or as .deb packages for a local pool.)
% This is part of the privilege seperation work, but the procd
% is required even if you're not turning privsep on.
% Questions should go to the privsep team: psilord, zmiller, etc.
\item As part of ongoing security enhancements, Condor now has a
new, required daemon: \Condor{procd}.  This daemon is
automatically started by the \Condor{master}, you do not need to
add it to \Macro{DAEMON\_LIST}.  
However, you must be certain to update the \Condor{master}
if you update any of the other Condor daemons.
%Commented out the below since the defaults are in the code.
%New installations should not
%need to do anything; the default configuration file is correctly
%set.  Installations upgrading to 6.9.2 from previous versions
%will need to ensure several things are done.  
%1. Be sure to install \Condor{procd} into your Condor \Macro{SBIN} directory. 
%2. Add ``\Code{PROCD = \$(SBIN)/condor\_procd}'' to your Condor configuration. 
%3. Add ``\Code{PROCD\_ADDRESS = \$(LOCK)/procd\_pipe}'' to your Condor configuration. 
%On Windows there are two additional steps:
%4. Be sure to install \Condor{softkill} into your Condor \Macro{SBIN} directory. 
%5. Add ``\Code{WINDOWS\_SOFTKILL = \$(SBIN)/condor\_softkill}'' to your Condor configuration. 

% This isn't quite so important, but it's not really a feature or
% a bug, just a change.  It is a change that may surprise some
% users.  The full list of settings impacted is
% pretty long.  So far the below is just a small fraction,
% primarily being added because an external user was suprised by
% this when testing a 6.9.2-prerelease. For anyone curious or
% inspired to flesh out the list, here's the checkin that caused
% this:
% http://bonsai.cs.wisc.edu/bonsai/cvsquery.cgi?who=danb&whotype=match&sortby=Date&date=explicit&mindate=02%2F23%2F2007+19%3A15&maxdate=02%2F23%2F2007+19%3A30
% (To do the search yourself, search for checkins by danb between
% 02/23/2007 19:15 and 02/23/2007 19:30 )
% To determine if a variable is impacted, look at the removed
% code and confirm that it used the default if the setting was 0.
% Then if the new code sets a minimum of 1 (the third argument to
% param_integer), it's impacted.
\item Some configuration settings that previously accepted 0 no
  longer do so.  Instead the daemon using the setting will exit
  with an error message listing the acceptable range to its log.
  For these settings 0 was equivalent to requesting the default.
  As this was undocumented and confusing behavior it is no longer
  present.  To request a setting use its default, either comment it
  out, or set it to nothing (``\Code{EXAMPLE\_SETTING=}'').
  Setting impacted include but are not limited to: 
  % From condor_master.V6/master.C 1.82 to 1.83:
  \Macro{MASTER\_BACKOFF\_CONSTANT},
  \Macro{MASTER\_BACKOFF\_CEILING},
  \Macro{MASTER\_RECOVER\_FACTOR},
  \Macro{MASTER\_UPDATE\_INTERVAL},
  \Macro{MASTER\_NEW\_BINARY\_DELAY},
  \Macro{PREEN\_INTERVAL},
  \Macro{SHUTDOWN\_FAST\_TIMEOUT},
  \Macro{SHUTDOWN\_GRACEFUL\_TIMEOUT},
  % From condor_master.V6/daemon.C 1.68 to 1.69:},
  \Macro{MASTER\_<name>\_BACKOFF\_CONSTANT},
  \Macro{MASTER\_<name>\_BACKOFF\_CEILING},

\item Version 1.4.1 of the Generic Connection Broker (GCB) is
  now used for building Condor.  This version of GCB fixes a timing bug
  where a client may incorrectly think a network connection has been established,
  and also guards against an unresponsive client from causing a denial of
  service by the broker.
  For more information about GCB, see section~\ref{sec:GCB} on
  page~\pageref{sec:GCB}. 

% I'm checking this in commented since I'm not surewhat disclosure
% policy we want to use. Only CDF (Igor) uses the GLEXEC_STARTER
% functionality, so I think it'd be wise to run it by him before
% documenting this publicly.
%
%\item Fixed a security vulnerability in the \Macro{GLEXEC\_STARTER}
%feature. In previous versions when the \Condor{startd} received the
%user proxy, it placed it in a temporary file that for a short window
%of time could be opened for reading by any user on the system.

\end{itemize}


\noindent New Features:

\begin{itemize}
\item On UNIX, an execute-side Condor installation can run without
root privileges and still execute jobs as different users, properly
clean up when a job exits, and correctly enforce policies specified by
the Condor administrator and resource owners. Privileged functionality
has been separated into a well-defined set of functions provided by a
setuid helper program. This feature currently does not work for the
standard or PVM universes.

%%\item added bogus ImageSize to bogus dedicated scheduler
%%jobAd used only for claiming.  This fixes some problems with
%%startd WANT_SUSPEND going to undefined, but we don't document
%%this bogus ad anywhere, so I'm not going to add it here.

\item Added support for EmailAttributes in the parallel universe.  
Previously, it was only valid in the vanilla and standard universes.

\item Added configuration parameter \Macro{DEDICATED\_SCHEDULER\_USE\_FIFO}
which defaults to true.  When false, the dedicated scheduler will
use a best-fit algorithm to schedule parallel jobs.  This setting is
not recommended, as it can cause starvation.  When true, the dedicated
scheduler will schedule jobs in a first-in, first-out manner.

\item Added \Opt{-dump} to \Condor{config\_val} which will print out
all of the macros defined in any of the configuration files found by
the program.
\Condor{config\_val} \Opt{-dump} \Opt{-v} will augment the output
with exactly what line and in what file each configuration variable
was found.
\Note: The output format of the \Opt{-dump} option will most likely
change in a future revision of Condor.

% Gnats PR 671
\item Node names in \Condor{dagman} DAG files can now be DAG
keywords, except for PARENT and CHILD.

\item Improved the log message when \Attr{OnExitRemove} or
\Attr{OnExitHold} evaluates to UNDEFINED.

% Gnats PR 796
\item Added the \Macro{DAGMAN\_ON\_EXIT\_REMOVE} configuration macro,
which allows customization of the \Attr{OnExitRemove} expression
generated by \Condor{submit\_dag}.

\item When using GCB, Condor can now be told to choose from a list of
brokers. \Macro{NET\_REMAP\_INAGENT} is now a space and comma separated
list of brokers. On start up, the \Condor{master} will query all of the
brokers and pick the least-used one for it and its children to use. If
none of the brokers are operational, then the \Condor{master} will wait
until one is working. This waiting can be disabled by setting 
\Macro{MASTER\_WAITS\_FOR\_GCB\_BROKER} to FALSE in the configuration file.
If the chosen broker fails and recovery is not possible or another broker
is available, the \Condor{master} will restart all of the daemons.

\item When using GCB, communications between parent and child
Condor daemons on the same host no longer use the GCB broker.
This improves scalability and also allows a single host to
continue functioning if the GCB broker is unavailable.

\item The \Condor{schedd} now uses non-blocking methods to send the
``alive'' message to the \Condor{startd} when renewing the job lease.
This prevents the \Condor{schedd} from blocking for 20 seconds while
trying to connect to a machine that has become disconnected from the
network.

\item \Condor{advertise} can read the classad to be advertised from
standard input.

\item Unix Condor daemons now reinitialize their DNS
configuration (e.g. IP addresses of the name servers) on reconfig.

% Gnats PR 777
\item A configuration file for \Condor{dagman} can now be specified
in a DAG file or on the \Condor{submit\_dag} command line.

\item Added \Condor{cod} option \Opt{-lease} for creation of COD claims
with a limited duration lease.  This provides automatic cleanup of COD
claims that are not renewed by the user.  The default lease is infinitely
long, so existing behavior is unchanged unless \Opt{-lease} is explicitly
specified.

\item Added \Condor{cod} command \Opt{delegate\_proxy} which will
delegate an x509 proxy to the requested COD claim.
This is primarily useful for sites wishing to use glexec to spawn the
\Condor{starter} used for COD jobs.
The new command optionally takes an \Opt{-x509proxy} argument to
specify the proxy file.
If this argument is not present, \Condor{cod} will search for the
proxy using the same logic as \Condor{submit} does.

% This is barely a feature, but it's definitely not a bug fix. It's
% more of a change in behavior.
\item \Macro{STARTD\_DEBUG} can now be empty, indicating a default, minimal
log level. It now defaults to empty.
Previously it had to be non-empty and defaulted to include D\_COMMAND.

\item The addition of the \Condor{procd} daemon means that all process
family monitoring and control logic is no longer replicated in each
Condor daemon that needs it. This improves Condor's scalability,
particularly on machines with many processes.

\end{itemize}

\noindent Bugs Fixed:

\begin{itemize}

\item Under various circumstances, condor 6.9.1 daemons would abort
with the message, ``ERROR: Unexpected pending status for fake message
delivery.''  A specific example is when \Attr{OnExitRemove} or
\Attr{OnExitHold} evaluated to UNDEFINED.  This caused the
\Condor{schedd} to abort.

\item In Condor 6.9.1, the \Condor{schedd} would die during startup
when trying to reconnect to running jobs for which the \Condor{schedd}
could not find a startd ClassAd.  This would happen shortly after
logging the following message: ``Could not find machine ClassAds for
one or more jobs.  May be flocking, or machine may be down.
Attempting to reconnect anyway.''

\item Improved Condor's validity checking of configuration values.
For example, in some cases where Condor was expecting an integer but
was given an expression such as 12*60, it would silently interpret
this as 12.  Such cases now result in the condor daemon exiting
after issuing an error message into the log file.

\item When sending a \Code{WM\_CLOSE} message to a process on Windows,
Condor daemons now invoke the helper program \Condor{softkill} to do
so. This prevents the daemon from needing to temporarily switch away
from its dedicated service Window Station and Desktop. It also fixes a
bug where daemons would leak Window Station and Desktop handles. This
was mainly a problem in the \Condor{schedd} when running many scheduler
universe jobs.

\end{itemize}

\noindent Known Bugs:

\begin{itemize}

\item \Condor{glidein} generates a default config file that sets
\Macro{PREEN\_INTERVAL} to an invalid value (0).  To fix this,
remove the setting of \MacroNI{PREEN\_INTERVAL}.

\item There are a couple of known issues with Condor's
\Macro{GLEXEC\_STARTER} feature when used in conjunction with
COD. First, the \Condor{cod} tool invoked with the
\Opt{delegate\_proxy} option will sometimes incorrectly report that the
operation has failed. In addition, the \MacroNI{GLEXEC\_STARTER}
feature will not work properly with COD unless the UID that the each
COD job runs as is different than the UID of the opportunistic job or
any other COD jobs that are running on the execute machine when the
COD claim is activated.

\item The \Macro{EXECUTE\_LOGIN\_IS\_DEDICATED} feature has been found
to be broken on UNIX platforms. Its use will cause the \Condor{procd}
to crash, bringing down the other Condor daemons with it.

\end{itemize}



%%%%%%%%%%%%%%%%%%%%%%%%%%%%%%%%%%%%%%%%%%%%%%%%%%%%%%%%%%%%%%%%%%%%%%
\subsection*{\label{sec:New-6-9-1}Version 6.9.1}
%%%%%%%%%%%%%%%%%%%%%%%%%%%%%%%%%%%%%%%%%%%%%%%%%%%%%%%%%%%%%%%%%%%%%%

\noindent Release Notes:

\begin{itemize}

\item The 6.9.1 release contains all of the bug fixes and enhancements
  from the 6.8.x series up to and including version 6.8.3.

\item Version 1.4.0 of the Generic Connection Broker (GCB) library is
  now used for building Condor, and it is the 1.4.0 versions of the
  \Prog{gcb\_broker} and \Prog{gcb\_relay\_server} programs that are
  included in this release.
  This version of GCB includes enhancements used by Condor
  along with a new GCB-related command-line tool:
  \Prog{gcb\_broker\_query}.
  Condor 6.9.1 will not work properly with older versions of the
  \Prog{gcb\_broker} or \Prog{gcb\_relay\_server}.
  For more information about GCB, see section~\ref{sec:GCB} on
  page~\pageref{sec:GCB}. 

\end{itemize}

\noindent New Features:

\begin{itemize}

\item Improved the performance of the ClassAd matching algorithm,
which speeds up the \Condor{schedd} and other daemons.

\item Improved the scalability of the algorithm used by 
the \Condor{schedd} daemon to find runnable jobs.
This makes a noticeable difference in \Condor{schedd} daemon performance,
when there are on the order of thousands of jobs in the queue.

\item the \Dflag{COMMAND} debugging level has been enhanced to
log many more messages. 

\item Updated the version of DRMAA, which contains several great
improvements regarding scalability and race conditions.

% Gnats PR 774
\item Added the \Macro{DAGMAN\_SUBMIT\_DEPTH\_FIRST} configuration macro,
which causes \Condor{dagman} to submit ready nodes in more-or-less depth-first
order, if set to \Expr{True}.  The default behavior is to submit
the ready nodes in breadth-first order.

\item Added configuration parameter \Macro{USE\_PROCESS\_GROUPS}.
If it is set to \Expr{False},
then Condor daemons on Unix machines will not create new 
sessions or process groups. This is intended for use with Glidein, as
we have had reports that some batch systems cannot properly track jobs that
create new process groups. The default value is \Expr{True}.

\item The default value for the submit file command
\SubmitCmd{copy\_to\_spool} has been changed to \Expr{False}, because copying
the executable to the spool directory for each job (or job cluster) is almost
never desired.  Previously, the default was \Expr{True} in all
cases, except for grid universe jobs and remote submissions.

\item More types of file transfer errors now result in the job going
on hold, with a specific error message about what went wrong.  The new
cases involve failures to write output files to disk on the submit
side (for example, when the disk is full).
As always, the specific error number is
recorded in \Attr{HoldReasonSubCode}, so you can enforce an automated
error handling policy using \SubmitCmd{periodic\_release} or
\SubmitCmd{periodic\_remove}.

\item Added the \Macro{<SUBSYS>\_DAEMON\_AD\_FILE}
configuration variable, which is similar to the 
\Macro{<SUBSYS>\_ADDRESS\_FILE}.
This new variable will be used in future versions of Condor, but is
not necessary for 6.9.1.


\end{itemize}

\noindent Bugs Fixed:

\begin{itemize}

\item Fixed a bug in the \Condor{master} so that it will now send obituary
e-mails when it kills child processes that it considers hung.

\item \Condor{configure} used to always make a personal Condor with
\Opt{--install} even when \Opt{--type} called for only execute or
submit types.  Now, \Condor{configure} honors the \Opt{--type}
argument, even when using \Opt{--install}.
If \Opt{--type} is not specified, the default is to still install a
full personal Condor with the following daemons: 
\Condor{master}, \Condor{collector},
\Condor{negotiator}, \Condor{schedd}, \Condor{startd}. 

\item While removing, putting on hold, or vacating a large number of
jobs, it was possible for the \Condor{schedd} and the \Condor{shadow} to
temporarily deadlock with each other.  This has been fixed under Unix,
but not yet under Windows.

\item Communication from a \Condor{schedd} to a \Condor{startd}
now occurs in a nonblocking manner.
This fixes the problem of the \Condor{schedd} blocking 
when the claimed machine running the \Condor{startd}
cannot be reached, for example because the machine is turned off.

\end{itemize}

\noindent Known Bugs:

\begin{itemize}

\item Under various circumstances, condor 6.9.1 daemons abort
with the message, ``ERROR: Unexpected pending status for fake message
delivery.''  A specific example is when \Attr{OnExitRemove} or
\Attr{OnExitHold} evaluated to UNDEFINED, which causes the
\Condor{schedd} to abort.

\item In Condor 6.9.1, the \Condor{schedd} will die during startup
when trying to reconnect to running jobs for which the \Condor{schedd}
can not find a startd ClassAd.  This happens shortly after
logging the following message: ``Could not find machine ClassAds for
one or more jobs.  May be flocking, or machine may be down.
Attempting to reconnect anyway.''

\end{itemize}

%%%%%%%%%%%%%%%%%%%%%%%%%%%%%%%%%%%%%%%%%%%%%%%%%%%%%%%%%%%%%%%%%%%%%%
\subsection*{\label{sec:New-6-9-0}Version 6.9.0}
%%%%%%%%%%%%%%%%%%%%%%%%%%%%%%%%%%%%%%%%%%%%%%%%%%%%%%%%%%%%%%%%%%%%%%

\noindent Release Notes:

\begin{itemize}

\item The 6.9.0 release contains all of the bug fixes and enhancements
  from the 6.8.x series up to and including version 6.8.2.

% and a few \condor{gridmanager} bug fixes from 6.8.3.  *sigh* we need
% a real solution to this problem (like pointing to issue node ids) ;)

\end{itemize}


\noindent New Features:

\begin{itemize}


\item Preliminary support for using \Prog{glexec} on execute machines
has been added.  This feature causes the \Condor{startd} to spawn the
\Condor{starter} as the user that \Prog{glexec} determines based on
the user's GSI credential.

\item A ``per-job history files'' feature has been added to the
\Condor{schedd}. When enabled, this will cause the \Condor{schedd} to
write out a copy of each job's ClassAd when it leaves the job
queue. The directory to place these files in is determined by the
parameter \Macro{PER\_JOB\_HISTORY\_DIR}. It is the responsibility of
whatever external entity (for example, an accounting or monitoring system) is
using these files to remove them as it completes its processing.

\item \Condor{chirp} command now supports writing messages to the user log.

\item \Condor{chirp} getattr and putattr now send all classad getattr
and putattr commands to the proc 0 classad, which allows multiple proc
parallel jobs to use proc 0 as a scratch pad.

\item Parallel jobs now support an \Attr{AllRemoteHosts} attribute,
which lists all the hosts across all procs in a cluster.

\item The \Macro{DAGMAN\_ABORT\_DUPLICATES} configuration macro (which causes
\Condor{dagman} to abort itself if it detects another \Condor{dagman}
running on the same DAG) now defaults to \Expr{True} instead of
\Expr{False}.

\end{itemize}

\noindent Bugs Fixed:

\begin{itemize}

\item None.

\end{itemize}

\noindent Known Bugs:

\begin{itemize}

\item None.

\end{itemize}

