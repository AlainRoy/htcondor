\begin{ManPage}{\label{man-condor-store-cred}\Condor{store\_cred}}{1}
{securely stash user's password}
\Synopsis
\SynProg{\Condor{store\_cred}}
\oOpt{-help}

\SynProg{\Condor{store\_cred}}
\Arg{add}
\Lbr
\Opt{-c} \verb@|@ \OptArg{-u}{username}
\Rbr 
\oOptArg{-p}{password}
\oOptArg{-n}{machinename}
\oOptArg{-f}{filename}

\SynProg{\Condor{store\_cred}}
\Arg{delete}
\Lbr
\Opt{-c} \verb@|@ \OptArg{-u}{username}
\Rbr 
\oOptArg{-n}{machinename}

\SynProg{\Condor{store\_cred}}
\Arg{query}
\Lbr
\Opt{-c} \verb@|@ \OptArg{-u}{username}
\Rbr 
\oOptArg{-n}{machinename}

\index{Condor commands!condor\_store\_cred}
\index{condor\_store\_cred command}

\Description 

On a Windows machine, \Condor{store\_cred} stores the password
of a user/domain pair securely in the Windows registry.
Using this stored password, Condor is able to
run jobs with the user ID of the submitting user.
In addition, Condor uses this password to
acquire the submitting user's credentials when writing output or log
files. The password is stored in the same manner as the system does when
setting or changing account passwords.
When \Condor{store\_cred} is invoked, it contacts the \Condor{schedd}
daemon to carry out the requested operations on behalf of the user.
This is necessary since registry keys are accessible only by the Windows
SYSTEM account, not by administrators or other users.

On a Unix machine, \Condor{store\_cred} is used to manage
the pool password,
placed in a file specified by the \MacroNI{SEC\_PASSWORD\_FILE}
configuration variable, and for use in password authentication
among Condor daemons.


The password is stashed in a persistent manner; it is maintained
across system reboots.

The \Arg{add} argument stores the current user's password securely
in the registry. The user is prompted to enter the password
twice for confirmation, and characters are not echoed.	If there
is already a password stashed, the old password will be
overwritten by the new password.

The \Arg{delete} deletes the current password,
if it exists.

The \Arg{query} reports whether the password is stored or not.

\begin{Options}

  \OptItem{\Opt{-c}}{Apply the option to the pool password.}

  \OptItem{\OptArg{-f}{filename}}{For Unix machines only,
  generates a pool password file named \Arg{filename} that may be used
  with the PASSWORD authentication method.}

  \OptItem{\Opt{-help}}{Displays a brief summary of command options.}

  \OptItem{\OptArg{-n}{machinename}}{Apply the command on the
  given machine.}

  \OptItem{\OptArg{-p}{password}}{Stores given password,
  rather than prompting.}

  \OptItem{\OptArg{-u}{username}}{Specify the user name.}

\end{Options}

\ExitStatus

\Condor{store\_cred} will exit with a status value of 0 (zero) upon success,
and it will exit with the value 1 (one) upon failure.

\end{ManPage}
