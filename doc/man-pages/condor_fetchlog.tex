\begin{ManPage}{\label{man-condor-fetchlog}\Condor{fetchlog}}{1}
{Retrieve a log file that is located on another computer}

\Synopsis 
\SynProg{\Condor{fetchlog}}
\ToolArgsBase

\SynProg{\Condor{fetchlog}}
\oOptArg{-pool}{centralmanagerhostname[:portnumber]}
\ToolArgsAffect
\Arg{machine-name}
\Arg{log-file-name}

\index{Condor commands!condor\_fetchlog}
\index{condor\_fetchlog command}

\Description 

\Condor{fetchlog} contacts Condor running on the machine specified
by \Arg{machine-name}, and asks it
to return a log file from that machine. This simplifies the
retrieval of log file by bypassing the need 
to log on to the machine in order to get the log file. 

For security purposes of authentication and authorization, 
this command requires an administrator's level of access.
See section~\ref{sec:Config-Security} 
on page~\pageref{sec:Config-Security} for more details about Condor's
security mechanisms.

The \Arg{log-file-name} argument identifies the log file to retrieve.
It is defined by the name
given in the configuration file, without the "\_LOG" suffix.
For \MacroNI{COLLECTOR\_LOG}, it is \MacroNI{COLLECTOR}.
The complete list of log files are defined by
\begin{verbatim}
  MASTER
  COLLECTOR
  NEGOTIATOR
  NEGOTIATOR_MATCH
  SCHEDD
  SHADOW
  STARTD
  STARTER
\end{verbatim}

\begin{Options}
    \ToolArgsBaseDesc
    \OptItem{\OptArg{-pool}{centralmanagerhostname[:portnumber]}}
    {Specify a pool by giving the central manager's host name
    and an optional port number}
    \ToolArgsAffectDesc
\end{Options}

\Examples
To get the \Condor{negotiator} daemon's log from a host named 
\File{head.example.com} from within the current pool:
\begin{verbatim}
condor_fetchlog head.example.com NEGOTIATOR
\end{verbatim}

To get the \Condor{startd} daemon's log from a host named
\File{execute.example.com} from within the current pool:
\begin{verbatim}
condor_fetchlog execute.example.com STARTD
\end{verbatim}

This command requested the \Condor{startd} daemon's log from the
\Condor{master}.
If the \Condor{master} has crashed or is unresponsive,
ask another daemon
running on that computer to return the log.
For example, ask the \Condor{startd} daemon to return the
\Condor{master}'s log:

\begin{verbatim}
condor_fetchlog -startd execute.example.com MASTER
\end{verbatim}

\ExitStatus
\Condor{fetchlog} will exit with a status value of 0 (zero) upon success,
and it will exit with the value 1 (one) upon failure.

\end{ManPage}
