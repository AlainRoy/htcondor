\begin{ManPage}{\label{man-condor-reconfig}\Condor{reconfig}}{1}
{Reconfigure Condor daemons}
\Synopsis \SynProg{\Condor{reconfig}}
\ToolArgsBase

\SynProg{\Condor{reconfig}}
\ToolDebugOption
\ToolArgsLocate
\ToolArgsAffect
\oOptnm{-full}

\SynProg{\Condor{reconfig}}
\ToolDebugOption
\ToolWhere
\ToolArgsAffect
\oOptnm{-full}

\index{Condor commands!condor\_reconfig}
\index{condor\_reconfig command}

\Description 

\Condor{reconfig} reconfigures all of the condor daemons in accordance with 
the current
status of the Condor configuration file(s).  
Once reconfiguration is complete, the daemons will behave according to
the policies stated in the configuration file(s).
The main exception is with the DAEMON\_LIST variable, which will only be
updated if the \Condor{restart} command is used.  
There are a few other configuration settings that can only be changed
if the Condor daemons are restarted.
Whenever this is the case, it will be mentioned in
section~\ref{sec:Configuring-Condor} on
page~\pageref{sec:Configuring-Condor} which lists all of the settings
used to configure Condor. 
In general, \Condor{reconfig} should be used when making changes to
the configuration files, since it is faster and more efficient than
restarting the daemons.

The command 
\Condor{reconfig}
with no arguments or with the \Opt{-master} argument specifying
a daemon will cause the reconfiguration of the \Condor{master}
daemon and all the child processes of the \Condor{master}.

For security purposes (authentication and authorization),
this command requires an administrator's level of access.
Note that changes to the \MacroNI{ALLOW\_*} and
\MacroNI{DENY\_*} configuration variables require the 
\Opt{-full} option.
See
section~\ref{sec:Config-Security} on page~\pageref{sec:Config-Security}
for further explanation.

\begin{Options}
    \ToolArgsBaseDesc
    \OptItem{\Opt{-full}}{Perform a full reconfiguration.
    In addition to re-reading the configuration files,
    a full reconfiguration will clear
    cached DNS information in the daemons.
    Use this option only when the DNS information needs to
    be reinitialized.}
    \ToolDebugDesc
    \ToolArgsLocateDesc
\end{Options}

\ExitStatus

\Condor{reconfig} will exit with a status value of 0 (zero) upon success,
and it will exit with the value 1 (one) upon failure.

\Examples
To reconfigure the \Condor{master} and all its children
on the local host:
\begin{verbatim}
% condor_reconfig
\end{verbatim}

To reconfigure only the \Condor{startd} on a named machine:
\begin{verbatim}
% condor_reconfig -name bluejay -startd
\end{verbatim}

To reconfigure a machine within a pool
other than the local pool, use the \Opt{-pool} option.
The argument is the name of the central manager for the pool.
Note that one or more machines within the pool must be
specified as the targets for the command.
This command reconfigures
the single machine named \Opt{cae17} within the
pool of machines that has \Opt{condor.cae.wisc.edu} as
its central manager:
\begin{verbatim}
% condor_reconfig -pool condor.cae.wisc.edu -name cae17
\end{verbatim}

\end{ManPage}
