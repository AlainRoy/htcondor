\begin{ManPage}{\label{man-condor-config-bind}\Condor{config\_bind}}{1}
{bind together a set of configuration files}

\Synopsis \SynProg{\Condor{config\_bind}}
\Opt{-help}

\SynProg{\Condor{config\_bind}}
\OptArg{-o}{outputfile}
\Arg{configfile1}
\Arg{configfile2}
\oArg{configfile3\ldots}

\index{Condor commands!condor\_config\_bind}
\index{condor\_config\_bind command}

\Description

\Condor{config\_bind} dynamically binds two or more Condor
configuration files through the use of a new configuration file.  The
purpose of this tool is to allow the user to dynamically bind a local
configuration file into an already created, and possible immutable,
configuration file.  This is particularly useful when the user wants to
modify a configuration but cannot actually make any changes to the
global configuration file (even to change the list of local configuration
files).  This program does not modify the given configuration files.
Rather, it creates a new configuration file that specifies the given
configuration files as local configuration files.  

Condor evaluates each of the configuration files in the given
command-line order (left to right).
A value defined 
in two or more of the configuration files results in
the last one evaluated defining the value. It overrides any others.
To bind a new local configuration into a global configuration, 
specify the local configuration second within the command-line
ordering.

\begin{Options}
  \OptItem{\Arg{configfile1}}{First configuration file to
    bind.}
  \OptItem{\Arg{configfile2}}{Second configuration file to
    bind.} 
  \OptItem{\Arg{configfile3\ldots}}{An optional list of other
    configuration files to bind.}
  \OptItem{\Opt{-help}}{Display brief usage information and exit}
  \OptItem{\OptArg{-o}{output\_file}} {
    Specifies the file name where this program should output the
    binding configuration. 
  }
\end{Options}

\ExitStatus

\Condor{config\_bind} will exit with a status value of 0 (zero) upon
success, and non-zero on error.

\end{ManPage}
