\begin{ManPage}{\label{man-condor-run}\Condor{run}}{1}
{Submit a shell command-line as a Condor job.}
\Synopsis \SynProg{\Condor{run}}
\Arg{``shell-cmd''}

\index{Condor commands!condor\_run}
\index{condor\_run command}

\Description
\Condor{run} is a simple front-end to the \Condor{submit} command for
submitting a shell command-line as a vanilla universe Condor job.  The
\Condor{run} command waits for the Condor job to complete, writes the
job's output to the terminal, and exits with the exit status of the
Condor job.  No output will appear until the job completes.  The shell
command-line should be enclosed in quotes so it is passed directly to
\Condor{run} without modification by the invoking shell.

\Condor{run} will not read any input from the terminal while the job
executes.  If the shell command-line requires input, you must
explicitly redirect the input from a file to the command, as
illustrated in the example.

You can specify where \Condor{run} should execute the shell
command-line with three environment variables:

\begin{description}
\item[CONDOR\_ARCH] Specifies the architecture of the execution
machine (from the ``Arch'' field in the output of \Condor{status}).
\item[CONDOR\_OPSYS] Specifies the operating system of the execution
machine (from the ``OpSys'' field in the output of \Condor{status}).
\item[CONDOR\_REQUIREMENTS] Specifies any additional requirements for
the Condor job (as described in manual page for \Condor{submit} on
page~\pageref{man-condor-submit}).  It is recommended that
\Macro{CONDOR\_REQUIREMENTS} always be enclosed in parenthesis.
\end{description}

If one or more of these environment variables is specified, the job is
submitted with:

\begin{verbatim}
  requirements = $CONDOR_REQUIREMENTS && Arch == $CONDOR_ARCH && \
                 OpSys == $CONDOR_OPSYS
\end{verbatim}

Otherwise, the job receives the default requirements expression, which
requests a machine of the same architecture and operating system of
the machine on which \Condor{run} is executed.

All environment variables set when \Condor{run} is executed will be
included in the environment of the Condor job.

\Condor{run} will remove the Condor job from the Condor queue and
delete its temporary files if it is killed before the Condor job
finishes.

\Examples

\Condor{run} can be used to compile jobs on architectures and
operating systems to which the user doesn't have login access.  For example:

\begin{verbatim}
$ setenv CONDOR_ARCH "SGI"
$ setenv CONDOR_OPSYS "IRIX65"
$ condor_run "f77 -O -o myprog myprog.f"
$ condor_run "make"
$ condor_run "condor_compile cc -o myprog.condor myprog.c"
\end{verbatim}

Since \Condor{run} does not read input from the terminal, you must
explicitly redirect input from a file to the shell command.  For
example:

\begin{verbatim}
$ condor_run "myprog < input.dat > output.dat"
\end{verbatim}

\Files

\Condor{run} creates the following temporary files in the user's
working directory (replacing ``pid'' with \Condor{run}'s process id):
\begin{description}
\item[.condor\_run.pid] This is the shell script containing the shell
command-line which is submitted to Condor.
\item[.condor\_submit.pid] This is the submit file passed to
\Condor{submit}. 
\item[.condor\_log.pid] This is the Condor log file monitored by
\Condor{run} to determine when the job exits.
\item[.condor\_out.pid] This file contains the output of the Condor
job (before it is copied to the terminal).
\item[.condor\_error.pid] This file contains any error messages for the Condor
job (before they are copied to the terminal).
\end{description}
The script removes these files when the job completes.  However, if
the script fails, it is possible that these files will remain in the
user's working directory and the Condor job will remain in the queue.

\GenRem

\Condor{run} is intended for submitting simple shell command-lines to
Condor.  It does not provide the full functionality of
\Condor{submit}.  We have attempted to make \Condor{run} as robust as
possible, but it is possible that it will not correctly handle some
possible \Condor{submit} errors or system failures.

\Condor{run} jobs have the same restrictions as other vanilla universe
jobs.  Specifically, the current working directory of the job must be
accessible on the machine where the job runs.  This typically means
that the job must be submitted from a network file system such as NFS
or AFS.  Also, since Condor does not manage AFS credentials,
permissions must be set to allow unauthenticated processes to access
any AFS directories used by the Condor job.

All processes on the command-line will be executed on the machine
where Condor runs the job.  Condor will not distribute multiple
processes of a command-line pipe across multiple machines.

\Condor{run} will use the shell specified in the \Macro{SHELL} environment
variable, if one exists.  Otherwise, it will use \Cmd{/bin/sh} to execute
the shell command-line.

By default, \Condor{run} expects perl to be installed in
\File{/usr/bin/perl}.  If perl is installed in another path, you can
ask your Condor administrator to edit the path in the \Condor{run}
script or explicitly call perl from the command line:

\begin{verbatim}
$ perl [path-to-condor]/bin/condor_run "shell-cmd"
\end{verbatim}

\ExitStatus

\Condor{run} exits with a status value of 0 (zero) upon complete success.
The exit status of \Condor{run} will be non-zero upon failure.
The exit status in the case of a single error due to a system call
will be the error number (\Code{errno}) of the failed call.

\end{ManPage}
