\index{ClassAd!Negotiator attributes}
\begin{description}

\index{ClassAd Negotiator attribute!DaemonStartTime}
\item[\AdAttr{DaemonStartTime}:] The time that this daemon was
  started, represented as the number of second elapsed since
    the Unix epoch (00:00:00 UTC, Jan 1, 1970).

\index{ClassAd Negotiator attribute!Machine}
\item[\AdAttr{Machine}:] A string with the machine's fully qualified 
  host name.

\index{ClassAd Negotiator attribute!MyAddress}
\item[\AdAttr{MyAddress}:] Description is not yet written.

\index{ClassAd Negotiator attribute!MyCurrentTime}
\item[\AdAttr{MyCurrentTime}:]  The time, represented as the number of 
  second elapsed since the Unix epoch (00:00:00 UTC, Jan 1, 1970),
  at which the \Condor{schedd} daemon last sent a ClassAd update to the
  \Condor{collector}.

\index{ClassAd Negotiator attribute!Name}
\item[\AdAttr{Name}:] The name of this resource; typically the same value as
  the \AdAttr{Machine} attribute, but could be customized by the site
  administrator.
  On SMP machines, the \Condor{startd} will divide the CPUs up into separate
  slots, each with with a unique name.
  These names will be of the form ``slot\#@full.hostname'', for example,
  ``slot1@vulture.cs.wisc.edu'', which signifies slot number 1 from
  vulture.cs.wisc.edu.

\index{ClassAd Negotiator attribute!NegotiatorIpAddr}
\item[\AdAttr{NegotiatorIpAddr}:] String with the IP and port address of the
\Condor{negotiator} daemon which is publishing this Negotiator ClassAd.

\index{ClassAd Negotiator attribute!PublicNetworkIpAddr}
\item[\AdAttr{PublicNetworkIpAddr}:] Description is not yet written.

\index{ClassAd Negotiator attribute!UpdateSequenceNumber}
\item[\AdAttr{UpdateSequenceNumber}:] An integer, starting at zero,
  and incremented with each ClassAd update sent to the \Condor{collector}.
  The \Condor{collector} uses this value to sequence the updates it
  receives.

\end{description}

