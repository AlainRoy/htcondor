%%%%%%%%%%%%%%%%%%%%%%%%%%%%%%%%%%%%
\subsection{\label{API-WebService} Web Service}
%%%%%%%%%%%%%%%%%%%%%%%%%%%%%%%%%%%%
\index{API!Web Services}
\index{SOAP!Web Service API}
\index{Simple Object Access Protocol(SOAP)}

Condor daemons understand and implement
the SOAP (Simple Object Access Protocol) XML API
to provide a web service interface for Condor job submission
and management.

The API utilizes a two-phase commit mechanism to provide
a transaction-based protocol.  
This structure enhances reliability when using the API.

%%%%%%%%%%%%%%%%%%%%%%%%%%%%%%%%%%%%
\subsubsection{\label{WebService-Implementation} Implementation Details}
%%%%%%%%%%%%%%%%%%%%%%%%%%%%%%%%%%%%

Condor daemons understand and communicate using the
SOAP XML protocol.
An application seeking to use this protocol
will require code that handles the communication.
The XML WSDL (Web Services Description Language)
that Condor implements is included with the
Condor distribution.
It is in \File{\MacroUNI{RELEASE\_DIR}/lib/webservice}.
The WSDL must be run through a toolkit to produce
language-specific routines that do communication.
The application is compiled with these routines.

Condor must be configured to enable responses to SOAP calls.
Please see 
section~\ref{sec:API-Config-File-Entries} for definitions of the
configuration variables related to the web services API.

The API's routines can be roughly categorized into ones that
deal with
\begin{itemize}
  \item Transactions
  \item Job Submission
  \item File Transfer
  \item Job Management
\end{itemize}
The routines for each of these categories is detailed.
Note that the signature provided will accurately 
reflect a routine's name, 
but that return values and parameter specification
will vary according  to the target programming language.

%%%%%%%%%%%%%%%%%%%%%%%%%%%%%%%%%%%%
\subsubsection{\label{WebService-Transactions} Methods for Transaction Management}
%%%%%%%%%%%%%%%%%%%%%%%%%%%%%%%%%%%%

\begin{description}
\item [\Code{StatusAndTransaction beginTransaction(int duration)}]
  Begin a transaction.
\item [\Code{Status commitTransaction(Transaction transaction)}]
  Commits a transaction.
\item [\Code{Status abortTransaction(Transaction transaction)}]
  Abort a transaction.
\item [\Code{StatusAndTransaction extendTransaction(Transaction transaction, int duration)}]
  Request an extension in duration for a specific transaction.
\end{description}

%%%%%%%%%%%%%%%%%%%%%%%%%%%%%%%%%%%%
\subsubsection{\label{WebService-Submission} Methods for Job Submission}
%%%%%%%%%%%%%%%%%%%%%%%%%%%%%%%%%%%%

\begin{description}
\item [\Code{Status submit(Transaction transaction, int clusterId, int jobId, ClassAd jobAd)}]
  Submit a job.
\item [\Code{StatusAndClassAd createJobTemplate(int clusterId, int jobId, String owner, UniverseType type, String command, String arguments, String requirements)}]
  Request a job Class Ad, given some of the job requirements.
  This job Class Ad will be suitable for use when submitting the job.

%%    StatusAndRequirements discoverJobRequirements(ClassAd jobAd)

\end{description}

%%%%%%%%%%%%%%%%%%%%%%%%%%%%%%%%%%%%
\subsubsection{\label{WebService-FileTransfer} Methods for File Transfer}
%%%%%%%%%%%%%%%%%%%%%%%%%%%%%%%%%%%%

\begin{description}
\item [\Code{Status declareFile(Transaction transaction, int clusterId, int jobId, String name, int size, HashType hashType, String hash)}]
  Declare a file that may be used by a job.
\item [\Code{Status sendFile(Transaction transaction, int clusterId, int jobId, String name, int offset, Base64 data)}]
  Send a file that a job may use.
\item [\Code{StatusAndBase64 getFile(Transaction transaction, int clusterId, int jobId, String name, int offset, int length)}]
  Get a file from a job's spool.
  Does not need to occur in a transaction.
\item [\Code{Status closeSpool(Transaction transaction, int clusterId, int jobId)}]
  Close a job's spool.
  Does not need to occur in a transaction.
  All the files in the job's spool can be deleted. 
  %%  deleted before or after this method returns/is invoked?
\item [\Code{StatusAndFileInfoArray listSpool(Transaction transaction, int clusterId, int jobId)}]
  List the files in a job's spool.
  Does not need to occur in a transaction.
\end{description}

%%%%%%%%%%%%%%%%%%%%%%%%%%%%%%%%%%%%
\subsubsection{\label{WebService-JobManagement} Methods for Job Management}
%%%%%%%%%%%%%%%%%%%%%%%%%%%%%%%%%%%%

\begin{description}
\item [\Code{StatusAndInteger newCluster(Transaction transaction)}]
  Create a new job cluster.
\item [\Code{Status removeCluster(Transaction transaction, int clusterId, String reason)}]
  Remove a job cluster, and all the jobs within it.
  %% What does it mean within Condor to remove a cluster?
  Does not need to occur in a transaction.
\item [\Code{StatusAndInteger newJob(Transaction transaction, int clusterId)}]
  Creates a new job within the most recently created job cluster.
  %% Why pass the clusterId if there's only 1 that we can use?
  %% Is this to mimic the behaviour of commands within a submit file
  %%     WRT the queue command?
\item [\Code{Status removeJob(Transaction transaction, int clusterId, int jobId, String reason, boolean forceRemoval)}]
  Remove a job, regardless of the job's state.
  Does not need to occur in a transaction.
\item [\Code{Status holdJob()}]
  %% Needs parameters.
  Put a job into the Hold state, regardless of the job's current state.
  Does not need to occur in a transaction.
\item [\Code{Status releaseJob(Transaction transaction, int clusterId, int jobId, String reason, boolean emailUser, boolean emailAdmin)}]
  Release a job that has been in the Hold state.
  Does not need to occur in a transaction.

\item [\Code{StatusAndClassAdArray getJobAds(Transaction transaction, String constraint)}]
  Find all the job ClassAds matching the given constraint.
  %% Find them from where?  The API's set of cluster/jobs or Condor's?
  Does not need to occur in a transaction.
\item [\Code{StatusAndClassAd getJobAd(Transaction transaction, int clusterId, int jobId)}]
   Find a specific job ClassAd. 
   This method does much the same as the first element from the array 
   returned by

\footnotesize
\begin{verbatim}
getJobAds(transaction, "(ClusterId==clusterId && JobId==jobId)")
\end{verbatim}
\normalsize


\item [\Code{Status  requestReschedule()}]
   Request a \Condor{reschedule} from the \Condor{schedd} daemon.
\end{description}

%% Not yet in this list:
